
\chapter{Introduction}

This document descibes the installation, configuration, execution and 
usage of the Polar Wind Outflow Model (PWOM).

Software usage.

\section{Acknowledgments}

The PWOM was developed at the Center for Space Environment Modeling
(CSEM) of the University of Michigan. 
The core software designers and code developers: Alex Glocer, G\'abor T\'oth

\section{The PWOM in a Few Paragraphs}

The Polar Wind Outflow Model (PWOM), solves the gyrotropic continuity, 
momentum and energy equations that describe the supersonic ion outflow 
along open magnetic field lines in the polar region. The PWOM can simulate 
the polar winds of Earth and Saturn. At Earth the behavior of three ion 
species, O$^{+}$, H$^{+}$, and He$^{+}$ are considered, while at Saturn 
only two species, H$^{+}$ and H$_{3}^{+}$, are considered. The model assumes 
a stationary neutral atmosphere. Ranging in altitude from 250 km to 8,000 km 
for the Earth version, or 1,400 to 61,000 km for the Saturn version, the PWOM 
has its lower boundary set in a reservoir at chemical and thermal equilibrium, 
while the top boundary is at considerably lower pressure, thus creating a 
transonic outflow to a low pressure external medium. The considerable 
altitude range covers two different regimes; the collision and chemistry 
dominated low altitude, and the expansion dominated high altitude. 
Furthermore, the ambipolar electric field is calculated at every time step, 
and is a major contributor to ion outflow. Other physical effects that are 
included in the PWOM, are topside electron heating, photo-ionization, and the 
expanding cross-sectional area of the magnetic flux tube.

Obtaining a single field line solution yields the vertical solution at one 
point. The field line convects as determined from the polar cap potential. 
By solving several field lines distributed  throughout the polar cap, 
the full solution can be obtained. Since each field line solution is 
independent, this problem is embarassingly parallel. Accordingly, 
we use the Message Passing Interface (MPI) to split the computational work 
evenly among processors. 

The PWOM can be compiled into standalone or coupled mode. In the standalone 
mode, the polar cap potential is read from a file or calculated from the 
Weimer empirical model, and the polar wind solution is obtained independantly 
of other models.In the coupled mode, the PWOM interacts with the SWMF 
via a coupling interface known as a wrapper. In this case the polar 
cap potential is obtained from the IE component, and the fluxes at the top 
of the PWOM are passed to the GM component. Most of this manual is devoted to 
the standalone operation of the PWOM, but chapter \ref{SWMF} discusses some 
details on how to utilize the coupled model.



The PWOM has been tested on the SGI Altix, on Linux Beowulf clusters with the 
NAG f95 compiler. We have also run the PWOM with reasonable success under
Mac OS Darwin using the XLF and NAG f95 compilers, and under Linux with
the PGF90 compiler.

\section{System Requirements}

In order to install and run the PWOM the following minimum system
requirements apply.

\begin{itemize}
\item The PWOM runs only under the UNIX/Linux operating systems.  This now
  includes Macintosh system 10.x because it is based on BSD UNIX.  The
  PWOM does not run under any Microsoft Windows operating system.
\item A FORTRAN 90 compiler must be installed.
\item The Perl interpreter must be installed.
\item A version of the Message Passing Interface (MPI) library must be
  installed.
\item Small tests such as single  field-line simulations can be done on a 
  single processor
\item Very large runs require many more processors.
\item In order to generate the documentation you must have LaTex installed on
your system.  The PDF generation requires the {\tt dvips} and {\tt ps2pdf}
utilities. 

\end{itemize}


In addition to the above requirements, the PWOM output is designed to
be visualized using either IDL or Tecplot.  You may be able to
visualize the output with other packages, but formats and scripts have
been designed for only these two visualization softwares.


%-----------------------------------------------------------------------
% Chapter 2
%-----------------------------------------------------------------------

\chapter{Quick Start}

\section{A Brief Description of the PWOM Distribution}

The top level directory contains the following subdirectories:
\begin{itemize}
\item {\tt src}       - Source code of planet independent code
\item {\tt srcEarth}  - Source code of Earth specific code      
\item {\tt srcSaturn} - Source code of Saturn specific code 
\item {\tt Scripts}   - shell and Perl scripts
\item {\tt bin}       - location of executable
\item {\tt doc}       - the documentation directory 
\item {\tt share}     - shared scripts and source code
\item {\tt util}      - general utilities such as TIMING and NOMPI
\end{itemize}
and the following files
\begin{itemize}
\item {\tt README}        - a short instruction on installation and usage
\item {\tt Makefile}      - the main makefile
\item {\tt Makefile.test} - the makefile containing the tests 
\item {\tt Config.pl}     - Perl script for (un)installation and configuration
\end{itemize}

\section{General Hints}

\subsubsection{Getting help with scripts and the Makefile}

Most of the Perl and shell scripts that are distributed with the PWOM
provide help which can be accessed as follows using the {\tt -h} flag.
For example, 
\begin{verbatim}
  Config.pl -h
\end{verbatim}
will provide a detailed listing of the options and capabilities of the
{\tt Config.pl} script.  In addition, you can find all the possible
targets  that can be built by typing
\begin{verbatim}
make help
\end{verbatim}

\subsubsection{Input commands: PARAM.XML}
A very useful set of files to become familiar with are the {\tt PARAM.XML}
files.

This file contains a complete list of all input commands for the
component as well as the allowed ranges for each of the input parameters.
Although the XML format makes the files a little hard to read, they are
extremely useful.  A typical usage is to cut and paste commands out of the
PARAM.XML file into the PARAM.in file for a run.


\section{Installing the Code}

The first step in installing the PWOM is untarring the distribution.
If the tar program knows about the -z flag, you can open the gzipped
tar files with a single UNIX command:
\begin{verbatim}
  tar xzf SOMETARFILE.tgz
\end{verbatim}
If the tar program does not recognize the -z flag, two steps are needed:
\begin{verbatim}
  gunzip SOMETARFILE.tgz
  tar xf SOMETARFILE.tar
\end{verbatim}
In the following descriptions the shorter form is shown, but you may
need to use the two step procedure on certain platforms.

Untar the distribution using the command:
\begin{verbatim}
  tar xzf PWOM.tgz
\end{verbatim}

Change directories into the distribution:
\begin{verbatim}
  cd PWOM
\end{verbatim}

The PWOM needs to know what architecture you are running the code on
and what FORTRAN compiler will be used.  For most platforms and compilers,
it can figure this out all by itself. To install PWOM run the command:
\begin{verbatim}
  Config.pl -install
\end{verbatim}
in the main directory. This creates {\tt Makefile.def} with
the correct absolute path to the base directory and {\tt Makefile.conf}
which contains the operating system and compiler specific part of
the Makefile. If the compiler is not the default one for a given
platform (e.g. not the NAG f95 compiler for a Linux platform) then
the compiler must be specified explicitly with the {\tt -compiler}
flag. If the MPI header file is not the default one, it can be
specified with the {\tt -mpi} flag. For example on a Mac machine
one can select the g95 compiler with the Intel version of the 
MPI library using
\begin{verbatim}
  Config.pl -install -compiler=g95 -mpi=Intel
\end{verbatim}
To uninstall PWOM type
\begin{verbatim}
  Config.pl -uninstall
\end{verbatim}
If the uninstallation fails (this can happen if some makefiles are missing)
do reinstallation with
\begin{verbatim}
  Config.pl -install
\end{verbatim}
and then try uninstalling the code again.
To get a complete description of the {\tt Config.pl}  script type
\begin{verbatim}
  Config.pl -h
\end{verbatim}

\section{Creating Documentation}

The documentation for PWOM can be generated from the distribution by
the command
\begin{verbatim}
  make PDF
\end{verbatim}
which creates the user manual
\begin{verbatim}
  doc/PWOM.pdf
\end{verbatim}
In order for this to work you must have
LaTex installed on your system (and dvips and ps2pdf).  

To clean the intermediate files type
\begin{verbatim}
  cd doc/Tex
  make clean
\end{verbatim}
To remove all the created documentation type
\begin{verbatim}
  cd doc/Tex
  make cleanall
\end{verbatim}

\section{Building and Running an Executable}


Compilation flags, such as the precision and optimization 
level are stored in {\tt Makefile.conf}. This file is created on
installation of the PWOM and has defaults which are appropriate for
your system architecture.  The precision of reals
can be changed to single precision (for example) by typing
\begin{verbatim}
  Config.pl -single
\end{verbatim}
while the compiler flags can be modified with
\begin{verbatim}
  Config.pl -debug -O0
\end{verbatim}
to debug the code with 0 optimization level, and
\begin{verbatim}
  Config.pl -nodebug -O4
\end{verbatim}
to run the code at maximum optimization level and without the debugging flags.

To specify which planet the PWOM should be set to use 
\begin{verbatim}
  Config.pl -Earth or Config.pl -Saturn
\end{verbatim}


To build the executable {\bf bin/PWOM.exe}, type:
\begin{verbatim}
  make
\end{verbatim} 
Depending on the configuration, the compiler settings and the machine 
that you are compiling on, this can take from 1 to up to 5 minutes.  

The {\tt PWOM.exe} executable should be run in a sub-directory, since a 
large number of files are created in each run.  
To create this directory use thecommand:
\begin{verbatim}
  make rundir
\end{verbatim} 
This command creates a directory called {\tt run}.  You can either
leave this directory as named, or {\tt mv} it to a different name.  It
is best to leave it in the same PWOM directory, since
keeping track of the code version associated with each run is quite
important.  The {\tt run} directory will contain links to the codes
which were created in the previous step as well as subdirectories
where input and output will reside.

Here we assume that the {\tt run} directory is still called {\tt
run}:
\begin{verbatim}
  cd run
\end{verbatim}
In order to run the PWOM you must have the input files: 
pw.input. The pw.input 
file contains the detailed commands for controlling what you want the
code to do during the run.  The default pw.input 
file in the run directory  suitable to perform a simple test.

To run the PWOM interactively on four processors:
\begin{verbatim}
cd run
mpirun -np 4 PWOM.exe
\end{verbatim}

To recompile the executable with different compiler settings you have
to use the command
\begin{verbatim}
make clean
\end{verbatim}
before recompiling the executables. It is possible to recompile
only a component or just one subdirectory if the {\tt make clean}
command is issued in the appropriate directory.

\section{Restarting a Run}

There are several reasons for restarting a run. A run may fail
due to a run time error, due to hardware failure, due to 
software failure (e.g. the machine crashes) or because the
queue limits are exceeded. In such a case the run can be continued from
the last saved state of PWOM. 

The restart files are saved at the frequency determined in the pw.input file.
Normally the restart files are saved into the PW/restartOUT directory. 
The files in that directory need only be copied into the PW/restartIN 
directory inorder to prepare a restarted run.

\section{What next?}

Hopefully this section has guided you through installing the PWOM and
given you a basic knowledge of how to run it.  So what next?

We suggest that you read all of chapter \ref{chapter:basics}, which
outlines the basic features of the PWOM as well as some things you
really must know in order to use the PWOM.  Once you have done this you
are ready to experiment.  Chapter \ref{chapter:examples} gives several 
examples which are intended to make you familiar with the use of the
PWOM.  We suggest that you try them!

%\end{document}
