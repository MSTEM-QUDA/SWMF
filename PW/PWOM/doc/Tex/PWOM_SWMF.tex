The polar wind outflow can have a significant impact on the solution in the 
magnetosphere. Likewise, changes in the field aligned currents and ionospheric 
convection can modify the polar wind. Because of the importance of including 
such connections in our model, we now turn to the issue of how to use the 
PWOM as a component of the SWMF. 

\section{Configuring the SWMF to use PWOM}
Configuring the SWMF to use the PWOM is straight forward. Simply type:
\begin{verbatim}
  Config.pl -v=PW/PWOM
\end{verbatim}
Doing so tells the SWMF that the PWOM will represent the PW component of the 
SWMF. This configuration allows the user to execute the PWOM through the 
SWMF without any other components.

Should the user wish to include, say, a global magnetosphere or the 
ionosphere electrodynamics the configuration is simply:
\begin{verbatim}
  Config.pl -v=PW/PWOM,GM/BATSRUS,IE/Ridley_serial
\end{verbatim}
Coupling with these other components is described in greater detail in the 
next section.

\section{Coupling the PW component with other components}
The PWOM represents the PW component of the SWMF. It can be directly coupled 
to the Global Magnetosphere (GM) and Ionosphere Electrodynamics (IE) components 
by using the
\begin{verbatim}
  #COUPLE1
\end{verbatim}
 or
\begin{verbatim}
  #COUPLE2
\end{verbatim}
commands (See the SWMF manual for details). The IE component is coupled 
in a unidirectional manner with the PW component, by which the polar cap 
potentials and field aligned currents are passed to PW but no information 
flows in the opposite direction. 

The PW-GM coupling is more complicated. The GM component can be coupled 
unidirectionally, with the PW component setting the densities and velocities at 
the GM inner boundary, or it can be set bidirectionally where the 
magnetospheric pressure is used to set the upper boundary of the PW model. 
The latter coupling is untested so we will focus on the first form in the 
documentation.

There are several options for how the coupling between GM and PW can proceed. 
They all, however, start with the same command in the PARAM file:
\begin{verbatim}
  #COUPLE1
  PW            COMP1
  GM            COMP2
  10.0          DTcoupling
\end{verbatim}
The default method for coupling the PW and GM components is to combine the 
individual fluid densities and velocities at the upper boundary of PW into 
a single fluid density and velocity. However, the user may want to 
preserve the component information so as to track magnetospheric 
composition. In this case, the GM component can be configured to 
use the multi-species equations:
\begin{verbatim}
  Config.pl -o=GM:e=MhdPw
\end{verbatim}
In this case the individual species densities are preserved, but the 
velocities are combined into a single fluid velocity. This is because the 
multi-species MHD representation of BATS-R-US solves separate continuity 
equations, but a single momentum and energy equation. 

The user also has the option to preserve velocity information as well as 
density information when putting the PW output into GM. This requires the 
user to configure the GM component to solve the full multi-fluid 
equations:
\begin{verbatim}
  Config.pl -o=GM:e=MultiIon
\end{verbatim}
In this case the individual species densities and velocities are conserved, as 
BATS-R-US solves a full continuity, momentum, and energy equation for each 
species.
