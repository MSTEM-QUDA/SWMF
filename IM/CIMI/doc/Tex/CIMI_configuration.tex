\section{Configuration of CIMI}

Configuration refers to several different ways of controlling how the 
CIMI model is compiled and run.  The most obvious is the setting of
compiler flags specific to the machine and version of FORTRAN
compiler.  The other methods refer to the grid (default vs expanded) and
species.

\subsection{Setting compiler flags in Makefile.conf and with Config.pl}

The compiler flags can be modified by editing
\begin{verbatim}
  Makefile.conf
\end{verbatim}
This makefile is created during installation, and it contains the
platform and compiler specific part of the makefile system.
The most usual changes can be easily done with the Config.pl script.
The precision of real numbers can be set with
\begin{verbatim}
  Config.pl -single
\end{verbatim}
or 
\begin{verbatim}
  Config.pl -double
\end{verbatim}
If the precision is modified, the script will execute 'make clean',
so that all the files are compiled with the new real precision.

The debugging flags can be switched on and off with
\begin{verbatim}
  Config.pl -debug
\end{verbatim}
and
\begin{verbatim}
  Config.pl -nodebug
\end{verbatim}
respectively. The maximum optimization level can be set to -O2 with
\begin{verbatim}
  Config.pl -O2
\end{verbatim}
The minimum level is 0, the maximum is 4 (Note that not all compilers support
level 4 optimization).

\subsection{Grid and Species Configuration}

The CIMI model is set to have a very general core of code,

\begin{verbatim}
  Config.pl -GridExpanded
\end{verbatim}
to configure for H
\begin{verbatim}
  Config.pl -EarthH
\end{verbatim}


