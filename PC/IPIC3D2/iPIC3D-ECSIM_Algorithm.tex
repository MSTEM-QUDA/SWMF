\documentclass[a4paper, 11pt]{article}
\usepackage{graphicx}
\usepackage{amsmath,array}
%\usepackage{tablefootnote}
\usepackage{epstopdf}
\author{Yuxi Chen}
\title{iPIC3D/ECSIM Algorithm}
\begin{document}
\maketitle
\newpage

\section{The algorithm of the filed solver}
The $\Theta$-scheme of the Maxwell's equation is:
\begin{eqnarray}
\nabla \times \mathbf{E}^{n+\theta} + \frac{1}{c}\frac{\mathbf{B}^{n+1}-\mathbf{B}^{n}}{\Delta t} = 0
\label{eq:theta-B}\\
\nabla \times \mathbf{B}^{n+\theta} - \frac{1}{c}\frac{\mathbf{E}^{n+1}-\mathbf{E}^{n}}{\Delta t} = \frac{4\pi}{c} \mathbf{J}^{n+1/2}
\label{eq:theta-E}
\end{eqnarray}
where the value at $n+\theta$ stage is defined as a linear combination of the values at $n$ and $n+1$ stage, for example:
\begin{eqnarray}
\mathbf{E}^{n+\theta} = (1-\theta)\mathbf{E}^{n} + \theta\mathbf{E}^{n+1}.
\label{eq:theta}
\end{eqnarray}
Replace $\mathbf{B}^{n+\theta}$ in eq.(\ref{eq:theta-E}) and the following equation is obtained:
\begin{eqnarray}
\mathbf{E}^{n+\theta} + \delta^2 \nabla \times \nabla \times \mathbf{E}^{n+\theta} = \mathbf{E}^{n} + \delta(\nabla \times \mathbf{B}^n - \frac{4\pi}{c}\mathbf{J}^{n+1/2})
\label{eq:E1}
\end{eqnarray}
where $\delta = c \theta \Delta t$. The $\nabla \times \nabla \times$ operator can be split into two operators:
\begin{eqnarray}
\mathbf{E}^{n+\theta} + \delta^2 \left[ \nabla(\nabla \cdot \mathbf{E}^{n+\theta}) - \nabla ^2 \mathbf{E}^{n+\theta}\right] = \mathbf{E}^{n} + \delta(\nabla \times \mathbf{B}^n - \frac{4\pi}{c}\mathbf{J}^{n+1/2})
\label{eq:E2}
\end{eqnarray}
For both iPIC3D and ECSIM, the 'future' current $\mathbf{J}^{n+1/2}$ can be written in the form of:
\begin{eqnarray}
\mathbf{J}^{n+1/2} = \hat{\mathbf{J}} + \mathbf{M} \cdot \mathbf{E}^{n+\theta}
\label{eq: J}
\end{eqnarray}
but iPIC3D and ECSIM have different matrix $\mathbf{M}$ and $\hat{\mathbf{J}}$ calculation algorithm.
In order to have better charge conserving properties, we introduce the term:
\begin{eqnarray}
\nabla (\nabla \cdot \mathbf{E}^{n+\theta}) - \nabla (4\pi \rho ^{n+\theta})
\label{eq:divE}
\end{eqnarray}
into the field solver. An diffusion term $\nabla^2 \mathbf{E}^{n+\theta}$ is also introduced to smooth the field. The field equation becomes:
\begin{equation}
\begin{split}
\mathbf{E}^{n+\theta} + \delta^2 \left[(1-c_1) \nabla(\nabla \cdot \mathbf{E}^{n+\theta}) -\nabla ^2 \mathbf{E}^{n+\theta}\right] - c_2\frac{\Delta x^2}{2} \nabla ^2 \mathbf{E}^{n+\theta} = \\
\mathbf{E}^{n} + \delta(\nabla \times \mathbf{B}^n - \frac{4\pi}{c}\mathbf{J}^{n+1/2})-c_1\delta^2\nabla(4\pi\rho^{n+\theta}).
\end{split}
\label{eq:E3}
\end{equation}
Based on the continuity equation, the 'future' density is:
\begin{eqnarray}
\mathbf{\rho}^{n+\theta} = \mathbf{\rho}^n - \Delta t \theta \nabla\cdot \mathbf{J}^{n+1/2} = \mathbf{\rho}^n - \Delta t \theta \nabla\cdot (\hat{\mathbf{J}} + \mathbf{M} \cdot \mathbf{E}^{n+\theta}) = \hat{\rho} - \Delta t \theta \nabla \cdot (\mathbf{M} \cdot \mathbf{E}^{n+\theta})
\label{eq:rhoh}
\end{eqnarray}
Substitute eq.(\ref{eq:rhoh}) and eq.(\ref{eq: J}) into eq.(\ref{eq:E3}):
\begin{equation}
\begin{split}
\mathbf{E}^{n+\theta} + \mathbf{D}^{n+\theta}+ \delta^2 \left[(1-c_1) \nabla(\nabla \cdot \mathbf{E}^{n+\theta}) -  c_1\nabla(\nabla \cdot \mathbf{D}^{n+\theta}) - \nabla ^2 \mathbf{E}^{n+\theta}\right]- c_2\frac{\Delta x^2}{2} \nabla ^2 \mathbf{E}^{n+\theta} =\\ 
\mathbf{E}^{n} + \delta(\nabla \times \mathbf{B}^n - \frac{4\pi}{c}\hat{\mathbf{J}})-c_1\delta^2\nabla(4\pi\hat{\rho}).
\end{split}
\label{eq:E4}
\end{equation}
where $\mathbf{D}^{n+\theta} = 4\pi\theta\Delta t\mathbf{M}\cdot \mathbf{E}^{n+\theta}$.

\subsection{iPIC3D}
For iPIC3D, $4\pi\theta\Delta t\mathbf{M}$ is the $\chi$ in the iPIC3D paper. Choose $c_1=1$, $c_2=0$ and $\theta = 1$, and we can obtain the field equation described in iPIC3D paper. 

\subsection{ECSIM}
With $c_1= 0$, $c_2=0$, $\theta = 0.5$, explicit particle mover (particle location is at the half time stage) and the mass matrix described in ECSIM papeer, the exact energy conserving (up to the iteration error of the field solver) scheme is obtained. In practice, $\theta = 0.51$ is a good choice most of the time. The role of $c_1$ and $c_2$ still needs to be clarified. 

\section{Reference}
\begin{itemize}
\item[1] Markidis, S. and Lapenta, G., 2010. Multi-scale simulations of plasma with iPIC3D. Mathematics and Computers in Simulation, 80(7), pp.1509-1519.
\item[2] Lapenta, G., 2017. Exactly energy conserving semi-implicit particle in cell formulation. Journal of Computational Physics, 334, pp.349-366.
\end{itemize}

\end{document}