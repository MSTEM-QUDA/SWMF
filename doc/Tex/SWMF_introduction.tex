%\documentclass[a4paper,11pt]{article}
%\author{\bf Center for Space Environment Modeling, The University of Michigan}
%\title{\bf \Large Release Notes for the Milestone 7I and Referance Manual}
%\maketitle



\chapter{Introduction}

This document describes a working prototype of the NASA-funded Space
Weather Modeling Framework (SWMF) delivered to NASA to fulfill the
Milestone ??? requirements. The SWMF was developed to provide flexible
"plug and play" type simulation capabilities serving the Sun-Earth
modeling community.  In its current form the SWMF links together eight
models from the surface of the Sun to the upper atmosphere of the Earth: 
\begin{enumerate}
\item SC -- Solar Corona which includes the Eruptive Event Generator,
\item IH -- Inner Heliosphere
\item SP -- Solar Energetic Particles 
\item GM -- Global Magnetosphere 
\item IM -- Inner Magnetosphere
\item RB -- Radiation Belts
\item IE -- Ionosphere Electrodynamics
\item UA -- Upper Atmosphere
\end{enumerate}
In the future the SWMF may be extended to include even more 
physics domains: Cometary Environment, Interstellar
Neutrals, Outer Heliosphere, Plasmasphere, Planetary Satellites and
Polar Wind. 

The SWMF implementation is based on the component technology and
Object-Oriented Programming emulated in Fortran 90.  The SWMF parallel
communications are based on the MPI standard.  In its current
implementation the SWMF creates a single executable.

\section{Acknowledgments}

The SWMF was developed at the Center for Space Environment Modeling
(CSEM) of the University of Michigan under the NASA Earth Science
Technology Office (ESTO) Computational Technologies (CT) Project (NASA
CAN NCC5-614). The project is entitled as ``A High-Performance
Adaptive Simulation Framework for Space-Weather Modeling (SWMF)''.
The Project Director is Professor Tamas Gombosi, and the Co-Principal
Investigators are Professors Quentin Stout and Kenneth Powell.

The SWMF and many of the physics components were developed at CSEM
by the following individuals (in alphabetic order):
David Chesney,
Darren DeZeeuw, Tamas Gombosi, Kenneth Hansen, Kevin Kane, Ward (Chip)
Manchester, Robert Oehmke, Kenneth Powell, Aaron Ridley, Ilia Roussev,
Quentin Stout, Igor Sokolov, G\'abor T\'oth and Ovsei Volberg.

The core design and code development was done by G\'abor
T\'oth, Igor Sokolov and Ovsei Volberg:
\begin{itemize}
\item Component registration and layout was designed and implemented by 
      Ovsei Volberg and G\'abor T\'oth.
\item The session and time management support was designed and
      developed by G\'abor T\'oth.
\item The SWMF coupling toolkit was developed by Igor Sokolov.
\end{itemize}
The physics models were developed by the following research groups:
\begin{itemize}
\item
The Solar Corona (SC), Inner Heliosphere (IH) and the Global Magnetosphere 
(GM) components are based on \BATSRUS\ MHD code developed at CSEM. 
\BATSRUS\ is a 3-dimensional block-adaptive Cartesian code which is 
highly parallel.

\item
The Solar Energetic Particle (SP) component has two versions:
K\'ota's SEP model was developed at the University of Arizona.
???
The FLAMPA model was developed at CSEM.
???

\item
The Inner Magnetosphere (IM) component is the Rice Convection Model
(RCM) developed Rice University.  This code is 2-dimensional and
serial.

\item
The Radiation Belt (RB) component is the Rice RBM
developed Rice University.  This code is 2-dimensional and
serial.

\item
The Ionospheric Electrodynamics (IE) component is a 2-processor,
2-dimensional spherical electric potential solver developed at CSEM
(termed the ``Ridley Ionosphere''.  
???
In addition, there are a number of
statistical models included with the distribution, which are described
further in the document ???

\item
The Upper Atmosphere (UA) component is the Global Ionosphere -
Thermosphere Model (GITM).  This model is a 3-dimensional spherical
model developed at CSEM.  It is fully parallel.

\end{itemize}
The transformation of physics models into physics components,
the coupling of components to the SWMF and each other and,
all the testing were done at CSEM.

\section{What is New in Version 2.0}

The SWMF has been developed extensively since the first release of
version 1.0. Here are some of the highlights:
\begin{itemize}
\item The SWMF now contains 3 new components:
      the Solar Corona (SC), the Solar Energetic Particles (SP) 
      and the Radiation Belt (RB)
\item Some components now support dynamic memory allocation which 
      reduces the total memory required by a processing element. 
\item The directory structure of the SWMF has been greatly improved and 
      reorganized. 
\item This reorganization allows components to be used as stand alone 
      physics models without any modification in the source code,
      only a small library needs to be linked.
\item The installation and configuration of SWMF has been greatly simplified
      with the aid of Perl scripts. 
\item Layout and input parameter can be checked with Scripts/TestParam.pl
\item User manual is produced from the XML description of the input parameters.
\item Unused components can be configured out completely.
\item The control module now fully supports steady state calculations 
      including component subcycling. 
\item The coupling toolkit provides means for extracting and following 
      the motion of field lines.
\end{itemize}

\section{The SWMF in a Few Paragraphs}

The SWMF is a structured collection of software building blocks that
can be used or customized to develop Sun-Earth system modeling
components, and to assemble them into applications. The SWMF consist
of utilities and data structures for coupling model components. The
SWMF contains a Control Module (CON), which is responsible for
component registration, processor layout for each component and
coupling schedules.  It controls initialization and execution of the
components. A component is adapted from user-supplied physics codes,
(for example \BATSRUS\ or RCM), by adding two relatively small units
of code:
\begin{itemize}
\item A wrapper, which provides the control functions, and
\item A coupling interface to perform the data exchange with other
components.
\end{itemize}
Both the wrapper and coupling interface are constructed from the
building blocks provided by the framework. From 
component software technology perspective both the wrapper and
coupling interface are component interfaces: the wrapper is an
interface with CON, and the coupling interface is an interface with
another component. In this release the wrappers and coupling
interfaces for all five physics components are included. A physics
model code and its wrapper, which comprise a component, share the
communication group.  The coupling interface uses the union
communicator of the two components that it links together.

An SWMF component is compiled into a separate library that resides in
the directory {\tt lib}, which is created as part of the installation
process described later in this document.  Currently the component
libraries are static libraries. The executable image is created in the
directory {\tt bin}, which is created during the compilation.  If a
user does not want to build some particular component, this component
should be substituted by an empty version of the component.

An important feature of the SWMF is the component registration.  A
component to be included in the run should be registered by the
framework.  Currently entering the line for the component in the input
file called {\tt LAYOUT.in} does the registration.  Thus the SWMF
performs the run-time registration of components.

The framework controls the initialization, execution, coupling and
finalization of components.  The execution is done in sessions. In
each session the parameters of the framework and the components can be
changed.  The parameters are read from the {\tt PARAM.in} file, which
may contain further included parameter files.  These parameters are
read and broadcast by CON and the component specific parameters are
sent to the components. The structure of the parameter file will be
described in detail.

If two components reside on different sets of processing elements
(PE-s) they can execute in an efficient parallel manner.
This is possible, because the coupling times are
known in advance.  The components advance to the time of coupling and
only the processors involved in the coupling need to communicate with
each other. The components are also allowed to share some processing elements.
The execution is sequential for the components with overlapping layouts.
Of course this still allows the individual components to execute in parallel.
For steady state calculations the components are allowed to progress
at different rates towards steady state. Each component can be called
at different frequencies by the control module.

The coupling of the components is realized either with plain MPI
calls, or via the SWMF coupling toolkit, which can couple components
based on the following types of parallel distributed grids:
\begin{itemize}
\item 3-D Block adaptive (AMR) parallel grid
\item 2-D Spherical grid
\item Logically Cartesian uniform grid
\item Logically Cartesian non-uniform grid 
\end{itemize}
The SWMF coupling toolkit performs an efficient N to M parallel
coupling based on a router. The router is calculated in advance using
the domain decomposition and grid description obtained from the
components.  The router is updated only when the domain decompositions
or the grids of the components change, or when the mapping geometry
changes.  The coupling toolkit takes care of linear interpolation in
space based on the grid descriptor.  Temporal interpolation is not
supported by the current implementation.

The framework has been tested on the SGI Origin 3000, SGI Altix and 
Compaq ES45 machines, and on Linux Beowulf clusters with the NAG f95 
compiler. We have also run earlier versions of the framework under
Mac OS Darwin using the NAG f95 compiler, and under Linux with
the PGF90 compiler.

%-----------------------------------------------------------------------
% Chapter 2
%-----------------------------------------------------------------------

\chapter{Quick Start}

\section{A Brief Description of the SWMF Distribution}

The distribution in the form of the compressed tar image
SWMF.tar.gz includes the SWMF source code and the tests.
The top level directory contains the following subdirectories:
\begin{itemize}
\item {\tt CON}     - the directory of the framework's main building blocks
\item {\tt GM}      - Global Magnetosphere component       %^CMP IF GM
\item {\tt IE}      - Ionosphere Electrodynamics Component %^CMP IF IE
\item {\tt IH}      - Inner Heliosphere component          %^CMP IF IH
\item {\tt IM}      - Inner Magnetosphere component        %^CMP IF IM
\item {\tt RB}      - Radiation Belt component             %^CMP IF RB
\item {\tt SC}      - Solar Corona component               %^CMP IF SC
\item {\tt SP}      - Solar Energetic Particles component  %^CMP IF SP
\item {\tt UA}      - Upper Atmosphere component           %^CMP IF UA
\item {\tt Copyrights} - copyright files
\item {\tt Param}   - description of CON parameters, parameter and layout files
\item {\tt Scripts} - shell and Perl scripts
\item {\tt bin}     - scripts for installation, configuration and testing
\item {\tt doc}     - the documentation directory %^CMP IF DOC
\item {\tt share}   - shared scripts and source code
\item {\tt test}    - input files for tests and reference output
\item {\tt util}    - general utilities such as TIMING and NOMPI
\end{itemize}
and the following files
\begin{itemize}
\item {\tt README}           - a short instruction on installation and usage
\item {\tt Makefile}         - the main makefile
\item {\tt Configure.pl}     - Perl script for configuration %^CMP IF CONFIGURE
\item {\tt Configure.options} - default configuration options %^CMP IF CONFIGURE
\item {\tt SetSWMF.pl}     - Perl script for (un)installation and configuration
\end{itemize}

\section{Installing the Code}

The first step in installing the SWMF is untarring the distribution.
If the tar program knows about the -z flag, you can open the gzipped
tar files with a single UNIX command:
\begin{verbatim}
  tar xzf SOMETARFILE.tgz
\end{verbatim}
If the tar program does not recognize the -z flag, two steps are needed:
\begin{verbatim}
  gunzip SOMETARFILE.tgz
  tar xf SOMETARFILE.tar
\end{verbatim}
In the following descriptions the shorter form is shown, but you may
need to use the two step procedure on certain platforms.

Untar the distribution using the command:
\begin{verbatim}
  tar xzf SWMF.tgz
\end{verbatim}

Change directories into the distribution:
\begin{verbatim}
  cd SWMF
\end{verbatim}

The SWMF needs to know what architecture you are running the code on.
It can figure this out all by itself, but you have to run the command:
\begin{verbatim}
  SetSWMF.pl -i
\end{verbatim}
in the main directory. This creates {\tt Makefile.def} with
the correct absolute path to the base directory and {\tt Makefile.conf}
which contains the operating system and compiler specific part of
the Makefile. If the compiler is not the default one for a given
platform (e.g. not the NAG f95 compiler for a Linux platform) then
the compiler must be specified explicitly with the {\tt -c}
flag. If the MPI header file is not the default one, it can be
specified with the {\tt -m} flag. For example on the Altix machines
SWMF should be installed as
\begin{verbatim}
  SetSWMF.pl -i -c=ifort -m=Altix
\end{verbatim}
To uninstall SWMF type
\begin{verbatim}
  SetSWMF.pl -uninstall
\end{verbatim}
If the uninstallation fails (this can happen if some makefiles are missing)
force installation with
\begin{verbatim}
  SetSWMF.pl -install
\end{verbatim}
and then try uninstalling the code again.
When SWMF is installed, its configuration can be checked with
\begin{verbatim}
  SetSWMF.pl -s
\end{verbatim}
To get a list of the available component versions type
\begin{verbatim}
  SetSWMF.pl -l
\end{verbatim}
To get a complete description of the {\tt SetSWMF.pl}  script type
\begin{verbatim}
  SetSWMF.pl -h
\end{verbatim}

\section{Creating Documentation}

The documentation for SWMF can be generated from the distribution by
the command
\begin{verbatim}
  make PDF
\end{verbatim}
which creates the documentation
\begin{verbatim}
  doc/SWMF.pdf
\end{verbatim}
in the Adobe PDF format.  An on-line version can be created by
\begin{verbatim}
  make HTML
\end{verbatim}
The top level HTML file is in
\begin{verbatim}
  doc/HTML/index.html
\end{verbatim}
to point at with the browser.  To clean the intermediate files type
\begin{verbatim}
  cd doc/Tex
  make clean
\end{verbatim}
To remove all the created documentation type
\begin{verbatim}
  cd doc/Tex
  make distclean
\end{verbatim}

\section{Building and Running an Executable}

The physics components can be selected with the {\tt -v} flag
of the SetSWMF.pl script. For example typing
\begin{verbatim}
  SetSWMF.pl -v=SC/BATSRUS,IH/BATSRUS,SP/Kota
  SetSWMF.pl -v=GM/Empty,RB/Empty,IM/Empty,IE/Empty,UA/Empty
\end{verbatim}
will select BATSRUS for the SC and IH components and K\'ota's SP model.
The other components are set to Empty versions, which contain emmpty
subroutines for compilation, but cannot be used.
The default configuration includes a working version for all components, 
which takes up more memory, but is the most general.
The only exception is SC, which requires configuration, so the 
default version is Empty for the Solar Corona component.

The grid size of several components can also be set with the {\tt -g}
flag of the {\tt SetSWMF.pl} script. For example the 
\begin{verbatim}
  SetSWMF.pl -g=GM:8,8,8,400,100
\end{verbatim}
command sets the block size for the GM component to $8\times 8\times 8$ cells, 
the maximum number of blocks per processor to 400, 
and the maximum number of implicit blocks to 100.
The SetSWMF.pl script actually runs the individual GridSize.pl
scripts in the component versions. These scripts can be run directly,
and they provide more options and more verbose information than SetSWMF.pl.
For example try
\begin{verbatim}
  cd GM/BATSRUS
  GridSize.pl -s
\end{verbatim}
Compilation flags, such as the precision and optimization 
level are stored in {\tt Makefile.conf}. The precision of reals
can be changed to single precision (for example) by typing
\begin{verbatim}
  SetSWMF.pl -p=single
\end{verbatim}
while the compiler flags can be editied in {\tt Makefile.conf} by hand.

Before compiling SWMF it is always a good idea to check its configuration
with
\begin{verbatim}
  SetSWMF.pl -s
\end{verbatim}

{\bf IMPORTANT NOTE:
On the Altix machine at NASA Ames (columbia),
the current (08/12/2004) 
default version of the Intel Fortran compiler 
cannot be used for SWMF. 
You should load the 8.0.066 version with the command
\begin{verbatim}
  module load intel-comp.8.0.066
\end{verbatim}
You may wish to insert this line into the .cshrc file
so it executes at login time. 
Selecting the correct compiler version is 
necessary both to compile and to run the code.
Therefore the above line is needed in the job scripts
as well.}

To build the executable {\bf bin/SWMF.exe}, type:
\begin{verbatim}
  make
\end{verbatim} 
Depending on the configuration, the compiler settings and the machine 
that you are compiling on, this can take from 2 to upto 30 minutes.  
In addition, you may want make the post processing
codes (for BATSRUS only) also:
\begin{verbatim}
  make PSPH
  make PIDL
\end{verbatim} 
These two commands will create the codes {\tt bin/PostSPH.exe}, for post
processing spherical Tecplot files, and {\tt bin/PostIDL.exe} 
for post processing IDL files.

The executable should be run in a sub-directory, since a large number
of files are created in each run.  To create this directory use the
command:
\begin{verbatim}
  make rundir
\end{verbatim} 
This command creates a directory called {\tt run}.  You can either
leave this directory as named, or {\tt mv} it to a different name.  It
is best to leave it in the same SWMF directory, since
keeping track of the code version associated with each run is quite
important.  The {\tt run} directory will contain links to the codes
which were created in the previous step as well as subdirectories
where input and output of the different components will reside.

Here we assume that the {\tt run} directory is still called {\tt
run}:
\begin{verbatim}
  cd run
\end{verbatim} 
Edit the processor map input file LAYOUT.in to define the processor
layout for the components involved in the future run.  Edit or create
the input parameter file PARAM.in.  The default LAYOUT.in and PARAM.in
files in the run directory are suitable to perform the ``Start'' test
on 16 PE-s. 

An example processor map file LAYOUT.in to run the executable with
five components on 16 processors is:
\begin{verbatim}
#COMPONENTMAP
GM    0    4    1
IE    5    6    1
IH    7   10    1
IM   11   11    1
UA   12   15    1
#END
\end{verbatim}
The file syntax is simple. It must start with the directive
\#COMPONENTMAP and end with another directive \#END. Each line between
these directives specifies the label for component, i.e. IE, GM and
etc., its first and last processor, all relatively to the world
communicator, and the stride. Thus GM will run on 5 processors from 0
to 4, and IM will run on only 1 processor, the processor 11.  If
stride is not equal to 1, the processors for the component will not be
neighboring processors.

It is useful to check the validity of the {\tt run/PARAM.in} and 
{\tt run/LAYOUT.in} files berfore running the code. If the
code will be run on 16 processors, type
\begin{verbatim}
Scripts/TestParam.pl -n=16
\end{verbatim}
in the main SWMF directory.
The Perl script reports inconsistencies and errors. 
If no errors are found, the script finishes silently.
Now you are ready to run the executable through submitting a batch job or, 
if it is possible on your computer, run the code interactively.  For example:
\begin{verbatim}
cd run
mpirun -np 16 SWMF.exe
\end{verbatim}
The SWMF provides example job scripts for several architectures and
machines used by the developers. These job scripts are found in 
\begin{verbatim}
CON/Scripts
\end{verbatim}
in the subdirectories named after the operating system. If the name
of the file in the appropriate subdirectory matches the 
name of the machine, the job script is copied into
the {\tt run} directory when it is created.
These job scripts serve as a starting point only, they must
be customized before they can be used for submitting a job.

To recompile the executable with different compiler settings you have
to use the command
\begin{verbatim}
make clean
\end{verbatim}
before recompiling the executables. It is possible to recompile
only a component or just one subdirectory if the {\tt make clean}
command is issued in the appropriate directory.

\section{Executing the Test Runs}

\subsection{The 8 Component Test}

This test demonstrates faster than real time execution of the
SWMF with all 8 components. The test was built up in six stages,
the last stage constitutes the actual performance test.
\begin{enumerate}
\item Create a steady state SC solution.
\item Create a steady state IH solution coupled to the SC.
\item Run SC, IH and SP for 4 hours real time with 
      a CME initiated in SC using an Eruptive Event generator.
\item Create a start up GM solution based on the IH solution.
\item Create a start up GM+IM+IE+UA solution.
\item Run all 8 components in time accurate mode restarting from the
      SC+IH and GM+IM+IE+UA solutions. 
      The SP and RB components start from scratch.
\end{enumerate}
We provide the parameter and layout files for all six stages,
but here we only describe how the final stage can be run using
the restart files saved from the previous stages.
This takes much less time than running all stages, 
and it tests the whole SWMF. The README file in the SWMF\_TEST
tar ball contains a detailed description of all stages.

\subsubsection{Configuration and Compilation}

For sake of simplicity one can configure SWMF with all 8 components
(no Empty versions) so that all 6 stages can be done with the same
executable. 
\begin{verbatim}
  SetSWMF.pl -v=SC/BATSRUS,IH/BATSRUS_share,GM/BATSRUS,SP/Kota,IM/RCM
  SetSWMF.pl -v=IE/Ridley_serial,UA/GITM,RB/RiceV5
\end{verbatim}
This is the default configuration except for the SC
component, which has the Empty version by default.
The source code for the SC/BATSRUS component is configured and 
from the GM/BATSRUS source code and the subroutines, functions and 
modules are renamed to avoid name conflicts. This automated code
generation takes some time.
Note that one can select the IH/BATSRUS version instead of IH/BATSRUS\_share.
In that case the source code for the IH/BATSRUS version is generated 
from the GM/BATSRUS source code with a similar renaming procedure.

The grid sizes of the components should be set to be sufficient for the test
for a given processor layout. The following grid sizes were used:
\begin{verbatim}
  SetSWMF.pl -g=GM:8,8,8,200,40 -g=SC:4,4,4,1000 
  SetSWMF.pl -g=UA:9,9,25,2,2   -g=SP:1000,10,150
\end{verbatim}
Note that the GM, SC and UA grid sizes differ from the defaults.
If the IH component is BATSRUS\_share, it uses the same source
code as GM/BATSRUS, and hence the same grid. If IH/BATSRUS is
selected, then one can use different grids, for example
\begin{verbatim}
  SetSWMF.pl -g=GM:8,8,8,100,40 -g=IH:8,8,8,200,1
\end{verbatim}
since GM needs only a 100 blocks per PE (if run on at least
25 processors), while IH does not need any implicit blocks 
(see the BATSRUS manual for details).
Note, however, that this will save memory only if the large arrays used
by a given component are allocated only on the processors which are dedicated 
for that component. The BATSRUS code uses static allocation by
default, but it can also use dynamic allocation. To make for example
the GM and SC components dynamic type
\begin{verbatim}
  cd GM/BATSRUS/src; make DYNAMIC
  cd ../../..
  cd SC/BATSRUS/src; make DYNAMIC
  cd ../../..
\end{verbatim}
Now compile the source code into bin/SWMF.exe 
and create the post processing executable bin/PostIDL.exe 
and the run directory with the following commands
\begin{verbatim}
  make
  make PIDL
  make rundir
\end{verbatim}

\subsubsection{Parameter, Layout and Restart Files}

Open the prepacked SC+IH and the GM+IM+IE+UA restart files in
the run directory.
On a ``little endian'' platform (Linux clusters, SGI Altix and Compaq)
\begin{verbatim}
  cd run
  tar xzf ../test/scih_4hr_restart.tgz
  tar xzf ../test/gmimieua_restart.tgz
\end{verbatim}
while on a ``big endian platform'' (SGI Origin, MAC) use
\begin{verbatim}
  cd run
  gunzip ../test/scih_4hr_restart_sgi.tgz
  gunzip ../test/gmimieua_restart_sgi.tgz
  tar xf ../test/scih_4hr_restart_sgi.tar
  tar xf ../test/gmimieua_restart_sgi.tar
\end{verbatim}
Note that the {\tt tar} program on the SGI does not recognize the 
{\tt -z flag}, so we have to decompress the files with {\tt gunzip} 
first.

Copy the PARAM and LAYOUT files into the run directory
\begin{verbatim}
  cp ../test/PARAM.in.8comp PARAM.in
  cp ../test/LAYOUT.in.8comp LAYOUT.in
\end{verbatim}
Edit the LAYOUT.in file according to the number of processors
available. The file in its original form is best suited for 
running on around 128 processors or more.
Note that for the above suggested grid sizes and the
selected component versions the following restrictions apply:
\begin{itemize}
\item The IM/RCM component can use 1 processor only
\item The RB/RiceV5 component can use 1 processor only
\item The SP/Kota component can use 1 processor only
\item The IE/Ridley\_serial component can use 1 or 2 processors.
\item The UA component can use at most 32 processors, 
      and needs at least 8 processors. The number of blocks should
      be the same on all processors, so use 8, 16 or 32 processors.
\item The GM/BATSRUS component needs at least 25 processors.
\item The SC/BATSRUS component needs at least 23 processors.
\item The IH/BATSRUS(\_share) component needs at least 25 processors.
\item The IH/BATSRUS\_share component cannot be overlapped with the
      GM/BATSRUS component.
\end{itemize}
Once the LAYOUT.in file has been edited, make sure that everything
is correctly setup by running the TestParam.pl script. 
If you plan to run the test on 128 processors, type
\begin{verbatim}
  Scripts/TestParam.pl -n=128
\end{verbatim}
in the main SWMF directory.
If any errors or warnings were reported, fix them until the script
runs silently. 

\subsubsection{Running the Test}

Now you are ready to run the test interactively as
\begin{verbatim}
cd run
mpirun -np 128 SWMF.exe
\end{verbatim}
or as a job. The test involves 2 sessions, each runs for 300 seconds
real time. 

On 128 processors of an SGI Altix machine (altix2 at NASA Ames) 
the 2 sessions finished in 278 and 240 seconds, 
which is 8\% and 25\% faster than real time, respectively.
Note that the timings do not include the set up time, which was 217 seconds.
For a longer run we expect that the performance observed in the
second session would persist.

\subsection{The 5 Component Test}

This test was developped for Milestone 7I. 
It can be done in the current version as well with minor
modifications. 
The tests are in the Tests directory for the released code, otherwise
they are in the SMWF\_7I\_TEST CVS repository.  The following tests
are available:
\begin{itemize}
\item {\tt StartTst.tgz}\\
      Steady state run for IH, GM and IE components.
      The run directory is created with the same files as in
      this test.

\item {\tt RestartTst.tgz}\\
      Time accurate run for IH-GM-IM-IE-UA component
      The restart files are for little endian platforms.

\item {RestartTstSGI.tgz}\\
      Same as {\tt RestartTst.tgz} 
      but for big endian platforms such as the SGI.
\end{itemize}
On 16 CPU-s of a Linux cluster (AMD Athlon 1900+, using a 100Mbs
network) the StartTst required 2872 seconds to complete, while the
restart test requires 182 seconds (set up time is not included).

Reference results are also  provided:
\begin{itemize}
\item {\tt StartResult.tgz}\\
       Results for the steady state run with IH, GM and IE

\item {\tt RestartResultGeneral.tgz}\\
      Results for the restarted run using the 
      LAYOUT.in.general and PARAM.in.general files.
\end{itemize}
Only the ASCII output files are included for sake of easy comparison.
Results may vary slightly from platform to platform due to round off errors.


\subsubsection{Running the startup test}

To run this test you need to install, configure and compile SWMF.
The {\tt IH/BATSRUS} (or {\tt BATSRUS\_share}), 
{\tt GM/BATSRUS} and {\tt IE/Ridley\_serial}
components are required, but it is useful to include all five
components so the same executable and run directory 
can be used for the next test too. For example set
\begin{verbatim}
  SetSWMF.pl -v=IH/BATSRUS_share,GM/BATSRUS,IM/RCM,IE/Ridley_serial,UA/GITM
  SetSWMF.pl -v=SC/Empty,SP/Empty,RB/Empty
  SetSWMF.pl -g=GM:8,8,8,400,1 -p=double
\end{verbatim}
The default parameter and layout files can be used. 
These are copied into the run directory during {\tt make rundir},
or can be copied with
\begin{verbatim}
  cd run
  cp Param/PARAM.DEFAULT  PARAM.in
  cp Param/LAYOUT.DEFAULT LAYOUT.in
\end{verbatim}
This test takes some time, and it creates the restart files
for the following test. 

After the run finished, you may post process the IDL plot files (the
{\tt .idl} and {\tt .h} files produced by GM and IH can be processed
into {\tt .out} files), and the northern and southern hemispheres
plots of the IE component can be put together: 
\begin{verbatim}
  cd run/GM
  ./pIDL
  cd ../IH
  ./pIDL
  cd ../IE
  ./pION -r
\end{verbatim}
The {\tt .out} files can be visualized with IDL using the scripts in
the Idl directory.  You can also compare the ASCII output files with
the ones given in the reference solutions.

\subsubsection{Running the short restart test}

To run this test you need to install, configure and compile SWMF
as described above and create the run directory. 
The {\tt IH/BATSRUS} (or {\tt BATSRUS\_share}),
{\tt GM/BATSRUS}, {\tt IM/RCM}, {\tt IE/Ridley\_serial} and
{\tt UA/GITM} components are required. The configuration shown
in the previous section can be used, for example.
After this you can install the test into the run directory.  
For example on a Linux or TrueUnix
(OSF1) machine the short restart test can be installed with the
following commands:
\begin{verbatim}
  cd run
  tar xzf ../Tests/RestartTst.tgz
  cp Param/PARAM.in.test.restart.IHGMIMIEUA PARAM.in
  cp Param/LAYOUT.in.test.restart.IHGMIMIEUA LAYOUT.in
\end{verbatim}
On a big endian platform use the {\tt RestartTstSGI.tgz} file instead.

Now you can run the code interactively or submit a job.  The default
layout assumes 16 PE-s. For example an interactive execution could be
like this:
\begin{verbatim}
  mpirun -np 16 SWMF.exe
\end{verbatim}
The post processing of the plot files is the same as described above. 
The IM/RCM and UA/GITM components do not require post processing.

\subsubsection{Test with the renamed IH code}

The restriction on the overlapping of the IH and GM components
can be eliminated by the use of the renamed IH source code. 
To test this, select the {\tt IH/BATSRUS} component:
\begin{verbatim}
  SetSWMF.pl -v=IH/BATSRUS
\end{verbatim}
This will automatically create the renamed source code for IH/BATSRUS.
which will take 2-5 minutes. Once the renaming
is done, reduce the number of the blocks for GM and IH:
\begin{verbatim}
  SetSWMF.pl -g=GM:8,8,8,200,1 -g=IH:8,8,8,200,1
\end{verbatim}
The test requires 520 blocks for GM and 456 blocks for IH on the PE-s
assigned to the components. So with at least 3 PE-s for GM and IH each,
the above numbers are sufficient. This test does not require the implicit
scheme, therefore the number of blocks with implicit time stepping can
be minimized. You can specify different number of blocks for GM and IH. 

Recompile the code:
\begin{verbatim}
  make
\end{verbatim}
Now it is allowed to overlap GM and IH. The renamed IH code has the 
version name IH\_BATSRUS, which is printed to the standard output at the
beginning of the execution.

\subsection{Testing GM/BATSRUS and IE/Ridley\_serial}

The \BATSRUS\ code is used in many components, so it comes
with an extensive test suite. The test suite resides in
\begin{verbatim}
  GM/BATSRUS/Param/TESTSUITE
\end{verbatim}
but it can be used from the SWMF as well. Two of the tests
involve the {\tt IE/Ridley\_serial} component. These two tests
can only be done in the framework, while the other tests should give
identical results in the framework and in the standalone
BATSRUS code.

The functionality testsuite requires two components only,
so the following configuration is recommended:
\begin{verbatim}
  SetSWMF.pl -v=GM/BATSRUS,IE/Ridley_serial
  SetSWMF.pl -v=SC/Empty,IH/Empty,SP/Empty,IM/Empty,RB/Empty,UA/Empty
  SetSWMF.pl -g=GM:8,8,8,400,100 -p=double
  make
  make PIDL
  make PSPH
  make rundir
\end{verbatim}
After compilation and creation of the run directory, the test suite
can be executed on 2 processors with
\begin{verbatim}
  Scripts/TestSuite func
\end{verbatim}
The results will be stored in the run directory
\begin{verbatim}
  run/test.000
  run/test.001
  ...
  run/test.035
\end{verbatim}
An individaul test can be run with the {\tt TestSWMF} script,
for example the first test of the test suite is 
\begin{verbatim}
  Scripts/TestSWMF -Limiter=beta -Plottype=idltecamr
\end{verbatim}
Two test suite results obtained with different versions of the code,
or with different platforms/compilers, can be compared for 
consistency and performance:
\begin{verbatim}
  Scripts/TestCompare run1 run2
  Scripts/TestCompare -speed run1 run2
\end{verbatim}
The configuration of BATSRUS to a 'covariant' (generalized coordinate) version
can be tested with
\begin{verbatim}
  Scripts/TestCovariant
\end{verbatim}
All these scripts provide a complete usage information when called with the 
{\tt -h} flag.
%\end{document}
