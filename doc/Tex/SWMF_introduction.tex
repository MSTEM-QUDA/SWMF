%\documentclass[a4paper,11pt]{article}
%\author{\bf Center for Space Environment Modeling, The University of Michigan}
%\title{\bf \Large Release Notes for the Milestone 7I and Reference Manual}
%\maketitle



\chapter{Introduction}

This document describes a working prototype of the 
Space Weather Modeling Framework (SWMF).
The SWMF was developed to provide a flexible tool serving the Sun-Earth
modeling community.  In its current form the SWMF contains many 
domains extending from the surface of the Sun to the upper atmosphere of the 
Earth: 
\begin{enumerate}
\item EE -- Eruptive Event generator (currently empirical models only)
\item GM -- Global Magnetosphere 
\item IE -- Ionosphere Electrodynamics
\item IM -- Inner Magnetosphere
\item IH -- Inner Heliosphere
\item LA -- Lower Atmosphere (under development)
\item LC -- Lower Corona
\item OH -- Outer Heliosphere
\item PS -- Plasma Sphere (under development)
\item PW -- Polar Wind
\item RB -- Radiation Belts
\item SC -- Solar Corona
\item SP -- Solar Energetic Particles 
\item UA -- Upper Atmosphere
\end{enumerate}
The core of the SWMF and most of the models are implemented in Fortran 90. 
The parallel communications use the Message Passing Interface (MPI) library. 
The SWMF creates a single executable. Note, however, that the
models in the SWMF can still be compiled into stand-alone executables.
This means that the models preserve their individuality while being
compatible with the SWMF.

\section{Acknowledgments}

The first version of the SWMF was developed at the Center for 
Space Environment Modeling (CSEM) of the University of Michigan under 
the NASA Earth Science Technology Office (ESTO) 
Computational Technologies (CT) Project (NASA
CAN NCC5-614). The project was entitled as ``A High-Performance
Adaptive Simulation Framework for Space-Weather Modeling (SWMF)''.
The Project Director was Professor Tamas Gombosi, and the Co-Principal
Investigators are Professors Quentin Stout and Kenneth Powell.

The first version of the SWMF and many of the physics components 
were developed at CSEM by the following individuals (in alphabetical order):
David Chesney, Yue Deng,
Darren DeZeeuw, Tamas Gombosi, Kenneth Hansen, Kevin Kane, Ward (Chip)
Manchester, Robert Oehmke, Kenneth Powell, Aaron Ridley, Ilia Roussev,
Quentin Stout, Igor Sokolov, G\'abor T\'oth and Ovsei Volberg.

The core design and code development was done by G\'abor
T\'oth, Igor Sokolov and Ovsei Volberg:
\begin{itemize}
\item Component registration and layout was designed and implemented by 
      Ovsei Volberg and G\'abor T\'oth.
\item The session and time management support as well as the various
      configuration scripts and the parameter editor GUI were designed and
      implemented by G\'abor T\'oth.
\item The SWMF coupling toolkit was developed by Igor Sokolov.
\item The SWMF GUI was designed and implemented by Darren De Zeeuw.
\end{itemize}
The SWMF has undergone major improvements over the years. 
The current version has a fully automated test suite designed
by Gabor Toth. Empirical models were added in a systematic way.
The shared libraries and utilities have been greatly extended.
The SWMF has been coupled to the Earth System Modeling Framework (ESMF).

The current physics 
models were developed by the following research groups
and individuals:
\begin{itemize}
\item
The Lower Corona (LC), Solar Corona (SC), Inner Heliosphere (IH), 
Outer Heliosphere (OH), and the Global Magnetosphere (GM) 
components are based on \BATSRUS\ MHD code developed at CSEM. 
\BATSRUS\ is highly parallel up to 3-dimensional block-adaptive 
hydrodynamic and MHD code. 
Currently the main developers of \BATSRUS\ are Gabor Toth,
Bart van der Holst, Igor Sokolov, and Darren De Zeeuw.
The Lower Corona specific part was developed by
Cooper Downs and Ilia Roussev at the University of Hawaii.
The current version of the Solar Corona model was developed
by Bart van der Holst, Chip Manchester, Igor Sokolov, and Ofer Cohen. 
The Inner Heliosphere model was mostly developed by Chip Manchester.
The Outer Heliosphere model was developed by Merav Opher and Gabor
Toth. The Global Magnetosphere model was developed by 
Darren De Zeeuw, Gabor Toth, Aaron Ridley and many others.
The physics based Eruptive Even Generator model is also based on
\BATSRUS. It is developed by Fang Fang, Chip Manchester and Bart van
der Holst. The EE model alrady works as a stand-alone code, and
it will by coupled to the SWMF in the future.

\item
The Ionospheric Electrodynamics (IE) model is the Ridley Ionosphere Model
developed by Aaron Ridley, Darren De Zeeuw and Gabor Toth at CSEM.
RIM  is a 2-dimensional spherical electric potential solver.
It has two versions. The Ridley\_serial version can run on up to 2 processors,
the RIM version is fully parallel with many new options, however it is still
being developed, and it is not yet fully functional.

\item
The first Inner Magnetosphere (IM) component in the SWMF was
the Rice Convection Model (RCM) developed Dick Wolf, Stan Sazykin and 
others at Rice University, also modified by Darren De Zeeuw at the
University of Michigan. The current version of the model with oxygen and loss
is named RCM2. The RCM code is 2-dimensional in space 
(plus one dimension for energy) and serial. There are 3 more IM models.
The CRCM model was developed by Mei-Ching Fok, Natasha Buzulukova and 
Alex Glocer at the NASA Goddard Space Flight Center.
The HEIDI model was developed by Mike Liemohn and Raluca Ilie 
at the University of Michigan.
The RAM-SCB model was developed by Vania Jordanova, Sorin Zaharia and
Dan Welling. RAM-SCB is currently only available at Los Alamos National
Laboratory. The CRCM, HEIDI and RAM-SCB models are also 2 dimensional
in space, but they resolve energy as well as pitch angle.
Currently only the RCM and CRCM models are 2-way coupled with the
GM component. 

\item
The Polar Wind (PW) component is the Polar Wind Outflow Model (PWOM)
developed by Alex Glocer, Gabor Toth and Tamas Gombosi
at the University of Michigan.  This code solves the
multifluid equations along multiple field lines and it is fully parallel.

\item
The Radiation Belt Environment (RBE) model is developed by Meiching Fok
and Alex Glocer at NASA Goddard. It is a spatially 2-dimensional code with 
extra two dimensions for pitch angle and energy. 

\item
The Solar Energetic Particle (SP) component is the
K\'ota's SEP model which was developed by Joseph Kota 
at the University of Arizona.
It solves the equations for the advection and acceleration of
energetic particles along a magnetic field line in a 3D phase space.

\item
The Upper Atmosphere (UA) model is the 
Global Ionosphere-Thermosphere Model (GITM) 
developed by Aaron Ridley, Yue Deng, and Gabor Toth at CSEM.
GITM is 3-dimensional spherical
fully parallel hydrodynamic model with ions, neutrals, chemistry etc.
The current version is the GITM2 model.

\end{itemize}
The following empirical models are available:
\begin{itemize}
\item[EEE] The Gibbson-Low and the Titov-Demoulin flux rope models
          can be used to initiate CME-s. The breakout model is
          also available.
\item[EGM] The Tsyganenko 1996 and 2004 models.
\item[EIE] The Weimer 1996 and 2000 models, and many more empirical
          ionospheric electrodynamics models.
\item[EUA] The MSIS and IRI models for the upper atmospher and the 
          ionosphere, respectively.
\end{itemize}

\section{The SWMF in a Few Paragraphs}

The SWMF is a structured collection of software building blocks that
can be used or customized to develop Sun-Earth system modeling
components, and to assemble them into applications. The SWMF consists
of utilities and data structures for coupling model components. The
SWMF contains a Control Module (CON), which is responsible for
component registration, processor layout for each component and
coupling schedules.  It controls initialization and execution of the
components. A component is adapted from user-supplied physics codes,
(for example \BATSRUS\ or RCM), by adding two relatively small units
of code:
\begin{itemize}
\item A wrapper, which provides the control functions, and
\item A coupling interface to perform the data exchange with other
components.
\end{itemize}
Both the wrapper and coupling interface are constructed from the
building blocks provided by the framework. From 
component software technology perspective both the wrapper and
coupling interface are component interfaces: the wrapper is an
interface with CON, and the coupling interface is an interface with
another component. A physics
model code and its wrapper, which comprise a component, share the
communication group.  The coupling interface uses the union
communicator of the two components that it links together.

An SWMF component is compiled into a separate library that resides in
the directory {\tt lib}, which is created as part of the installation
process described later in this document.  Currently the component
libraries are static libraries. The executable image is created in the
directory {\tt bin}, which is created during the compilation.  If a
user does not want to build some particular component, this component
should be substituted by an empty version of the component.

An important feature of the SWMF is the component registration.  A
component to be included in the run should be registered by the
framework.  Currently entering the line for the component in the input
file called {\tt LAYOUT.in} does the registration.  Thus the SWMF
performs the run-time registration of components.

The framework controls the initialization, execution, coupling and
finalization of components.  The execution is done in sessions. In
each session the parameters of the framework and the components can be
changed.  The parameters are read from the {\tt PARAM.in} file, which
may contain further included parameter files.  These parameters are
read and broadcast by CON and the component specific parameters are
sent to the components. The structure of the parameter file will be
described in detail.

If two components reside on different sets of processing elements
(PE-s) they can execute in an efficient concurrent manner.
This is possible, because the coupling times (in terms of the simulation time
or number of iterations) are known in advance.  
The components advance to the time of coupling and
only the processors involved in the coupling need to communicate with
each other. The components are also allowed to share some processing elements.
The execution is sequential for the components with overlapping layouts.
This can be useful when the execution time of the components vary a lot
during the run, or when a component needs a lot of processors 
for memory storage, but it requires little CPU time.
Of course this still allows the individual components to execute in parallel.
For steady state calculations the components are allowed to progress
at different rates towards steady state. Each component can be called
at different frequencies by the control module.

The coupling of the components is realized either with plain MPI
calls, or via the SWMF coupling toolkit, which can couple components
based on the following types of parallel distributed grids:
\begin{itemize}
\item 3D block adaptive (AMR) parallel grid
\item 2D spherical grid
\item Logically Cartesian uniform grid
\item Logically Cartesian non-uniform grid 
\end{itemize}
The SWMF coupling toolkit performs an efficient N to M parallel
coupling based on a router. The router is calculated in advance using
the domain decomposition and grid description obtained from the
components.  The router is updated only when the domain decompositions
or the grids of the components change, or when the mapping geometry
changes.  The coupling toolkit takes care of linear interpolation in
space based on the grid descriptor.  Temporal interpolation is not
supported by the current implementation.

The framework has been ported to many platforms using many different
compilers and MPI libraries. The list of compilers includes
Absoft, gfortran, ifort, Lahey, NAG, PGF90, and XLF90.
The MPI libraries used so far include many versions of MPICH, MVAPICH and 
OpenMpi. 

\section{System Requirements}

In order to install and run the SMWF the following minimum system
requirements apply.

\begin{itemize}
\item The SWMF runs only under the UNIX/Linux operating systems.  This now
  includes Macintosh system 10.x because it is based on BSD UNIX.  The
  SWMF does not run under any Microsoft Windows operating system.
\item A FORTRAN 77 and FORTRAN 90 compiler must be installed.
\item The Perl interpreter must be installed.
\item A version of the Message Passing Interface (MPI) library must be
  installed for parallel execution.
\item You may be able to compile the code and do very small test
runs on 1 or 2 processor machines.  However, to do most physically
meaningful runs the SWMF requires a parallel processor machine with a 
minimum of 8 processors and a minimum of 8GB of memory.
\item Very large runs require many more processors.
\item In order to generate the documentation you must have LaTex installed on
your system.  The PDF generation requires the {\tt dvips} and {\tt ps2pdf}
utilities.  To generate the HTML version you also must install the
{\tt latex2html} package. 

\end{itemize}

In addition to the above requirements, the SWMF output is designed to
be visualized using either IDL or Tecplot.  You may be able to
visualize the output with other packages, but formats and scripts have
been designed for only these two visualization softwares.




%-----------------------------------------------------------------------
% Chapter 2
%-----------------------------------------------------------------------

\chapter{Quick Start}

\section{A Brief Description of the SWMF Distribution}

The distribution in the form of the compressed tar image
includes the SWMF source code.
The top level directory contains the following subdirectories:
\begin{itemize}\itemsep=0pt
\item {\tt CON}     - the source code for the control module of the SWMF
\item {\tt Copyrights} - copyright files
\item {\tt ESMF}    - the ESMF wrapper for the SWMF
\item {\tt GM, IE, ... UA} - component directories
\item {\tt Param}   - description of CON parameters, parameter and layout files
\item {\tt Scripts} - shell and Perl scripts
\item {\tt doc}     - the documentation directory %^CMP IF DOC
\item {\tt gui}     - the SWMF graphical user interface
\item {\tt output}  - reference test results for the SWMF tests
\item {\tt share}   - shared scripts and source code
\item {\tt util}    - utilities such as TIMING, NOMPI, empirical models, etc.
\end{itemize}
and the following files
\begin{itemize}\itemsep=0pt
\item {\tt README}        - a short instruction on installation and usage
\item {\tt Makefile}      - the main makefile
\item {\tt Makefile.test} - the makefile containing the tests %#^CMP IF TESTING
\item {\tt Config.pl}     - Perl script for (un)installation and configuration
\end{itemize}

\section{General Hints}

\subsubsection{Getting help with scripts and the Makefile}

Most of the Perl and shell scripts that are distributed with the SWMF
provide help which can be accessed as follows using the {\tt -h} flag.
For example, 
\begin{verbatim}
  Config.pl -h
\end{verbatim}
will provide a detailed listing of the options and capabilities of the
{\tt Config.pl} script.  In addition, you can find all the possible
targets  that can be built by typing
\begin{verbatim}
make help
\end{verbatim}

\subsubsection{Input commands: PARAM.XML}

A very useful set of files to become familiar with are the {\tt PARAM.XML}
files.  Such a file exists for the SWMF itself and for each of the
physics components.  The file for the SWMF is found at
\begin{verbatim}
Param/PARAM.XML
\end{verbatim}
while the files for the physics components are found in the component's
subdirectory.  For example, the file for the GM/BATSRUS component can
be found at
\begin{verbatim}
GM/BATSRUS/PARAM.XML
\end{verbatim}
This file contains a complete list of all input commands for the
component as well as the type, the allowed ranges and default values
for each of the input parameters.
Although the XML format makes the files a little hard to read, they are
extremely useful.  A typical usage is to cut and paste commands out of the
PARAM.XML file into the PARAM.in file for a run. 

An alternative approach is to use the web browser based parameter editor 
to edit the PARAM.in file for the SWMF 
(also for the stand-alone models that have PARAM.XML files).
The editor GUI can be started as
\begin{verbatim}
share/Scripts/ParamEditor.pl
\end{verbatim}
This editor allows constructing PARAM.in files with pull down menus, 
shows the manual for the edited commands, and checks the correctness of
the parameter file and highlights the errors. All this functionality 
is based on the PARAM.XML files.

\subsubsection{Have the working directory in your path}

In order to run executable files in the UNIX environment you must have
the current working directory either your path or in the filename you
want to execute. In UNIX the current working directory is represented
by the period (.).  For example
\begin{verbatim} 
./Config.pl -s
\end{verbatim}
will execute the Config.pl script if it is in your current directory.  
If you add the `.' to your path using for example
\begin{verbatim}
set path = (${path} .)
\end{verbatim}
then you can simply type
\begin{verbatim} 
Config.pl -s
\end{verbatim}
Setting the path is best done in the .cshrc or equivalent Unix shell 
customization file located in the user's home directory.

\section{Installation}

The installation instructions are described in the README file.
To keep this user manual more up-to-date and consistent, 
the README file is quoted verbatim below.

\input ../../README

\section{Platform specific information}

\subsection{Pleiades machine at NASA Ames}

Currently on pleiades you should load the following modules
\begin{verbatim}
   module load intel-comp.11.1.072
   module load mpi/mpt.1.25
\end{verbatim}
It is best adding the above statements into the .cshrc file in the 
home directory.
Note that ifort 12.0 does not work correctly.
The module versions keep changing. Use
\begin{verbatim}
   module avail
\end{verbatim}
to see the list of all available modules.

\subsection{Nyx and Flux at the University of Michigan}

Note that on nyx/flux the tcsh shell has to be started manually after login
by typing 'tcsh'. You can either use the pgf90, NAG or ifort compilers. 
For the pgf90 compiler use
\begin{verbatim}
  Config.pl -install -compiler=pgf90
  module load pgi
  module swap openmpi openmpi/1.3.2-pgi
\end{verbatim}
For the NAG compiler use
\begin{verbatim}
  Config.pl -install -compiler=f95
  module load nag/5.1
  module swap openmpi/1.3.2-nag
\end{verbatim}
For the Intel ifort compiler use
\begin{verbatim}
  Config.pl -install -compiler=ifort
  module load intel-comp
  module swap openmpi openmpi/1.4.3/intel/11.0
\end{verbatim}
Currently the most reliable compiler on nyx/flux is pgf90, 
but this may change. The ifort compiler often gives better performance.
The NAG compiler is recommended for debugging.
Note that gfortran 4.1.2 (current version on nyx/flux) 
does not work properly with the SWMF.
The module versions keep changing. Use
\begin{verbatim}
  module avail
\end{verbatim}
to see the list of all available modules.

\section{Building and Running an Executable}

At compile time, the user can select which physics components should be
compiled.  
Any component not compiled will not be available for
use at run time.  The physics components can be selected with the {\tt -v} flag
of the Config.pl script. For example typing
\begin{verbatim}
  Config.pl -v=Empty,SC/BATSRUS,IH/BATSRUS,SP/Kota
\end{verbatim}
will select BATSRUS for the SC and IH components and K\'ota's model for
the SP component and the other components are set to Empty versions
that contain empty subroutines for compilation, but cannot be used.
The default configuration includes a working version for all components, 
which takes up more memory, but is the most general.
The only exception is SC, which requires configuration, so the 
default version is Empty for the Solar Corona component.

The grid size of several components can also be set with the {\tt -g}
flag of the {\tt Config.pl} script. For example the 
\begin{verbatim}
  Config.pl -g=GM:8,8,8,400,100
\end{verbatim}
command sets the block size for the GM component to $8\times 8\times 8$ cells, 
the maximum number of blocks per processor to 400, 
and the maximum number of implicit blocks per processor to 100.
The main SWMF Config.pl script actually runs the individual Config.pl
scripts in the component versions. These scripts can be run directly,
For example try
\begin{verbatim}
  cd GM/BATSRUS
  Config.pl -show
\end{verbatim}
Compilation rules, library definitions, debugging flags, and optimization 
level are stored in {\tt Makefile.conf}. This file is created during
installation of the SWMF and contains default settings for production runs.
The compiler flags can be modified with
\begin{verbatim}
  Config.pl -debug -O0
\end{verbatim}
to debug the code with 0 optimization level, and
\begin{verbatim}
  Config.pl -nodebug -O4
\end{verbatim}
to run the code at maximum optimization level and without the debugging flags.

Before compiling SWMF it is always a good idea to check its configuration
with
\begin{verbatim}
  Config.pl -show
\end{verbatim}

To build the executable {\bf bin/SWMF.exe}, type:
\begin{verbatim}
  make
\end{verbatim} 
Depending on the configuration, the compiler settings and the machine 
that you are compiling on, this can take from 2 to up to 30 minutes.  
In addition, you may want to make the post processing
codes (for BATSRUS only) also:
\begin{verbatim}
  make PSPH
  make PIDL
\end{verbatim} 
These two commands will create the codes {\tt bin/PostSPH.exe}, for post
processing spherical Tecplot files, and {\tt bin/PostIDL.exe} 
for post processing IDL files.

The {\tt SWMF.exe} executable should be run in a sub-directory, 
since a large number of files are created in each run.  
To create this directory use the command:
\begin{verbatim}
  make rundir
\end{verbatim} 
This command creates a directory called {\tt run}.  You can either
leave this directory as named, or {\tt mv} it to a different name.  It
is best to leave it in the same SWMF directory, since
keeping track of the code version associated with each run is quite
important. On some platforms, however, the runs should be done on a
parallel file system (often called scratch or nobackup), while the
source code is better kept in the home directory. In this case move
the run directory to the scratch disk and create a symbolic link to it, 
for example
\begin{verbatim}
  mv rundir /p/scratch/MYNAME/SWMF/run_halloween2
  ln -s /p/scratch/MYNAME/SWMF/run_halloween2 .
\end{verbatim}
The {\tt run} directory will contain links to the codes
which were created in the previous step as well as subdirectories
where input and output of the different components will reside.
On some systems the compute nodes cannot access symbolic links
accross different file systems. In this case the executable should be 
copied instead of linked, so in our example the following commands
should be done every time after the SWMF has been (re)compiled:
\begin{verbatim}
  rm -f run_halloween2/SWMF.exe
  cp bin/SWMF.exe run_halloween2/
\end{verbatim}
To run the SWMF change directory into the {\tt run} directory 
(or the symbolic link to it):
\begin{verbatim}
  cd run_halloween2
\end{verbatim}
In order to run the SWMF you must have two input files:  LAYOUT.in and
PARAM.in.  The LAYOUT.in file defines the processor
layout for the components involved in the future run.  The PARAM.in
file contains the detailed commands for controlling what you want the
code to do during the run.  The {\tt Param} directory contains many
example input files. Many of these are used by the nightly test suite.

An example processor map file LAYOUT.in to run the executable with
five components on 16 processors is:
\begin{verbatim}
#COMPONENTMAP
GM    0    4    1
IE    5    6    1
IH    7   10    1
IM   11   11    1
UA   12   15    1
#END
\end{verbatim}
The file syntax is simple. It must start with the directive
\#COMPONENTMAP and end with another directive \#END. Each line between
these directives specifies the label for component, i.e. IE, GM and
etc., its first and last processor, all relatively to the world
communicator, and the stride. Thus GM will run on 5 processors from 0
to 4, and IM will run on only 1 processor, the processor 11.  If
stride is not equal to 1, the processors for the component will not be
neighboring processors.

It is strongly recommended to check the validity of the {\tt PARAM.in} and 
{\tt LAYOUT.in} files before running the code. If the
code will be run on 16 processors, type
\begin{verbatim}
Scripts/TestParam.pl -n=16 run_halloween2/PARAM.in
\end{verbatim}
in the main SWMF directory.
The Perl script reports inconsistencies and errors. 
If no errors are found, the script finishes silently.
Now you are ready to run the executable through submitting a batch job or, 
if it is possible on your computer, you can run the code interactively.  For
example, to run the SWMF interactively:
\begin{verbatim}
cd run_halloween2
mpirun -np 16 SWMF.exe
\end{verbatim}
The SWMF provides example job scripts for several machines. 
These job scripts are found in 
\begin{verbatim}
share/JobScripts
\end{verbatim}
in the subdirectories named after the operating system. If the name
of the file in the appropriate subdirectory matches the 
name of the machine, the job script is copied into
the {\tt run} directory when it is created.
These job scripts serve as a starting point only, they must
be customized before they can be used for submitting a job.

To recompile the executable with different compiler settings you have
to use the command
\begin{verbatim}
make clean
\end{verbatim}
before recompiling the executables. It is possible to recompile
only a component or just one subdirectory if the {\tt make clean}
command is issued in the appropriate directory.

\section{Post-Processing the Output Files}

Several components produce output files (plot files) that require
some post-processing before they can be visualized. The post-processing
collects data written out by different processors, and it can also
process and transform the data. 

The PostProc.pl script greatly simplifies the post-processing and
it also helps to collect the run results in a well contained directory tree.
The script can also be used to do post-processing while the code is running.
Usually the processed output files are much smaller than the raw output file,
so post-processing during the run can limit the amount of disk space used
by the raw data files. It also avoids the need to wait for a long time 
for the post-processing after the run is done. 

The PostProc.pl script is copied into the run directory and it should
be executed from the run directory.
To demonstrate the use of the script, here are a few simple examples.
After or during a run, you may simply type
\begin{verbatim}
cd run_halloween2
./PostProc.pl
\end{verbatim}
to post-process the available output files. The series of individual 
IDL plot files can be concatenated into single movie files with
\begin{verbatim}
./PostProc.pl -M
\end{verbatim}
Repeat the post-processing every 360 seconds during the run,
and gzip large ASCII files:
\begin{verbatim}
./PostProc.pl -r=360 -g >& PostProc.log &
\end{verbatim}
After the run is finished, create IDL movie files and concatenate
various log and satellite files (for restarted runs),
and create a directory tree with the output
of all the components, the input parameter file and the 'runlog' file
(if present)
\begin{verbatim}
./PostProc.pl -M -cat -o RESULTS/NewRun
\end{verbatim}
The RESULTS/NewRun directory will contain the PARAM.in file, the
runlog file (the standard output should be piped into that file),
and the output files for each component in a subdirectory named
accordingly (eg. RESULTS/NewRun/GM/). The output directories of
the components (e.g. GM/IO2/) will be empty.

To see all the options of the script, including parallel processing and
syncing results to a remote computer, type
\begin{verbatim}
./PostProc.pl -h
\end{verbatim}

\section{Restarting a Run}

There are several reasons for restarting a run. A run may fail
due to a run time error, due to hardware failure, due to 
software failure (e.g. the machine crashes) or because the
queue limits are exceeded. In such a case the run can be continued from
the last saved state of SWMF. 

It is also possible that one builds up a complex simulation from multiple 
runs. For example the first run creates a steady state for the SC component.
The second run includes both the SC and IH components and it 
restarts from the results of the first run and creates a steady state
for both components. A third run may restart from this solution and include
the GM component, etc. 

The restart files are saved at the frequency determined in the PARAM.in file.
Normally the restart files are saved into the output restart directories
of the individual components and subsequent saves overwrite the previous ones
(to reduce the required disk space). A restart requires the modification
of the PARAM.in file: one needs to include the restart file for the
control module of SWMF as well as ask for restart by all the components.

The Restart.pl script simplifies the work of the restart in several ways:
\begin{enumerate}
\item The SWMF restart file and the individual output restart 
directories of the components are collected into a single directory tree, 
the {\bf restart tree}.
\item The default input restart file of SWMF and the default 
      input directories of the components can be linked to an existing
      restart tree.
\item The script can run continuously in the background and create
      multiple restart trees while SWMF is running. 
\item The script does extensive checking of the consistency 
      of the restart files.
\end{enumerate}
The Restart.pl script is copied into the run directory and it should
be executed in the run directory. Note that the PARAM.in file is not
modified by the script: it has to be modified with an editor as needed.

To demonstrate the use of the script, here are a few simple examples.
After a successful or failed run which should be continued, simply type
\begin{verbatim}
cd run_halloween2
./Restart.pl
\end{verbatim}
to create a restart tree from the final output and to link to the tree for the
next run. The default name of the restart tree is based on the simulation time
for time accurate runs, or the time step for non-time accurate runs.
But you can also specify a name explicitly, for example
\begin{verbatim}
./Restart.pl RESTART_SC_steady_state
\end{verbatim}
If you wish to continue the run in another run directory, or on another
machine, transfer the restart tree as a whole into the new run
directory and type
\begin{verbatim}
./Restart.pl -i=RESTART_SC_steady_state
\end{verbatim}
where the {\tt -i} stands for ``input only'', i.e. the script links to
the tree, but it does not attempt to create the restart tree.

To save multiple restart trees repeatedly at an hourly frequency of 
wall clock time while the SWMF is running, type
\begin{verbatim}
./Restart.pl -r=3600 &
\end{verbatim}
To see all the options of the script type
\begin{verbatim}
./Restart.pl -h
\end{verbatim}

\section{What's next?}

Hopefully this section has guided you through installing the SWMF and
given you a basic knowledge of how to run it.  However it has probably
also convinced you that the SWMF is quite a complex tool and that there
are many more things for you to learn.  So, what next?

We suggest that you read all of chapter \ref{chapter:basics}, which
outlines the basic features of the SWMF as well as some things you
really must know in order to use the SWMF.  Once you have done this you
are ready to experiment.  Chapter \ref{chapter:examples} gives several 
examples which are intended to make you familiar with the use of the
SWMF.  We suggest that you try them!

%\end{document}
