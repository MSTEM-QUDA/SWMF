%  Copyright (C) 2002 Regents of the University of Michigan, portions used with permission 
%  For more information, see http://csem.engin.umich.edu/tools/swmf
\documentclass[twoside,10pt]{article}

\title{Testing Procedures for the Space Weather Modeling Framework\\
       \hfill \\
       \SWMFLOGOVERSION}

\author{G\'abor T\'oth\\
  {\it Center for Space Environment Modeling}\\
  {\it The University of Michigan}}

\input HEADER

\section{Description of the Testing Philosophy}

The Space Weather Modeling Framework (SWMF) is
composed of the core of the framework and the various components
modeling the physics domains. Since the components are developed 
independently, it is neither possible nor desirable to enforce a
rigorous and comprehensive testing procedure for each component.
As a minimum, however, we require that new components are provided
with at least one functionality test before they can be integrated
into the SWMF.

On the other hand we must ensure that the core of the SWMF and
the key components developed at the Center for Space Environment
Modeling (CSEM) are well tested and reliable. 
We also established some base-line testing for the whole
framework involving all the components to verify that the SWMF
works as expected for a more or less typical space weather simulation
run. This test can be run on different platforms, and it serves as
the portability testing for the SWMF.

In summary our testing philosophy follows a layered approach:
\begin{itemize}
\item The SWMF core: individual unit testing
\item Key components: extensive functionality test suite
\item New components: at least one functionality test
\item The SWMF as a whole: comprehensive portability tests
\end{itemize}
The core of the SWMF is tested by unit testing, by the functionality
test suite and also by the portability tests. The key components
are tested by the functionality test suite and the portability tests.
New components are tested by their functionality test and the 
portability tests. Finally all components and couplers are tested
by the portability tests.

\section{Description of Unit Testing}

The core of the SWMF consists of two layers:
\begin{itemize}
\item The super structure:  CON/Control, CON/Interface, CON/Stubs
\item The infra structure: CON/Library, CON/Coupler, share/, util/
\end{itemize}
The super structure, as the name suggests, can only be tested
together with the components and the infra structure. 
This complexity can be somewhat reduced if the components are replaced
with {\it stubs}. The stub components do not do any computation,
they simply advance the simulation time and use up some CPU time.
The stub components are implemented in the CON/Stubs directory.

\subsection{Testing CON/Control with CON/Stubs}

The SWMF can be compiled with the stub components if the 
\begin{verbatim}
INT_VERSION = Stubs
\end{verbatim}
is selected in {\tt Makefile.def}. All component versions
can be set to 'Empty'. With this choice the core
of the SWMF can be tested without running the components.
An example parameter file is provided in
\begin{verbatim}
Param/PARAM.in.test.stubs
\end{verbatim}
To run the test, compile the SWMF with
\begin{verbatim}
./Config.pl -v=Empty
make -j SWMF INT_VERSION=Stubs
\end{verbatim}
then create the run directory and run the test:
\begin{verbatim}
make rundir
cd run
cp Param/PARAM.in.test.stubs PARAM.in
mpiexec -n 2 SWMF.exe
\end{verbatim}
This test is intended for developers only, so the output is
relatively complicated. Other than printing to the screen,
the test creates several files
\begin{verbatim}
cd run
ls STDOUT/*.log ??restart_*
\end{verbatim}
The stub components can also be used to predict the parallel 
execution time for various layouts and control parameters.
An alternative approach is to use the Scripts/Performance.pl script.

\subsection{Testing CON/Library}

There is a test for the registration of components.
The list and layout of registered components is described by the 
\#COMPONENTMAP command in the PARAM.in file.
An example file is provided in CON/Library/src.
To test the reading of this file and the various functions provided
by CON\_world, CON\_comp\_info and CON\_comp\_param, run 
\begin{verbatim}
cd CON/Library/src
make test NP=4 NTHREAD=2
\end{verbatim}
This test is intended for developers only, 
so the output is relatively complicated.
One can change the PARAM.in file or the number of processors to
do more extensive testing.

The other modules in this directory (CON\_time and CON\_physics) are
relatively simple and they do not have a unit tester. These modules
are tested in the functionality and portability tests.

\subsection{Testing CON/Interface}

The CON/Interface directory contains the couplers between the 
components of the SWMF. These couplers cannot be tested by themselves.
The interfaces are tested by the portability tests 
(see section~\ref{sec:portability}).

\subsection{Testing CON/Coupler}

The CON/Coupler directory contains the parallel coupling toolkit of the SWMF.
This toolkit is used in some of the component couplers. 
The unit tester for the coupling toolkit is CON\_test\_global\_message\_pass.
This module has been used in the past to test the SWMF coupling toolkit. 
To avoid various problems with the compilers on the SGI Altix machines, 
the unit tester has been removed recently. 
The coupling toolkit is tested by the portability tests
(see section~\ref{sec:portability}).


\subsection{Testing share/Library}

This library is used by the SWMF as well as the stand alone components.
It is crucial to thoroughly test all methods provided by this library.
To run the unit tests
\begin{verbatim}
cd share/Library/test
make test
\end{verbatim}
Although the output looks rather complex, it is mostly caused by the
compiler messages and the verbose information provided by the make
program. To get a cleaner output, rerun the tests as
\begin{verbatim}
make -s tests
\end{verbatim}
There should be now very limited output reporting the tests
of the various modules and methods. Some of the tests may show small
differences relative to the expected results due to round off errors.


\subsection{Testing util/TIMING and util/NOMPI}

The TIMING utility provides a simple and compiler independent utility
to measure the CPU time spent on various parts of a Fortran code.
The NOMPI utility allows to compile an MPI parallel F90 code with the
NOMPI library instead of the MPI library. The resulting executable
can run on a single processor. The NOMPI utility is useful for debugging 
purposes.

The TIMING can be tested with
\begin{verbatim}
cd util/TIMING/src
make tests
\end{verbatim}
An example output is provided in 'tests.log'. To compare with this,
rerun the tests like this
\begin{verbatim}
make -s tests > tmp.log
diff tmp.log tests.log
\end{verbatim}
Note that the timings and the order of the output can vary from
test run to test run. Two of the four tests involve the NOMPI
library, although only a few of the NOMPI methods are used.

\section{Description of Functionality Testing}

The functionality tests check the functionality of 
the software in typical configurations. The SWMF uses a hierarchical
test suite. The top level Makefile.test contains various SWMF
tests that exercise various subsets of the models running together.
Each model inside the SWMF has (or should have) functionality tests that
check the model in stand-alone mode. In addition, there are tests
for the shared library and some of the utilities. All tests can be executed
with
\begin{verbatim}
make -j test NP=4 NTHREAD=2 >& test_swmf.log
\end{verbatim}
where NP sets the number of cores and NTHREAD sets the number OpenMP threads to be used.
The tests are typically run from 1 to 8 cores and 1 to 2 threads. 

The results of the tests are written into .diff files. An empty .diff
file means that the test passed. The results can be collected with
\begin{verbatim}
make test_check
\end{verbatim}
Use
\begin{verbatim}
make test_help
\end{verbatim}
to see a complete list of functionality tests for the SWMF.
For individual models or libraries, go into the model directory:
\begin{verbatim}
cd GM/BATSRUS
make test_help
cd ../../PW/PWOM
make help
cd  ../../share/Library/test
make help
\end{verbatim}
The functionality tests are executed on various machines with various compilers and different
number of processurs every night. The results are collected to a website, currently available at
\begin{verbatim}
http://herot.engin.umich.edu/~gtoth/
\end{verbatim}

\end{document}
