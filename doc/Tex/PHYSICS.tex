%^CMP COPYRIGHT UM
\documentclass[twoside,10pt]{article}

\title{Physics of the Space Weather Modeling Framework\\
  \hfill \\
  \SWMFLOGOVERSION}

\author{Center for Space Environment Modeling\\
  {\it The University of Michigan}\\
  \hfill\\
  {\bf The text of this document}\\
  {\bf is based on a section of a paper}\\
  {\bf submitted to J. Geophys. Res.}
}

\newcommand{\RE}{{\mathrm R}_{\mathrm{E}}}
\newcommand{\RS}{{\mathrm R}_{\mathrm{S}}}
\newcommand{\apj}{{\it Astrophys. J.,}}
\newcommand{\jgr}{{\it J. Geophys. Res.,}}
\newcommand{\grl}{{\it Geophys. Res. Lett.,}}

\input HEADER

\section{Introduction}


The Sun-Earth system is a complex natural system of many different,
interconnecting elements. The solar wind transfers significant mass,
momentum and energy to the magnetosphere, ionosphere, and upper
atmosphere, and dramatically affects the physical processes in each of
these physical domains.  The ability to simulate and eventually
predict space-weather phenomena is important for many applications,
for instance, the success of spacecraft missions and the reliability
of satellite communication equipment. In extreme cases, the magnetic
storms may have significant effects on the power grids used by
millions of households.

The various domains of the Sun-Earth system can be simulated with
stand-alone models if simplifying assumptions are made about the
interaction of a particular domain with the rest of the system.
Sometimes the effects of the other domains can be taken into account
by the use of satellite and ground based measurements.  In other
cases, statistical and/or phenomenological models can be used. For the
prediction of space-weather events, however, one must use
first-principles-based physics models for all of the involved domains
and these models must execute and be coupled in an efficient manner,
so that the simulation can run faster than real-time.

\section{Physics Domains and Their Couplings \label{sec:domain}}

The current version of the SWMF includes nine physics domains
ranging from the surface of the Sun to the surface of a planet
(usually the Earth). The nine physics domains are the following:
\begin{itemize}
\item Solar Corona (SC),
\item Eruptive Event Generator (EE),
\item Inner Heliosphere (IH),
\item Solar Energetic Particles (SP),
\item Global Magnetosphere (GM),
\item Inner Magnetosphere (IM),
\item Radiation Belt (RB),
\item Ionosphere Electrodynamics (IE),
\item Upper Atmosphere (UA).
\end{itemize}
Each physics domain corresponds to a component of the framework.  Each
component can have multiple versions.  A {\it component version} is
based on a particular {\it physics model}, which is implemented by a
particular {\it physics module}.

Below we briefly describe all nine physics domains, the
typical coordinate systems, the equations to be solved, the
boundary conditions, and the couplings with the other domains.
A component is responsible for solving the dynamical equations 
in its domain, and it is also responsible for receiving and
providing information as needed. The most computationally
challenging couplings are described in more detail, 
since these present additional
tasks to be accomplished by the components.

\subsection{Solar Corona -- SC}

The Solar Corona domain extends from the low corona at $\approx1\,\RS$
(solar radius) to approximately
$24\,\RS$. The physics of this domain is well approximated with
the equations of magnetohydrodynamics, however, 
additional source terms are required to take into account
the heating and acceleration of the solar wind \cite{Groth2000, Usmanov2000}.
Alternative models mimic the coronal heating by incorporating a variable
adiabatic  index \cite{Wu1999}, or solve for one extra equation that
describes the energy interchange between the solar wind plasma and
the large-scale MHD turbulence \cite{Roussev2003b}.
The SC component can be in an inertial (e.g. HGI -- Heliographic Inertial)
frame or in a rotating (e.g. HGR -- Heliographic Rotating) frame.
In a rotating frame the inertial forces must be included.

The inner boundary of the SC component is driven by the 
density, pressure, velocity and magnetic field defined just above
the photosphere. The magnetic field may be obtained from synoptic magnetograms,
or a simple dipole (possibly with a few higher order terms) 
may be assumed. The boundary conditions for the
temperature and mass density at the Sun may vary with longitude and
latitude to achieve the most realistic solar wind near the Sun and at 1AU. 
The velocity components at the inner boundary should maintain 
line-tying of the magnetic field.
The flow at the outer boundary is usually superfast (faster than 
the fast magnetosonic speed of the plasma), so no information
is propagating inward. Sometimes, however, when a coronal mass ejection 
(CME) passes the boundary, the solar wind speed may become subfast 
for short periods of time. 
During such periods, the SC component needs to receive the outer
boundary condition from the Inner Heliosphere. 

The Solar Corona provides the plasma variables at the inner boundary 
of the Inner Heliosphere. The inner boundary of the IH component 
does not have to coincide with the outer boundary of the Solar Corona, i.e.
the two domains are allowed to overlap. Such an overlap is actually 
numerically advantageous when the flow becomes subfast for a short time.
The overlap can reduce reflections or other numerical artifacts at the
inner boundary of IH.
The Solar Corona also provides information to the Solar Energetic
Particle domain: the geometry of one or multiple field lines
and the plasma parameters along each field line are provided to the
SP component.

\subsection{Eruptive Event Generator -- EE}

The EE domain is embedded in the Solar Corona, and it is restricted
to the region of the eruptive event, which is typically in the form of
a coronal mass ejection (CME).
To date, we lack good understanding of
the actual physical processes by which a CME is initiated,
and it is still an active field of research.
One group of models \cite{Forbes1991, Gibson1998, Roussev2003a}
assume that a magnetic flux rope exists prior to the eruption.
Flux ropes may suddenly lose mechanical equilibrium and erupt due to:
foot-point motions \cite{Wu2000}; injection of magnetic helicity 
\cite{Chen1993}; or draining of heavy prominence material
\cite{Low2001}. Another group of models \cite{Antiochos1999,
Manchester2003, Roussev2004}
relies on the existence of sheared magnetic arcades,
which become unstable and erupt once some critical state is reached.
Here a flux rope is formed by reconnection between the opposite polarity 
feet of the arcade during the eruption process.

The EE component can be represented as a boundary condition for
the SC component, or it can be a (non-linear) perturbation of
the SC solution. In short, the EE component interacts with the SC component
only. Due to the multitude of possibilities, the EE component is integrated
into the SC component in the current implementation of the SWMF. 
Multiple EE versions are possible, but all the EE versions
belong to one SC version only.

\subsection{Inner Heliosphere -- IH}

The IH domain extends from around 20 solar radii all the way to the planet.
It does not have to cover a spherical region, it may be rectangular
and asymmetric with respect to the center of the Sun.
The physics of this domain is well approximated with the equations of
ideal MHD. The IH component is usually in an inertial (e.g. HGI) frame.

The inner boundary conditions of the IH component are obtained from 
the SC component or measurements. The flow at the
outer boundary of the IH component is always superfast
(the interaction with the interstellar medium is outside of the IH).
The Inner Heliosphere provides the same information to the SP component
as the Solar Corona. The IH component also provides the outer boundaries
for the SC component when the flow at the outer boundary of SC is
not superfast. Finally the Inner Heliosphere provides the upstream boundary
conditions for the Global Magnetosphere. The IH and GM domains overlap:
the upstream boundary of GM is typically at about $30\,\RE$ (Earth radii) 
from the Earth towards the Sun, which is inside the IH domain.

\subsection{Solar Energetic Particles -- SP}

The SP domain consists of one or more one dimensional field lines,
which are assumed to advect with the plasma.
The physics of this domain is responsible for the acceleration
of the solar energetic particles along the field lines. 
There are various mathematical models that approximate this 
physical system. They include the effects of acceleration and spatial
diffusion, and can be averaged \cite{Sokolov2004} 
or non-averaged \cite{Kota1999,Kota2005} with respect to pitch angle.

The geometry of the field line and the plasma parameters along
the field line are obtained from the SC and IH components.
The boundary conditions can be zero particle flux at the ends of 
the field line(s).
The SP component does not currently provide information to other components.

\subsection{Global Magnetosphere -- GM}

The GM domain contains the bow shock, magnetopause and 
magnetotail of the planet. The GM domain
typically extends to about $30\,\RE$
on the day side, hundreds of $\RE$ on the night side, and 
50 to 100$\,\RE$ in the directions orthogonal to the Sun-Earth line.
The physics of this domain is well approximated with the resistive
MHD equations except near the planet, where it overlaps with the
Inner Magnetosphere. 
The GM component typically uses Geocentric Solar Magnetic (GSM), 
Geocentric Solar Ecliptic (GSE) or possibly Solar Magnetic (SM) 
coordinate system.

The upstream boundary conditions are obtained from the IH component or from 
satellite measurements. At the other outer boundaries one can usually assume
zero gradient for the plasma variables since these boundaries 
are far enough from the planet 
to have no significant effect on the dynamics near the planet. 
The inner boundary of the Global Magnetosphere is 
at some distance from the center of the planet, usually at 1 to 3 planet radii.
The inner boundary conditions are partially determined 
by the Ionosphere Electrodynamics, which provides the electric potential
at the inner boundary of the GM. 
The potential is used to calculate the electric field and 
the corresponding plasma velocities, which are used as 
the inner boundary condition for the GM.  The GM component also
receives pressure and possibly density corrections from the
Inner Magnetosphere along the closed magnetic field lines
(field lines connected to the planet at both ends).
These are used to 'nudge' the MHD solution towards the 
more accurate inner magnetosphere values \cite{DeZeeuw2004}.

The GM component provides the field aligned currents to the
IE component. These currents are mapped from the 
GM down to the ionosphere along
the magnetic field lines. The Global Magnetosphere provides the
Inner Magnetosphere with the field line volume, average density
and pressure along closed field lines. Depending on the needs of the
IM component, the GM could also provide the geometry of the closed field
lines and the distribution of plasma parameters along field lines.

\subsection{Inner Magnetosphere -- IM}

The IM domain consists of the closed field line region around the planet.
This component solves equations describing the motion of keV-energy ions
and electrons. Kinetic effects are important for these particles, and
several types of theoretical models have been developed to describe them.
At least five different groups have developed codes that calculate the
distribution function of the ring current ions and associated electrons
given an inputted electric and magnetic field distribution (see review by
\cite{Ebihara2002}. 
The Rice Convection Model (RCM) (e.g., \cite{Wolf1982,Toffoletto2003}
computes field-aligned currents and ionospheric
potentials self-consistently, but still requires an inputted magnetic field
and assumes that the particles have an isotropic pitch-angle distribution
(consistent with MHD). Models developed by \cite{Fok2001},
\cite{Liemohn2004} and \cite{Ridley2002} 
compute full pitch-angle distributions as well as
field-aligned currents and ionospheric equipotentials.
A different approach uses test-particle Monte
Carlo models \cite{Chen2003, Ebihara2004}.
The IM component typically uses Solar Magnetic (SM) coordinates.

%% The boundary conditions of the IM component???
The Inner Magnetosphere obtains the geometrical and plasma information about
the closed field lines from the Global Magnetosphere. It also obtains
the electric potential solution from the Ionosphere Electrodynamics.
The IM component provides the density and pressure corrections along
the closed field lines to the GM component.
The IM may also provide field aligned currents along the closed
magnetic field lines to the IE component (but this is not done in the
current implementation of the SWMF).

\subsection{Radiation Belt -- RB}

The RB spatial domain is coincident with that of the Inner Magnetosphere
component. This component solves equations for the relativistic electron
distribution near the Earth, which are responsible for some of the most
detrimental space-weather effects. Gradient and curvature drift dominate the
motion of these particles around the Earth, and the kinetic equation is
sometimes drift-shell averaged as well as gyration and bounce averaged.
Diffusion is the primary transport mechanism left in the equation. The
physics of this domain can be solved with the same two techniques mentioned
for the Inner Magnetosphere, that is, numerical discretization of the
kinetic equation \cite{Beutier1995, Shprits2004} or test particle tracking
\cite{Elkington1999}.  The RB component typically uses Solar Magnetic (SM)
coordinates, or simply equatorial plane radial distance.

The Radiation Belt receives similar information from the Global
Magnetosphere as does the Inner Magnetosphere. 
The RB component does not provide information to the other components.  

\subsection{Ionosphere Electrodynamics -- IE}

The IE domain is a two dimensional height-integrated spherical surface
at a nominal ionospheric altitude (at around $110\,$km for the Earth).
There are several mathematical models that can describe this domain:
empirical models such as the \cite{Weimer1996} electric potential
maps, and the \cite{Fuller1987} particle precipitation and auroral
conductance maps; the assimilative mapping of ionospheric
electrodynamics \cite{Richmond1988}; and the height averaged
electric potential solver, which uses the field-aligned currents to
calculate particle precipitation and conductances 
\cite{Ridley2004,Ridley2002}.  
In the current version of the SWMF, the IE component is
a potential solver, but there is nothing in the design that would
exclude the incorporation of other types of IE models.  Usually the IE
component uses the Solar Magnetic (SM) coordinates.

The Ionosphere Electrodynamics obtains the field aligned currents from
the Global Magnetosphere and Upper Atmosphere, which is used to
generate an auroral precipitation pattern.  The UA component also
provides IE with the Hall and Pedersen conductivities. In case the UA
component is not used, the auroral pattern and the solar illumination
are used to generate Hall and Pedersen conductances.  This is done
through the use of the \cite{Robinson1987} and \cite{Moen1993}
formulation, which takes the average and total electron energy flux
and converts them to Hall and Pedersen conductances based on a simple
formula.  The IE component provides the electric potential to the GM,
IM and UA components. In addition, it provides the particle
precipitation to the UA component.

\subsection{Upper Atmosphere -- UA}

The UA domain includes the thermosphere and the ionosphere and it
extends from around $90\,$km to about $600\,$km altitude for the
Earth.  The physics of the Upper Atmosphere is rather complicated.  It
can be approximated with the equations of multi-species hydrodynamics
including viscosity, thermal conduction, chemical reactions,
ion-neutral friction, coupling of the ions to the electric field,
source terms due to solar radiation, etc.  In such a complex system
there are many possible choices even at the level of the mathematical
model. For example, one can approximate the system with the assumption
of hydrostatic equilibrium \cite{Richmond1992} or use a compressible
hydrodynamic description \cite{Ridley2005}.  The UA component is
typically in a planet-centric rotating frame, i.e. the Geocentric
(GEO) coordinate system for the Earth.

The lower and upper boundaries of the UA domain are approximated
with physically motivated boundary conditions. 
The Upper Atmosphere obtains the electric potential along 
the magnetic field lines and the particle precipitation 
from the Ionosphere Electrodynamics.
The gradient of the potential provides the electric field, which is 
used to drive the ion motion, while the auroral precipitation 
is used to calculate ionization rates.
The UA component provides field aligned currents and the Hall and
Pedersen conductivities to the IE component. The conductivities
are calculated from the electron density and integrated along field lines.

\subsection{Coupling the Inner Magnetosphere to the Global Magnetosphere}

The IM to GM coupling is the most challenging computationally. 
The GM component 
needs to know where each of its 3D grid points are mapped onto the IM grid 
along the closed magnetic field lines in order
to apply the pressure and density corrections. 
This means that field lines must be traced from possibly 
millions of grid points.
In addition, the magnetic field information is typically distributed
over many processors of the GM component.
Since the GM grid structure and the magnetic field is inherently 
known by the GM component, it is the responsibility of the GM
component to find the mapping of its 3D grid along the closed field lines.
For our implementation of the GM component, we have developed a highly parallel
method, which can accomplish this task in a few seconds \cite{Toth2005c}.

\subsection{Coupling the Global Magnetosphere to the Inner Magnetosphere}

The GM to IM coupling is also challenging computationally.  The IM
needs the magnetic field line flux tube volumes and the average
density and pressure in the flux tubes connected to its 2D spherical
grid points. This requires the integration along many (thousands of)
magnetic field lines in GM. Field line integration is an inherently
serial procedure. A further complication is that the domain
decomposition of the GM component may distribute each field line over
several processors.  We have developed an efficient parallel algorithm
\cite{Toth2005c}, which can trace and integrate along the field
lines in a fraction of a second of CPU time.  The framework provides a
library, which takes care of the information exchange and the
collection of data among the processors of GM, but the GM component is
responsible for the tracing and integration along field lines within
the subdomain corresponding to one GM processor.

\subsection{Coupling the Solar Corona and the Inner Heliosphere to
            the Solar Energetic Particles}

The SP component needs the geometry of one or more magnetic field lines,
and it also needs the plasma parameters along these lines.
This is not a computationally
intensive procedure due to the small number of field lines.
On the other hand it is an algorithmically
non-trivial problem, especially when the SP component uses
a Lagrangian grid. In our implementation of the SWMF, the 
field line is traced through the SC and IH components
by the core of the framework, and the components only need
to provide the plasma variables for the moving grid points 
when requested.

\begin{thebibliography}{99}

\bibitem[\textit{Antiochos et al.\/}(1999)]{Antiochos1999}
  Antiochos, S.~K., C.~R. DeVore, J.~A. Klimchuk (1999),
  A Model for Solar Coronal Mass Ejections,
  \textit{\apj 510}, 485

\bibitem[\textit{Beutier and Boscher\/}(1995)]{Beutier1995}
  Beutier, T. and D. Boscher (1995), 
  A three-dimensional analysis of the electron
  radiation belt by the Salammbo code, 
  \textit{\jgr 100}, 14,853

\bibitem[\textit{Buis et al.\/}(2004)]{palm2004}
Buis, S., D. Declat, E. Gondet, S. Massart, T. Morel, O. Thual (2003)
PALM: A dynamic parallel coupler for Data Assimilation,
\textit{EGS - AGU - EUG Joint Assembly, Nice, France, 2003}, abstract \#54

\bibitem[\textit{Chen and Garren\/}(1993)]{Chen1993}
Chen, J. and D. Garren,  
Interplanetary magnetic clouds: topology and driving mechanism (1993), 
\textit{\grl 20}, 2319 %-2322

\bibitem[\textit{Chen et al.\/}(2003)]{Chen2003}
Chen, M. W., M. Schulz, G. Lu, and L. R. Lyons (2003), 
Quasi-steady drift paths in
a model magnetosphere with AMIE electric field: Implications for ring
current formation, 
\textit{\jgr 108(A5)}, 1180, doi:10.1029/2002JA009584

\bibitem[\textit{De Zeeuw et al.\/}(2004)]{DeZeeuw2004}
De Zeeuw, D. L., S. Sazykin, R. A. Wolf, T. I. Gombosi,
A. J. Ridley, G. T\'oth (2004),
Coupling of a Global MHD Code and an
Inner Magnetosphere Model: Initial Results,
\textit{\jgr 109}, 12219
doi:10.1029/2003JA010366

\bibitem[\textit{Ebihara and Ejiri\/}(2002)]{Ebihara2002}
Ebihara, Y., and M. Ejiri (2002), 
Numerical simulation of the ring current: Review, 
\textit{Space Sci. Rev., 105}, 377 %-452.

\bibitem[\textit{Ebihara et al.\/}(2004)]{Ebihara2004}
Ebihara, Y., M. Ejiri, H. Nilsson, I. Sandahl, M. Grande, J. F. Fennell, J.
L. Roeder, D. R. Weimer, and T. A. Fritz (2004), 
Multiple discrete-energy ion
features in the inner magnetosphere: 9 February 1998, event, 
\textit{Ann. Geophys., 22}, 1297

\bibitem[\textit{Elkington et al.\/}(1999)]{Elkington1999}
Elkington, S. R., M. K. Hudson, and A. A. Chan, 
Acceleration of relativistic
electrons via drift-resonant interaction with toroidal-mode Pc-5 ULF
oscillations (1999), 
\textit{\grl 26}, 3273

\bibitem[\textit{Fok et al.\/}(2001)]{Fok2001}
Fok, M.-C., R. A. Wolf, R. W. Spiro, and T. E. Moore (2001), 
Comprehensive computational model of the Earth's ring current, 
\textit{\jgr 106}, 8417

\bibitem[\textit{Forbes and Isenberg\/}(1991)]{Forbes1991}
  Forbes, T.~G. and P.~A. Isenberg (1991), 
  A catastrophe mechanism for coronal mass ejections,
\textit{\apj 373}, 294

\bibitem[\textit{Fuller-Rowell and Evans\/}(1987)]{Fuller1987}
  Fuller-Rowell T. J. and D. S. Evans (1987),
  Height-integrated {P}edersen and {H}all conductivity
  patterns inferred from {TIROS--NOAA} satellite data,
  \textit{\jgr 92}, 7606

\bibitem[\textit{Gibson and Low\/}(1998)]{Gibson1998}
  Gibson, S.~E., B.~C. Low (1998), 
  A Time-Dependent Three-Dimensional Magnetohydrodynamic Model of the 
  Coronal Mass Ejection,
  \textit{\apj 493}, 460, doi:10.1086/305107

\bibitem[\textit{Goodman\/}(1995)]{Goodman1995}
Goodman, M.~L. (1995),
  A three-dimensional, iterative mapping procedure for the
  implementation of an ionosphere-magnetosphere anisotropic {Ohm's} law
  boundary condition in global magnetohydrodynamic simulations,
\textit{Ann. Geophys., 13}, 843

\bibitem[\textit{Groth et al.\/}(2000)]{Groth2000}
  Groth, C. P. T., D. L. De Zeeuw, T. I. Gombosi and
  K. G. Powell (2000), 
  Global three-dimensional MHD simulation of a space weather event: 
  CME formation, interplanetary propagation, and interaction with the 
  magnetosphere,
  \textit{\jgr 105}, 25,053, doi:10.1029/2000JA900093

\bibitem[\textit{K\'ota and Jokipii\/}(1999)]{Kota1999}
K\'ota, J. and J. R. Jokipii (1999),
The Transport of CIR accelerated particles,
{\it Proc. 26th ICRC, (Salt Lake City, 6}, 512

\bibitem[\textit{K\'ota et al.\/}(2005)]{Kota2005}
K\'ota, J., W. B. Manchester, D.~L. De Zeeuw, J. R. Jokipii, 
and T.~I. Gombosi (2005),
Acceleration and Transport of Solar Energetic Particles in
a Simulated CME Environment,
\textit{to be submitted to \apj}

\bibitem[\textit{Liemohn et al.\/}(2004)]{Liemohn2004}
  Liemohn, M. W., A. J. Ridley, D. L. Gallagher, D. M. Ober, and J. U. Kozyra
  (2004),
  Dependence of plasmaspheric morphology on the electric field description
  during the recovery phase of the April 17, 2002 magnetic storm, 
  \textit{\jgr  109(A3)}, A03209, doi:10.1029/2003JA010304

\bibitem[\textit{Low\/}(2001)]{Low2001} 
  Low, B.C. (2001),
  Coronal mass ejections, magnetic flux ropes, and solar magnetism, 
  \textit{\jgr 106}, 25141 %-25163

\bibitem[\textit{Manchester\/}(2003)]{Manchester2003} 
  Manchester IV, W. B. (2003)
  Buoyant disruption of magnetic arcades with self-induced shearing,
  \textit{\jgr 108(A4)}, 1162, doi:10.1029/2002JA009252

\bibitem[\textit{Moen and Brekke\/}(1993)]{Moen1993}
  Moen, J., and A. Brekke (1993),
  The solar flux influence of quiet-time conductances in the auroral 
  ionosphere,
  \textit{\grl 20}, 971

\bibitem[\textit{Richmond and Kamide\/}(1988)]{Richmond1988}
  Richmond, A. D. and Y. Kamide (1988),
  Mapping Electrodynamic features of the high-latitude
  ionosphere from localized observations: Technique,
  \textit{\jgr 93}, 5741

\bibitem[\textit{Richmond et al.}\/(1992)]{Richmond1992}
  Richmond, A. D., E. C. Ridley, and R. G. Roble (1992),
  A thermosphere/ionosphere general circulation model with 
  coupled electrodynamics,
  \textit{\jgr 96}, 1071

\bibitem[\textit{Ridley and Liemohn\/}(2002)]{Ridley2002}
Ridley, A. J. and M. W. Liemohn (2002),
A model-derived stormtime asymmetric ring 
current driven electric field description,
\textit{\jgr 107(A8)}, 1290, doi:10.1029/2001JA000051

\bibitem[\textit{Ridley et al.\/}(2004)]{Ridley2004}
Ridley, A. J., T. I. Gombosi, and D. L. De~Zeeuw (2004),
Ionospheric control of the magnetospheric configuration: Conductance
\textit{Ann. Geophys., 22}, 567

\bibitem[\textit{Ridley et al.\/}(2005)]{Ridley2005}
Ridley, A. J., Y. Deng, G. T\'oth (2005),
A general ionospheric-thermospheric model (GITM),
\textit{in preparation}

\bibitem[\textit{Robinson et al.\/}(1987)]{Robinson1987}
  Robinson, R.M., R. R. Vondrak, K. Miller, T. Dabbs, and D.A. Hardy (1987),
  On calculating ionospheric conductances from the flux and energy of
  precipitating electrons,
  \textit{\jgr 92}, 2565

\bibitem[\textit{Roussev et al.\/}(2003a)]{Roussev2003a}
  Roussev, I. I., T. G. Forbes, T. I. Gombosi, I. V. Sokolov, 
  D. L. De Zeeuw and J. Birn (2003), 
  A Three-dimensional Flux Rope Model for Coronal Mass Ejections Based 
  on a Loss of Equilibrium,
  \textit{\apj 588}, L45

\bibitem[\textit{Roussev et al.\/}(2003b)]{Roussev2003b}
  Roussev, I.~I., T. I. Gombosi, I. V. Sokolov, M. Velli, W. Manchester,
  D. L. De Zeeuw, P. Liewer, G. Toth and J. Luhmann (2003), 
  A Three-dimensional Model of the Solar Wind Incorporating Solar 
  Magnetogram Observations,
  \textit{\apj 595}, L57

\bibitem[\textit{Roussev et al.\/}(2004)]{Roussev2004}
  Roussev, I. I., I. V. Sokolov, T. G. Forbes, T. I. Gombosi, M. A, Lee, 
  and J. I. Sakai (2004), 
  A Numerical Model of a Coronal Mass Ejection: 
  Shock Development with Implications for the Acceleration of GeV Protons
  \textit{\apj 605}, L73

\bibitem[\textit{Shprits and Thorne\/}(2004)]{Shprits2004}
Shprits, Y. Y., and R. M. Thorne (2004), 
Time dependent radial diffusion modeling
of relativistic electrons with realistic loss rates, 
\textit{\grl 31}, L08805, doi:10.1029/2004GL019591

\bibitem[\textit{Sokolov et al.\/}(2004)]{Sokolov2004}
Sokolov, I. V.,  I. I. Roussev, T. I. Gombosi, 
M. A. Lee, J. K\'ota, T. G. Forbes, W. B. Manchester and J. I. Sakai (2004),
A New Field Line Advection Model for Solar Particle Acceleration 
\textit{\apj 616:L171}, L174.

\bibitem[\textit{Toffoletto et al.\/}(2003)]{Toffoletto2003}
Toffoletto, F., S. Sazykin, R. Spiro, and R. Wolf (2003), 
Inner magnetospheric modeling with the Rice Convection Model, 
\textit{Space Sci. Rev., 107}, 175 %-196.

\bibitem[\textit{T\'oth et al.\/}(2005c)]{Toth2005c}
T\'oth, G., D. L. De Zeeuw, G. Monostori (2005),
Parallel Field Line and Streamline Tracing Algorithms for
Space Physics Applications,
\textit{in preparation}

\bibitem[\textit{Usmanov et al.\/}(2000)]{Usmanov2000}
  Usmanov, A. V., M. L. Goldstein, B. P. Besser, and J. M. Fritzer (2000), 
\textit{\jgr 105}, 12,675

\bibitem[\textit{Weimer\/}(1996)]{Weimer1996}
Weimer, D. R. (1996),
A flexible, {IMF} dependent model of high-latitude
electric potential having ``space weather'' applications,
\textit{\grl 23}, 2549

\bibitem[\textit{Wolf et al.\/}(1982)]{Wolf1982}
Wolf, R. A., M. Harel, R.W. Spiro, G.-H. Voigt, P. H. Reiff, and C. K. Chen
(1982),
Computer simulation of inner magnetospheric dynamics for the magnetic storm
of July 29, 1977, 
\textit{\jgr 87}, 5949 %�5962

\bibitem[\textit{Wu et al.\/}(2000)]{Wu2000}
Wu, S. T., W. P. Guo, S. P. Plunkett, B. Schmieder and G. M. Simnett (2000),
Coronal mass ejections (CMEs) initiation: models and observations,
\textit{J. Atmos. Solar-Terr. Phys., 62}, 1489 %-1498

\bibitem[\textit{Wu et al.\/}(1999)]{Wu1999}
Wu, S. T., W. P. Guo, D. J. Michels and L. F. Bulgara (1999),
MHD description of the dynamical relationships between a flux rope, 
streamer, coronal mass ejection, and magnetic cloud: 
An analysis of the January 1997 Sun-Earth connection event,
\textit{\jgr 104(A7)}, 14789, doi:10.1029/1999JA900099

\end{thebibliography}

\end{document}


