%                **** IMPORTANT NOTICE *****
% This LaTeX file has been automatically produced by ProTeX v. 2.1
% Any changes made to this file will likely be lost next time
% this file is regenerated from its source.
 
\parskip        0pt
\parindent      0pt
\baselineskip  11pt
 
%--------------------- SHORT-HAND MACROS ----------------------
\def\bv{\begin{verbatim}}
\def\ev{\end{verbatim}}
\def\be{\begin{equation}}
\def\ee{\end{equation}}
\def\bea{\begin{eqnarray}}
\def\eea{\end{eqnarray}}
\def\bi{\begin{itemize}}
\def\ei{\end{itemize}}
\def\bn{\begin{enumerate}}
\def\en{\end{enumerate}}
\def\bd{\begin{description}}
\def\ed{\end{description}}
\def\({\left (}
\def\){\right )}
\def\[{\left [}
\def\]{\right ]}
\def\<{\left  \langle}
\def\>{\right \rangle}
\def\cI{{\cal I}}
\def\diag{\mathop{\rm diag}}
\def\tr{\mathop{\rm tr}}
%-------------------------------------------------------------
 
%/////////////////////////////////////////////////////////////
\newpage

\markboth{Source File: PM\_XY\_coupler.f90}{Source File: PM\_XY\_coupler.f90}

\subsubsection{ XY\_get\_for\_pm - get XY data for sending it to PM }


\bigskip
{\sf INTERFACE:}
\begin{verbatim} subroutine XY_get_for_pm(Buffer_II, iSize, jSize, NameVar)
 \end{verbatim}
{\em USES:}
\begin{verbatim}   use XY_ModField, ONLY: XY_calc_field, XY_Field_II, XY_FieldUnit
 
   implicit none
 \end{verbatim}
{\em INPUT ARGUMENTS:}
\begin{verbatim}   integer, intent(in)           :: iSize,jSize            ! Size of buffer
   character (len=*),intent(in)  :: NameVar                ! Name of data
 \end{verbatim}
{\em OUTPUT ARGUMENTS:}
\begin{verbatim}   real, intent(out)             :: Buffer_II(iSize,jSize) ! Data buffer
 \end{verbatim}
{\sf LOCAL VARIABLES:}
\begin{verbatim}   character (len=*),parameter :: NameSub='XY_get_for_pm'
 \end{verbatim}
{\sf DESCRIPTION:}\\
 
     This subroutine shows an example for getting data for another component.
     This subroutine should be in the srcInterface directory of the
     specific XY component version. The subroutine is called by the
     couple\_xy\_pm method of the CON\_couple\_pm\_xy module in 
     CON/Interface/src.
    
     Calculate field for the positions needed by the PM component.
     We assume here that the name of the variable, its dimension,
     and the grid descriptor of the PM component are 
     sufficient for extracting the data.

%/////////////////////////////////////////////////////////////

\bigskip
{\sf CONTENTS:}
\begin{verbatim}
  ! Allocate and set XY_Field_II for the positions needed by the PM component
  call XY_calc_field(NameVar, iSize, jSize)

  ! Put appropriate part of the XY_Field_II array into the buffer
  ! Convert to SI units
  select case(NameVar)
  case('FieldNorth')
     Buffer_II = XY_FieldUnit * XY_Field_II(1:iSize,:)
  case('FieldSouth')
     Buffer_II = XY_FieldUnit * XY_Field_II(iSize+1:2*iSize)
  case default
     call CON_stop(NameSub//' invalid NameVar='//NameVar)
  end select
 
\end{verbatim}
 
%/////////////////////////////////////////////////////////////
 
\mbox{}\hrulefill\ 
 
\newpage

\markboth{Source File: PM\_XY\_coupler.f90}{Source File: PM\_XY\_coupler.f90}

\subsubsection{ PM\_put\_from\_xy - put into PM the data received from XY }


\bigskip
{\sf INTERFACE:}
\begin{verbatim} subroutine PM_put_from_xy(Buffer_II, iSize, jSize, NameVar)
 \end{verbatim}
{\em USES:}
\begin{verbatim}   use PM_ModProc, ONLY: PM_iProc, PM_nProc
   use PM_ModMain, ONLY: PM_FieldNorth_II, PM_FieldSouth_II, PM_FieldUnit
 
   implicit none\end{verbatim}
{\em INPUT ARGUMENTS:}
\begin{verbatim}   integer,          intent(in) :: iSize, jSize            ! Size of buffer
   real,             intent(in) :: Buffer_II(iSize, jSize) ! Data buffer
   character(len=*), intent(in) :: NameVar                 ! Name of data
 \end{verbatim}
{\sf LOCAL VARIABLES:}
\begin{verbatim}   character (len=*), parameter :: NameSub = 'PM_put_from_xy'
 \end{verbatim}
{\sf DESCRIPTION:}\\
 
     This subroutine shows an example for receiving data from another component.
     This subroutine should be in the srcInterface directory of the 
     specific PM component version. The subroutine is called by the
     couple\_xy\_pm method of the CON\_couple\_pm\_xy module in
     CON/Interface/src.
     
     The data is assumed to be on a spherical grid of size iSize by jSize.
     Based on the NameVar argument the data corresponds either to the 
     northern or to the southern hemisphere. The northern hemisphere is
     always solved with processor 0, while the southern hemisphere with 
     processor PM\_nProc-1 where PM\_nProc is the number of processors (1 or 2)
     used by component PM.

%/////////////////////////////////////////////////////////////

\bigskip
{\sf CONTENTS:}
\begin{verbatim}
  select case(NameVar)
  case('FieldNorth')
     if (PM_iProc /= 0) RETURN
     PM_FieldNorth_II = Buffer_II / PM_FieldUnit
  case('FieldSouth')
     if (PM_iProc /= PM_nProc-1) RETURN
     PM_FieldSouth_II = Buffer_II / PM_FieldUnit
  case default
     call CON_stop(NameSub//' invalid NameVar='//NameVar)
  end select
 
\end{verbatim}
 
%/////////////////////////////////////////////////////////////
 
\mbox{}\hrulefill\ 
 
\newpage

\markboth{Source File: PM\_XY\_coupler.f90}{Source File: PM\_XY\_coupler.f90}

\subsection{Module  CON\_couple\_pm\_xy - couple PM and XY components }


   
   This is an example for a module for the data exchange between two
   components. This module should be in the CON/Interface/src directory.
   The order of the component ID-s in the name of the module is alphabetical.
  
   This example provides methods for coupling the XY component to the PM 
   component (one way only). A single field is passed on the spherical 
   grid of the PM component. The communication assumes that the XY component 
   collects all data on its root processor, while the grid of the PM component
   consists of two blocks (two hemispheres) and it runs on 1 or 2 processors.
  
\bigskip
{\sf INTERFACE:}
\begin{verbatim} module CON_couple_pm_xy
 \end{verbatim}
{\em USES:}
\begin{verbatim}   use CON_coupler ! provides all the CON methods needed for the coupling
 
   implicit none
 
   private ! except
 \end{verbatim}
{\sf PUBLIC MEMBER FUNCTIONS:}
\begin{verbatim}   public :: couple_pm_xy_init   ! initialize coupling
   public :: couple_xy_pm        ! couple XY to PM
 \end{verbatim}
{\sf LOCAL VARIABLES:}
\begin{verbatim}   logical, save :: UseMe        ! logical for participating processors
   integer, save :: iSize, jSize ! size of the 2D spherical structured PM grid
 \end{verbatim}
 
%/////////////////////////////////////////////////////////////
 
\mbox{}\hrulefill\ 
 
\subsubsection{ couple\_pm\_xy\_init - initialize XY-PM couplings}


\bigskip
{\sf INTERFACE:}
\begin{verbatim}   subroutine couple_pm_xy_init
 \end{verbatim}
{\sf LOCAL VARIABLES:}
\begin{verbatim}     logical :: IsInitialized = .false.  ! logical to store initialization
     integer :: nCells_D(2)              ! temporary array for grid size
 \end{verbatim}
{\sf DESCRIPTION:}\\
 
       This subroutine initializes the data exchange between the XY and PM
       components. The subroutine must be called by the couple\_all\_init
       method of the CON\_couple\_all module.
       This particular initialization stores the PM grid size iSiza and jSize
       and sets the logical UseMe to .true. for participating PE-s.

%/////////////////////////////////////////////////////////////

\bigskip
{\sf CONTENTS:}
\begin{verbatim}
    if(IsInitialized) RETURN
    IsInitialized = .true.

    ! Get the block size iSize and jSize from the PM grid descriptor 
    ! using the ncell_id method of CON_coupler
    nCells_D = ncell_id(PM_)
    iSize = nCells_D(1); jSize = nCells_D(2)

    ! Set UseMe to .true. for the participating PE-s
    ! using the is_proc() function of CON_coupler
    UseMe = is_proc(PM_) .or. is_proc(XY_)
 
\end{verbatim}
 
%/////////////////////////////////////////////////////////////
 
\mbox{}\hrulefill\ 
 
\subsubsection{ couple\_xy\_pm - couple XY to PM}


\bigskip
{\sf INTERFACE:}
\begin{verbatim}   subroutine couple_xy_pm(tSimulation)
 \end{verbatim}
{\em INPUT ARGUMENTS:}
\begin{verbatim}     real, intent(in) :: tSimulation  ! simulation time
 \end{verbatim}
{\sf LOCAL VARIABLES:}
\begin{verbatim}     character (len=*), parameter :: NameSub='couple_xy_pm'
 
     character (len=*), parameter, dimension(2) :: &   ! names of variables
          NameVar_B = (/ 'FieldNorth', 'FieldSouth' /) ! for both blocks
 
     real, dimension(:,:), allocatable :: Buffer_II    ! buffer for 2D field
 
     integer :: nSize                      ! MPI message size
     integer :: iStatus_I(MPI_STATUS_SIZE) ! MPI status variable
     integer :: iError                     ! MPI error code
 
     integer :: iBlock                     ! block index
     integer :: iProcTo                    ! rank of the receiving processor
 \end{verbatim}
{\sf DESCRIPTION:}\\
 
       This subroutine is called by the couple\_two\_comp method
       of the CON\_couple\_all module.
       This particular coupler send data Field from XY to PM.

%/////////////////////////////////////////////////////////////

\bigskip
{\sf CONTENTS:}
\begin{verbatim}
    ! Exclude PEs which are not involved
    if(.not.UseMe) RETURN

    ! Do Northern and then Southern hemispheres
    do iBlock = 1, 2

       ! Get the rank of the PM processor for this block
       ! using the pe_decomposition() method of CON_coupler
       iProcTo = pe_decomposition(PM_, iBlock)

       ! Allocate buffers for the variables both in XY and PM
       allocate(Buffer_II(iSize, jSize), stat=iError)
       call check_allocate(iError, NameSub//": "//NameVar_B(iBlock))

       ! Calculate Field on XY.
       ! The result will be on the root processor of XY.
       if( is_proc(XY_) ) &
            call XY_get_for_pm(Buffer_II, iSize, jSize, NameVar_B(iBlock))

       ! Transfer variables from the root processor of XY to PM.
       ! Identify the root processor of PM with the i_proc0() method
       ! and the receiving processor with the i_proc() method of CON_coupler.
       if( iProcTo /= i_proc0(XY_))then
          nSize = iSize*jSize
          if(is_proc0(XY_)) &
               call MPI_send(Buffer_II, nSize, MPI_REAL, iProcTo,&
               1, i_comm(), iError)
          if(i_proc() == iProcTo) &
               call MPI_recv(Buffer_II, nSize, MPI_REAL, i_proc0(XY_),&
               1,i_comm(), iStatus_I, iError)
       end if

       ! Put variables into PM
       if( i_proc() == iProcTo )&
            call PM_put_from_xy(Buffer_II, iSize, jSize, NameVar_B(iBlock))

       ! Deallocate buffer to save memory
       deallocate(Buffer_II)
    end do

 
\end{verbatim}

%...............................................................
