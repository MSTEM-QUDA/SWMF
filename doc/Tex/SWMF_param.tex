%^CFG COPYRIGHT UM
%\documentclass{article}
%\begin{document}

\section{PARAM.in \label{section:param.in}}

The input parameters for the SWMF are read from the 
{\tt PARAM.in} file which must be located in the run directory.
This file, together with the LAYOUT.in file, controls the SWMF
and its components.
The component specific parameters are given between
the \#BEGIN\_COMP ID and \#END\_COMP ID commands,
where the ID is the two character identifier of the component.
For example the GM parameters are enclosed between the 
\begin{verbatim}
#BEGIN_COMP GM
...
#END_COMP GM
\end{verbatim}
commands. We refer to the lines starting with a \# character as commands.

For example if the command string 
\begin{verbatim}
#END
\end{verbatim}
is present, it indicates the end of the run and lines following
this command are ignored. If the \#END command is not
present, the end of the file indicates the end of the run.

There are several features of the input parameter file syntax
that allow the user to easily run the code
in a variety of modes while at the same time being able to 
keep a library of useful parameter files that can be used
again and again.

The syntax and the content of the input parameter files
is defined in the PARAM.XML files. The commands controlling
the whole SWMF are described in the 
\begin{verbatim}
  Param/PARAM.XML
\end{verbatim}
file. The component parameters are described by the PARAM.XML
file in the component version directory. For example the
input parameters for the GM/BATSRUS component are described in
\begin{verbatim}
  GM/BATSRUS/PARAM.XML
\end{verbatim}
These files can be read (and edited) in a normal editor.
The same files are used to produce much of this
manual with the aid of the {\tt share/Scripts/XmlToTex.pl} script. 
The {\tt Scripts/TestParam.pl} script also uses these files
to check the PARAM.in file.
Copying small segments of the {\tt PARAM.XML} files
into {\tt PARAM.in} can speed up the creation or modification of a 
parameter file. 

\subsection{Included Files, {\tt \#INCLUDE} \label{section:include}}

The {\tt PARAM.in} file can include other parameter files with the 
command
\begin{verbatim}
#INCLUDE
include_parameter_filename
\end{verbatim}
The include files serve two purposes: (i) they help
to group the parameters; (ii) the included files can be reused
for other parameter files. 
An include file can include another file itself.
Up to 10 include files can be nested.
The include files have exactly the same structure as {\tt PARAM.in}. 
The only difference is that the
\begin{verbatim}
#END
\end{verbatim}
command in an included file means only the end of the include file, 
and not the end of the run, as it does in {\tt PARAM.in}.

The user can place his/her
included parameter files into the main run directory or in any subdirectory
as long as the correct path to the file from the run directory is
included in the {\tt \#INCLUDE} command.
There are many include files in the {\tt Param} directory. These
can be included into the {\tt PARAM.in} files, or they can serve as
examples. 

\subsection{Commands, Parameters, and Comments \label{section:commands}}

As can be seen from the above examples, the parameters are entered
with a combination of a {\bf command} followed by specific {\bf parameter(s)},
if any.
The {\bf command} must start with a hashmark (\#), which 
is followed by capital letters and underscores without space in between. 
Any characters behind the first space or TAB character are ignored
(the \#BEGIN\_COMP and \#END\_COMP commands are the only exception,
but these are markers rather than commands).
The parameters, which follow, must conform to 
requirements of the command. They can be of four types: logical, integer,
real, or character string. Logical parameters can be entered as 
{\tt .true.} or {\tt .false.} or simply {\tt T} or {\tt F}.
Integers and reals can be in any of the usual Fortran formats.
All these can be followed by arbitrary comments, typically separated
by space or TAB characters. In case of the character type input
parameters (which may contain spaces themselves), the comments must
be separated by a TAB or by at least 3 consecutive space characters.
Comments can be freely put anywhere between two commands as long
as they don't start with a hashmark.

Here are some examples of valid commands, parameters, and comments:
\begin{verbatim}

#TIMEACCURATE
F                       DoTimeAccurate

Here is a comment between two commands...

#DESCRIPTION
My first run            StringDescription (3 spaces or TAB before the comment)

#STOP
-1.                     tSimulationMax
100		        MaxIteration

#RUN ------------ last command of this session -----------------

#TIMEACCURATE
T                       DoTimeAccurate

#STOP
10.0                    tSimulationMax
-1                      MaxIteration

\end{verbatim}

\subsection{Sessions \label{section:sessions}}

A single parameter file can control consecutive {\bf sessions}
of the run. Each session looks like
\begin{verbatim}
#SOME_COMMAND
param1
param2

...

#STOP
max_simulation_time_for_this_session
max_iter_for_this_session

#RUN
\end{verbatim}
while the final session ends like
\begin{verbatim}
#STOP
max_simulation_time_for_final_session
max_iter_for_final_session

#END
\end{verbatim}
The purpose of using multiple sessions is to be able to change parameters 
during the run. For example one can use only a subset of the
components in the first session, and add more components in the
later session. Or one can obtain a coarse steady state solution
on a coarse grid with a component in one session, and improve on the solution
with a finer grid in the next session. Or one can switch from 
steady state mode to time accurate mode. The SWMF remembers parameter
settings from all previous sessions, so in each session one should only
set those parameters which change relative to the previous session.
Note that the maximum number of iterations given in the {\tt \#STOP} command 
is meant for the entire run, and not for the individual sessions. 
On the other hand, when a restart file is read, the iterations prior to 
the current run do not count.

The {\tt PARAM.in} file and all included parameter files are read into 
a buffer at the beginning of the run, so even for multi-session runs, 
changes in the parameter files have no effect once {\tt PARAM.in} 
has been read. 

\subsection{The Order of Commands \label{section:order}}

In essence, the order of parameter commands within a
session is arbitrary, but there are some important restrictions.  
We should note that the order of the parameters following 
the command is not arbitrary and must exactly match what the code requires.  
Here we restrict ourselves to the restrictions on the commands read by
the control module of SWMF. There may be (and are) restrictions
for the commands read by the components, but those are described
in the documentation of the components.

The only strict restriction on the SWMF commands is related
to the 'planet' commands. The default values of the 
planet parameters are defined by the \#PLANET command.
For example the parameters of Earth can be selected with the
\begin{verbatim}
#PLANET
Earth            NamePlanet
\end{verbatim}
command. The true parameters of Earth can be modified or simplified
with a number of other commands which {\bf must occur after the
\#PLANET command}. These commands are (without showing their parameters)
\begin{verbatim}
\#IDEALAXES
\#ROTATIONAXIS
\#MAGNETICAXIS
\#ROTATION
\#DIPOLE
\end{verbatim}
Other than this strict rule, it makes sense to follow a 'natural'
order of commands. This will help in interpreting, maintaining
and reusing parameter files.

If you want all the input parameters to be echoed back, the first
command in {\tt PARAM.in} should be
\begin{verbatim}
#ECHO
T                 DoEcho
\end{verbatim}
If the code starts from restart files, it usually reads in a
file which was saved by SWMF. The default name of the saved
file is RESTART.out and it is written into the run directory.
It should be renamed, for example to RESTART.in, so that it
does not get overwritten during the run. It can be included as
\begin{verbatim}
#INCLUDE
RESTART.in
\end{verbatim}
The SWMF will read the following commands (the parameter values are
examples only) from the included file:
\begin{verbatim}
#DESCRIPTION
Create startup for GM-IM-IE-UA from GM stady state.

#PLANET
EARTH                        NamePlanet

#STARTTIME
    1998                     iYear
       5                     iMonth
       1                     iDay
       0                     iHour
       0                     iMinute
       0                     iSecond
 0.000000000000              FracSecond
 
#NSTEP
    4000                     nStep
 
#TIMESIMULATION
 0.00000000E+00              tSimulation
 
#VERSION
 2.00                        VersionSwmf
 
#PRECISION
8                              nByteReal
\end{verbatim}
The \#PLANET command defines the selected planet.
The \#STARTTIME command defines the starting date and time of the whole
simulation. The current simulation time (which is relative to
the starting date and time) and the step number are
given by the \#TIMESIMULATION and \#NSTEP commands. Finally
the \#VERSION and \#PRECISION commands check the consistency
of the current version and real precision with the run which
is being continued. As it was explained above, 
all modifications of the planet parameters should follow 
the \#PLANET command, i.e. they should be after 
the \#INCLUDE RESTART.in command. In case the description is
changed it should also follow, e.g.
\begin{verbatim}
#INCLUDE
RESTART.in

#DESCRIPTION
We continue the run for another 2 hours

#IDEALAXES
\end{verbatim}
When the run starts from scratch, the PARAM.in file
should start similarly with the 
\begin{verbatim}
\#DESCRIPTION
This is the start up run

\#PLANET
SATURN

\#STARTTIME
    2004                     iYear
       8                     iMonth
      15                     iDay
       1                     iHour
      25                     iMinute
       0                     iSecond
 0.000000000000              FracSecond
\end{verbatim}
commands (the parameters are examples only).
These commands are typically followed by the planet parameter
modifying commands, if any, and setting time accurate mode
(if changed from default true to false or relative to the previous session).
For example:
\begin{verbatim}
! Align the rotation and magnetic axes with Z_GSE
#IDEALAXES

#TIMEACCURATE
F                           DoTimeAccurate
\end{verbatim}
All the commands which are written into the RESTART.out file and all 
the planet modifying commands can only occur in the first session.
These commands contain parameters which should not change during a run.
In the Param/PARAM.XML file these commands are marked with an 
{\tt if="\$\_IsFirstSession"} conditional.
If any of these parameters are attempted to be changed in later sessions, 
a warning is printed on the screen and the code stops running
(except when the code is in non-strict mode).

Most command parameters have sensible default values.
These are described in the PARAM.XML files,
and in this manual (which was produced from them).
The {\tt PARAM.XML} file also defines which commands are required
with the {\tt required="T"} attribute of the {\tt <command...>} tag.
For the control module the only required command is the \#STOP command,
which defines the final time step or the final time of the run.

\subsection{Iterations, Time Steps and Time \label{section:frequency}}

In several commands the frequency or 'time' of some action has
to be defined. This is usually done with a pair of parameters.
The first defines the frequency or time in terms of the number of time steps,
and the second one in terms of the simulation time.
A negative value for the frequency means that it should not be taken 
into account. For example, in time accurate mode,
\begin{verbatim}
#SAVERESTART
T            DoSaveRestart
2000         DnSaveRestart
-1.          DtSaveRestart
\end{verbatim}
means that a restart file should be saved after every 2000th time step, while
\begin{verbatim}
#SAVERESTART
T            DoSaveRestart
-1           DnSaveRestart
100.0        DtSaveRestart
\end{verbatim}
means that it should be saved every 100 second in terms of physical time.
Defining positive values for both frequencies might be useful
when switching from steady state mode to time accurate mode.
In the steady state mode the DnSaveRestart parameter is used,
while in time accurate mode the DtSaveRestart if it is positive.
But it is more typical and more intuitive 
to explicitly repeat the command in the first 
time accurate session with the time frequency set.

The purpose of this subsection is to try to help the user understand 
the difference between the iteration number used for stopping the code
and the time step which is used to define the frequencies of various
actions. After using \BATSRUS\ over several years, it is clear to the
authors that this distinction is useful and the
most reasonable implementation. The SWMF has inherited these
features from the \BATSRUS\ code.

We begin by defining several different quantities and the variables that 
represent them in the code.  The variable {\tt nIteration}, 
represents the number of ``iterations'' 
that the simulation has taken since it began running.  
This number starts at zero every time the code is run, even if beginning 
from a restart file.
This is reasonable since most users know how many iterations the code can take
in a certain amount of CPU time and it is this number that is needed when 
running in a queue.
The quantity {\tt nStep} is a number of ``time steps'' that the code has 
taken in total.  This number starts at zero when the code is started from 
scratch, but when started from a restart file, this
number will start with the time step at which the restart file was written.
This implementation lets the user output data files at a regular interval, even
when a restart happens at an odd number of iterations.
The quantity {\tt tSimulation} is the amount of simulated, or physical, 
time that the code has run.  
This time starts when time accurate time stepping begins.
When restarting, it starts from the physical time for the restart.
Of course the time should be cumulative since it is the physically meaningful
quantity.  We will 
use these three phrases( ``iteration'', ``time step'', ``time'') 
with the meanings outlined above.

Now, what happens when the user has more than one session and he or she
changes the frequencies.  Let us examine what would happen in the following
sample of part of a {\tt PARAM.in} file.  For the following example we will
assume that when in time accurate mode, 1 iteration simulates 1 second of time.
Columns to the right indicate the values of {\tt nITER}, {\tt n\_step} and
{\tt time\_simulation} at which restart files will be written in each session.

\clearpage

\begin{alltt}
                                             Restart Files Written at:
==SESSION 1       \hfill        Session   nITER   nStep   time_simulation
#TIMEACCURATE	  \hfill        --------  ------  -------  --------------
F            DoTimeAccurate  

#SAVERESTART                      \hfill  1       200      200              0.0  
T            DoSaveRestart     \hfill  1       400      400              0.0
200          DnSaveRestart
-1.0         DtSaveRestart

#STOP
400          MaxIteration
-1.          tSimulationMax

#RUN ==END OF SESSION 1== 
                         
#SAVERESTART			  \hfill  2       600      600              0.0
T            DoSaveRestart	  \hfill  2       900      900              0.0
300          DnSaveRestart
-1.0         DtSaveRestart
				
#STOP				
1000         MaxIteration				
-1.          tSimulationMax
				
#RUN ==END OF SESSION 2== 

#TIMEACCURATE
T            DoTimeAccurate  		
				
#SAVERESTART			  \hfill  3      1100     1100            100.0
T            DoSaveRestart	  \hfill  3      1200     1200            200.0
-1           DnSaveRestart  \hfill  3      1300     1300            300.0
100.0        DtSaveRestart
				
#STOP				
-1           MaxIteration				
300.0        tSimulationMax			
				
#RUN ==END OF SESSION 3== 
                          
#SAVERESTART                \hfill   4      1400     1400            400.0
T            DoSaveRestart  \hfill   4      1800     1800            800.0
-1           DnSaveRestart  \hfill   4      2000     2000           1000.0
400.0        DtSaveRestart
 				
#STOP				
-1           MaxIteration				
1000.0       tSimulationMax				
				
#END  ==END OF SESSION 4== 
\end{alltt}
Now the question is how many iterations will be taken and when will restart
file be written out.  In session 1 the code will make 400 iterations and will
write a restart file at time steps 200 and 400.  Since the iterations 
in the {\tt \#STOP}
command are cumulative, the {\tt \#STOP} command in the second session will
have the code make 600 more iterations for a total of 1000.  Since the timing
of output is also cumulative, a restart file will be written at time step 600
and at 900.
After session 2, the code is switched to time accurate mode.  Since we
have not run in this mode yet the simulated (or physical) time is cumulatively
0.  The third session will run for 300.0 simulated seconds (which for the
sake of this example is 300 iterations).  The restart file will be written
after every 100.0 simulated seconds.
The {\tt \#STOP} command in Session 4 tells the code to simulate  700.0 more 
seconds for a total of 1000.0 seconds.  The code will make a restart file
when the time is a multiple of 400.0 seconds or at 400.0 and 800.0 seconds.
Note that a restart file will also be written at time 1000.0
seconds since this is the end of a run.

In the next example ee want to restart from 1000.0 seconds 
and continue with a time accurate run.
\begin{alltt}
                                             Restart Files Written at:
==SESSION 1      \hfill        Session   nITER   nStep    time_simulation
                 \hfill        --------  ------  -------  --------------
#INCLUDE                         \hfill            0     2000           1000.0
RESTART.in

#TIMEACCURATE
T            DoTimeAccurate  

#SAVERESTART                      \hfill  1       200     2200           1200.0
T            DoSaveRestart
-1           DnSaveRestart
600.0        DtSaveRestart

#STOP
-1           MaxIteration
1500.0       tSimulationMax

#RUN ==END OF SESSION 1== 

#SAVERESTART                      \hfill  2       700     2700           1500.0
T            DoSaveRestart	  \hfill  2      1000     3000           2000.0
-1           DnSaveRestart
750.0        DtSaveRestart

#STOP
-1           MaxIteration
2000.0       tSimulationMax

#END ==END OF SESSION 2 = 
                          
\end{alltt}
In this example, we see that in time accurate mode the simulated, or
physical, time is always cumulative.  To make 500.0 seconds more simulation,
the original 1000.0 seconds must be taken into account.  In this example,
since each second is 1 iteration, the restart file would be written at the
same time steps as in the previous example.  The final output 
at 2000.0 seconds is written because the run ended.

Throughout this subsection, we have used the frequency of writing restart files
as an example.  The frequencies of coupling components and checking stop
files work similarly. In the SWMF, and potentially in any of the
components, the frequencies are handled by the general
\begin{verbatim}
  share/Library/src/ModFreq
\end{verbatim}
module which is described in the reference manual.

%\end{document}
