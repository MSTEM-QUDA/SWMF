%  Copyright (C) 2002 Regents of the University of Michigan, portions used with permission 
%  For more information, see http://csem.engin.umich.edu/tools/swmf
\documentclass[twoside,10pt]{book}

\usepackage{sverb}

\title{Space Weather Modeling Framework User Manual \\ 
       \hfill \\
       \SWMFLOGO}

\author{Center for Space Environment Modeling\\
  {\it The University of Michigan}\\
  \hfill \\
  \CSEMLOGO}

\makeindex

\input HEADER


% Introduction
%\documentclass[a4paper,11pt]{article}
%\author{\bf Center for Space Environment Modeling, The University of Michigan}
%\title{\bf \Large Release Notes for the Milestone 7I and Reference Manual}
%\maketitle



\chapter{Introduction}

This document describes a working prototype of the NASA-funded Space
Weather Modeling Framework (SWMF) delivered to NASA to fulfill the
Milestone 11K requirements. The SWMF was developed to provide flexible
``plug and play" type simulation capabilities serving the Sun-Earth
modeling community.  In its current form the SWMF links together eight
models from the surface of the Sun to the upper atmosphere of the Earth: 
\begin{enumerate}
\item SC -- Solar Corona which includes the Eruptive Event Generator,
\item IH -- Inner Heliosphere
\item SP -- Solar Energetic Particles 
\item GM -- Global Magnetosphere 
\item IM -- Inner Magnetosphere
\item RB -- Radiation Belts
\item IE -- Ionosphere Electrodynamics
\item UA -- Upper Atmosphere
\end{enumerate}
In the future the SWMF may be extended to include even more 
physics domains: Cometary Environment, Interstellar
Neutrals, Outer Heliosphere, Plasmasphere, Planetary Satellites and
Polar Wind. 

The SWMF implementation is based on the component technology and
Object-Oriented Programming emulated in Fortran 90.  The SWMF parallel
communications are based on the MPI standard.  In its current
implementation the SWMF creates a single executable.

\section{Acknowledgments}

The SWMF was developed at the Center for Space Environment Modeling
(CSEM) of the University of Michigan under the NASA Earth Science
Technology Office (ESTO) Computational Technologies (CT) Project (NASA
CAN NCC5-614). The project is entitled as ``A High-Performance
Adaptive Simulation Framework for Space-Weather Modeling (SWMF)''.
The Project Director is Professor Tamas Gombosi, and the Co-Principal
Investigators are Professors Quentin Stout and Kenneth Powell.

The SWMF and many of the physics components were developed at CSEM
by the following individuals (in alphabetical order):
David Chesney,
Darren DeZeeuw, Tamas Gombosi, Kenneth Hansen, Kevin Kane, Ward (Chip)
Manchester, Robert Oehmke, Kenneth Powell, Aaron Ridley, Ilia Roussev,
Quentin Stout, Igor Sokolov, G\'abor T\'oth and Ovsei Volberg.

The core design and code development was done by G\'abor
T\'oth, Igor Sokolov and Ovsei Volberg:
\begin{itemize}
\item Component registration and layout was designed and implemented by 
      Ovsei Volberg and G\'abor T\'oth.
\item The session and time management support was designed and
      developed by G\'abor T\'oth.
\item The SWMF coupling toolkit was developed by Igor Sokolov.
\end{itemize}
The physics models were developed by the following research groups:
\begin{itemize}
\item
The Solar Corona (SC), Inner Heliosphere (IH) and the Global Magnetosphere 
(GM) components are based on \BATSRUS\ MHD code developed at CSEM. 
\BATSRUS\ is a 3-dimensional block-adaptive Cartesian code which is 
highly parallel.

\item
The Solar Energetic Particle (SP) component is the
K\'ota's SEP model which was developed at the University of Arizona.
It solves the equations for the advection and acceleration of
energetic particles along a magnetic field line in a 3D phase space.

\item
The Inner Magnetosphere (IM) component is the Rice Convection Model
(RCM) developed at Rice University.  This code is 2-dimensional and
serial.

\item
The Radiation Belt (RB) component is the Rice RBM
developed at Rice University.  This code is 2-dimensional and
serial.

\item
The Ionospheric Electrodynamics (IE) component is a 2-processor,
2-dimensional spherical electric potential solver developed at CSEM
(termed the ``Ridley Ionosphere'').  

\item
There are two versions of the Upper Atmosphere (UA) component:
the Global Ionosphere - Thermosphere Model (GITM) and its newer
version GITM2. Both versions are 3-dimensional spherical
models developed at CSEM.  They are fully parallel.

\end{itemize}
The transformation of physics models into physics components,
the coupling of components to the SWMF and each other and
all the testing were done at CSEM.

\section{What is New in Version 2.1}

The SWMF has been developed further since the second release of
version 2.0. Here is a partial list of improvements:
\begin{itemize}
\item The documentation has been improved and split into smaller parts.
\item The SWMF contains a new version for the UA component: GITM2.
\item Restarting SWMF is made easier with the Scripts/Restart.pl script.
\item The Solar Corona can be solved in a corotating frame.
\item The coupling between the GM and IE components has been rewritten
      with simpler and more accurate algorithms.
\item The coupling between the GM and IM components has been made more robust.
\item The geometry based grid refinement can be modified easily with
      the \#GRIDRESOLUTION command in the
      parameter file for the GM,IH,SC/BATSRUS components.

\end{itemize}

\section{What is New in Version 2.0}

The SWMF has been developed extensively since the first release of
version 1.0. Here are some of the highlights:
\begin{itemize}
\item The SWMF now contains 3 new components:
      the Solar Corona (SC), the Solar Energetic Particles (SP) 
      and the Radiation Belt (RB)
\item Some components now support dynamic memory allocation which 
      reduces the total memory required by a processing element. 
\item The directory structure of the SWMF has been greatly improved and 
      reorganized. 
\item This reorganization allows components to be used as stand alone 
      physics models without any modification in the source code.  The
      stand alone version only links to a small SWMF library.
\item The installation and configuration of SWMF has been greatly simplified
      with the aid of Perl scripts. 
\item Layout and input parameter can be checked with Scripts/TestParam.pl
\item User manual is produced from the XML description of the input parameters.
\item Unused components can be configured out completely.
\item The control module now fully supports steady state calculations 
      including component subcycling. 
\item The coupling toolkit provides means for extracting and following 
      the motion of field lines.
\end{itemize}

\section{The SWMF in a Few Paragraphs}

The SWMF is a structured collection of software building blocks that
can be used or customized to develop Sun-Earth system modeling
components, and to assemble them into applications. The SWMF consists
of utilities and data structures for coupling model components. The
SWMF contains a Control Module (CON), which is responsible for
component registration, processor layout for each component and
coupling schedules.  It controls initialization and execution of the
components. A component is adapted from user-supplied physics codes,
(for example \BATSRUS\ or RCM), by adding two relatively small units
of code:
\begin{itemize}
\item A wrapper, which provides the control functions, and
\item A coupling interface to perform the data exchange with other
components.
\end{itemize}
Both the wrapper and coupling interface are constructed from the
building blocks provided by the framework. From 
component software technology perspective both the wrapper and
coupling interface are component interfaces: the wrapper is an
interface with CON, and the coupling interface is an interface with
another component. A physics
model code and its wrapper, which comprise a component, share the
communication group.  The coupling interface uses the union
communicator of the two components that it links together.

An SWMF component is compiled into a separate library that resides in
the directory {\tt lib}, which is created as part of the installation
process described later in this document.  Currently the component
libraries are static libraries. The executable image is created in the
directory {\tt bin}, which is created during the compilation.  If a
user does not want to build some particular component, this component
should be substituted by an empty version of the component.

An important feature of the SWMF is the component registration.  A
component to be included in the run should be registered by the
framework.  Currently entering the line for the component in the input
file called {\tt LAYOUT.in} does the registration.  Thus the SWMF
performs the run-time registration of components.

The framework controls the initialization, execution, coupling and
finalization of components.  The execution is done in sessions. In
each session the parameters of the framework and the components can be
changed.  The parameters are read from the {\tt PARAM.in} file, which
may contain further included parameter files.  These parameters are
read and broadcast by CON and the component specific parameters are
sent to the components. The structure of the parameter file will be
described in detail.

If two components reside on different sets of processing elements
(PE-s) they can execute in an efficient concurrent manner.
This is possible, because the coupling times are
known in advance.  The components advance to the time of coupling and
only the processors involved in the coupling need to communicate with
each other. The components are also allowed to share some processing elements.
The execution is sequential for the components with overlapping layouts.
Of course this still allows the individual components to execute in parallel.
For steady state calculations the components are allowed to progress
at different rates towards steady state. Each component can be called
at different frequencies by the control module.

The coupling of the components is realized either with plain MPI
calls, or via the SWMF coupling toolkit, which can couple components
based on the following types of parallel distributed grids:
\begin{itemize}
\item 3-D Block adaptive (AMR) parallel grid
\item 2-D Spherical grid
\item Logically Cartesian uniform grid
\item Logically Cartesian non-uniform grid 
\end{itemize}
The SWMF coupling toolkit performs an efficient N to M parallel
coupling based on a router. The router is calculated in advance using
the domain decomposition and grid description obtained from the
components.  The router is updated only when the domain decompositions
or the grids of the components change, or when the mapping geometry
changes.  The coupling toolkit takes care of linear interpolation in
space based on the grid descriptor.  Temporal interpolation is not
supported by the current implementation.

The framework has been tested on the SGI Origin 3000, SGI Altix and 
Compaq ES45 machines, and on Linux Beowulf clusters with the NAG f95 
compiler. We have also run the framework with reasonable success under
Mac OS Darwin using the XLF and NAG f95 compilers, and under Linux with
the PGF90 compiler.

\section{System Requirements}

In order to install and run the SMWF the following minimum system
requirements apply.

\begin{itemize}
\item The SWMF runs only under the UNIX/Linux operating systems.  This now
  includes Macintosh system 10.x because it is based on BSD UNIX.  The
  SWMF does not run under any Microsoft Windows operating system.
\item A FORTRAN 77 and FORTRAN 90 compiler must be installed.
\item The Perl interpreter must be installed.
\item A version of the Message Passing Interface (MPI) library must be
  installed.
\item You may be able to compile the code and do very small test
runs on 1 or 2 processor machines.  However, to do most physically
meaningful runs the SWMF requires a
parallel processor machine with a minimum of 8 processors and a minimum of 8GB of
memory.
\item Very large runs require many more processors.
\item In order to generate the documentation you must have LaTex installed on
your system.  The PDF generation requires the {\tt dvips} and {\tt ps2pdf}
utilities.  To generate the HTML version you also must install the
{\tt latex2html} package. 

\end{itemize}


In addition to the above requirements, the SWMF output is designed to
be visualized using either IDL or Tecplot.  You may be able to
visualize the output with other packages, but formats and scripts have
been designed for only these two visualization softwares.




%-----------------------------------------------------------------------
% Chapter 2
%-----------------------------------------------------------------------

\chapter{Quick Start}

\section{A Brief Description of the SWMF Distribution}

The distribution in the form of the compressed tar image
includes the SWMF source code.
The top level directory contains the following subdirectories:
\begin{itemize}
\item {\tt CON}     - the directory of the framework's main building blocks
\item {\tt GM}      - Global Magnetosphere component       %^CMP IF GM
\item {\tt IE}      - Ionosphere Electrodynamics Component %^CMP IF IE
\item {\tt IH}      - Inner Heliosphere component          %^CMP IF IH
\item {\tt IM}      - Inner Magnetosphere component        %^CMP IF IM
\item {\tt RB}      - Radiation Belt component             %^CMP IF RB
\item {\tt SC}      - Solar Corona component               %^CMP IF SC
\item {\tt SP}      - Solar Energetic Particles component  %^CMP IF SP
\item {\tt UA}      - Upper Atmosphere component           %^CMP IF UA
\item {\tt Copyrights} - copyright files
\item {\tt Param}   - description of CON parameters, parameter and layout files
\item {\tt Scripts} - shell and Perl scripts
\item {\tt bin}     - scripts for installation, configuration and testing
\item {\tt doc}     - the documentation directory %^CMP IF DOC
\item {\tt share}   - shared scripts and source code
\item {\tt util}    - general utilities such as TIMING and NOMPI
\end{itemize}
and the following files
\begin{itemize}
\item {\tt README}           - a short instruction on installation and usage
\item {\tt Makefile}         - the main makefile
\item {\tt Configure.pl}     - Perl script for configuration %^CMP IF CONFIGURE
\item {\tt Configure.options} - default configuration options %^CMP IF CONFIGURE
\item {\tt SetSWMF.pl}     - Perl script for (un)installation and configuration
\end{itemize}

\section{General Hints}

\subsubsection{Getting help with scripts and the Makefile}
Most of the Perl and shell scripts that are distributed with the SWMF
provide help which can be accessed as follows using the {\tt -h} flag.
For example, 
\begin{verbatim}
  SetSWMF.pl -h
\end{verbatim}
will provide a detailed listing of the options and capabilities of the
{\tt SetSWMF.pl} script.  In addition, you can find all the possible
targets  that can be built by typing
\begin{verbatim}
make help
\end{verbatim}

\subsubsection{Input commands: PARAM.XML}
A very useful set of files to become familiar with are the {\tt PARAM.XML}
files.  Such a file exists for the SWMF itself and for each of the
physics components.  The file for the SWMF is found at
\begin{verbatim}
Param/PARAM.XML
\end{verbatim}
while the files for the physics components are found in the component's
subdirectory.  For example, the file for the GM/BATSRUS component can
be found at
\begin{verbatim}
GM/BATSRUS/PARAM.XML
\end{verbatim}
This file contains a complete list of all input commands for the
component as well as the allowed ranges for each of the input parameters.
Although the XML format makes the files a little hard to read, they are
extremely useful.  A typical usage is to cut and paste commands out of the
PARAM.XML file into the PARAM.in file for a run.

\subsubsection{Have the working directory in your path}
In order to run executable files in the UNIX environment you must have
the current working directory either your path or in the filename you
want to execute.  In UNIX the current working directory is represented
by the period (.).  For example
\begin{verbatim} 
./SWMF.exe
\end{verbatim}
will execute the SWMF.exe program if it is in your current directory.  If you
add the `.' to your path using
\begin{verbatim}
set path = (~/bin /usr/local/mpi/bin /usr/local/bin ${path} .)
\end{verbatim}
then you can simply type
\begin{verbatim} 
SWMF.exe
\end{verbatim}

\section{Installing the Code}

The first step in installing the SWMF is untarring the distribution.
If the tar program knows about the -z flag, you can open the gzipped
tar files with a single UNIX command:
\begin{verbatim}
  tar xzf SOMETARFILE.tgz
\end{verbatim}
If the tar program does not recognize the -z flag, two steps are needed:
\begin{verbatim}
  gunzip SOMETARFILE.tgz
  tar xf SOMETARFILE.tar
\end{verbatim}
In the following descriptions the shorter form is shown, but you may
need to use the two step procedure on certain platforms.

Untar the distribution using the command:
\begin{verbatim}
  tar xzf SWMF.tgz
\end{verbatim}

Change directories into the distribution:
\begin{verbatim}
  cd SWMF
\end{verbatim}

The SWMF needs to know what architecture you are running the code on
and what FORTRAN compiler will be used.  For most platforms and compilers,
tt can figure this out all by itself, but you have to run the command:
\begin{verbatim}
  SetSWMF.pl -i
\end{verbatim}
in the main directory. This creates {\tt Makefile.def} with
the correct absolute path to the base directory and {\tt Makefile.conf}
which contains the operating system and compiler specific part of
the Makefile. If the compiler is not the default one for a given
platform (e.g. not the NAG f95 compiler for a Linux platform) then
the compiler must be specified explicitly with the {\tt -c}
flag. If the MPI header file is not the default one, it can be
specified with the {\tt -m} flag. For example on the Altix machines
SWMF should be installed as
\begin{verbatim}
  SetSWMF.pl -i -c=ifort -m=Altix
\end{verbatim}
To uninstall SWMF type
\begin{verbatim}
  SetSWMF.pl -uninstall
\end{verbatim}
If the uninstallation fails (this can happen if some makefiles are missing)
force installation with
\begin{verbatim}
  SetSWMF.pl -install
\end{verbatim}
and then try uninstalling the code again.
When SWMF is installed, its configuration can be checked with
\begin{verbatim}
  SetSWMF.pl -s
\end{verbatim}
To get a list of the available component versions type
\begin{verbatim}
  SetSWMF.pl -l
\end{verbatim}
To get a complete description of the {\tt SetSWMF.pl}  script type
\begin{verbatim}
  SetSWMF.pl -h
\end{verbatim}

\section{Creating Documentation}

The documentation for SWMF can be generated from the distribution by
the command
\begin{verbatim}
  make PDF
\end{verbatim}
which creates the user manual
\begin{verbatim}
  doc/SWMF.pdf
\end{verbatim}
and several other documents in the Adobe PDF format.  
In order for this to work you must have
LaTex installed on your system (and dvips and ps2pdf).  
An on-line version can be created by
\begin{verbatim}
  make HTML
\end{verbatim}
The HTML version is generated from the LaTex using the command {\tt
  latex2html}.  You will have to install this if it does not already exist
  on your system.
The top level HTML file is in
\begin{verbatim}
  doc/HTML/index.html
\end{verbatim}
to point at with the browser.  This html file list the different
documentation files and what they contain.  To clean the intermediate files type
\begin{verbatim}
  cd doc/Tex
  make clean
\end{verbatim}
To remove all the created documentation type
\begin{verbatim}
  cd doc/Tex
  make cleanall
\end{verbatim}

\section{Building and Running an Executable}

At compile time, the user can select which physics components should be
compiled.  
Any component not compiled will not be available for
use at run time.  The physics components can be selected with the {\tt -v} flag
of the SetSWMF.pl script. For example typing
\begin{verbatim}
  SetSWMF.pl -v=SC/BATSRUS,IH/BATSRUS,SP/Kota
  SetSWMF.pl -v=GM/Empty,RB/Empty,IM/Empty,IE/Empty,UA/Empty
\end{verbatim}
will select BATSRUS for the SC and IH components and K\'ota's model for
the SP component.
The other components are set to Empty versions, which contain empty
subroutines for compilation, but cannot be used.
The default configuration includes a working version for all components, 
which takes up more memory, but is the most general.
The only exception is SC, which requires configuration, so the 
default version is Empty for the Solar Corona component.

The grid size of several components can also be set with the {\tt -g}
flag of the {\tt SetSWMF.pl} script. For example the 
\begin{verbatim}
  SetSWMF.pl -g=GM:8,8,8,400,100
\end{verbatim}
command sets the block size for the GM component to $8\times 8\times 8$ cells, 
the maximum number of blocks per processor to 400, 
and the maximum number of implicit blocks per processor to 100.
The SetSWMF.pl script actually runs the individual GridSize.pl
scripts in the component versions. These scripts can be run directly,
and they provide more options and more verbose information than SetSWMF.pl.
For example try
\begin{verbatim}
  cd GM/BATSRUS
  GridSize.pl -s
\end{verbatim}
Compilation flags, such as the precision and optimization 
level are stored in {\tt Makefile.conf}. This file is created on
installation of the SWMF and has defaults which are appropriate for
your system architecture.  The precision of reals
can be changed to single precision (for example) by typing
\begin{verbatim}
  SetSWMF.pl -p=single
\end{verbatim}
while the compiler flags can be edited in {\tt Makefile.conf} by hand.

Before compiling SWMF it is always a good idea to check its configuration
with
\begin{verbatim}
  SetSWMF.pl -s
\end{verbatim}

{\bf IMPORTANT NOTE:
On the Altix machine at NASA Ames (columbia)
you should load the 8.0.070 version 
of the Intel Fortran compiler 
with the command
\begin{verbatim}
  module load intel-comp.8.0.070
\end{verbatim}
You may wish to insert this line into the .cshrc file
so it executes at login time. 
Selecting the correct compiler version is 
necessary both to compile and to run the code.
Therefore the above line is needed in the job scripts
as well.}

To build the executable {\bf bin/SWMF.exe}, type:
\begin{verbatim}
  make
\end{verbatim} 
Depending on the configuration, the compiler settings and the machine 
that you are compiling on, this can take from 2 to up to 30 minutes.  
In addition, you may want to make the post processing
codes (for BATSRUS only) also:
\begin{verbatim}
  make PSPH
  make PIDL
\end{verbatim} 
These two commands will create the codes {\tt bin/PostSPH.exe}, for post
processing spherical Tecplot files, and {\tt bin/PostIDL.exe} 
for post processing IDL files.

The {\tt SWMF.exe} executable should be run in a sub-directory, since a large number
of files are created in each run.  To create this directory use the
command:
\begin{verbatim}
  make rundir
\end{verbatim} 
This command creates a directory called {\tt run}.  You can either
leave this directory as named, or {\tt mv} it to a different name.  It
is best to leave it in the same SWMF directory, since
keeping track of the code version associated with each run is quite
important.  The {\tt run} directory will contain links to the codes
which were created in the previous step as well as subdirectories
where input and output of the different components will reside.

Here we assume that the {\tt run} directory is still called {\tt
run}:
\begin{verbatim}
  cd run
\end{verbatim}
In order to run the SWMF you must have two input files:  LAYOUT.in and
PARAM.in.  The LAYOUT.in file defines the processor
layout for the components involved in the future run.  The PARAM.in
file contains the detailed commands for controlling what you want the
code to do during the run.  The default LAYOUT.in and PARAM.in
files in the run directory are suitable to perform the ``Start'' test
on 16 processors (PE-s). 

An example processor map file LAYOUT.in to run the executable with
five components on 16 processors is:
\begin{verbatim}
#COMPONENTMAP
GM    0    4    1
IE    5    6    1
IH    7   10    1
IM   11   11    1
UA   12   15    1
#END
\end{verbatim}
The file syntax is simple. It must start with the directive
\#COMPONENTMAP and end with another directive \#END. Each line between
these directives specifies the label for component, i.e. IE, GM and
etc., its first and last processor, all relatively to the world
communicator, and the stride. Thus GM will run on 5 processors from 0
to 4, and IM will run on only 1 processor, the processor 11.  If
stride is not equal to 1, the processors for the component will not be
neighboring processors.

It is strongly recommended to check the validity of the {\tt run/PARAM.in} and 
{\tt run/LAYOUT.in} files before running the code. If the
code will be run on 16 processors, type
\begin{verbatim}
Scripts/TestParam.pl -n=16
\end{verbatim}
in the main SWMF directory.
The Perl script reports inconsistencies and errors. 
If no errors are found, the script finishes silently.
Now you are ready to run the executable through submitting a batch job or, 
if it is possible on your computer, run the code interactively.  For
example, to run the SWMF interactively:
\begin{verbatim}
cd run
mpirun -np 16 SWMF.exe
\end{verbatim}
The SWMF provides example job scripts for several architectures and
machines used by the developers. These job scripts are found in 
\begin{verbatim}
CON/Scripts
\end{verbatim}
in the subdirectories named after the operating system. If the name
of the file in the appropriate subdirectory matches the 
name of the machine, the job script is copied into
the {\tt run} directory when it is created.
These job scripts serve as a starting point only, they must
be customized before they can be used for submitting a job.

To recompile the executable with different compiler settings you have
to use the command
\begin{verbatim}
make clean
\end{verbatim}
before recompiling the executables. It is possible to recompile
only a component or just one subdirectory if the {\tt make clean}
command is issued in the appropriate directory.

\section{Restarting a Run}

There are several reasons for restarting a run. A run may fail
due to a run time error, due to hardware failure, due to 
software failure (e.g. the machine crashes) or because the
queue limits are exceeded. In such a case the run can be continued from
the last saved state of SWMF. 

It is also possible that one builds up a complex simulation from multiple 
runs. For example the first run creates a steady state for the SC component.
The second run includes both the SC and IH components and it 
restarts from the results of the first run and creates a steady state
for both components. A third run may restart from this solution and include
the GM component, etc. 

The restart files are saved at the frequency determined in the PARAM.in file.
Normally the restart files are saved into the output restart directories
of the individual components and subsequent saves overwrite the previous ones
(to reduce the required disk space). A restart requires the modification
of the PARAM.in file: one needs to include the restart file for the
control module of SWMF as well as ask for restart by all the components.

The Scripts/Restart.pl script simplifies the work of the restart in
several ways:
\begin{enumerate}
\item The SWMF restart file and the individual output restart 
directories of the components are collected into a single directory tree, 
the {\bf restart tree}.
\item The default input restart file of SWMF and the default 
      input directories of the components can be linked to an existing
      restart tree.
\item The script can run continuously in the background and create
      multiple restart trees while SWMF is running. 
\item The script does extensive checking of the consistency 
      of the restart files.
\end{enumerate}
The Restart.pl script is copied into the run directory and it should
be executed in the run directory. Note that the PARAM.in file is not
modified by the script: it has to be modified with an editor as needed.

To demonstrate the use of the script, here are a few simple examples.
After a successful or failed run which should be continued, simply type
\begin{verbatim}
cd run
./Restart.pl
\end{verbatim}
to create a restart tree from the final output and to link to the tree for the
next run. The default name of the restart tree is based on the simulation time
for time accurate runs, or the time step for non-time accurate runs.
But you can also specify a name explicitly, for example
\begin{verbatim}
./Restart.pl RESTART_SC_steady_state
\end{verbatim}
If you wish to continue the run in another run directory, or on another
machine, transfer the restart tree as a whole into the new run
directory and type
\begin{verbatim}
./Restart.pl -i=RESTART_SC_steady_state
\end{verbatim}
where the {\tt -i} stands for ``input only'', i.e. the script links to
the tree, but it does not attempt to create the restart tree.

To save multiple restart trees repeatedly at an hourly frequency of 
wall clock time while the SWMF is running, type
\begin{verbatim}
./Restart.pl -r=3600 &
\end{verbatim}
To see all the options of the script type
\begin{verbatim}
./Restart.pl -h
\end{verbatim}

\section{What next?}

Hopefully this section has guided you through installing the SWMF and
given you a basic knowledge of how to run it.  However it has probably
also convinced you that the SWMF is quite a complex tool and that there
are many more things for you to learn.  So, what next?

We suggest that you read all of chapter \ref{chapter:basics}, which
outlines the basic features of the SWMF as well as some things you
really must know in order to use the SWMF.  Once you have done this you
are ready to experiment.  Chapter \ref{chapter:examples} gives several 
examples which are intended to make you familiar with the use of the
SWMF.  We suggest that you try them!

%\end{document}


% Basic User manual
\chapter{The Basics \label{chapter:basics}}

%  Copyright (C) 2002 Regents of the University of Michigan, portions used with permission 
%  For more information, see http://csem.engin.umich.edu/tools/swmf
\section{Configuration of SWMF}

Configuration refers to several different ways of controlling how the 
SWMF is compiled and run.  The most obvious is the setting of
compiler flags specific to the machine and version of FORTRAN
compiler.  The other methods refer to ways in which different physics
components are chosen to participate in or not participate in a run.
Inclusion of components can be controlled using one of several methods:

\begin{itemize}
\item The source code can modified so that all references %^CMP IF CONFIGURE
      to a subset of the components is removed. %^CMP IF CONFIGURE
      This method uses the Scripts/Configure.pl script. %^CMP IF CONFIGURE
      In a similar way, some physics components can be individually
      configured.
\item The user may select which version of a physics component,
      including the Empty version,
      should be compiled.  This is controlled using the Config.pl script.
\item When submitting a run, a subset of the non-empty (compiled) 
      components can be
      registered to participate in the run in the LAYOUT.in file.
\item Registered components can be turned off and on with the \#COMPONENT
      command in the PARAM.in file.
\end{itemize}
Each of these options have their useful application.

Finally, each physics component may have some settings which need to
(or can) be individually
configured, such as selecting user routines for the IH/BATSRUS or
GM/BATSRUS components.

%^CMP IF CONFIGURE BEGIN
\subsection{Scripts/Configure.pl}

The Scripts/Configure.pl script can build a new software package which
contains only a subset of the components. It is a simple interface
for the general share/Scripts/Configure.pl script. The configuration
can remove a whole component directory and all references to the component 
in the source code, in the scripts and the Makefiles.
This type of configuration results in a smaller software package.
The main use of this type of configuration is to distribute
a part of SWMF to users. For example one can create a 
software distribution which includes GM, IE and UA only by typing
\begin{verbatim}
  Scripts/Configure.pl -on=GM,IE,UA -off=SC,IH,SP,IM,PW,RB
\end{verbatim}
The configured package will be in the Build directory.  Type
\begin{verbatim}
  Scripts/Configure.pl -h
\end{verbatim}
to get complete usage information or read about this script 
in the reference manual.
%^CMP END CONFIGURE

\subsection{Selecting physics models with Config.pl}

The physics models (component versions) reside in the component 
directories CZ, EE, GM, IE, IH, IM, OH, RB, PS, PT, PW, SC, SP and UA.
Most components have only one working version and one empty version.
The empty version consists of a single wrapper file, which contains 
empty subroutines required by CON\_wrapper and the couplers.
These empty subroutines are needed for the compilation of the code,
and they also show the interface of the working versions.

The appropriate version can be selected with the {\tt -v} flag
of the {\tt Config.pl} script, which edits the Makefile.def file.
For example
\begin{verbatim}
  Config.pl -v=GM/BATSRUS,IM/RCM2,IE/Ridley_serial
\end{verbatim}
selects the BATSRUS, RCM2 and Ridley\_seriel models for
the GM, IM and IE components, respectively.
To see the current selectoin and the available models for all
the components type
\begin{verbatim}
  Config.pl -l
\end{verbatim}
The first column shows the currently selected models, the rest are the 
available alternatives.

If a physics component is not needed for a particular run, 
an Empty version of the component can be compiled.
Selecting the Empty version for unused components reduces
compilation time and memory usage during run time.
It may also improve performance slightly.
This is achieved with the {\tt -v} flag of the Config.pl script. 
For example the Empty UA component can be selected with
\begin{verbatim}
  Config.pl -v=UA/Empty
\end{verbatim}
It is also possible to select the Empty version for all components
with a few exceptions. For example
\begin{verbatim}
  Config.pl -v=Empty,GM/BATSRUS,IE/Ridley_serial
\end{verbatim}
will select the Empty version for all components except for GM and IE.
Note that the 'Empty' item has to be the first one.

\subsection{Clone Components}

The EE/BATSRUS, IH/BATSRUS, OH/BATSRUS and SC/BATSRUS models are special, 
since they use the same source code as GM/BATSRUS, which is stored 
in the CVS repository. We call the other BATSRUS models
{\bf clones} of the GM/BATSRUS code. The source code of the clone models
is copied over from the original files and then all modules,
external subroutines and functions are renamed. For example
ModMain.f90 is renamed to IH\_ModMain.f90 in IH/BATSRUS.
These steps are performed automatically when the clone model is selected
for the first time, for example by typing
\begin{verbatim}
Config.pl -v=IH/BATSRUS
\end{verbatim}
Once the source code is copied and renamed, the clone models work
just like any model. They can be configured, compiled, and used in runs.

It is important to realize that code development is always done
in the original source code, i.e. in GM/BATSRUS and in 
IH/BATSRUS/srcInterface/IH\_wrapper.f90.
If the source code of the clones should be refreshed, for example
after an update from the CVS respository, type
\begin{verbatim}
make cleanclones
Config.pl
\end{verbatim}
and the source code will be copied and renamed for the selected clones.
The source code of the clones is removed fully when the SWMF is
uninstalled with the
\begin{verbatim}
Config.pl -uninstall
\end{verbatim}
command. 

\subsection{Registering components with LAYOUT.in}

The components used in particular run has to be listed (registered)
in the LAYOUT.in file. 
Note that empty component versions cannot be registered at all.
Component registration allows to run the same executable with different 
subsets of the components. For example the GM and IE components 
can be selected with the following LAYOUT.in file
\begin{verbatim}
ID   first last  stride
#COMPONENTMAP
IE     0      1     1
GM     2   9999     1
#END
\end{verbatim}
The first column contains the component ID, the second is the index
of the first (root) processor for the component, the third column is the
last processor and the last column contains the stride 
that is typically set to 1.
In the example above IE will run on the first 2 PE-s,
while GM will run on the rest of the available PE-s.
Changing the LAYOUT.in file to
\begin{verbatim}
ID   first last  stride
#COMPONENTMAP
GM     0    999     1
#END
\end{verbatim}
will still use the same executable, but will not allow the IE 
physics component to participate in the run.

\subsection{Switching models on and off with PARAM.in}

Registered components can be switched on and off during a run
with the \#COMPONENT command in the PARAM.in file. 
This approach allows the component to be switched on in a later 
'session' of the run. For example, in the first session only GM 
is running, while in the second session it is coupled to IE. 
In this example the IE component can be switched off with the
\begin{verbatim}
#COMPONENT
IE              NameComp
F               UseComp
\end{verbatim}
in the first session and it can be switched on with the
\begin{verbatim}
#COMPONENT
IE              NameComp
T               UseComp
\end{verbatim}
command in the second session.

\subsection{Setting compiler flags}

The debugging flags can be switched on and off with
\begin{verbatim}
  Config.pl -debug
\end{verbatim}
and
\begin{verbatim}
  Config.pl -nodebug
\end{verbatim}
respectively. The maximum optimization level can be set to -O2 with
\begin{verbatim}
  Config.pl -O2
\end{verbatim}
The minimum level is 0, the maximum is 5. Note that not all compilers support
level 5 optimization. As already mentioned, the code needs to be cleaned 
and recompiled after changing the compiler flags:
\begin{verbatim}
  make clean
  make -j
\end{verbatim}
Note that not all the components take into account the selected
compiler flags. For example the IM/RCM2 component has to be compiled 
with the -save (or similar) flag, thus it uses the flags defined in the 
{\tt CFLAGS} variable. Also some of the compilers produce incorrect
code if they compile certain source files with high optimization level.
Such exceptions are described in the 
\begin{verbatim}
  Makefile.RULES.all
\end{verbatim}
files in the source code directories. The content of this file
is processed by {\tt Config.pl} into {\tt Makefile.RULES}
(according to the selected compiler and other parameters),  
which is then included into the main Makefile of the source
directory.

\subsection{Configuration of individual components}

Some of the components can be configured individually. 
The {\tt GM/BATSRUS} code, for example, can be configured to
use specific equation and user modules.
For example
\begin{verbatim}
cd GM/BATSRUS
Config.pl -e=MhdIonsPe
\end{verbatim}
will select the equation module for multiple ion fluids and separate
electron pressure. The same can be done with the {\tt Config.pl} script
in the main SWMF directory
\begin{verbatim}
Config.pl -o=GM:e=MhdIonsPe
\end{verbatim}
The grid sizes of the various components can be set with the 
{\tt -g} flag of the {\tt Config.pl} script.
For example the
\begin{verbatim}
  Config.pl -g=UA:36,36,50,16
\end{verbatim}
will set the blocks size to $36\times 36\times 50$ and the number of blocks to 
16 for the UA/GITM2 component. This command runs the {\tt Config.pl}
script of the selected UA component. 
On machines with limited memory it is especially important to
set the number of blocks correctly. 

Of course, the SWMF code has to be recompiled after any of these changes with
\begin{verbatim}
  make -j
\end{verbatim}
Note that in this case there is no need to type 'make clean', 
because the {\tt make} command knows which files need to be recompiled.

\subsection{Using stubs for all components}

It is possible to compile and run the SWMF without the physics components
but with place holders (stubs) for them that mimic their behavior.
This can be used as a test tool for the CON component, but it may
also serve as an inexpensive testbed for getting the optimal layout
and coupling schedule for a simulation. To configure SWMF with 
stub components, select the Empty version for all physics components
(with Config.pl -v=...) and edit the {\tt Makefile.def} file to
contain
\begin{verbatim}
#INT_VERSION = Interface
INT_VERSION = Stubs
\end{verbatim}
for the interface so that the real interface in {\tt CON/Interface}
is replaced with {\tt CON/Stubs}.
The resulting executable will run CON with 
the stubs for the physics components. For the stubs one can
specify the time step size in terms of simulation time and the
CPU time needed for the time step. The stub components communicate
at the coupling time, so the PE-s need to synchronize, but 
(at least in the current implementation) there is no net time taken
for the coupling itself. 

The stub components help development of the SWMF core, but it also
allows an efficient optimization of the LAYOUT and coupling
schedules for an actual run, where the physical time steps
and the CPU time needed by the components is approximately known.
In the test runs with the Stubs, one can reduce the CPU times by 
a fixed factor, so it takes less CPU time to see the efficiency of the 
SWMF for a given layout and coupling scheme.

An alternative way to test performance with different configurations is
to use the Scripts/Performance.pl script. See the help message of the
script for information on usage.

%  Copyright (C) 2002 Regents of the University of Michigan, portions used with permission 
%  For more information, see http://csem.engin.umich.edu/tools/swmf
%\documentclass{article}
%\begin{document}

\section{PARAM.in \label{section:param.in}}

The input parameters for the SWMF are read from the 
{\tt PARAM.in} file which must be located in the run directory.
This file, together with the LAYOUT.in file, controls the SWMF
and its components.
There are many include files in the {\tt Param} directory. These
can be included into the {\tt PARAM.in} files, or they can serve as
examples. 

In the PARAM.in file, 
the parameters specific to a component are given between
the \#BEGIN\_COMP ID and \#END\_COMP ID commands,
where the ID is the two character identifier of the component.
For example the GM parameters are enclosed between the 
\begin{verbatim}
#BEGIN_COMP GM
...
#END_COMP GM
\end{verbatim}
commands. We refer to the lines starting with a \# character as commands.
For example if the command string 
\begin{verbatim}
#END
\end{verbatim}
is present, it indicates the end of the run and lines following
this command are ignored. If the \#END command is not
present, the end of the PARAM.in file indicates the end of the run.

There are several features of the input parameter file syntax
that allow the user to easily run the code
in a variety of modes while at the same time being able to 
keep a library of useful parameter files that can be used
again and again.

The syntax and the content of the input parameter files
is defined in the PARAM.XML files. The commands controlling
the whole SWMF are described in the main directory in the
\begin{verbatim}
  PARAM.XML
\end{verbatim}
file. The component parameters are described by the PARAM.XML
file in the component version directory. For example the
input parameters for the GM/BATSRUS component are described in
\begin{verbatim}
  GM/BATSRUS/PARAM.XML
\end{verbatim}
These files can be read (and edited) in a normal editor.
The same files are used to produce much of this
manual with the aid of the {\tt share/Scripts/XmlToTex.pl} script. 
The {\tt Scripts/TestParam.pl} script also uses these files
to check the PARAM.in file.
Copying small segments of the {\tt PARAM.XML} files
into {\tt PARAM.in} can speed up the creation or modification of a 
parameter file. 

\subsection{Included Files, {\tt \#INCLUDE} \label{section:include}}

The {\tt PARAM.in} file can include other parameter files with the 
command
\begin{verbatim}
#INCLUDE
include_parameter_filename
\end{verbatim}
The include files serve two purposes: (i) they help
to group the parameters; (ii) the included files can be reused
for other parameter files. 
An include file can include another file itself.
Up to 10 include files can be nested.
The include files have exactly the same structure as {\tt PARAM.in}. 
The only difference is that the
\begin{verbatim}
#END
\end{verbatim}
command in an included file means only the end of the include file, 
and not the end of the run, as it does in {\tt PARAM.in}.

The user can place his/her
included parameter files into the main run directory or in any subdirectory
as long as the correct path to the file from the run directory is
included in the {\tt \#INCLUDE} command.

\subsection{Commands, Parameters, and Comments \label{section:commands}}

As can be seen from the above examples, the parameters are entered
with a combination of a {\bf command} followed by specific {\bf parameter(s)},
if any.
The {\bf command} must start with a hashmark (\#), which 
is followed by capital letters and underscores without space in between. 
Any characters behind the first space or TAB character are ignored
(the \#BEGIN\_COMP and \#END\_COMP commands are the only exception,
but these are markers rather than commands).
The parameters, which follow, must conform to 
requirements of the command. They can be of four types: logical, integer,
real, or character string. Logical parameters can be entered as 
{\tt .true.} or {\tt .false.} or simply {\tt T} or {\tt F}.
Integers and reals can be in any of the usual Fortran formats.  In
addition, real numbers can be entered as fractions (5/3 for example).
All these can be followed by arbitrary comments, typically separated
by space or TAB characters. In case of the character type input
parameters (which may contain spaces themselves), the comments must
be separated by a TAB or by at least 3 consecutive space characters.
Comments can be freely put anywhere between two commands as long
as they don't start with a hashmark.

Here are some examples of valid commands, parameters, and comments:
\begin{verbatim}

#TIMEACCURATE
F                       DoTimeAccurate

Here is a comment between two commands...

#DESCRIPTION
My first run            StringDescription (3 spaces or TAB before the comment)

#STOP
-1.                     tSimulationMax
100                     MaxIteration

#RUN ------------ last command of this session -----------------

#TIMEACCURATE
T                       DoTimeAccurate

#STOP
10.0                    tSimulationMax
-1                      MaxIteration

#BEGIN_COMP IH

#GAMMA
5/3                     Gamma

#END_COMP IH

\end{verbatim}

\subsection{Sessions \label{section:sessions}}

A single parameter file can control consecutive {\bf sessions}
of the run. Each session looks like
\begin{verbatim}
#SOME_COMMAND
param1
param2

...

#STOP
max_simulation_time_for_this_session
max_iter_for_this_session

#RUN
\end{verbatim}
while the final session ends like
\begin{verbatim}
#STOP
max_simulation_time_for_final_session
max_iter_for_final_session

#END
\end{verbatim}
The purpose of using multiple sessions is to be able to change parameters 
during the run. For example one can use only a subset of the
components in the first session, and add more components in the
later session. Or one can obtain a coarse steady state solution
on a coarse grid with a component in one session, and improve on the solution
with a finer grid in the next session. Or one can switch from 
steady state mode to time accurate mode. The SWMF remembers parameter
settings from all previous sessions, so in each session one should only
set those parameters which change relative to the previous session.
Note that the maximum number of iterations given in the {\tt \#STOP} command 
is meant for the entire run, and not for the individual sessions. 
On the other hand, when a restart file is read, the iterations prior to 
the current run do not count.

The {\tt PARAM.in} file and all included parameter files are read into 
a buffer at the beginning of the run, so even for multi-session runs, 
changes in the parameter files have no effect once {\tt PARAM.in} 
has been read. 

\subsection{The Order of Commands \label{section:order}}

In essence, the order of parameter commands within a
session is arbitrary, but there are some important restrictions.  
We should note that the order of the parameters following 
the command is not arbitrary and must exactly match what the code requires.  
Here we restrict ourselves to the restrictions on the commands read by
the control module of SWMF. There may be (and are) restrictions
for the commands read by the components, but those are described
in the documentation of the components.

The only strict restriction on the SWMF commands is related
to the 'planet' commands. The default values of the 
planet parameters are defined by the \#PLANET command.
For example the parameters of Earth can be selected with the
\begin{verbatim}
#PLANET
Earth            NamePlanet
\end{verbatim}
command. The true parameters of Earth can be modified or simplified
with a number of other commands which {\bf must occur after the
\#PLANET command}. These commands are (without showing their parameters)
\begin{verbatim}
#IDEALAXES
#ROTATIONAXIS
#MAGNETICAXIS
#MAGNETICCENTER
#ROTATION
#DIPOLE
\end{verbatim}
Other than this strict rule, it makes sense to follow a 'natural'
order of commands. This will help in interpreting, maintaining
and reusing parameter files.

If you want all the input parameters to be echoed back, the first
command in {\tt PARAM.in} should be
\begin{verbatim}
#ECHO
T                 DoEcho
\end{verbatim}
If the code starts from restart files, it usually reads in a
file which was saved by SWMF. The default name of the saved
file is RESTART.out and it is written into the run directory.
It should be renamed, for example to RESTART.in, so that it
does not get overwritten during the run. It can be included as
\begin{verbatim}
#INCLUDE
RESTART.in
\end{verbatim}
The SWMF will read the following commands (the parameter values are
examples only) from the included file:
\begin{verbatim}
#DESCRIPTION
Create startup for GM-IM-IE-UA from GM steady state.

#PLANET
EARTH                        NamePlanet

#STARTTIME
    1998                     iYear
       5                     iMonth
       1                     iDay
       0                     iHour
       0                     iMinute
       0                     iSecond
 0.000000000000              FracSecond
 
#NSTEP
    4000                     nStep
 
#TIMESIMULATION
 0.00000000E+00              tSimulation
 
#VERSION
 2.00                        VersionSwmf
 
#PRECISION
8                              nByteReal
\end{verbatim}
The \#PLANET command defines the selected planet.
The \#STARTTIME command defines the starting date and time of the whole
simulation. The current simulation time (which is relative to
the starting date and time) and the step number are
given by the \#TIMESIMULATION and \#NSTEP commands. Finally
the \#VERSION and \#PRECISION commands check the consistency
of the current version and real precision with the run which
is being continued. For sake of convenience, the \#IDEALAXES,
\#ROTATEHGR and \#ROTATEHGI commands are also saved 
into the restart file if they were set in the run.

As it was explained above, all modifications of the planet 
parameters should follow the \#PLANET command, i.e. they should be after 
the \#INCLUDE RESTART.in command. In case the description is
changed it should also follow, e.g.
\begin{verbatim}
#INCLUDE
RESTART.in

#DESCRIPTION
We continue the run for another 2 hours
\end{verbatim}
When the run starts from scratch, the PARAM.in file
should start similarly with the 
\begin{verbatim}
#DESCRIPTION
This is the start up run

#PLANET
SATURN

#STARTTIME
    2004                     iYear
       8                     iMonth
      15                     iDay
       1                     iHour
      25                     iMinute
       0                     iSecond
 0.000000000000              FracSecond
\end{verbatim}
commands (the parameters are examples only).
These commands are typically followed by the planet parameter
modifying commands, if any, and setting time accurate mode
(if changed from default true to false or relative to the previous session).
For example:
\begin{verbatim}
! Align the rotation and magnetic axes with Z_GSE
#IDEALAXES

#TIMEACCURATE
F                           DoTimeAccurate
\end{verbatim}
All the commands which are written into the RESTART.out file and all 
the planet modifying commands can only occur in the first session.
These commands contain parameters which should not change during a run.
In the PARAM.XML file these commands are marked with an 
{\tt if="\$\_IsFirstSession"} conditional.
If any of these parameters are attempted to be changed in later sessions, 
a warning is printed on the screen and the code stops running
(except when the code is in non-strict mode).

Most command parameters have sensible default values.
These are described in the PARAM.XML files,
and in chapter \ref{chapter:commands} (which was produced from them).
The {\tt PARAM.XML} file also defines which commands are required
with the {\tt required="T"} attribute of the {\tt <command...>} tag.
For the control module the only required command in every
session is the \#STOP command
(or this can be replaced with the \#ENDTIME command in the last session), 
which defines the final time step in steady state mode 
or the final time of the session in time accurate mode.

\subsection{Iterations, Time Steps and Time \label{section:frequency}}

In several commands the frequency or `time' of some action has
to be defined. This is usually done with a pair of parameters.
The first defines the frequency or time in terms of the number of time steps,
and the second in terms of the simulation time.
A negative value for the frequency means that it should not be taken 
into account. For example, in time accurate mode,
\begin{verbatim}
#SAVERESTART
T            DoSaveRestart
2000         DnSaveRestart
-1.          DtSaveRestart
\end{verbatim}
means that a restart file should be saved after every 2000th time step, while
\begin{verbatim}
#SAVERESTART
T            DoSaveRestart
-1           DnSaveRestart
100.0        DtSaveRestart
\end{verbatim}
means that it should be saved every 100 seconds in terms of physical time.
Defining positive values for both frequencies might be useful
when switching from steady state mode to time accurate mode.
In the steady state mode the DnSaveRestart parameter is used,
while in time accurate mode the DtSaveRestart if it is positive.
But it is more typical and more intuitive 
to explicitly repeat the command in the first 
time accurate session with the time frequency set.

The purpose of this subsection is to try to help the user understand 
the difference between the iteration number used for stopping the code
and the time step which is used to define the frequencies of various
actions. After using \BATSRUS\ over several years, it is clear to the
authors that this distinction is useful and the
most reasonable implementation. The SWMF has inherited these
features from the \BATSRUS\ code.

We begin by defining several different quantities and the variables that 
represent them in the code.  The variable {\tt nIteration}, 
represents the number of ``iterations'' 
that the simulation has taken since it began running.  
This number starts at zero every time the code is run, even if beginning 
from a restart file.
This is reasonable since most users know how many iterations the code can take
in a certain amount of CPU time and it is this number that is needed when 
running in a queue.
The quantity {\tt nStep} is a number of ``time steps'' that the code has 
taken in total.  This number starts at zero when the code is started from 
scratch, but when started from a restart file, this
number will start with the time step at which the restart file was written.
This implementation lets the user output data files at a regular interval, even
when a restart happens at an odd number of iterations.
The quantity {\tt tSimulation} is the amount of simulated, or physical, 
time that the code has run.  
This time starts when time accurate time stepping begins.
When restarting, it starts from the physical time for the restart.
Of course the time should be cumulative since it is the physically meaningful
quantity.  We will 
use these three phrases( ``iteration'', ``time step'', ``time'') 
with the meanings outlined above.

Now, what happens when the user has more than one session and he or she
changes the frequencies.  Let us examine what would happen in the following
sample of part of a {\tt PARAM.in} file.  For the following example we will
assume that when in time accurate mode, 1 iteration simulates 1 second of time.
Columns to the right indicate the values of {\tt nITER}, {\tt n\_step} and
{\tt time\_simulation} at which restart files will be written in each session.

\clearpage

\begin{verbatim}
                                             Restart Files Written at:
==SESSION 1                         Session   nITER   nStep    time_simulation
#TIMEACCURATE                       --------  ------  -------  --------------
F            DoTimeAccurate  

#SAVERESTART                             1     200      200             0.0  
T            DoSaveRestart               1     400      400             0.0
200          DnSaveRestart
-1.0         DtSaveRestart

#STOP
400          MaxIteration
-1.          tSimulationMax

#RUN ==END OF SESSION 1== 
                         
#SAVERESTART                             2     600      600             0.0
T            DoSaveRestart               2     900      900             0.0
300          DnSaveRestart
-1.0         DtSaveRestart
				
#STOP				
1000         MaxIteration				
-1.          tSimulationMax
				
#RUN ==END OF SESSION 2== 

#TIMEACCURATE
T            DoTimeAccurate  		
				
#SAVERESTART                             3    1100     1100           100.0
T            DoSaveRestart               3    1200     1200           200.0
-1           DnSaveRestart               3    1300     1300           300.0
100.0        DtSaveRestart
				
#STOP				
-1           MaxIteration				
300.0        tSimulationMax			
				
#RUN ==END OF SESSION 3== 
                          
#SAVERESTART                             4    1400     1400           400.0
T            DoSaveRestart               4    1800     1800           800.0
-1           DnSaveRestart               4    2000     2000          1000.0
400.0        DtSaveRestart
 				
#STOP				
-1           MaxIteration				
1000.0       tSimulationMax				
				
#END  ==END OF SESSION 4== 
\end{verbatim}
Now the question is how many iterations will be taken and when will restart
file be written out.  In session 1 the code will make 400 iterations and will
write a restart file at time steps 200 and 400.  Since the iterations 
in the {\tt \#STOP}
command are cumulative, the {\tt \#STOP} command in the second session will
have the code make 600 more iterations for a total of 1000.  Since the timing
of output is also cumulative, a restart file will be written at time step 600
and at 900.
After session 2, the code is switched to time accurate mode.  Since we
have not run in this mode yet the simulated (or physical) time is cumulatively
0.  The third session will run for 300.0 simulated seconds (which for the
sake of this example is 300 iterations).  The restart file will be written
after every 100.0 simulated seconds.
The {\tt \#STOP} command in Session 4 tells the code to simulate  700.0 more 
seconds for a total of 1000.0 seconds.  The code will make a restart file
when the time is a multiple of 400.0 seconds or at 400.0 and 800.0 seconds.
Note that a restart file will also be written at time 1000.0
seconds since this is the end of a run.

In the next example we want to restart from 1000.0 seconds 
and continue with a time accurate run.
\begin{verbatim}
                                             Restart Files Written at:
==SESSION 1                         Session   nITER   nStep    time_simulation
                                    --------  ------  -------  --------------
#INCLUDE                                 1       0     2000          1000.0
RESTART.in

#TIMEACCURATE
T            DoTimeAccurate  

#SAVERESTART                             1     200     2200          1200.0
T            DoSaveRestart
-1           DnSaveRestart
600.0        DtSaveRestart

#STOP
-1           MaxIteration
1400.0       tSimulationMax

#RUN ==END OF SESSION 1== 

#SAVERESTART                             2     700     2700          1500.0
T            DoSaveRestart               2    1000     3000          2000.0
-1           DnSaveRestart
750.0        DtSaveRestart

#STOP
-1           MaxIteration
2000.0       tSimulationMax

#END ==END OF SESSION 2 = 
                          
\end{verbatim}
In this example, we see that in time accurate mode the simulated, or
physical, time is always cumulative.  To make 400.0 seconds more simulation,
the original 1000.0 seconds must be taken into account.  The final output 
at 2000.0 seconds is written because the run ended.

Throughout this subsection, we have used the frequency of writing restart files
as an example.  The frequencies of coupling components and checking stop
files work similarly. In the SWMF, and potentially in any of the
components, the frequencies are handled by the general
\begin{verbatim}
  share/Library/src/ModFreq
\end{verbatim}
module which is described in the reference manual.

%\end{document}

%  Copyright (C) 2002 Regents of the University of Michigan, portions used with permission 
%  For more information, see http://csem.engin.umich.edu/tools/swmf
\section{Execution and Coupling Control}

The control module of SWMF controls the execution and coupling of
components. The control module is controlled by the user through the
input parameter file PARAM.in.
Defining the most efficient component layout, execution and coupling control
is not an obvious task. In the current version of SWMF the processor
layout of the components is static. This restriction is somewhat
mitigated by the possibility of restart, which allows to change
the processor layout from one run to another.

\subsection{Processor Layout}

Within one run the layout is determined by the \#COMPONENTMAP
command in the PARAM.in file. The command
is documented in the PARAM.XML file.
Here we provide several examples which will help to develop
a sense of using optimal layouts.  An optimal layout is one that 
maximizes the use of all processors and does not leave processors
with nothing do while waiting for other processors to finish their work.

First of all we have to define the processor rank:
it is a number ranging from 0 to $N-1$, where
$N$ is the total number of processors in the run. 
A component can run on a subset of the processors,
which is defined by the rank of the first (root) processor,
the rank of the last processor, and the stride. For
example if the root processor has rank 4, the last processor
has rank 8, and the stride is 2 than the component will
run on 3 processors with ranks 4, 6 and 8.

\subsubsection{One component}

In the simplest case a single component, say the Global
Magnetosphere (GM) is running. The layout should be
the following
\begin{verbatim}
ID   Proc0 ProcEnd Stride
#COMPONENTMAP
GM   0     -1    1
\end{verbatim}
Here the -1 is interpreted as the rank of the last processor,
which is $N-1$ if the SWMF is running on $N$ processors.

\subsubsection{One serial and one parallel component}

When two components are used, their layouts may or may not overlap.
An example for overlapping the layouts of the GM and the
Inner Magnetosphere (IM) components is
\begin{verbatim}
ID   proc0 last stride
#COMPONENTMAP
IM    0       0    1
GM    0      -1    1
\end{verbatim}
When the component layouts overlap, the two components can run
sequentially only. Since IM is using a single processor only
(because it is not a parallel code), all the other processors 
will be idling while IM is running. This can be rather inefficient,
especially if the CPU time required by IM is not negligible.
A more efficient execution can be achieved with a non-overlapping layout:
\begin{verbatim}
ID   proc0 last stride
#COMPONENTMAP
IM    0       0    1
GM    1      -1    1
\end{verbatim}
Note that this layout file will work for any number 
of processors from 2 and up.

\subsubsection{Two parallel components with different speeds}

It is not always possible, or even efficient to use non-overlapping
layouts. For example both the SC and IH components require a lot of memory,
but the IH component runs much faster (say 100 times faster) 
in terms of cpu time than the SC component (this is due to the 
larger cells and smaller wave speeds in IH).
If we tried to use concurrent execution on 101 processors,
SC should run on 100 and IH on 1 processors to get good load balancing.
However the IH component needs much more memory than available
on a single processor. It is therefore not possible to use a non-overlapping
layout for SC and IH on a reasonable number of processors.

Fortunately both the Solar Corona (SC) and Inner Heliosphere (IH)
components are modeled by \BATSRUS, which is a highly parallel code
with good scaling. The following layout can be optimal:
\begin{verbatim}
ID   proc0 last stride
#COMPONENTMAP
IH    0    -1    1
SC    0    -1    1
\end{verbatim}
Although IH and SC will execute sequentially, they both
use all the available CPU-s, so no CPU is left waiting for the others.

\subsubsection{Two parallel components with similar speeds}

If two parallel components need about the same CPU time/real time
on the same number of processors, the optimal layout can be
\begin{verbatim}
ID   proc0 last stride
#COMPONENTMAP
GM    0     -1    2
SC    1     -1    2
\end{verbatim}
Here GM is running on the processors with even rank,
while SC is running on the processors with odd ranks.
By using the processor stride, this layout works on
an arbitrary number of processors.

When more serial and parallel codes are executing together,
finding the optimal layout may not be trivial. 
It may take some experimentation to see which component
is running slower or faster, how much time is spent
on coupling two components, etc. It may be a good idea
to test the components separately or in smaller groups
to see how fast they can execute.

\subsubsection{A complex example with four components}

Here is an example with 4 components: the Ionospheric
Electrodynamics (IE) component can run on 2 processors and 
runs about 3 times faster than real time.
The serial Inner Magnetosphere (IM) component runs even faster,
on the other hand the coupling of GM and IM is rather
computationally expensive. The Upper Atmosphere (UA) component
can run on up to 32 processors, and it runs twice as fast
as real time. The Global Magnetosphere model (GM) needs
at least 32 processor to run faster than real time.
If we have a lot of CPU-s, we may simply create a non-overlapping
layout. Since GM has no restriction on the number of processors,
it can be the last component in the map
\begin{verbatim}
ID   proc0 last stride
#COMPONENTMAP
IM    0       0    1
IE    0       1    1
UA    2      33    1
GM   34     999    1
\end{verbatim}
This layout will be optimal in terms of speed for a large 
(more than 100) number of PE-s, and actually the maximum
speed is going to be limited by the components which do
not scale. On a more modest number of PE-s one can try
to overlap UA and GM:
\begin{verbatim}
ID   proc0 last stride
#COMPONENTMAP
IM    0       0    1
IE    1       2    1
UA    0      31    1
GM    3     999    1
\end{verbatim}

\subsubsection{Using OpenMP threads}

Some of the models, such as \BATSRUS, can use OpenMP threads
in addition to the MPI paralelization. Typically one should
run one OpenMP thread on each core, and the number of MPI
processes should be 1 or 2 (or possibly more) for each node.
The most efficient arrangement depends on the hardware architecture
and the model. The number of maximum threads MaxThread is set by the
environment variable OMP\_NUM\_THREADS. Typically one wants
to use nThread=MaxThread threads for the components that can use
OpenMP. This can be easily achieved by setting the stried and the
number of threads in the last (optional) column to -1:
\begin{verbatim}
ID   proc0 last stride nthread
#COMPONENTMAP
GM    0      -1    -1      -1
\end{verbatim}
For example, if the node has 56 cores split to two independent
slots, the optimal setting is likely to be OMP\_NUM\_THREADS=28.
In this case both stride and nthread will be 28.

If OMP\_NUM\_THREADS is not in advance, it is best to set
the root of the multithreaded component to proc0=0, so
that the stride is properly aligned with the cores of
the nodes. This means that other components that can
only use a fixed number of processors should be put
to the last processors, for example
\begin{verbatim}
ID   proc0 last stride nthread
#COMPONENTMAP
GM    0      -3    -1      -1
IE   -2      -1     1
\end{verbatim}
In this layout GM is running with multiple threads on
cores 0 to $N-3$, while IE is using cores $N-2$ and $N-1$.

\subsection{Steady State vs. Time Accurate Execution}

The SWMF can run both in time accurate (default) and
steady state mode. This sounds surprising first, 
since many of the components can run in time accurate 
mode only. Nevertheless, the SWMF can improve the convergence
towards a steady state by allowing the different components
to run at different speeds in terms of the physical time.
In \BATSRUS\ the same idea is used on a much smaller scale:
local time stepping allows each cell to converge towards
steady state as fast as possible, limited only by the local
numerical stability limit.

\subsubsection{Steady state session}

The steady state mode should be signaled with the 
\begin{verbatim}
#TIMEACCURATE
F                DoTimeAccurate
\end{verbatim}
command, usually placed somewhere at the beginning of the session.
Since the SWMF runs in time accurate mode by default,
this command is required in the first steady state session of the run.

When SWMF runs in steady state mode, the SWMF time is not
advanced and tSimulation usually keeps its default initial value,
which is zero. 
The components may or may not advance their own
internal times. The execution is controlled by the 
step number {\tt nStep}, which goes from its initial value 
to the final step allowed by the MaxIteration parameter
of the \#STOP command. The components are called at
the frequency defined by the \#CYCLE command. For example
\begin{verbatim}
#CYCLE
GM               NameComp
1                DnRun

#CYCLE
IM               NameComp
2                DnRun
\end{verbatim}
means that IM runs in every second time step of the SWMF.
By defining the DnRun parameter for all the components,
an arbitrary relative calling frequency can be obtained,
which can optimize the global convergence rate to steady state.
The default frequency is DnRun=1, i.e. the component is
run in every SWMF time step. 

The relative frequency can be important for numerical
stability too. When GM and IM are to be relaxed
to a steady state, the GM/BATSRUS code is running in 
local time stepping mode, while IM/RCM runs in time 
accurate mode internally. Since GM and IM are coupled
both ways, an instability can occur if both GM and IM
are run every time step, because the GM physical time
step is very small, and the MHD solution cannot relax
while being continuously pushed by the IM coupling.
This unphysical instability can be avoided by calling the
IM component less frequently.

The coupling frequencies should be set to be optimal
for reaching the steady state. If the components are
coupled too frequently, a lot of CPU time is spent
on the couplings. If they are coupled very infrequently,
the solution may become oscillatory instead of relaxing
into a (quasi-)steady state solution. For example
we used the
\begin{verbatim}
#COUPLE2
IM                      NameComp1
GM                      NameComp2
10                      DnCouple
-1.                     DtCouple
\end{verbatim}
command to couple the GM and IM components in both directions
in every 10-th SWMF iteration.
Note that according to the above \#CYCLE commands,
GM and IM do 10 and 5 steps between two couplings,
respectively. GM/BATSRUS uses 10 local time steps,
while IM advances by 5 five-second time steps.

Another example is the relaxation of SC and IH components.
Under usual conditions the solar wind is supersonic at the 
inner boundary of the IH component, thus the steady state SC
solution can be obtained first, and then IH can converge
to a steady state using the SC solution as the inner boundary 
condition. In this second stage SC does not need to run
(assuming that it has reached a good steady state solution),
it is only needed for providing the inner boundary condition for IH.
This can be achieved by
\begin{verbatim}
! No need to run SC too often, it is already in steady state
#CYCLE
SC                      NameComp
1000                    DnRun

! No need to couple SC to IH too often
#COUPLE1
SC                      NameSource
IH                      NameTarget
1000                    DnCouple
-1.0                    DtCouple
\end{verbatim}
Since SC and IH are always coupled at the beginning of the session,
further couplings are not necessary.

\subsubsection{Time accurate session}

The SWMF runs in time accurate mode by default. The
\begin{verbatim}
#TIMEACCURATE
T                       DoTimeAccurate
\end{verbatim}
command is only needed in a time accurate session following a 
steady state session.
In time accurate mode the components advance in time at
approximately the same rate. The component times are
only synchronized when necessary, i.e. when they
are coupled, when restart files are written, or 
at the end of session and execution. Since the time
steps (in terms of physical and/or CPU time) of the components can be 
vastly different, this minimal synchronization provides the 
most possibilities for efficient concurrent execution.

In time accurate mode the coupling times have to be defined
with the DtCouple arguments. For example
\begin{verbatim}
#COUPLE2
GM                      NameComp1
IM                      NameComp2
-1                      DnCouple
10.0                    DtCouple
\end{verbatim}
will couple the GM and IM components every 10 seconds. 

In some cases the models have to be coupled every time step.
An example is the coupling between the MHD model GM/BATSRUS and 
the Particle-in-Cell model PC/IPIC3D. This can be achieved with
\begin{verbatim}
#COUPLE2TIGHT
GM                      NameMaster
PC                      NameSlave
T                       DoCouple
\end{verbatim}
command. In this case the master component (GM) tells the slave
component (PC) the time step to be used. The tight 
coupling requires models and couplers that support this option.

By default the component time steps are limited by the
time of couplings. This means that if GM can take 4 second
times steps, and it is coupled with IE every 5 seconds,
then every second GM time step will be truncated to 1 second.
There are two ways to avoid this. One is to choose the
coupling frequencies to be round multiples of the time steps
of the two components involved. This works well if both components
have fixed time steps and/or much smaller time steps than the 
coupling frequency.

In certain cases the efficiency can be improved with the
\#COUPLETIME command, which can allow a component to 
step through the coupling time. For example
\begin{verbatim}
#COUPLETIME
GM                      NameComp
F                       DoCoupleOnTime
\end{verbatim}
will allow the GM component to use 4 second time steps even
if it is coupled at every 5 seconds. Of course this will
make the data transferred during the coupling be 
first order accurate in time.

\subsection{Coupling order}

The default coupling order is usually optimal for accuracy
and consistency, but it may not be optimal for speed.
In particular, the IE/Ridley\_serial component solves
a Poisson type equation for the data received from the 
other components (GM and UA). For sake of accuracy
IE always uses the latest data received from the other
components. If GM, UA and IE are coupled
in the default order
\begin{verbatim}
#COUPLEORDER
4             nCouple	  
GM IE         NameSourceTarget
UA IE         NameSourceTarget
IE UA         NameSourceTarget
IE GM         NameSourceTarget
\end{verbatim}
and the to-IE and from-IE coupling times coincide, e.g.
\begin{verbatim}
#COUPLE2
GM            NameComp1
IE            NameComp2
10.0          DtCouple
-1            DnCouple

#COUPLE2
UA            NameComp1
IE            NameComp2
10.0          DtCouple
-1            DnCouple
\end{verbatim}
then GM and UA will have to wait until IE solves
the Poisson equation, because IE receives new data
and it is required to produce results immediately.
With the reversed coupling order
\begin{verbatim}
#COUPLEORDER
4             nCouple	  
IE UA	      NameSourceTarget
IE GM	      NameSourceTarget
GM IE	      NameSourceTarget
UA IE	      NameSourceTarget
\end{verbatim}
IE will provide the solution from the previously received data,
and it will have time to work on the new data while GM and UA
are working on their time steps. The reversed coupling order
allows the concurrent execution of IE with other components.
The temporal accuracy, on the other hand, will be somewhat worse.

To demonstrate that the coupling order is important, here
is a very {\bf inefficient} coupling order
\begin{verbatim}
#COUPLEORDER
4             nCouple	  
GM IE         NameSourceTarget
IE GM         NameSourceTarget
UA IE         NameSourceTarget
IE UA         NameSourceTarget
\end{verbatim}
in case the coupling times with GM and UA coincide (always at the beginning
of a the sessions).
With this coupling order, IE first receives information from GM,
then solves the Poisson equation and returns the information based
on the solution to GM while GM is waiting. Then IE receives extra
information from UA, solves the Poisson equation again, and sends
back information to UA, while UA is waiting. 

An alternative way to achieve concurrent execution is to
stagger the coupling times. For example the
\begin{verbatim}
#COUPLE2SHIFT
GM                 NameComp1
IE                 NameComp2
-1                 DnCouple
10.0               DtCouple
-1                 nNext12
0.0                tNext12
-1                 nNext21
5.0                tNext21
\end{verbatim}
will schedule a GM to IE coupling at 0, 10, 20, 30, \ldots seconds,
and the IE to GM coupling at 5, 15, 25, \ldots seconds.
This provides IE half the GM time to solve the Poisson equations.
If IE runs at least twice as fast as GM, this solution will
produce concurrent execution. The temporal accuracy is
somewhat better than in the reversed coupling case.
Note that GM and IE will be synchronized at 0, 5, 10, \ldots seconds,
which works best if the GM time step is an integer fraction of 5 seconds.


% Example runs
%\chapter{Example Runs \label{chapter:examples}}
%The examples in this chapter are intended to make you familiar
with the use of the SWMF. By carefully following the steps you should
be able to do the tests as described. It is a good idea to read the
provided PARAM.in and LAYOUT.in files and try to understand how
the examples work. You may also experiment by changing these files
after copied from the originals. These examples should help you
in setting up your own runs.

For sake of simplicity we describe how to do all the example runs with the
same executable. In actual runs one would streamline the configuration
of the SWMF to reduce compilation time and memory use. If you work
on a machine with limited resources, you may wish to configure SWMF
differently. For example, you may set the Empty version for the unused
components. If the number of processors is limited, you will have to
change the LAYOUT.in file and overlap the components. You may also have to 
increase the allowed grid size per processor, or you can run the problem with 
a coarser grid resolution. To reduce the CPU time, you may shorten the 
run by changing the number of iterations or the final time in the 
PARAM.in file.

\section{Configuration and Compilation for the Examples}

Select the SC/BATSRUS component version. When this is done the first time
after installation, the SC/BATSRUS source code is created from the GM/BATSRUS
source code, which takes a few minutes:
\begin{verbatim}
Config.pl -v=SC/BATSRUS -g=SC:4,4,4,1000
\end{verbatim}
Set the grid size for the GM/BATSRUS and IH/BATSRUS\_share components:
\begin{verbatim}
Config.pl -g=GM:8,8,8,200,40
\end{verbatim}
You can select either the UA/GITM version with
\begin{verbatim}
Config.pl -v=UA/GITM -g=UA:9,9,25,2,2
\end{verbatim}
or the new (default) UA/GITM2 version with
\begin{verbatim}
Config.pl -v=UA/GITM2 -g=UA:9,9,25,4
\end{verbatim}
Note the difference in the grid size parameters.
You should also take care of using the parameter files appropriate for
the selected UA component version.  GITM2 is the default UA module
currently in the SWMF and is the most up to date.

Check the current settings with
\begin{verbatim}
Config.pl -show
\end{verbatim}
You should see the current directory, the operating system, the name
of the compiler and the following settings
\begin{verbatim}
The default precision for reals is double precision.
The selected component versions and grid sizes are:
GM/BATSRUS                   grid: 8,8,8,200,40
IE/Ridley_serial             
IH/BATSRUS                   grid: 8,8,8,200,40
IM/RCM                       
RB/RiceV5                    
SC/BATSRUS                   grid: 4,4,4,1000
SP/Kota                      grid: 1000,10,150
UA/GITM2                     grid: 9,9,25,4
\end{verbatim}
In case you selected the UA/GITM version, the last line will read
\begin{verbatim}
UA/GITM                      grid: 9,9,25,2,2
\end{verbatim}
If the settings differ you can change them with the Config.pl script.

Compile the main executable code bin/SWMF.exe. With the above
settings this may take an hour, but if you select Empty versions
for the unused components, it will take much less.
You should also compile the post processing code bin/PostIDL.exe,
and finally create a run directory and change to that directory
\begin{verbatim}
make
make PIDL
make rundir
cd run
\end{verbatim}
Note that the run directory contains subdirectories for all the
non-empty components. There is also a link to the Param directory
in the main SWMF directory. The Param directory contains all the
parameter and layout files used in the example runs.

\section{Example 1: Create Steady State for the Solar Corona}

This example involves the SC component only. It demonstrates
how a steady state solar corona can be obtained from
a magnetogram. The convergence to steady state is accelerated by 
a gradual grid refinement and a gradual application of the
more-and-more accurate numerical schemes. The final state
is a steady state to high accuracy. 

You can use the SWMF with the settings recommended above
but you only need the SC component and
all other component versions can be Empty.

Copy the PARAM.in and LAYOUT.in files
\begin{verbatim}
rm -f PARAM.in LAYOUT.in
cp Param/PARAM.in.test.start.SC.AMR8 PARAM.in
cp Param/LAYOUT.in.test.start.SC LAYOUT.in
\end{verbatim}
We recommend the AMR8 version of the parameter file because it creates
a smaller grid and reaches steady state in smaller number of iterations
than the higher resolution version (AMR9). For the milestone problem
we used the higher resolution.

Check the parameter and layout files with the TestParam.pl script. 
This script should be run from the main directory.
Define the number of CPU-s you plan to use with the -n=NUMBER flag.
For example
\begin{verbatim}
cd ..
Scripts/TestParam.pl -n=32
\end{verbatim}
The script should return without any warnings and error messages.
If there are error messages, fix them. 
For example if the grid is not large enough there will be an 
error message from the command \#CHECKGRIDSIZE with respect
to the parameter  MinBlockALL, which contains the minimum number
of blocks required. To fix the problem, you have to increase the number 
of CPU-s or increase the grid size for the SC component with the 
Config.pl script. For example set
\begin{verbatim}
Config.pl -g=SC:4,4,4,1500
\end{verbatim}
Keep fixing the problems until the Scripts/TestParam.pl runs silently.
Remember to recompile SWMF.exe if you change the grid size.

Run the code by submitting a job or do it interactively
\begin{verbatim}
cd run
mpirun -np 32 SWMF.exe | tee runlog.SC
\end{verbatim}
Here 'tee' is a Unix command which splits the output to the screen as
well as pipes it to the file 'runlog.SC'. 

Depending on the number of processors and the speed of the machine,
this run should take a few to several hours to complete.
You may check the progress on the screen, or look at the 
runlog.SC file, or look at the log file written by the code
\begin{verbatim}
tail SC/IO2/log*
\end{verbatim}
This file contains information determined by the \#SAVELOGFILE command
in the PARAM.in file. In this case the file contains the time step, the
time (which is zero, because this is a steady state run) and the
magnetic, kinetic and thermal energies integrated over the surface of 
two spheres of radii 4 and 10. As the code approaches steady state,
the integrated energies will change less and less. 

When the run finishes successfully (or even while the code is running), 
you can postprocess the plot files
\begin{verbatim}
cd SC
pIDL
pTEC p r
cd ..
\end{verbatim}
The 'p' and 'r' flags for the pTEC script mean that the raw ASCII data
files are preprocessed with preplot into .plt files and the ASCII data
files are removed. If the machine does not have the 'preplot' code
installed (preplot is a script that comes with the Tecplot software),
you can gzip the .dat files to save disk space and transfer time
\begin{verbatim}
pTEC g
\end{verbatim}
You can visualize the output files in run/SC/IO2 with your favored 
visualization software. For visualization with IDL you should read the
chapter on ``IDL visualization'' in the user manual in GM/BATSRUS/Doc.

The restart files were saved into RESTART.SC and the SC/restartIN directory.
These are not the default names, they were set in the PARAM.in file with
the \#RESTARTFILE command for the control module of the SWMF and 
with the \#RESTARTOUTDIR command for the SC/BATSRUS component.
This makes it easier to use the run in the following example run.

The alternative and more typical approach would be to write into the
default restart file RESTART.out and the default directory SC/restartOUT,
and use the Restart.pl script to create a restart tree, e.g.
\begin{verbatim}
Restart.pl RESTART_SC
\end{verbatim}

\section{Example 2: Create SC-IH Steady State}

This example run is built on the previous example. We restart SC from the
steady state created in the previous run and start the IH component from 
scratch and run the two components coupled until the IH component reaches
a steady state. 

First copy in the prepared parameter and layout files
\begin{verbatim}
rm -f PARAM.in LAYOUT.in
cp Param/PARAM.in.test.restart.SCIH  PARAM.in
cp Param/LAYOUT.in.test.restart.SCIH LAYOUT.in
\end{verbatim}
Look at the PARAM.in file to see how the convergence to 
steady state is accelerated.
First of all the SC component only provides the boundary conditions for IH,
so it only runs in every 100th iteration (see the \#CYCLE command in
the PARAM.in file). Second, the IH grid is built
up with a gradual grid refinement, and third the 
more-and-more accurate and expensive numerical schemes are 
applied in an optimal sequence. The final state
is a steady state for SC and IH to high accuracy. 

You can use the SWMF with the settings recommended for all the examples,
but you only need the SC and IH components so 
all other component versions can be Empty.

As usual, check the input parameters and the layout for the
number of CPU-s you plan to use, for example
\begin{verbatim}
cd ..
Scripts/TestParam.pl -n=32
\end{verbatim}
If there are error messages, fix them until the script runs silently.

Run the code by submitting a job, or interactively
\begin{verbatim}
cd run
mpirun -np 32 SWMF.exe | tee runlog.SCIH
\end{verbatim}
When the run finishes, postprocess the plot files for both components
\begin{verbatim}
cd SC
pIDL
pTEC
cd ../IH
pIDL
pTEC p r
cd ..
\end{verbatim}
Due to the commands \#RESTARTFILE and \#RESTARTOUTDIR (in the IH section)
the output restart file is saved into RESTART.SCIH and the IH component
saved the restart state into the IH/restartIN. The SC component saved
the restart state into the default SC/restartOUT. For a continuation run
one could move and link the directories in SC with the following Unix commands
\begin{verbatim}
cd SC
mv restartIN restart_SConly
mv restartOUT restart_SCIH
mkdir restartOUT
ln -s  restart_SCIH restartIN
cd ..
\end{verbatim}
As you can see this is a rather cumbersome and error prone procedure.
It is better to save the restart state into their default file and
directories and use the Restart.pl script to collect them into 
a restart tree and to link the input file and directories to them.

\section{Example 3: Create Initial Conditions for the Global Magnetosphere}

This example involves the GM component only. It demonstrates
how a reasonable global magnetosphere can be obtained.
The upwind boundary conditions are based on the solution obtained
for the IH component in the 2nd example run, but they
are intentionally modified to contain a discontinuity.
This is for demonstration purposes only to make a subsequent
GM-IH coupled run more dynamic. In a more typical run one would
use the upwind boundary condition based on satellite measurements 
or by coupling to the IH code.

The convergence to steady state is accelerated by 
a gradual grid refinement, and a gradual application of the
more-and-more accurate numerical schemes. The final state
is a reasonable global magnetosphere solution, but it is not 
a perfect steady state (that would require many more iterations).

First copy in the prepared parameter and layout files and check them
\begin{verbatim}
rm -f PARAM.in LAYOUT.in
cp Param/PARAM.in.test.start.GM  PARAM.in
cp Param/LAYOUT.in.test.start.GM LAYOUT.in
\end{verbatim}
The parameters of the \#SOLARWIND command were obtained
from the result of the SC-IH steady state. 
Otherwise this test can be run independently, 
no restart files are read.
You can use the SWMF with the settings recommended for all the tests, 
but you only need the GM component, so all other component versions 
can be set to Empty.

Check the parameters and layout before running the SWMF
\begin{verbatim}
cd ..
Scripts/TestParam.pl -n=16
\end{verbatim}
If there are error messages, fix them until the script reports no errors.
Run the code by submitting a job, or interactively
\begin{verbatim}
cd run
mpirun -np 16 SWMF.exe | tee runlog.GM
\end{verbatim}
Postprocess the plot files
\begin{verbatim}
cd GM
pIDL
pTEC p r
cd ..
\end{verbatim}
The output restart file is RESTART.GM and the GM component saved
its restart files into GM/restart.IN as determined by the PARAM.in file.

\section{Example 4: Create Initial Conditions for the GM, IM, IE 
         and UA Components}

This example involves the GM, IM, IE and UA components. It demonstrates
how to obtain a reasonable solution for the 
coupled global/inner magnetosphere, ionosphere and upper atmosphere.
The initial global magnetosphere solution is taken from Example 3.

Copy the layout file into the run directory
\begin{verbatim}
rm -f PARAM.in LAYOUT.in
cp Param/LAYOUT.in.test.restart.GMIMIEUA LAYOUT.in
\end{verbatim}
If you are using the UA/GITM2 component version, copy this parameter file:
\begin{verbatim}
cp Param/PARAM.in.test.restart.GMIMIEUA.GITM2 PARAM.in
\end{verbatim}
If you are using the UA/GITM component version, copy this parameter file:
\begin{verbatim}
cp Param/PARAM.in.test.restart.GMIMIEUA PARAM.in
\end{verbatim}
Have a look at the PARAM.in file.
The convergence to steady state is accelerated by component subcycling
(see the \#CYCLE commands in PARAM.in).
The components are called at different frequencies, which
allows the components to reach a reasonable solution
approximately at the same rate. Another optimization trick is the changing of 
the coupling order such that the IE component sends information before
it would receive it (see the \#COUPLEORDER command in PARAM.in),
so that IE component can solve for the electric potential concurrently with the
other components running. To reduce the time spent on field line
tracing, which is needed for the IM to GM coupling, the \#RAYTRACE command
in the GM section limits the frequency of traces to every 80th iteration.

Now have a look at the LAYOUT file:
\begin{verbatim}
#COMPONENTMAP
UA   0    31    1
IM   32   32    1
IE   33   34    1
GM   35  999    1
#END
\end{verbatim}
As you can see all the components are running concurrently.
Unless you have access to at least 64 processors, 
you probably want to modify the layout file.
For example for 32 processors you can overlap the UA component 
with the other components and run IE on a single processor:
\begin{verbatim}
#COMPONENTMAP
UA    0   31    1
IE    0    0    1
IM    1    1    1
GM    2   31    1
#END
\end{verbatim}
You can use the SWMF with the settings recommended for all the tests, 
but you only need the GM, IM, IE and UA components,
all other component versions can be Empty.

Check the parameters and layout for the number of processors you plan to use
\begin{verbatim}
cd ..
Scripts/TestParam.pl -n=32
\end{verbatim}
If there are error messages, fix them. 
Run the code by submitting a job, or interactively
\begin{verbatim}
cd run
mpirun -np 32 SWMF.exe | tee runlog.GMIMIEUA
\end{verbatim}
This run should take an hour or so on 32 processors.
After the run postprocess the plot files
\begin{verbatim}
cd GM
pIDL
pTEC p r
cd ../IE
pION
cd ..
\end{verbatim}
The pION script concatenates the northern and southern output files
produced by the component IE/Ridley\_serial.

In this run all output restart information is saved into the
default file (RESTART.out) and default directories.
You can use the RESTART.pl script to collect them into a restart tree.
For example
\begin{verbatim}
Restart.pl -o RESTART_GMIMIEUA
\end{verbatim}

\section{Example 5: Time Accurate Run with SC, IH and SP Components}

This example involves the Solar Corona, Inner Heliosphere
and Solar Energetic Particles components.
The run starts from a steady state of the SC and IH components
which was obtained in Example 2.

Copy in the prepared parameter and layout files
\begin{verbatim}
rm -f PARAM.in LAYOUT.in
cp Param/PARAM.in.test.restart.SCIHSP PARAM.in
cp Param/LAYOUT.in.test.restart.SCIHSP LAYOUT.in
\end{verbatim}
At the beginning of this time accurate run a CME is 
generated in the SC component. This demonstrates the
use of an Eruptive Event generator.
The SP component is coupled, and during the 4 hour evolution
of the CME, substantial particle acceleration is observed. 
By the end of the run the CME reaches the boundary 
between the SC and IH components, and partially enters the IH domain.
The test demonstrates the coupling between the SC-IH and SP
components in a challenging simulation.

In case you have limited computational resources, you can 
shorten the run by editing the PARAM.in file and changing
the \#STOP command at the end of the file. For example 
\begin{verbatim}
#STOP
-1                     MaxIteration
1800.0                 tSimulationMax
\end{verbatim}
will reduce the final simulation time to 30 minutes.

Also have a look at the layout file:
\begin{verbatim}
#COMPONENTMAP
SC    1 999    1
IH    1 999    1
SP    0   0    1 
#END
\end{verbatim}
The SC and IH components are overlapped while the SP component
runs concurrently. This is probably the optimal layout for any
number of CPU-s.

You can use the SWMF with the settings for all the tests, 
but you only need the SC, IH and SP components,
all other component versions can be Empty.

Check the parameters and layout before running the SWMF
\begin{verbatim}
cd ..
Scripts/TestParam.pl -n=32
\end{verbatim}
If there are error messages, fix them, then run the SWMF by submitting a job
or interactively
\begin{verbatim}
cd run
mpirun -np 32 SWMF.exe | tee runlog.SCIHSP
\end{verbatim}
This run may take a long time if run all the way to 4 hours.
After the run finishes, postprocess the plot files
\begin{verbatim}
cd SC
pIDL
pTEC p r
cd ../IH
pIDL
pTEC p r
cd ..
\end{verbatim}
The restart files can be collected into a restart tree 
with the Restart.pl script. For the full 4 hour run you could use
\begin{verbatim}
Restart.pl -o RESTART_SCIHSP_4hr
\end{verbatim}

\section{Example 6: Time Accurate Run with All Eight Components}

This run is described in the TESTING manual. 
The initial conditions were created with the runs described in
this chapter using the higher resolution SC grid (AMR9) and the full 4 hour
time accurate run with the SC-IH and SP components.

For sake of learning SWMF you can do this example with lower SC resolution
(AMR8) and do a shorter time accurate run with the SC, IH and SP components.

Copy the layout file into the run directory
\begin{verbatim}
rm -f PARAM.in LAYOUT.in
cp Param/LAYOUT.in.test.restart.8comp LAYOUT.in
\end{verbatim}
If you are using the UA/GITM2 component version, copy this parameter file:
\begin{verbatim}
cp Param/PARAM.in.test.restart.8comp.GITM2 PARAM.in
\end{verbatim}
If you are using the UA/GITM component version, copy this parameter file:
\begin{verbatim}
cp Param/PARAM.in.test.restart.8comp PARAM.in
\end{verbatim}
This example is desinged to run for 600 seconds only.
It is assumed that the input restart file RESTART.in contains
the simulation time corresponding to 4 hours (14400 seconds),
so the final time in the \#STOP command is 15000 seconds
in the PARAM.in file.
In case you are restarting from a different simulation time and/or
you wish to change the length of the run, edit the maximum
simulation time in the \#STOP command in the PARAM.in file.

To simplify this example run, you can remove some of the components.
To remove a component you need to edit the LAYOUT.in file and remove the
appropriate line, and edit the PARAM.in file, and remove the 
section belonging to the component and all the couplings with the 
component. In fact, in some cases it may not be necessary to remove 
anything from the PARAM.in file. You may simply add an \#END command
after the section belonging to the last included component,
since the lines following the \#END command are ignored.
For example to run with the SC and IH components only, you can put an
\#END command after the \#END\_COMP IH command but before the coupling
with the SP component:
\begin{verbatim}
#END_COMP IH ---------------------------------------

#END

#COUPLE1
SC                      NameSource
SP                      NameTarget
-1                      DnCouple
60.0                    DtCouple
...

\end{verbatim}
Test the parameters and the layout as usual
\begin{verbatim}
cd ..
Scripts/TestParam.pl -n=64
\end{verbatim}
and run the code
\begin{verbatim}
cd run
mpirun -np 32 SWMF.exe | tee runlog.8comp
\end{verbatim}
After the run finishes, postprocess the plot files
\begin{verbatim}
cd SC
pIDL
pTEC p r
cd ../IH
pIDL
pTEC p r
cd ..
cd ../GM
pIDL
pTEC p r
cd ../IE
pION
cd ..
\end{verbatim}
Since this example run is very short, no restart files are saved
(see the \#SAVERESTART command).


% Complete List of Input Commands
\chapter{Complete List of Input Commands \label{chapter:commands}}

The content of this chapter is generated from the PARAM.XML file
of the CON module and the PARAM.XML files
in the component version directories (e.g. UA/GITM2/PARAM.XML). 
These XML files can be read with an editor and they can be used
for creating PARAM.in files by copying small parts from them.

The XML files have been written by several developers at the
Center of Space Environment Modeling. The transformation of the XML
format into LaTex is done with the share/Scripts/XmlToTex.pl
script. This script generates index terms for all commands,
which are used to create an alphabetical index at the end of this chapter.

\input{SWMFXML}

\newpage
\addcontentsline{toc}{chapter}{{\bf Index of Input Commands}}
% Index for the commands
%^CFG COPYRIGHT UM
\documentclass[twoside,10pt]{book}

\input HEADER

\title{Space Weather Modeling Framework User Manual \\ 
       \hfill \\
       \large Code Version \SWMFVERSION}
\author{G\'abor T\'oth, Ovsei Volberg, Aaron Ridley\\
       \hfill \\
       {\it Center for Space Environment Modeling}\\
       {\it The University of Michigan}}

\makeindex

\begin{document}

\pagestyle{fancy}
\lhead[\fancyplain{}{\bfseries\thepage}]{\fancyplain{}{\bfseries\rightmark}}
\rhead[\fancyplain{}{\bfseries\leftmark}]{\fancyplain{}{\bfseries\thepage}}
\cfoot{}
%\pagestyle{headings} % if fancy heading does not work

\maketitle

%\noindent COPYRIGHT (c) 1996 \\
THE REGENTS OF THE UNIVERSITY OF MICHIGAN \\
ALL RIGHTS RESERVED \\

PERMISSION IS GRANTED TO A SINGLE USER TO USE THIS SOFTWARE FOR
NONCOMMERCIAL EDUCATION AND RESEARCH PURPOSES.  THE USER IS GIVEN
PERMISSION TO INSTALL AND RUN THE SOFTWARE ON MULTIPLE PLATFORMS.  NO
PERMISSION IS GIVEN TO COPY OR REDISTRIBUTE THIS SOFTWARE IN ANY FORM
INCLUDING, BUT NOT LIMITED TO, COPYING SECTIONS OF THE SOURCE CODE OR
DOCUMENTATION AND DISTRIBUTING THE CODE TO OTHER USERS.  ALTHOUGH
SOURCE CODE IS DISTRIBUTED, USERS ARE RESTRICTED FROM MODIFYING THE
SOURCE CODE IN ANYWAY WITHOUT PRIOR CONSENT OF THE CENTER FOR SPACE
ENVIRONMENT MODELING AT THE UNIVERSITY OF MIGHIGAN.

THIS SOFTWARE IS PROVIDED AS IS, WITHOUT REPRESENTATION AS TO ITS
FITNESS FOR ANY PURPOSE, AND WITHOUT WARRANTY OF ANY KIND, EITHER
EXPRESS OR IMPLIED, INCLUDING WITHOUT LIMITATION THE IMPLIED WARRANTIES
OF MERCHANTABILITY AND FITNESS FOR A PARTICULAR PURPOSE. THE REGENTS OF
THE UNIVERSITY OF MICHIGAN SHALL NOT BE LIABLE FOR ANY DAMAGES,
INCLUDING SPECIAL, INDIRECT, INCIDENTAL, OR CONSEQUENTIAL DAMAGES, WITH
RESPECT TO ANY CLAIM ARISING OUT OF OR IN CONNECTION WITH THE USE OF
THE SOFTWARE, EVEN IF IT HAS BEEN OR IS HEREAFTER ADVISED OF THE
POSSIBILITY OF SUCH DAMAGES.




\tableofcontents

% Introduction
%\documentclass[a4paper,11pt]{article}
%\author{\bf Center for Space Environment Modeling, The University of Michigan}
%\title{\bf \Large Release Notes for the Milestone 7I and Reference Manual}
%\maketitle



\chapter{Introduction}

This document describes a working prototype of the NASA-funded Space
Weather Modeling Framework (SWMF) delivered to NASA to fulfill the
Milestone 11K requirements. The SWMF was developed to provide flexible
``plug and play" type simulation capabilities serving the Sun-Earth
modeling community.  In its current form the SWMF links together eight
models from the surface of the Sun to the upper atmosphere of the Earth: 
\begin{enumerate}
\item SC -- Solar Corona which includes the Eruptive Event Generator,
\item IH -- Inner Heliosphere
\item SP -- Solar Energetic Particles 
\item GM -- Global Magnetosphere 
\item IM -- Inner Magnetosphere
\item RB -- Radiation Belts
\item IE -- Ionosphere Electrodynamics
\item UA -- Upper Atmosphere
\end{enumerate}
In the future the SWMF may be extended to include even more 
physics domains: Cometary Environment, Interstellar
Neutrals, Outer Heliosphere, Plasmasphere, Planetary Satellites and
Polar Wind. 

The SWMF implementation is based on the component technology and
Object-Oriented Programming emulated in Fortran 90.  The SWMF parallel
communications are based on the MPI standard.  In its current
implementation the SWMF creates a single executable.

\section{Acknowledgments}

The SWMF was developed at the Center for Space Environment Modeling
(CSEM) of the University of Michigan under the NASA Earth Science
Technology Office (ESTO) Computational Technologies (CT) Project (NASA
CAN NCC5-614). The project is entitled as ``A High-Performance
Adaptive Simulation Framework for Space-Weather Modeling (SWMF)''.
The Project Director is Professor Tamas Gombosi, and the Co-Principal
Investigators are Professors Quentin Stout and Kenneth Powell.

The SWMF and many of the physics components were developed at CSEM
by the following individuals (in alphabetical order):
David Chesney,
Darren DeZeeuw, Tamas Gombosi, Kenneth Hansen, Kevin Kane, Ward (Chip)
Manchester, Robert Oehmke, Kenneth Powell, Aaron Ridley, Ilia Roussev,
Quentin Stout, Igor Sokolov, G\'abor T\'oth and Ovsei Volberg.

The core design and code development was done by G\'abor
T\'oth, Igor Sokolov and Ovsei Volberg:
\begin{itemize}
\item Component registration and layout was designed and implemented by 
      Ovsei Volberg and G\'abor T\'oth.
\item The session and time management support was designed and
      developed by G\'abor T\'oth.
\item The SWMF coupling toolkit was developed by Igor Sokolov.
\end{itemize}
The physics models were developed by the following research groups:
\begin{itemize}
\item
The Solar Corona (SC), Inner Heliosphere (IH) and the Global Magnetosphere 
(GM) components are based on \BATSRUS\ MHD code developed at CSEM. 
\BATSRUS\ is a 3-dimensional block-adaptive Cartesian code which is 
highly parallel.

\item
The Solar Energetic Particle (SP) component is the
K\'ota's SEP model which was developed at the University of Arizona.
It solves the equations for the advection and acceleration of
energetic particles along a magnetic field line in a 3D phase space.

\item
The Inner Magnetosphere (IM) component is the Rice Convection Model
(RCM) developed at Rice University.  This code is 2-dimensional and
serial.

\item
The Radiation Belt (RB) component is the Rice RBM
developed at Rice University.  This code is 2-dimensional and
serial.

\item
The Ionospheric Electrodynamics (IE) component is a 2-processor,
2-dimensional spherical electric potential solver developed at CSEM
(termed the ``Ridley Ionosphere'').  

\item
There are two versions of the Upper Atmosphere (UA) component:
the Global Ionosphere - Thermosphere Model (GITM) and its newer
version GITM2. Both versions are 3-dimensional spherical
models developed at CSEM.  They are fully parallel.

\end{itemize}
The transformation of physics models into physics components,
the coupling of components to the SWMF and each other and
all the testing were done at CSEM.

\section{What is New in Version 2.1}

The SWMF has been developed further since the second release of
version 2.0. Here is a partial list of improvements:
\begin{itemize}
\item The documentation has been improved and split into smaller parts.
\item The SWMF contains a new version for the UA component: GITM2.
\item Restarting SWMF is made easier with the Scripts/Restart.pl script.
\item The Solar Corona can be solved in a corotating frame.
\item The coupling between the GM and IE components has been rewritten
      with simpler and more accurate algorithms.
\item The coupling between the GM and IM components has been made more robust.
\item The geometry based grid refinement can be modified easily with
      the \#GRIDRESOLUTION command in the
      parameter file for the GM,IH,SC/BATSRUS components.

\end{itemize}

\section{What is New in Version 2.0}

The SWMF has been developed extensively since the first release of
version 1.0. Here are some of the highlights:
\begin{itemize}
\item The SWMF now contains 3 new components:
      the Solar Corona (SC), the Solar Energetic Particles (SP) 
      and the Radiation Belt (RB)
\item Some components now support dynamic memory allocation which 
      reduces the total memory required by a processing element. 
\item The directory structure of the SWMF has been greatly improved and 
      reorganized. 
\item This reorganization allows components to be used as stand alone 
      physics models without any modification in the source code.  The
      stand alone version only links to a small SWMF library.
\item The installation and configuration of SWMF has been greatly simplified
      with the aid of Perl scripts. 
\item Layout and input parameter can be checked with Scripts/TestParam.pl
\item User manual is produced from the XML description of the input parameters.
\item Unused components can be configured out completely.
\item The control module now fully supports steady state calculations 
      including component subcycling. 
\item The coupling toolkit provides means for extracting and following 
      the motion of field lines.
\end{itemize}

\section{The SWMF in a Few Paragraphs}

The SWMF is a structured collection of software building blocks that
can be used or customized to develop Sun-Earth system modeling
components, and to assemble them into applications. The SWMF consists
of utilities and data structures for coupling model components. The
SWMF contains a Control Module (CON), which is responsible for
component registration, processor layout for each component and
coupling schedules.  It controls initialization and execution of the
components. A component is adapted from user-supplied physics codes,
(for example \BATSRUS\ or RCM), by adding two relatively small units
of code:
\begin{itemize}
\item A wrapper, which provides the control functions, and
\item A coupling interface to perform the data exchange with other
components.
\end{itemize}
Both the wrapper and coupling interface are constructed from the
building blocks provided by the framework. From 
component software technology perspective both the wrapper and
coupling interface are component interfaces: the wrapper is an
interface with CON, and the coupling interface is an interface with
another component. A physics
model code and its wrapper, which comprise a component, share the
communication group.  The coupling interface uses the union
communicator of the two components that it links together.

An SWMF component is compiled into a separate library that resides in
the directory {\tt lib}, which is created as part of the installation
process described later in this document.  Currently the component
libraries are static libraries. The executable image is created in the
directory {\tt bin}, which is created during the compilation.  If a
user does not want to build some particular component, this component
should be substituted by an empty version of the component.

An important feature of the SWMF is the component registration.  A
component to be included in the run should be registered by the
framework.  Currently entering the line for the component in the input
file called {\tt LAYOUT.in} does the registration.  Thus the SWMF
performs the run-time registration of components.

The framework controls the initialization, execution, coupling and
finalization of components.  The execution is done in sessions. In
each session the parameters of the framework and the components can be
changed.  The parameters are read from the {\tt PARAM.in} file, which
may contain further included parameter files.  These parameters are
read and broadcast by CON and the component specific parameters are
sent to the components. The structure of the parameter file will be
described in detail.

If two components reside on different sets of processing elements
(PE-s) they can execute in an efficient concurrent manner.
This is possible, because the coupling times are
known in advance.  The components advance to the time of coupling and
only the processors involved in the coupling need to communicate with
each other. The components are also allowed to share some processing elements.
The execution is sequential for the components with overlapping layouts.
Of course this still allows the individual components to execute in parallel.
For steady state calculations the components are allowed to progress
at different rates towards steady state. Each component can be called
at different frequencies by the control module.

The coupling of the components is realized either with plain MPI
calls, or via the SWMF coupling toolkit, which can couple components
based on the following types of parallel distributed grids:
\begin{itemize}
\item 3-D Block adaptive (AMR) parallel grid
\item 2-D Spherical grid
\item Logically Cartesian uniform grid
\item Logically Cartesian non-uniform grid 
\end{itemize}
The SWMF coupling toolkit performs an efficient N to M parallel
coupling based on a router. The router is calculated in advance using
the domain decomposition and grid description obtained from the
components.  The router is updated only when the domain decompositions
or the grids of the components change, or when the mapping geometry
changes.  The coupling toolkit takes care of linear interpolation in
space based on the grid descriptor.  Temporal interpolation is not
supported by the current implementation.

The framework has been tested on the SGI Origin 3000, SGI Altix and 
Compaq ES45 machines, and on Linux Beowulf clusters with the NAG f95 
compiler. We have also run the framework with reasonable success under
Mac OS Darwin using the XLF and NAG f95 compilers, and under Linux with
the PGF90 compiler.

\section{System Requirements}

In order to install and run the SMWF the following minimum system
requirements apply.

\begin{itemize}
\item The SWMF runs only under the UNIX/Linux operating systems.  This now
  includes Macintosh system 10.x because it is based on BSD UNIX.  The
  SWMF does not run under any Microsoft Windows operating system.
\item A FORTRAN 77 and FORTRAN 90 compiler must be installed.
\item The Perl interpreter must be installed.
\item A version of the Message Passing Interface (MPI) library must be
  installed.
\item You may be able to compile the code and do very small test
runs on 1 or 2 processor machines.  However, to do most physically
meaningful runs the SWMF requires a
parallel processor machine with a minimum of 8 processors and a minimum of 8GB of
memory.
\item Very large runs require many more processors.
\item In order to generate the documentation you must have LaTex installed on
your system.  The PDF generation requires the {\tt dvips} and {\tt ps2pdf}
utilities.  To generate the HTML version you also must install the
{\tt latex2html} package. 

\end{itemize}


In addition to the above requirements, the SWMF output is designed to
be visualized using either IDL or Tecplot.  You may be able to
visualize the output with other packages, but formats and scripts have
been designed for only these two visualization softwares.




%-----------------------------------------------------------------------
% Chapter 2
%-----------------------------------------------------------------------

\chapter{Quick Start}

\section{A Brief Description of the SWMF Distribution}

The distribution in the form of the compressed tar image
includes the SWMF source code.
The top level directory contains the following subdirectories:
\begin{itemize}
\item {\tt CON}     - the directory of the framework's main building blocks
\item {\tt GM}      - Global Magnetosphere component       %^CMP IF GM
\item {\tt IE}      - Ionosphere Electrodynamics Component %^CMP IF IE
\item {\tt IH}      - Inner Heliosphere component          %^CMP IF IH
\item {\tt IM}      - Inner Magnetosphere component        %^CMP IF IM
\item {\tt RB}      - Radiation Belt component             %^CMP IF RB
\item {\tt SC}      - Solar Corona component               %^CMP IF SC
\item {\tt SP}      - Solar Energetic Particles component  %^CMP IF SP
\item {\tt UA}      - Upper Atmosphere component           %^CMP IF UA
\item {\tt Copyrights} - copyright files
\item {\tt Param}   - description of CON parameters, parameter and layout files
\item {\tt Scripts} - shell and Perl scripts
\item {\tt bin}     - scripts for installation, configuration and testing
\item {\tt doc}     - the documentation directory %^CMP IF DOC
\item {\tt share}   - shared scripts and source code
\item {\tt util}    - general utilities such as TIMING and NOMPI
\end{itemize}
and the following files
\begin{itemize}
\item {\tt README}           - a short instruction on installation and usage
\item {\tt Makefile}         - the main makefile
\item {\tt Configure.pl}     - Perl script for configuration %^CMP IF CONFIGURE
\item {\tt Configure.options} - default configuration options %^CMP IF CONFIGURE
\item {\tt SetSWMF.pl}     - Perl script for (un)installation and configuration
\end{itemize}

\section{General Hints}

\subsubsection{Getting help with scripts and the Makefile}
Most of the Perl and shell scripts that are distributed with the SWMF
provide help which can be accessed as follows using the {\tt -h} flag.
For example, 
\begin{verbatim}
  SetSWMF.pl -h
\end{verbatim}
will provide a detailed listing of the options and capabilities of the
{\tt SetSWMF.pl} script.  In addition, you can find all the possible
targets  that can be built by typing
\begin{verbatim}
make help
\end{verbatim}

\subsubsection{Input commands: PARAM.XML}
A very useful set of files to become familiar with are the {\tt PARAM.XML}
files.  Such a file exists for the SWMF itself and for each of the
physics components.  The file for the SWMF is found at
\begin{verbatim}
Param/PARAM.XML
\end{verbatim}
while the files for the physics components are found in the component's
subdirectory.  For example, the file for the GM/BATSRUS component can
be found at
\begin{verbatim}
GM/BATSRUS/PARAM.XML
\end{verbatim}
This file contains a complete list of all input commands for the
component as well as the allowed ranges for each of the input parameters.
Although the XML format makes the files a little hard to read, they are
extremely useful.  A typical usage is to cut and paste commands out of the
PARAM.XML file into the PARAM.in file for a run.

\subsubsection{Have the working directory in your path}
In order to run executable files in the UNIX environment you must have
the current working directory either your path or in the filename you
want to execute.  In UNIX the current working directory is represented
by the period (.).  For example
\begin{verbatim} 
./SWMF.exe
\end{verbatim}
will execute the SWMF.exe program if it is in your current directory.  If you
add the `.' to your path using
\begin{verbatim}
set path = (~/bin /usr/local/mpi/bin /usr/local/bin ${path} .)
\end{verbatim}
then you can simply type
\begin{verbatim} 
SWMF.exe
\end{verbatim}

\section{Installing the Code}

The first step in installing the SWMF is untarring the distribution.
If the tar program knows about the -z flag, you can open the gzipped
tar files with a single UNIX command:
\begin{verbatim}
  tar xzf SOMETARFILE.tgz
\end{verbatim}
If the tar program does not recognize the -z flag, two steps are needed:
\begin{verbatim}
  gunzip SOMETARFILE.tgz
  tar xf SOMETARFILE.tar
\end{verbatim}
In the following descriptions the shorter form is shown, but you may
need to use the two step procedure on certain platforms.

Untar the distribution using the command:
\begin{verbatim}
  tar xzf SWMF.tgz
\end{verbatim}

Change directories into the distribution:
\begin{verbatim}
  cd SWMF
\end{verbatim}

The SWMF needs to know what architecture you are running the code on
and what FORTRAN compiler will be used.  For most platforms and compilers,
tt can figure this out all by itself, but you have to run the command:
\begin{verbatim}
  SetSWMF.pl -i
\end{verbatim}
in the main directory. This creates {\tt Makefile.def} with
the correct absolute path to the base directory and {\tt Makefile.conf}
which contains the operating system and compiler specific part of
the Makefile. If the compiler is not the default one for a given
platform (e.g. not the NAG f95 compiler for a Linux platform) then
the compiler must be specified explicitly with the {\tt -c}
flag. If the MPI header file is not the default one, it can be
specified with the {\tt -m} flag. For example on the Altix machines
SWMF should be installed as
\begin{verbatim}
  SetSWMF.pl -i -c=ifort -m=Altix
\end{verbatim}
To uninstall SWMF type
\begin{verbatim}
  SetSWMF.pl -uninstall
\end{verbatim}
If the uninstallation fails (this can happen if some makefiles are missing)
force installation with
\begin{verbatim}
  SetSWMF.pl -install
\end{verbatim}
and then try uninstalling the code again.
When SWMF is installed, its configuration can be checked with
\begin{verbatim}
  SetSWMF.pl -s
\end{verbatim}
To get a list of the available component versions type
\begin{verbatim}
  SetSWMF.pl -l
\end{verbatim}
To get a complete description of the {\tt SetSWMF.pl}  script type
\begin{verbatim}
  SetSWMF.pl -h
\end{verbatim}

\section{Creating Documentation}

The documentation for SWMF can be generated from the distribution by
the command
\begin{verbatim}
  make PDF
\end{verbatim}
which creates the user manual
\begin{verbatim}
  doc/SWMF.pdf
\end{verbatim}
and several other documents in the Adobe PDF format.  
In order for this to work you must have
LaTex installed on your system (and dvips and ps2pdf).  
An on-line version can be created by
\begin{verbatim}
  make HTML
\end{verbatim}
The HTML version is generated from the LaTex using the command {\tt
  latex2html}.  You will have to install this if it does not already exist
  on your system.
The top level HTML file is in
\begin{verbatim}
  doc/HTML/index.html
\end{verbatim}
to point at with the browser.  This html file list the different
documentation files and what they contain.  To clean the intermediate files type
\begin{verbatim}
  cd doc/Tex
  make clean
\end{verbatim}
To remove all the created documentation type
\begin{verbatim}
  cd doc/Tex
  make cleanall
\end{verbatim}

\section{Building and Running an Executable}

At compile time, the user can select which physics components should be
compiled.  
Any component not compiled will not be available for
use at run time.  The physics components can be selected with the {\tt -v} flag
of the SetSWMF.pl script. For example typing
\begin{verbatim}
  SetSWMF.pl -v=SC/BATSRUS,IH/BATSRUS,SP/Kota
  SetSWMF.pl -v=GM/Empty,RB/Empty,IM/Empty,IE/Empty,UA/Empty
\end{verbatim}
will select BATSRUS for the SC and IH components and K\'ota's model for
the SP component.
The other components are set to Empty versions, which contain empty
subroutines for compilation, but cannot be used.
The default configuration includes a working version for all components, 
which takes up more memory, but is the most general.
The only exception is SC, which requires configuration, so the 
default version is Empty for the Solar Corona component.

The grid size of several components can also be set with the {\tt -g}
flag of the {\tt SetSWMF.pl} script. For example the 
\begin{verbatim}
  SetSWMF.pl -g=GM:8,8,8,400,100
\end{verbatim}
command sets the block size for the GM component to $8\times 8\times 8$ cells, 
the maximum number of blocks per processor to 400, 
and the maximum number of implicit blocks per processor to 100.
The SetSWMF.pl script actually runs the individual GridSize.pl
scripts in the component versions. These scripts can be run directly,
and they provide more options and more verbose information than SetSWMF.pl.
For example try
\begin{verbatim}
  cd GM/BATSRUS
  GridSize.pl -s
\end{verbatim}
Compilation flags, such as the precision and optimization 
level are stored in {\tt Makefile.conf}. This file is created on
installation of the SWMF and has defaults which are appropriate for
your system architecture.  The precision of reals
can be changed to single precision (for example) by typing
\begin{verbatim}
  SetSWMF.pl -p=single
\end{verbatim}
while the compiler flags can be edited in {\tt Makefile.conf} by hand.

Before compiling SWMF it is always a good idea to check its configuration
with
\begin{verbatim}
  SetSWMF.pl -s
\end{verbatim}

{\bf IMPORTANT NOTE:
On the Altix machine at NASA Ames (columbia)
you should load the 8.0.070 version 
of the Intel Fortran compiler 
with the command
\begin{verbatim}
  module load intel-comp.8.0.070
\end{verbatim}
You may wish to insert this line into the .cshrc file
so it executes at login time. 
Selecting the correct compiler version is 
necessary both to compile and to run the code.
Therefore the above line is needed in the job scripts
as well.}

To build the executable {\bf bin/SWMF.exe}, type:
\begin{verbatim}
  make
\end{verbatim} 
Depending on the configuration, the compiler settings and the machine 
that you are compiling on, this can take from 2 to up to 30 minutes.  
In addition, you may want to make the post processing
codes (for BATSRUS only) also:
\begin{verbatim}
  make PSPH
  make PIDL
\end{verbatim} 
These two commands will create the codes {\tt bin/PostSPH.exe}, for post
processing spherical Tecplot files, and {\tt bin/PostIDL.exe} 
for post processing IDL files.

The {\tt SWMF.exe} executable should be run in a sub-directory, since a large number
of files are created in each run.  To create this directory use the
command:
\begin{verbatim}
  make rundir
\end{verbatim} 
This command creates a directory called {\tt run}.  You can either
leave this directory as named, or {\tt mv} it to a different name.  It
is best to leave it in the same SWMF directory, since
keeping track of the code version associated with each run is quite
important.  The {\tt run} directory will contain links to the codes
which were created in the previous step as well as subdirectories
where input and output of the different components will reside.

Here we assume that the {\tt run} directory is still called {\tt
run}:
\begin{verbatim}
  cd run
\end{verbatim}
In order to run the SWMF you must have two input files:  LAYOUT.in and
PARAM.in.  The LAYOUT.in file defines the processor
layout for the components involved in the future run.  The PARAM.in
file contains the detailed commands for controlling what you want the
code to do during the run.  The default LAYOUT.in and PARAM.in
files in the run directory are suitable to perform the ``Start'' test
on 16 processors (PE-s). 

An example processor map file LAYOUT.in to run the executable with
five components on 16 processors is:
\begin{verbatim}
#COMPONENTMAP
GM    0    4    1
IE    5    6    1
IH    7   10    1
IM   11   11    1
UA   12   15    1
#END
\end{verbatim}
The file syntax is simple. It must start with the directive
\#COMPONENTMAP and end with another directive \#END. Each line between
these directives specifies the label for component, i.e. IE, GM and
etc., its first and last processor, all relatively to the world
communicator, and the stride. Thus GM will run on 5 processors from 0
to 4, and IM will run on only 1 processor, the processor 11.  If
stride is not equal to 1, the processors for the component will not be
neighboring processors.

It is strongly recommended to check the validity of the {\tt run/PARAM.in} and 
{\tt run/LAYOUT.in} files before running the code. If the
code will be run on 16 processors, type
\begin{verbatim}
Scripts/TestParam.pl -n=16
\end{verbatim}
in the main SWMF directory.
The Perl script reports inconsistencies and errors. 
If no errors are found, the script finishes silently.
Now you are ready to run the executable through submitting a batch job or, 
if it is possible on your computer, run the code interactively.  For
example, to run the SWMF interactively:
\begin{verbatim}
cd run
mpirun -np 16 SWMF.exe
\end{verbatim}
The SWMF provides example job scripts for several architectures and
machines used by the developers. These job scripts are found in 
\begin{verbatim}
CON/Scripts
\end{verbatim}
in the subdirectories named after the operating system. If the name
of the file in the appropriate subdirectory matches the 
name of the machine, the job script is copied into
the {\tt run} directory when it is created.
These job scripts serve as a starting point only, they must
be customized before they can be used for submitting a job.

To recompile the executable with different compiler settings you have
to use the command
\begin{verbatim}
make clean
\end{verbatim}
before recompiling the executables. It is possible to recompile
only a component or just one subdirectory if the {\tt make clean}
command is issued in the appropriate directory.

\section{Restarting a Run}

There are several reasons for restarting a run. A run may fail
due to a run time error, due to hardware failure, due to 
software failure (e.g. the machine crashes) or because the
queue limits are exceeded. In such a case the run can be continued from
the last saved state of SWMF. 

It is also possible that one builds up a complex simulation from multiple 
runs. For example the first run creates a steady state for the SC component.
The second run includes both the SC and IH components and it 
restarts from the results of the first run and creates a steady state
for both components. A third run may restart from this solution and include
the GM component, etc. 

The restart files are saved at the frequency determined in the PARAM.in file.
Normally the restart files are saved into the output restart directories
of the individual components and subsequent saves overwrite the previous ones
(to reduce the required disk space). A restart requires the modification
of the PARAM.in file: one needs to include the restart file for the
control module of SWMF as well as ask for restart by all the components.

The Scripts/Restart.pl script simplifies the work of the restart in
several ways:
\begin{enumerate}
\item The SWMF restart file and the individual output restart 
directories of the components are collected into a single directory tree, 
the {\bf restart tree}.
\item The default input restart file of SWMF and the default 
      input directories of the components can be linked to an existing
      restart tree.
\item The script can run continuously in the background and create
      multiple restart trees while SWMF is running. 
\item The script does extensive checking of the consistency 
      of the restart files.
\end{enumerate}
The Restart.pl script is copied into the run directory and it should
be executed in the run directory. Note that the PARAM.in file is not
modified by the script: it has to be modified with an editor as needed.

To demonstrate the use of the script, here are a few simple examples.
After a successful or failed run which should be continued, simply type
\begin{verbatim}
cd run
./Restart.pl
\end{verbatim}
to create a restart tree from the final output and to link to the tree for the
next run. The default name of the restart tree is based on the simulation time
for time accurate runs, or the time step for non-time accurate runs.
But you can also specify a name explicitly, for example
\begin{verbatim}
./Restart.pl RESTART_SC_steady_state
\end{verbatim}
If you wish to continue the run in another run directory, or on another
machine, transfer the restart tree as a whole into the new run
directory and type
\begin{verbatim}
./Restart.pl -i=RESTART_SC_steady_state
\end{verbatim}
where the {\tt -i} stands for ``input only'', i.e. the script links to
the tree, but it does not attempt to create the restart tree.

To save multiple restart trees repeatedly at an hourly frequency of 
wall clock time while the SWMF is running, type
\begin{verbatim}
./Restart.pl -r=3600 &
\end{verbatim}
To see all the options of the script type
\begin{verbatim}
./Restart.pl -h
\end{verbatim}

\section{What next?}

Hopefully this section has guided you through installing the SWMF and
given you a basic knowledge of how to run it.  However it has probably
also convinced you that the SWMF is quite a complex tool and that there
are many more things for you to learn.  So, what next?

We suggest that you read all of chapter \ref{chapter:basics}, which
outlines the basic features of the SWMF as well as some things you
really must know in order to use the SWMF.  Once you have done this you
are ready to experiment.  Chapter \ref{chapter:examples} gives several 
examples which are intended to make you familiar with the use of the
SWMF.  We suggest that you try them!

%\end{document}


% Basic User manual
\chapter{The Basics}

%  Copyright (C) 2002 Regents of the University of Michigan, portions used with permission 
%  For more information, see http://csem.engin.umich.edu/tools/swmf
\section{Configuration of SWMF}

Configuration refers to several different ways of controlling how the 
SWMF is compiled and run.  The most obvious is the setting of
compiler flags specific to the machine and version of FORTRAN
compiler.  The other methods refer to ways in which different physics
components are chosen to participate in or not participate in a run.
Inclusion of components can be controlled using one of several methods:

\begin{itemize}
\item The source code can modified so that all references %^CMP IF CONFIGURE
      to a subset of the components is removed. %^CMP IF CONFIGURE
      This method uses the Scripts/Configure.pl script. %^CMP IF CONFIGURE
      In a similar way, some physics components can be individually
      configured.
\item The user may select which version of a physics component,
      including the Empty version,
      should be compiled.  This is controlled using the Config.pl script.
\item When submitting a run, a subset of the non-empty (compiled) 
      components can be
      registered to participate in the run in the LAYOUT.in file.
\item Registered components can be turned off and on with the \#COMPONENT
      command in the PARAM.in file.
\end{itemize}
Each of these options have their useful application.

Finally, each physics component may have some settings which need to
(or can) be individually
configured, such as selecting user routines for the IH/BATSRUS or
GM/BATSRUS components.

%^CMP IF CONFIGURE BEGIN
\subsection{Scripts/Configure.pl}

The Scripts/Configure.pl script can build a new software package which
contains only a subset of the components. It is a simple interface
for the general share/Scripts/Configure.pl script. The configuration
can remove a whole component directory and all references to the component 
in the source code, in the scripts and the Makefiles.
This type of configuration results in a smaller software package.
The main use of this type of configuration is to distribute
a part of SWMF to users. For example one can create a 
software distribution which includes GM, IE and UA only by typing
\begin{verbatim}
  Scripts/Configure.pl -on=GM,IE,UA -off=SC,IH,SP,IM,PW,RB
\end{verbatim}
The configured package will be in the Build directory.  Type
\begin{verbatim}
  Scripts/Configure.pl -h
\end{verbatim}
to get complete usage information or read about this script 
in the reference manual.
%^CMP END CONFIGURE

\subsection{Selecting physics models with Config.pl}

The physics models (component versions) reside in the component 
directories CZ, EE, GM, IE, IH, IM, OH, RB, PS, PT, PW, SC, SP and UA.
Most components have only one working version and one empty version.
The empty version consists of a single wrapper file, which contains 
empty subroutines required by CON\_wrapper and the couplers.
These empty subroutines are needed for the compilation of the code,
and they also show the interface of the working versions.

The appropriate version can be selected with the {\tt -v} flag
of the {\tt Config.pl} script, which edits the Makefile.def file.
For example
\begin{verbatim}
  Config.pl -v=GM/BATSRUS,IM/RCM2,IE/Ridley_serial
\end{verbatim}
selects the BATSRUS, RCM2 and Ridley\_seriel models for
the GM, IM and IE components, respectively.
To see the current selectoin and the available models for all
the components type
\begin{verbatim}
  Config.pl -l
\end{verbatim}
The first column shows the currently selected models, the rest are the 
available alternatives.

If a physics component is not needed for a particular run, 
an Empty version of the component can be compiled.
Selecting the Empty version for unused components reduces
compilation time and memory usage during run time.
It may also improve performance slightly.
This is achieved with the {\tt -v} flag of the Config.pl script. 
For example the Empty UA component can be selected with
\begin{verbatim}
  Config.pl -v=UA/Empty
\end{verbatim}
It is also possible to select the Empty version for all components
with a few exceptions. For example
\begin{verbatim}
  Config.pl -v=Empty,GM/BATSRUS,IE/Ridley_serial
\end{verbatim}
will select the Empty version for all components except for GM and IE.
Note that the 'Empty' item has to be the first one.

\subsection{Clone Components}

The EE/BATSRUS, IH/BATSRUS, OH/BATSRUS and SC/BATSRUS models are special, 
since they use the same source code as GM/BATSRUS, which is stored 
in the CVS repository. We call the other BATSRUS models
{\bf clones} of the GM/BATSRUS code. The source code of the clone models
is copied over from the original files and then all modules,
external subroutines and functions are renamed. For example
ModMain.f90 is renamed to IH\_ModMain.f90 in IH/BATSRUS.
These steps are performed automatically when the clone model is selected
for the first time, for example by typing
\begin{verbatim}
Config.pl -v=IH/BATSRUS
\end{verbatim}
Once the source code is copied and renamed, the clone models work
just like any model. They can be configured, compiled, and used in runs.

It is important to realize that code development is always done
in the original source code, i.e. in GM/BATSRUS and in 
IH/BATSRUS/srcInterface/IH\_wrapper.f90.
If the source code of the clones should be refreshed, for example
after an update from the CVS respository, type
\begin{verbatim}
make cleanclones
Config.pl
\end{verbatim}
and the source code will be copied and renamed for the selected clones.
The source code of the clones is removed fully when the SWMF is
uninstalled with the
\begin{verbatim}
Config.pl -uninstall
\end{verbatim}
command. 

\subsection{Registering components with LAYOUT.in}

The components used in particular run has to be listed (registered)
in the LAYOUT.in file. 
Note that empty component versions cannot be registered at all.
Component registration allows to run the same executable with different 
subsets of the components. For example the GM and IE components 
can be selected with the following LAYOUT.in file
\begin{verbatim}
ID   first last  stride
#COMPONENTMAP
IE     0      1     1
GM     2   9999     1
#END
\end{verbatim}
The first column contains the component ID, the second is the index
of the first (root) processor for the component, the third column is the
last processor and the last column contains the stride 
that is typically set to 1.
In the example above IE will run on the first 2 PE-s,
while GM will run on the rest of the available PE-s.
Changing the LAYOUT.in file to
\begin{verbatim}
ID   first last  stride
#COMPONENTMAP
GM     0    999     1
#END
\end{verbatim}
will still use the same executable, but will not allow the IE 
physics component to participate in the run.

\subsection{Switching models on and off with PARAM.in}

Registered components can be switched on and off during a run
with the \#COMPONENT command in the PARAM.in file. 
This approach allows the component to be switched on in a later 
'session' of the run. For example, in the first session only GM 
is running, while in the second session it is coupled to IE. 
In this example the IE component can be switched off with the
\begin{verbatim}
#COMPONENT
IE              NameComp
F               UseComp
\end{verbatim}
in the first session and it can be switched on with the
\begin{verbatim}
#COMPONENT
IE              NameComp
T               UseComp
\end{verbatim}
command in the second session.

\subsection{Setting compiler flags}

The debugging flags can be switched on and off with
\begin{verbatim}
  Config.pl -debug
\end{verbatim}
and
\begin{verbatim}
  Config.pl -nodebug
\end{verbatim}
respectively. The maximum optimization level can be set to -O2 with
\begin{verbatim}
  Config.pl -O2
\end{verbatim}
The minimum level is 0, the maximum is 5. Note that not all compilers support
level 5 optimization. As already mentioned, the code needs to be cleaned 
and recompiled after changing the compiler flags:
\begin{verbatim}
  make clean
  make -j
\end{verbatim}
Note that not all the components take into account the selected
compiler flags. For example the IM/RCM2 component has to be compiled 
with the -save (or similar) flag, thus it uses the flags defined in the 
{\tt CFLAGS} variable. Also some of the compilers produce incorrect
code if they compile certain source files with high optimization level.
Such exceptions are described in the 
\begin{verbatim}
  Makefile.RULES.all
\end{verbatim}
files in the source code directories. The content of this file
is processed by {\tt Config.pl} into {\tt Makefile.RULES}
(according to the selected compiler and other parameters),  
which is then included into the main Makefile of the source
directory.

\subsection{Configuration of individual components}

Some of the components can be configured individually. 
The {\tt GM/BATSRUS} code, for example, can be configured to
use specific equation and user modules.
For example
\begin{verbatim}
cd GM/BATSRUS
Config.pl -e=MhdIonsPe
\end{verbatim}
will select the equation module for multiple ion fluids and separate
electron pressure. The same can be done with the {\tt Config.pl} script
in the main SWMF directory
\begin{verbatim}
Config.pl -o=GM:e=MhdIonsPe
\end{verbatim}
The grid sizes of the various components can be set with the 
{\tt -g} flag of the {\tt Config.pl} script.
For example the
\begin{verbatim}
  Config.pl -g=UA:36,36,50,16
\end{verbatim}
will set the blocks size to $36\times 36\times 50$ and the number of blocks to 
16 for the UA/GITM2 component. This command runs the {\tt Config.pl}
script of the selected UA component. 
On machines with limited memory it is especially important to
set the number of blocks correctly. 

Of course, the SWMF code has to be recompiled after any of these changes with
\begin{verbatim}
  make -j
\end{verbatim}
Note that in this case there is no need to type 'make clean', 
because the {\tt make} command knows which files need to be recompiled.

\subsection{Using stubs for all components}

It is possible to compile and run the SWMF without the physics components
but with place holders (stubs) for them that mimic their behavior.
This can be used as a test tool for the CON component, but it may
also serve as an inexpensive testbed for getting the optimal layout
and coupling schedule for a simulation. To configure SWMF with 
stub components, select the Empty version for all physics components
(with Config.pl -v=...) and edit the {\tt Makefile.def} file to
contain
\begin{verbatim}
#INT_VERSION = Interface
INT_VERSION = Stubs
\end{verbatim}
for the interface so that the real interface in {\tt CON/Interface}
is replaced with {\tt CON/Stubs}.
The resulting executable will run CON with 
the stubs for the physics components. For the stubs one can
specify the time step size in terms of simulation time and the
CPU time needed for the time step. The stub components communicate
at the coupling time, so the PE-s need to synchronize, but 
(at least in the current implementation) there is no net time taken
for the coupling itself. 

The stub components help development of the SWMF core, but it also
allows an efficient optimization of the LAYOUT and coupling
schedules for an actual run, where the physical time steps
and the CPU time needed by the components is approximately known.
In the test runs with the Stubs, one can reduce the CPU times by 
a fixed factor, so it takes less CPU time to see the efficiency of the 
SWMF for a given layout and coupling scheme.

An alternative way to test performance with different configurations is
to use the Scripts/Performance.pl script. See the help message of the
script for information on usage.

%  Copyright (C) 2002 Regents of the University of Michigan, portions used with permission 
%  For more information, see http://csem.engin.umich.edu/tools/swmf
%\documentclass{article}
%\begin{document}

\section{PARAM.in \label{section:param.in}}

The input parameters for the SWMF are read from the 
{\tt PARAM.in} file which must be located in the run directory.
This file, together with the LAYOUT.in file, controls the SWMF
and its components.
There are many include files in the {\tt Param} directory. These
can be included into the {\tt PARAM.in} files, or they can serve as
examples. 

In the PARAM.in file, 
the parameters specific to a component are given between
the \#BEGIN\_COMP ID and \#END\_COMP ID commands,
where the ID is the two character identifier of the component.
For example the GM parameters are enclosed between the 
\begin{verbatim}
#BEGIN_COMP GM
...
#END_COMP GM
\end{verbatim}
commands. We refer to the lines starting with a \# character as commands.
For example if the command string 
\begin{verbatim}
#END
\end{verbatim}
is present, it indicates the end of the run and lines following
this command are ignored. If the \#END command is not
present, the end of the PARAM.in file indicates the end of the run.

There are several features of the input parameter file syntax
that allow the user to easily run the code
in a variety of modes while at the same time being able to 
keep a library of useful parameter files that can be used
again and again.

The syntax and the content of the input parameter files
is defined in the PARAM.XML files. The commands controlling
the whole SWMF are described in the main directory in the
\begin{verbatim}
  PARAM.XML
\end{verbatim}
file. The component parameters are described by the PARAM.XML
file in the component version directory. For example the
input parameters for the GM/BATSRUS component are described in
\begin{verbatim}
  GM/BATSRUS/PARAM.XML
\end{verbatim}
These files can be read (and edited) in a normal editor.
The same files are used to produce much of this
manual with the aid of the {\tt share/Scripts/XmlToTex.pl} script. 
The {\tt Scripts/TestParam.pl} script also uses these files
to check the PARAM.in file.
Copying small segments of the {\tt PARAM.XML} files
into {\tt PARAM.in} can speed up the creation or modification of a 
parameter file. 

\subsection{Included Files, {\tt \#INCLUDE} \label{section:include}}

The {\tt PARAM.in} file can include other parameter files with the 
command
\begin{verbatim}
#INCLUDE
include_parameter_filename
\end{verbatim}
The include files serve two purposes: (i) they help
to group the parameters; (ii) the included files can be reused
for other parameter files. 
An include file can include another file itself.
Up to 10 include files can be nested.
The include files have exactly the same structure as {\tt PARAM.in}. 
The only difference is that the
\begin{verbatim}
#END
\end{verbatim}
command in an included file means only the end of the include file, 
and not the end of the run, as it does in {\tt PARAM.in}.

The user can place his/her
included parameter files into the main run directory or in any subdirectory
as long as the correct path to the file from the run directory is
included in the {\tt \#INCLUDE} command.

\subsection{Commands, Parameters, and Comments \label{section:commands}}

As can be seen from the above examples, the parameters are entered
with a combination of a {\bf command} followed by specific {\bf parameter(s)},
if any.
The {\bf command} must start with a hashmark (\#), which 
is followed by capital letters and underscores without space in between. 
Any characters behind the first space or TAB character are ignored
(the \#BEGIN\_COMP and \#END\_COMP commands are the only exception,
but these are markers rather than commands).
The parameters, which follow, must conform to 
requirements of the command. They can be of four types: logical, integer,
real, or character string. Logical parameters can be entered as 
{\tt .true.} or {\tt .false.} or simply {\tt T} or {\tt F}.
Integers and reals can be in any of the usual Fortran formats.  In
addition, real numbers can be entered as fractions (5/3 for example).
All these can be followed by arbitrary comments, typically separated
by space or TAB characters. In case of the character type input
parameters (which may contain spaces themselves), the comments must
be separated by a TAB or by at least 3 consecutive space characters.
Comments can be freely put anywhere between two commands as long
as they don't start with a hashmark.

Here are some examples of valid commands, parameters, and comments:
\begin{verbatim}

#TIMEACCURATE
F                       DoTimeAccurate

Here is a comment between two commands...

#DESCRIPTION
My first run            StringDescription (3 spaces or TAB before the comment)

#STOP
-1.                     tSimulationMax
100                     MaxIteration

#RUN ------------ last command of this session -----------------

#TIMEACCURATE
T                       DoTimeAccurate

#STOP
10.0                    tSimulationMax
-1                      MaxIteration

#BEGIN_COMP IH

#GAMMA
5/3                     Gamma

#END_COMP IH

\end{verbatim}

\subsection{Sessions \label{section:sessions}}

A single parameter file can control consecutive {\bf sessions}
of the run. Each session looks like
\begin{verbatim}
#SOME_COMMAND
param1
param2

...

#STOP
max_simulation_time_for_this_session
max_iter_for_this_session

#RUN
\end{verbatim}
while the final session ends like
\begin{verbatim}
#STOP
max_simulation_time_for_final_session
max_iter_for_final_session

#END
\end{verbatim}
The purpose of using multiple sessions is to be able to change parameters 
during the run. For example one can use only a subset of the
components in the first session, and add more components in the
later session. Or one can obtain a coarse steady state solution
on a coarse grid with a component in one session, and improve on the solution
with a finer grid in the next session. Or one can switch from 
steady state mode to time accurate mode. The SWMF remembers parameter
settings from all previous sessions, so in each session one should only
set those parameters which change relative to the previous session.
Note that the maximum number of iterations given in the {\tt \#STOP} command 
is meant for the entire run, and not for the individual sessions. 
On the other hand, when a restart file is read, the iterations prior to 
the current run do not count.

The {\tt PARAM.in} file and all included parameter files are read into 
a buffer at the beginning of the run, so even for multi-session runs, 
changes in the parameter files have no effect once {\tt PARAM.in} 
has been read. 

\subsection{The Order of Commands \label{section:order}}

In essence, the order of parameter commands within a
session is arbitrary, but there are some important restrictions.  
We should note that the order of the parameters following 
the command is not arbitrary and must exactly match what the code requires.  
Here we restrict ourselves to the restrictions on the commands read by
the control module of SWMF. There may be (and are) restrictions
for the commands read by the components, but those are described
in the documentation of the components.

The only strict restriction on the SWMF commands is related
to the 'planet' commands. The default values of the 
planet parameters are defined by the \#PLANET command.
For example the parameters of Earth can be selected with the
\begin{verbatim}
#PLANET
Earth            NamePlanet
\end{verbatim}
command. The true parameters of Earth can be modified or simplified
with a number of other commands which {\bf must occur after the
\#PLANET command}. These commands are (without showing their parameters)
\begin{verbatim}
#IDEALAXES
#ROTATIONAXIS
#MAGNETICAXIS
#MAGNETICCENTER
#ROTATION
#DIPOLE
\end{verbatim}
Other than this strict rule, it makes sense to follow a 'natural'
order of commands. This will help in interpreting, maintaining
and reusing parameter files.

If you want all the input parameters to be echoed back, the first
command in {\tt PARAM.in} should be
\begin{verbatim}
#ECHO
T                 DoEcho
\end{verbatim}
If the code starts from restart files, it usually reads in a
file which was saved by SWMF. The default name of the saved
file is RESTART.out and it is written into the run directory.
It should be renamed, for example to RESTART.in, so that it
does not get overwritten during the run. It can be included as
\begin{verbatim}
#INCLUDE
RESTART.in
\end{verbatim}
The SWMF will read the following commands (the parameter values are
examples only) from the included file:
\begin{verbatim}
#DESCRIPTION
Create startup for GM-IM-IE-UA from GM steady state.

#PLANET
EARTH                        NamePlanet

#STARTTIME
    1998                     iYear
       5                     iMonth
       1                     iDay
       0                     iHour
       0                     iMinute
       0                     iSecond
 0.000000000000              FracSecond
 
#NSTEP
    4000                     nStep
 
#TIMESIMULATION
 0.00000000E+00              tSimulation
 
#VERSION
 2.00                        VersionSwmf
 
#PRECISION
8                              nByteReal
\end{verbatim}
The \#PLANET command defines the selected planet.
The \#STARTTIME command defines the starting date and time of the whole
simulation. The current simulation time (which is relative to
the starting date and time) and the step number are
given by the \#TIMESIMULATION and \#NSTEP commands. Finally
the \#VERSION and \#PRECISION commands check the consistency
of the current version and real precision with the run which
is being continued. For sake of convenience, the \#IDEALAXES,
\#ROTATEHGR and \#ROTATEHGI commands are also saved 
into the restart file if they were set in the run.

As it was explained above, all modifications of the planet 
parameters should follow the \#PLANET command, i.e. they should be after 
the \#INCLUDE RESTART.in command. In case the description is
changed it should also follow, e.g.
\begin{verbatim}
#INCLUDE
RESTART.in

#DESCRIPTION
We continue the run for another 2 hours
\end{verbatim}
When the run starts from scratch, the PARAM.in file
should start similarly with the 
\begin{verbatim}
#DESCRIPTION
This is the start up run

#PLANET
SATURN

#STARTTIME
    2004                     iYear
       8                     iMonth
      15                     iDay
       1                     iHour
      25                     iMinute
       0                     iSecond
 0.000000000000              FracSecond
\end{verbatim}
commands (the parameters are examples only).
These commands are typically followed by the planet parameter
modifying commands, if any, and setting time accurate mode
(if changed from default true to false or relative to the previous session).
For example:
\begin{verbatim}
! Align the rotation and magnetic axes with Z_GSE
#IDEALAXES

#TIMEACCURATE
F                           DoTimeAccurate
\end{verbatim}
All the commands which are written into the RESTART.out file and all 
the planet modifying commands can only occur in the first session.
These commands contain parameters which should not change during a run.
In the PARAM.XML file these commands are marked with an 
{\tt if="\$\_IsFirstSession"} conditional.
If any of these parameters are attempted to be changed in later sessions, 
a warning is printed on the screen and the code stops running
(except when the code is in non-strict mode).

Most command parameters have sensible default values.
These are described in the PARAM.XML files,
and in chapter \ref{chapter:commands} (which was produced from them).
The {\tt PARAM.XML} file also defines which commands are required
with the {\tt required="T"} attribute of the {\tt <command...>} tag.
For the control module the only required command in every
session is the \#STOP command
(or this can be replaced with the \#ENDTIME command in the last session), 
which defines the final time step in steady state mode 
or the final time of the session in time accurate mode.

\subsection{Iterations, Time Steps and Time \label{section:frequency}}

In several commands the frequency or `time' of some action has
to be defined. This is usually done with a pair of parameters.
The first defines the frequency or time in terms of the number of time steps,
and the second in terms of the simulation time.
A negative value for the frequency means that it should not be taken 
into account. For example, in time accurate mode,
\begin{verbatim}
#SAVERESTART
T            DoSaveRestart
2000         DnSaveRestart
-1.          DtSaveRestart
\end{verbatim}
means that a restart file should be saved after every 2000th time step, while
\begin{verbatim}
#SAVERESTART
T            DoSaveRestart
-1           DnSaveRestart
100.0        DtSaveRestart
\end{verbatim}
means that it should be saved every 100 seconds in terms of physical time.
Defining positive values for both frequencies might be useful
when switching from steady state mode to time accurate mode.
In the steady state mode the DnSaveRestart parameter is used,
while in time accurate mode the DtSaveRestart if it is positive.
But it is more typical and more intuitive 
to explicitly repeat the command in the first 
time accurate session with the time frequency set.

The purpose of this subsection is to try to help the user understand 
the difference between the iteration number used for stopping the code
and the time step which is used to define the frequencies of various
actions. After using \BATSRUS\ over several years, it is clear to the
authors that this distinction is useful and the
most reasonable implementation. The SWMF has inherited these
features from the \BATSRUS\ code.

We begin by defining several different quantities and the variables that 
represent them in the code.  The variable {\tt nIteration}, 
represents the number of ``iterations'' 
that the simulation has taken since it began running.  
This number starts at zero every time the code is run, even if beginning 
from a restart file.
This is reasonable since most users know how many iterations the code can take
in a certain amount of CPU time and it is this number that is needed when 
running in a queue.
The quantity {\tt nStep} is a number of ``time steps'' that the code has 
taken in total.  This number starts at zero when the code is started from 
scratch, but when started from a restart file, this
number will start with the time step at which the restart file was written.
This implementation lets the user output data files at a regular interval, even
when a restart happens at an odd number of iterations.
The quantity {\tt tSimulation} is the amount of simulated, or physical, 
time that the code has run.  
This time starts when time accurate time stepping begins.
When restarting, it starts from the physical time for the restart.
Of course the time should be cumulative since it is the physically meaningful
quantity.  We will 
use these three phrases( ``iteration'', ``time step'', ``time'') 
with the meanings outlined above.

Now, what happens when the user has more than one session and he or she
changes the frequencies.  Let us examine what would happen in the following
sample of part of a {\tt PARAM.in} file.  For the following example we will
assume that when in time accurate mode, 1 iteration simulates 1 second of time.
Columns to the right indicate the values of {\tt nITER}, {\tt n\_step} and
{\tt time\_simulation} at which restart files will be written in each session.

\clearpage

\begin{verbatim}
                                             Restart Files Written at:
==SESSION 1                         Session   nITER   nStep    time_simulation
#TIMEACCURATE                       --------  ------  -------  --------------
F            DoTimeAccurate  

#SAVERESTART                             1     200      200             0.0  
T            DoSaveRestart               1     400      400             0.0
200          DnSaveRestart
-1.0         DtSaveRestart

#STOP
400          MaxIteration
-1.          tSimulationMax

#RUN ==END OF SESSION 1== 
                         
#SAVERESTART                             2     600      600             0.0
T            DoSaveRestart               2     900      900             0.0
300          DnSaveRestart
-1.0         DtSaveRestart
				
#STOP				
1000         MaxIteration				
-1.          tSimulationMax
				
#RUN ==END OF SESSION 2== 

#TIMEACCURATE
T            DoTimeAccurate  		
				
#SAVERESTART                             3    1100     1100           100.0
T            DoSaveRestart               3    1200     1200           200.0
-1           DnSaveRestart               3    1300     1300           300.0
100.0        DtSaveRestart
				
#STOP				
-1           MaxIteration				
300.0        tSimulationMax			
				
#RUN ==END OF SESSION 3== 
                          
#SAVERESTART                             4    1400     1400           400.0
T            DoSaveRestart               4    1800     1800           800.0
-1           DnSaveRestart               4    2000     2000          1000.0
400.0        DtSaveRestart
 				
#STOP				
-1           MaxIteration				
1000.0       tSimulationMax				
				
#END  ==END OF SESSION 4== 
\end{verbatim}
Now the question is how many iterations will be taken and when will restart
file be written out.  In session 1 the code will make 400 iterations and will
write a restart file at time steps 200 and 400.  Since the iterations 
in the {\tt \#STOP}
command are cumulative, the {\tt \#STOP} command in the second session will
have the code make 600 more iterations for a total of 1000.  Since the timing
of output is also cumulative, a restart file will be written at time step 600
and at 900.
After session 2, the code is switched to time accurate mode.  Since we
have not run in this mode yet the simulated (or physical) time is cumulatively
0.  The third session will run for 300.0 simulated seconds (which for the
sake of this example is 300 iterations).  The restart file will be written
after every 100.0 simulated seconds.
The {\tt \#STOP} command in Session 4 tells the code to simulate  700.0 more 
seconds for a total of 1000.0 seconds.  The code will make a restart file
when the time is a multiple of 400.0 seconds or at 400.0 and 800.0 seconds.
Note that a restart file will also be written at time 1000.0
seconds since this is the end of a run.

In the next example we want to restart from 1000.0 seconds 
and continue with a time accurate run.
\begin{verbatim}
                                             Restart Files Written at:
==SESSION 1                         Session   nITER   nStep    time_simulation
                                    --------  ------  -------  --------------
#INCLUDE                                 1       0     2000          1000.0
RESTART.in

#TIMEACCURATE
T            DoTimeAccurate  

#SAVERESTART                             1     200     2200          1200.0
T            DoSaveRestart
-1           DnSaveRestart
600.0        DtSaveRestart

#STOP
-1           MaxIteration
1400.0       tSimulationMax

#RUN ==END OF SESSION 1== 

#SAVERESTART                             2     700     2700          1500.0
T            DoSaveRestart               2    1000     3000          2000.0
-1           DnSaveRestart
750.0        DtSaveRestart

#STOP
-1           MaxIteration
2000.0       tSimulationMax

#END ==END OF SESSION 2 = 
                          
\end{verbatim}
In this example, we see that in time accurate mode the simulated, or
physical, time is always cumulative.  To make 400.0 seconds more simulation,
the original 1000.0 seconds must be taken into account.  The final output 
at 2000.0 seconds is written because the run ended.

Throughout this subsection, we have used the frequency of writing restart files
as an example.  The frequencies of coupling components and checking stop
files work similarly. In the SWMF, and potentially in any of the
components, the frequencies are handled by the general
\begin{verbatim}
  share/Library/src/ModFreq
\end{verbatim}
module which is described in the reference manual.

%\end{document}

%  Copyright (C) 2002 Regents of the University of Michigan, portions used with permission 
%  For more information, see http://csem.engin.umich.edu/tools/swmf
\section{Execution and Coupling Control}

The control module of SWMF controls the execution and coupling of
components. The control module is controlled by the user through the
input parameter file PARAM.in.
Defining the most efficient component layout, execution and coupling control
is not an obvious task. In the current version of SWMF the processor
layout of the components is static. This restriction is somewhat
mitigated by the possibility of restart, which allows to change
the processor layout from one run to another.

\subsection{Processor Layout}

Within one run the layout is determined by the \#COMPONENTMAP
command in the PARAM.in file. The command
is documented in the PARAM.XML file.
Here we provide several examples which will help to develop
a sense of using optimal layouts.  An optimal layout is one that 
maximizes the use of all processors and does not leave processors
with nothing do while waiting for other processors to finish their work.

First of all we have to define the processor rank:
it is a number ranging from 0 to $N-1$, where
$N$ is the total number of processors in the run. 
A component can run on a subset of the processors,
which is defined by the rank of the first (root) processor,
the rank of the last processor, and the stride. For
example if the root processor has rank 4, the last processor
has rank 8, and the stride is 2 than the component will
run on 3 processors with ranks 4, 6 and 8.

\subsubsection{One component}

In the simplest case a single component, say the Global
Magnetosphere (GM) is running. The layout should be
the following
\begin{verbatim}
ID   Proc0 ProcEnd Stride
#COMPONENTMAP
GM   0     -1    1
\end{verbatim}
Here the -1 is interpreted as the rank of the last processor,
which is $N-1$ if the SWMF is running on $N$ processors.

\subsubsection{One serial and one parallel component}

When two components are used, their layouts may or may not overlap.
An example for overlapping the layouts of the GM and the
Inner Magnetosphere (IM) components is
\begin{verbatim}
ID   proc0 last stride
#COMPONENTMAP
IM    0       0    1
GM    0      -1    1
\end{verbatim}
When the component layouts overlap, the two components can run
sequentially only. Since IM is using a single processor only
(because it is not a parallel code), all the other processors 
will be idling while IM is running. This can be rather inefficient,
especially if the CPU time required by IM is not negligible.
A more efficient execution can be achieved with a non-overlapping layout:
\begin{verbatim}
ID   proc0 last stride
#COMPONENTMAP
IM    0       0    1
GM    1      -1    1
\end{verbatim}
Note that this layout file will work for any number 
of processors from 2 and up.

\subsubsection{Two parallel components with different speeds}

It is not always possible, or even efficient to use non-overlapping
layouts. For example both the SC and IH components require a lot of memory,
but the IH component runs much faster (say 100 times faster) 
in terms of cpu time than the SC component (this is due to the 
larger cells and smaller wave speeds in IH).
If we tried to use concurrent execution on 101 processors,
SC should run on 100 and IH on 1 processors to get good load balancing.
However the IH component needs much more memory than available
on a single processor. It is therefore not possible to use a non-overlapping
layout for SC and IH on a reasonable number of processors.

Fortunately both the Solar Corona (SC) and Inner Heliosphere (IH)
components are modeled by \BATSRUS, which is a highly parallel code
with good scaling. The following layout can be optimal:
\begin{verbatim}
ID   proc0 last stride
#COMPONENTMAP
IH    0    -1    1
SC    0    -1    1
\end{verbatim}
Although IH and SC will execute sequentially, they both
use all the available CPU-s, so no CPU is left waiting for the others.

\subsubsection{Two parallel components with similar speeds}

If two parallel components need about the same CPU time/real time
on the same number of processors, the optimal layout can be
\begin{verbatim}
ID   proc0 last stride
#COMPONENTMAP
GM    0     -1    2
SC    1     -1    2
\end{verbatim}
Here GM is running on the processors with even rank,
while SC is running on the processors with odd ranks.
By using the processor stride, this layout works on
an arbitrary number of processors.

When more serial and parallel codes are executing together,
finding the optimal layout may not be trivial. 
It may take some experimentation to see which component
is running slower or faster, how much time is spent
on coupling two components, etc. It may be a good idea
to test the components separately or in smaller groups
to see how fast they can execute.

\subsubsection{A complex example with four components}

Here is an example with 4 components: the Ionospheric
Electrodynamics (IE) component can run on 2 processors and 
runs about 3 times faster than real time.
The serial Inner Magnetosphere (IM) component runs even faster,
on the other hand the coupling of GM and IM is rather
computationally expensive. The Upper Atmosphere (UA) component
can run on up to 32 processors, and it runs twice as fast
as real time. The Global Magnetosphere model (GM) needs
at least 32 processor to run faster than real time.
If we have a lot of CPU-s, we may simply create a non-overlapping
layout. Since GM has no restriction on the number of processors,
it can be the last component in the map
\begin{verbatim}
ID   proc0 last stride
#COMPONENTMAP
IM    0       0    1
IE    0       1    1
UA    2      33    1
GM   34     999    1
\end{verbatim}
This layout will be optimal in terms of speed for a large 
(more than 100) number of PE-s, and actually the maximum
speed is going to be limited by the components which do
not scale. On a more modest number of PE-s one can try
to overlap UA and GM:
\begin{verbatim}
ID   proc0 last stride
#COMPONENTMAP
IM    0       0    1
IE    1       2    1
UA    0      31    1
GM    3     999    1
\end{verbatim}

\subsubsection{Using OpenMP threads}

Some of the models, such as \BATSRUS, can use OpenMP threads
in addition to the MPI paralelization. Typically one should
run one OpenMP thread on each core, and the number of MPI
processes should be 1 or 2 (or possibly more) for each node.
The most efficient arrangement depends on the hardware architecture
and the model. The number of maximum threads MaxThread is set by the
environment variable OMP\_NUM\_THREADS. Typically one wants
to use nThread=MaxThread threads for the components that can use
OpenMP. This can be easily achieved by setting the stried and the
number of threads in the last (optional) column to -1:
\begin{verbatim}
ID   proc0 last stride nthread
#COMPONENTMAP
GM    0      -1    -1      -1
\end{verbatim}
For example, if the node has 56 cores split to two independent
slots, the optimal setting is likely to be OMP\_NUM\_THREADS=28.
In this case both stride and nthread will be 28.

If OMP\_NUM\_THREADS is not in advance, it is best to set
the root of the multithreaded component to proc0=0, so
that the stride is properly aligned with the cores of
the nodes. This means that other components that can
only use a fixed number of processors should be put
to the last processors, for example
\begin{verbatim}
ID   proc0 last stride nthread
#COMPONENTMAP
GM    0      -3    -1      -1
IE   -2      -1     1
\end{verbatim}
In this layout GM is running with multiple threads on
cores 0 to $N-3$, while IE is using cores $N-2$ and $N-1$.

\subsection{Steady State vs. Time Accurate Execution}

The SWMF can run both in time accurate (default) and
steady state mode. This sounds surprising first, 
since many of the components can run in time accurate 
mode only. Nevertheless, the SWMF can improve the convergence
towards a steady state by allowing the different components
to run at different speeds in terms of the physical time.
In \BATSRUS\ the same idea is used on a much smaller scale:
local time stepping allows each cell to converge towards
steady state as fast as possible, limited only by the local
numerical stability limit.

\subsubsection{Steady state session}

The steady state mode should be signaled with the 
\begin{verbatim}
#TIMEACCURATE
F                DoTimeAccurate
\end{verbatim}
command, usually placed somewhere at the beginning of the session.
Since the SWMF runs in time accurate mode by default,
this command is required in the first steady state session of the run.

When SWMF runs in steady state mode, the SWMF time is not
advanced and tSimulation usually keeps its default initial value,
which is zero. 
The components may or may not advance their own
internal times. The execution is controlled by the 
step number {\tt nStep}, which goes from its initial value 
to the final step allowed by the MaxIteration parameter
of the \#STOP command. The components are called at
the frequency defined by the \#CYCLE command. For example
\begin{verbatim}
#CYCLE
GM               NameComp
1                DnRun

#CYCLE
IM               NameComp
2                DnRun
\end{verbatim}
means that IM runs in every second time step of the SWMF.
By defining the DnRun parameter for all the components,
an arbitrary relative calling frequency can be obtained,
which can optimize the global convergence rate to steady state.
The default frequency is DnRun=1, i.e. the component is
run in every SWMF time step. 

The relative frequency can be important for numerical
stability too. When GM and IM are to be relaxed
to a steady state, the GM/BATSRUS code is running in 
local time stepping mode, while IM/RCM runs in time 
accurate mode internally. Since GM and IM are coupled
both ways, an instability can occur if both GM and IM
are run every time step, because the GM physical time
step is very small, and the MHD solution cannot relax
while being continuously pushed by the IM coupling.
This unphysical instability can be avoided by calling the
IM component less frequently.

The coupling frequencies should be set to be optimal
for reaching the steady state. If the components are
coupled too frequently, a lot of CPU time is spent
on the couplings. If they are coupled very infrequently,
the solution may become oscillatory instead of relaxing
into a (quasi-)steady state solution. For example
we used the
\begin{verbatim}
#COUPLE2
IM                      NameComp1
GM                      NameComp2
10                      DnCouple
-1.                     DtCouple
\end{verbatim}
command to couple the GM and IM components in both directions
in every 10-th SWMF iteration.
Note that according to the above \#CYCLE commands,
GM and IM do 10 and 5 steps between two couplings,
respectively. GM/BATSRUS uses 10 local time steps,
while IM advances by 5 five-second time steps.

Another example is the relaxation of SC and IH components.
Under usual conditions the solar wind is supersonic at the 
inner boundary of the IH component, thus the steady state SC
solution can be obtained first, and then IH can converge
to a steady state using the SC solution as the inner boundary 
condition. In this second stage SC does not need to run
(assuming that it has reached a good steady state solution),
it is only needed for providing the inner boundary condition for IH.
This can be achieved by
\begin{verbatim}
! No need to run SC too often, it is already in steady state
#CYCLE
SC                      NameComp
1000                    DnRun

! No need to couple SC to IH too often
#COUPLE1
SC                      NameSource
IH                      NameTarget
1000                    DnCouple
-1.0                    DtCouple
\end{verbatim}
Since SC and IH are always coupled at the beginning of the session,
further couplings are not necessary.

\subsubsection{Time accurate session}

The SWMF runs in time accurate mode by default. The
\begin{verbatim}
#TIMEACCURATE
T                       DoTimeAccurate
\end{verbatim}
command is only needed in a time accurate session following a 
steady state session.
In time accurate mode the components advance in time at
approximately the same rate. The component times are
only synchronized when necessary, i.e. when they
are coupled, when restart files are written, or 
at the end of session and execution. Since the time
steps (in terms of physical and/or CPU time) of the components can be 
vastly different, this minimal synchronization provides the 
most possibilities for efficient concurrent execution.

In time accurate mode the coupling times have to be defined
with the DtCouple arguments. For example
\begin{verbatim}
#COUPLE2
GM                      NameComp1
IM                      NameComp2
-1                      DnCouple
10.0                    DtCouple
\end{verbatim}
will couple the GM and IM components every 10 seconds. 

In some cases the models have to be coupled every time step.
An example is the coupling between the MHD model GM/BATSRUS and 
the Particle-in-Cell model PC/IPIC3D. This can be achieved with
\begin{verbatim}
#COUPLE2TIGHT
GM                      NameMaster
PC                      NameSlave
T                       DoCouple
\end{verbatim}
command. In this case the master component (GM) tells the slave
component (PC) the time step to be used. The tight 
coupling requires models and couplers that support this option.

By default the component time steps are limited by the
time of couplings. This means that if GM can take 4 second
times steps, and it is coupled with IE every 5 seconds,
then every second GM time step will be truncated to 1 second.
There are two ways to avoid this. One is to choose the
coupling frequencies to be round multiples of the time steps
of the two components involved. This works well if both components
have fixed time steps and/or much smaller time steps than the 
coupling frequency.

In certain cases the efficiency can be improved with the
\#COUPLETIME command, which can allow a component to 
step through the coupling time. For example
\begin{verbatim}
#COUPLETIME
GM                      NameComp
F                       DoCoupleOnTime
\end{verbatim}
will allow the GM component to use 4 second time steps even
if it is coupled at every 5 seconds. Of course this will
make the data transferred during the coupling be 
first order accurate in time.

\subsection{Coupling order}

The default coupling order is usually optimal for accuracy
and consistency, but it may not be optimal for speed.
In particular, the IE/Ridley\_serial component solves
a Poisson type equation for the data received from the 
other components (GM and UA). For sake of accuracy
IE always uses the latest data received from the other
components. If GM, UA and IE are coupled
in the default order
\begin{verbatim}
#COUPLEORDER
4             nCouple	  
GM IE         NameSourceTarget
UA IE         NameSourceTarget
IE UA         NameSourceTarget
IE GM         NameSourceTarget
\end{verbatim}
and the to-IE and from-IE coupling times coincide, e.g.
\begin{verbatim}
#COUPLE2
GM            NameComp1
IE            NameComp2
10.0          DtCouple
-1            DnCouple

#COUPLE2
UA            NameComp1
IE            NameComp2
10.0          DtCouple
-1            DnCouple
\end{verbatim}
then GM and UA will have to wait until IE solves
the Poisson equation, because IE receives new data
and it is required to produce results immediately.
With the reversed coupling order
\begin{verbatim}
#COUPLEORDER
4             nCouple	  
IE UA	      NameSourceTarget
IE GM	      NameSourceTarget
GM IE	      NameSourceTarget
UA IE	      NameSourceTarget
\end{verbatim}
IE will provide the solution from the previously received data,
and it will have time to work on the new data while GM and UA
are working on their time steps. The reversed coupling order
allows the concurrent execution of IE with other components.
The temporal accuracy, on the other hand, will be somewhat worse.

To demonstrate that the coupling order is important, here
is a very {\bf inefficient} coupling order
\begin{verbatim}
#COUPLEORDER
4             nCouple	  
GM IE         NameSourceTarget
IE GM         NameSourceTarget
UA IE         NameSourceTarget
IE UA         NameSourceTarget
\end{verbatim}
in case the coupling times with GM and UA coincide (always at the beginning
of a the sessions).
With this coupling order, IE first receives information from GM,
then solves the Poisson equation and returns the information based
on the solution to GM while GM is waiting. Then IE receives extra
information from UA, solves the Poisson equation again, and sends
back information to UA, while UA is waiting. 

An alternative way to achieve concurrent execution is to
stagger the coupling times. For example the
\begin{verbatim}
#COUPLE2SHIFT
GM                 NameComp1
IE                 NameComp2
-1                 DnCouple
10.0               DtCouple
-1                 nNext12
0.0                tNext12
-1                 nNext21
5.0                tNext21
\end{verbatim}
will schedule a GM to IE coupling at 0, 10, 20, 30, \ldots seconds,
and the IE to GM coupling at 5, 15, 25, \ldots seconds.
This provides IE half the GM time to solve the Poisson equations.
If IE runs at least twice as fast as GM, this solution will
produce concurrent execution. The temporal accuracy is
somewhat better than in the reversed coupling case.
Note that GM and IE will be synchronized at 0, 5, 10, \ldots seconds,
which works best if the GM time step is an integer fraction of 5 seconds.


% Example runs
\chapter{Example Runs}
The examples in this chapter are intended to make you familiar
with the use of the SWMF. By carefully following the steps you should
be able to do the tests as described. It is a good idea to read the
provided PARAM.in and LAYOUT.in files and try to understand how
the examples work. You may also experiment by changing these files
after copied from the originals. These examples should help you
in setting up your own runs.

For sake of simplicity we describe how to do all the example runs with the
same executable. In actual runs one would streamline the configuration
of the SWMF to reduce compilation time and memory use. If you work
on a machine with limited resources, you may wish to configure SWMF
differently. For example, you may set the Empty version for the unused
components. If the number of processors is limited, you will have to
change the LAYOUT.in file and overlap the components. You may also have to 
increase the allowed grid size per processor, or you can run the problem with 
a coarser grid resolution. To reduce the CPU time, you may shorten the 
run by changing the number of iterations or the final time in the 
PARAM.in file.

\section{Configuration and Compilation for the Examples}

Select the SC/BATSRUS component version. When this is done the first time
after installation, the SC/BATSRUS source code is created from the GM/BATSRUS
source code, which takes a few minutes:
\begin{verbatim}
Config.pl -v=SC/BATSRUS -g=SC:4,4,4,1000
\end{verbatim}
Set the grid size for the GM/BATSRUS and IH/BATSRUS\_share components:
\begin{verbatim}
Config.pl -g=GM:8,8,8,200,40
\end{verbatim}
You can select either the UA/GITM version with
\begin{verbatim}
Config.pl -v=UA/GITM -g=UA:9,9,25,2,2
\end{verbatim}
or the new (default) UA/GITM2 version with
\begin{verbatim}
Config.pl -v=UA/GITM2 -g=UA:9,9,25,4
\end{verbatim}
Note the difference in the grid size parameters.
You should also take care of using the parameter files appropriate for
the selected UA component version.  GITM2 is the default UA module
currently in the SWMF and is the most up to date.

Check the current settings with
\begin{verbatim}
Config.pl -show
\end{verbatim}
You should see the current directory, the operating system, the name
of the compiler and the following settings
\begin{verbatim}
The default precision for reals is double precision.
The selected component versions and grid sizes are:
GM/BATSRUS                   grid: 8,8,8,200,40
IE/Ridley_serial             
IH/BATSRUS                   grid: 8,8,8,200,40
IM/RCM                       
RB/RiceV5                    
SC/BATSRUS                   grid: 4,4,4,1000
SP/Kota                      grid: 1000,10,150
UA/GITM2                     grid: 9,9,25,4
\end{verbatim}
In case you selected the UA/GITM version, the last line will read
\begin{verbatim}
UA/GITM                      grid: 9,9,25,2,2
\end{verbatim}
If the settings differ you can change them with the Config.pl script.

Compile the main executable code bin/SWMF.exe. With the above
settings this may take an hour, but if you select Empty versions
for the unused components, it will take much less.
You should also compile the post processing code bin/PostIDL.exe,
and finally create a run directory and change to that directory
\begin{verbatim}
make
make PIDL
make rundir
cd run
\end{verbatim}
Note that the run directory contains subdirectories for all the
non-empty components. There is also a link to the Param directory
in the main SWMF directory. The Param directory contains all the
parameter and layout files used in the example runs.

\section{Example 1: Create Steady State for the Solar Corona}

This example involves the SC component only. It demonstrates
how a steady state solar corona can be obtained from
a magnetogram. The convergence to steady state is accelerated by 
a gradual grid refinement and a gradual application of the
more-and-more accurate numerical schemes. The final state
is a steady state to high accuracy. 

You can use the SWMF with the settings recommended above
but you only need the SC component and
all other component versions can be Empty.

Copy the PARAM.in and LAYOUT.in files
\begin{verbatim}
rm -f PARAM.in LAYOUT.in
cp Param/PARAM.in.test.start.SC.AMR8 PARAM.in
cp Param/LAYOUT.in.test.start.SC LAYOUT.in
\end{verbatim}
We recommend the AMR8 version of the parameter file because it creates
a smaller grid and reaches steady state in smaller number of iterations
than the higher resolution version (AMR9). For the milestone problem
we used the higher resolution.

Check the parameter and layout files with the TestParam.pl script. 
This script should be run from the main directory.
Define the number of CPU-s you plan to use with the -n=NUMBER flag.
For example
\begin{verbatim}
cd ..
Scripts/TestParam.pl -n=32
\end{verbatim}
The script should return without any warnings and error messages.
If there are error messages, fix them. 
For example if the grid is not large enough there will be an 
error message from the command \#CHECKGRIDSIZE with respect
to the parameter  MinBlockALL, which contains the minimum number
of blocks required. To fix the problem, you have to increase the number 
of CPU-s or increase the grid size for the SC component with the 
Config.pl script. For example set
\begin{verbatim}
Config.pl -g=SC:4,4,4,1500
\end{verbatim}
Keep fixing the problems until the Scripts/TestParam.pl runs silently.
Remember to recompile SWMF.exe if you change the grid size.

Run the code by submitting a job or do it interactively
\begin{verbatim}
cd run
mpirun -np 32 SWMF.exe | tee runlog.SC
\end{verbatim}
Here 'tee' is a Unix command which splits the output to the screen as
well as pipes it to the file 'runlog.SC'. 

Depending on the number of processors and the speed of the machine,
this run should take a few to several hours to complete.
You may check the progress on the screen, or look at the 
runlog.SC file, or look at the log file written by the code
\begin{verbatim}
tail SC/IO2/log*
\end{verbatim}
This file contains information determined by the \#SAVELOGFILE command
in the PARAM.in file. In this case the file contains the time step, the
time (which is zero, because this is a steady state run) and the
magnetic, kinetic and thermal energies integrated over the surface of 
two spheres of radii 4 and 10. As the code approaches steady state,
the integrated energies will change less and less. 

When the run finishes successfully (or even while the code is running), 
you can postprocess the plot files
\begin{verbatim}
cd SC
pIDL
pTEC p r
cd ..
\end{verbatim}
The 'p' and 'r' flags for the pTEC script mean that the raw ASCII data
files are preprocessed with preplot into .plt files and the ASCII data
files are removed. If the machine does not have the 'preplot' code
installed (preplot is a script that comes with the Tecplot software),
you can gzip the .dat files to save disk space and transfer time
\begin{verbatim}
pTEC g
\end{verbatim}
You can visualize the output files in run/SC/IO2 with your favored 
visualization software. For visualization with IDL you should read the
chapter on ``IDL visualization'' in the user manual in GM/BATSRUS/Doc.

The restart files were saved into RESTART.SC and the SC/restartIN directory.
These are not the default names, they were set in the PARAM.in file with
the \#RESTARTFILE command for the control module of the SWMF and 
with the \#RESTARTOUTDIR command for the SC/BATSRUS component.
This makes it easier to use the run in the following example run.

The alternative and more typical approach would be to write into the
default restart file RESTART.out and the default directory SC/restartOUT,
and use the Restart.pl script to create a restart tree, e.g.
\begin{verbatim}
Restart.pl RESTART_SC
\end{verbatim}

\section{Example 2: Create SC-IH Steady State}

This example run is built on the previous example. We restart SC from the
steady state created in the previous run and start the IH component from 
scratch and run the two components coupled until the IH component reaches
a steady state. 

First copy in the prepared parameter and layout files
\begin{verbatim}
rm -f PARAM.in LAYOUT.in
cp Param/PARAM.in.test.restart.SCIH  PARAM.in
cp Param/LAYOUT.in.test.restart.SCIH LAYOUT.in
\end{verbatim}
Look at the PARAM.in file to see how the convergence to 
steady state is accelerated.
First of all the SC component only provides the boundary conditions for IH,
so it only runs in every 100th iteration (see the \#CYCLE command in
the PARAM.in file). Second, the IH grid is built
up with a gradual grid refinement, and third the 
more-and-more accurate and expensive numerical schemes are 
applied in an optimal sequence. The final state
is a steady state for SC and IH to high accuracy. 

You can use the SWMF with the settings recommended for all the examples,
but you only need the SC and IH components so 
all other component versions can be Empty.

As usual, check the input parameters and the layout for the
number of CPU-s you plan to use, for example
\begin{verbatim}
cd ..
Scripts/TestParam.pl -n=32
\end{verbatim}
If there are error messages, fix them until the script runs silently.

Run the code by submitting a job, or interactively
\begin{verbatim}
cd run
mpirun -np 32 SWMF.exe | tee runlog.SCIH
\end{verbatim}
When the run finishes, postprocess the plot files for both components
\begin{verbatim}
cd SC
pIDL
pTEC
cd ../IH
pIDL
pTEC p r
cd ..
\end{verbatim}
Due to the commands \#RESTARTFILE and \#RESTARTOUTDIR (in the IH section)
the output restart file is saved into RESTART.SCIH and the IH component
saved the restart state into the IH/restartIN. The SC component saved
the restart state into the default SC/restartOUT. For a continuation run
one could move and link the directories in SC with the following Unix commands
\begin{verbatim}
cd SC
mv restartIN restart_SConly
mv restartOUT restart_SCIH
mkdir restartOUT
ln -s  restart_SCIH restartIN
cd ..
\end{verbatim}
As you can see this is a rather cumbersome and error prone procedure.
It is better to save the restart state into their default file and
directories and use the Restart.pl script to collect them into 
a restart tree and to link the input file and directories to them.

\section{Example 3: Create Initial Conditions for the Global Magnetosphere}

This example involves the GM component only. It demonstrates
how a reasonable global magnetosphere can be obtained.
The upwind boundary conditions are based on the solution obtained
for the IH component in the 2nd example run, but they
are intentionally modified to contain a discontinuity.
This is for demonstration purposes only to make a subsequent
GM-IH coupled run more dynamic. In a more typical run one would
use the upwind boundary condition based on satellite measurements 
or by coupling to the IH code.

The convergence to steady state is accelerated by 
a gradual grid refinement, and a gradual application of the
more-and-more accurate numerical schemes. The final state
is a reasonable global magnetosphere solution, but it is not 
a perfect steady state (that would require many more iterations).

First copy in the prepared parameter and layout files and check them
\begin{verbatim}
rm -f PARAM.in LAYOUT.in
cp Param/PARAM.in.test.start.GM  PARAM.in
cp Param/LAYOUT.in.test.start.GM LAYOUT.in
\end{verbatim}
The parameters of the \#SOLARWIND command were obtained
from the result of the SC-IH steady state. 
Otherwise this test can be run independently, 
no restart files are read.
You can use the SWMF with the settings recommended for all the tests, 
but you only need the GM component, so all other component versions 
can be set to Empty.

Check the parameters and layout before running the SWMF
\begin{verbatim}
cd ..
Scripts/TestParam.pl -n=16
\end{verbatim}
If there are error messages, fix them until the script reports no errors.
Run the code by submitting a job, or interactively
\begin{verbatim}
cd run
mpirun -np 16 SWMF.exe | tee runlog.GM
\end{verbatim}
Postprocess the plot files
\begin{verbatim}
cd GM
pIDL
pTEC p r
cd ..
\end{verbatim}
The output restart file is RESTART.GM and the GM component saved
its restart files into GM/restart.IN as determined by the PARAM.in file.

\section{Example 4: Create Initial Conditions for the GM, IM, IE 
         and UA Components}

This example involves the GM, IM, IE and UA components. It demonstrates
how to obtain a reasonable solution for the 
coupled global/inner magnetosphere, ionosphere and upper atmosphere.
The initial global magnetosphere solution is taken from Example 3.

Copy the layout file into the run directory
\begin{verbatim}
rm -f PARAM.in LAYOUT.in
cp Param/LAYOUT.in.test.restart.GMIMIEUA LAYOUT.in
\end{verbatim}
If you are using the UA/GITM2 component version, copy this parameter file:
\begin{verbatim}
cp Param/PARAM.in.test.restart.GMIMIEUA.GITM2 PARAM.in
\end{verbatim}
If you are using the UA/GITM component version, copy this parameter file:
\begin{verbatim}
cp Param/PARAM.in.test.restart.GMIMIEUA PARAM.in
\end{verbatim}
Have a look at the PARAM.in file.
The convergence to steady state is accelerated by component subcycling
(see the \#CYCLE commands in PARAM.in).
The components are called at different frequencies, which
allows the components to reach a reasonable solution
approximately at the same rate. Another optimization trick is the changing of 
the coupling order such that the IE component sends information before
it would receive it (see the \#COUPLEORDER command in PARAM.in),
so that IE component can solve for the electric potential concurrently with the
other components running. To reduce the time spent on field line
tracing, which is needed for the IM to GM coupling, the \#RAYTRACE command
in the GM section limits the frequency of traces to every 80th iteration.

Now have a look at the LAYOUT file:
\begin{verbatim}
#COMPONENTMAP
UA   0    31    1
IM   32   32    1
IE   33   34    1
GM   35  999    1
#END
\end{verbatim}
As you can see all the components are running concurrently.
Unless you have access to at least 64 processors, 
you probably want to modify the layout file.
For example for 32 processors you can overlap the UA component 
with the other components and run IE on a single processor:
\begin{verbatim}
#COMPONENTMAP
UA    0   31    1
IE    0    0    1
IM    1    1    1
GM    2   31    1
#END
\end{verbatim}
You can use the SWMF with the settings recommended for all the tests, 
but you only need the GM, IM, IE and UA components,
all other component versions can be Empty.

Check the parameters and layout for the number of processors you plan to use
\begin{verbatim}
cd ..
Scripts/TestParam.pl -n=32
\end{verbatim}
If there are error messages, fix them. 
Run the code by submitting a job, or interactively
\begin{verbatim}
cd run
mpirun -np 32 SWMF.exe | tee runlog.GMIMIEUA
\end{verbatim}
This run should take an hour or so on 32 processors.
After the run postprocess the plot files
\begin{verbatim}
cd GM
pIDL
pTEC p r
cd ../IE
pION
cd ..
\end{verbatim}
The pION script concatenates the northern and southern output files
produced by the component IE/Ridley\_serial.

In this run all output restart information is saved into the
default file (RESTART.out) and default directories.
You can use the RESTART.pl script to collect them into a restart tree.
For example
\begin{verbatim}
Restart.pl -o RESTART_GMIMIEUA
\end{verbatim}

\section{Example 5: Time Accurate Run with SC, IH and SP Components}

This example involves the Solar Corona, Inner Heliosphere
and Solar Energetic Particles components.
The run starts from a steady state of the SC and IH components
which was obtained in Example 2.

Copy in the prepared parameter and layout files
\begin{verbatim}
rm -f PARAM.in LAYOUT.in
cp Param/PARAM.in.test.restart.SCIHSP PARAM.in
cp Param/LAYOUT.in.test.restart.SCIHSP LAYOUT.in
\end{verbatim}
At the beginning of this time accurate run a CME is 
generated in the SC component. This demonstrates the
use of an Eruptive Event generator.
The SP component is coupled, and during the 4 hour evolution
of the CME, substantial particle acceleration is observed. 
By the end of the run the CME reaches the boundary 
between the SC and IH components, and partially enters the IH domain.
The test demonstrates the coupling between the SC-IH and SP
components in a challenging simulation.

In case you have limited computational resources, you can 
shorten the run by editing the PARAM.in file and changing
the \#STOP command at the end of the file. For example 
\begin{verbatim}
#STOP
-1                     MaxIteration
1800.0                 tSimulationMax
\end{verbatim}
will reduce the final simulation time to 30 minutes.

Also have a look at the layout file:
\begin{verbatim}
#COMPONENTMAP
SC    1 999    1
IH    1 999    1
SP    0   0    1 
#END
\end{verbatim}
The SC and IH components are overlapped while the SP component
runs concurrently. This is probably the optimal layout for any
number of CPU-s.

You can use the SWMF with the settings for all the tests, 
but you only need the SC, IH and SP components,
all other component versions can be Empty.

Check the parameters and layout before running the SWMF
\begin{verbatim}
cd ..
Scripts/TestParam.pl -n=32
\end{verbatim}
If there are error messages, fix them, then run the SWMF by submitting a job
or interactively
\begin{verbatim}
cd run
mpirun -np 32 SWMF.exe | tee runlog.SCIHSP
\end{verbatim}
This run may take a long time if run all the way to 4 hours.
After the run finishes, postprocess the plot files
\begin{verbatim}
cd SC
pIDL
pTEC p r
cd ../IH
pIDL
pTEC p r
cd ..
\end{verbatim}
The restart files can be collected into a restart tree 
with the Restart.pl script. For the full 4 hour run you could use
\begin{verbatim}
Restart.pl -o RESTART_SCIHSP_4hr
\end{verbatim}

\section{Example 6: Time Accurate Run with All Eight Components}

This run is described in the TESTING manual. 
The initial conditions were created with the runs described in
this chapter using the higher resolution SC grid (AMR9) and the full 4 hour
time accurate run with the SC-IH and SP components.

For sake of learning SWMF you can do this example with lower SC resolution
(AMR8) and do a shorter time accurate run with the SC, IH and SP components.

Copy the layout file into the run directory
\begin{verbatim}
rm -f PARAM.in LAYOUT.in
cp Param/LAYOUT.in.test.restart.8comp LAYOUT.in
\end{verbatim}
If you are using the UA/GITM2 component version, copy this parameter file:
\begin{verbatim}
cp Param/PARAM.in.test.restart.8comp.GITM2 PARAM.in
\end{verbatim}
If you are using the UA/GITM component version, copy this parameter file:
\begin{verbatim}
cp Param/PARAM.in.test.restart.8comp PARAM.in
\end{verbatim}
This example is desinged to run for 600 seconds only.
It is assumed that the input restart file RESTART.in contains
the simulation time corresponding to 4 hours (14400 seconds),
so the final time in the \#STOP command is 15000 seconds
in the PARAM.in file.
In case you are restarting from a different simulation time and/or
you wish to change the length of the run, edit the maximum
simulation time in the \#STOP command in the PARAM.in file.

To simplify this example run, you can remove some of the components.
To remove a component you need to edit the LAYOUT.in file and remove the
appropriate line, and edit the PARAM.in file, and remove the 
section belonging to the component and all the couplings with the 
component. In fact, in some cases it may not be necessary to remove 
anything from the PARAM.in file. You may simply add an \#END command
after the section belonging to the last included component,
since the lines following the \#END command are ignored.
For example to run with the SC and IH components only, you can put an
\#END command after the \#END\_COMP IH command but before the coupling
with the SP component:
\begin{verbatim}
#END_COMP IH ---------------------------------------

#END

#COUPLE1
SC                      NameSource
SP                      NameTarget
-1                      DnCouple
60.0                    DtCouple
...

\end{verbatim}
Test the parameters and the layout as usual
\begin{verbatim}
cd ..
Scripts/TestParam.pl -n=64
\end{verbatim}
and run the code
\begin{verbatim}
cd run
mpirun -np 32 SWMF.exe | tee runlog.8comp
\end{verbatim}
After the run finishes, postprocess the plot files
\begin{verbatim}
cd SC
pIDL
pTEC p r
cd ../IH
pIDL
pTEC p r
cd ..
cd ../GM
pIDL
pTEC p r
cd ../IE
pION
cd ..
\end{verbatim}
Since this example run is very short, no restart files are saved
(see the \#SAVERESTART command).


% Complete List of Input Commands
\chapter{Complete List of Input Commands}

\input{SWMFXML}

% Index for the commands
%^CFG COPYRIGHT UM
\documentclass[twoside,10pt]{book}

\input HEADER

\title{Space Weather Modeling Framework User Manual \\ 
       \hfill \\
       \large Code Version \SWMFVERSION}
\author{G\'abor T\'oth, Ovsei Volberg, Aaron Ridley\\
       \hfill \\
       {\it Center for Space Environment Modeling}\\
       {\it The University of Michigan}}

\makeindex

\begin{document}

\pagestyle{fancy}
\lhead[\fancyplain{}{\bfseries\thepage}]{\fancyplain{}{\bfseries\rightmark}}
\rhead[\fancyplain{}{\bfseries\leftmark}]{\fancyplain{}{\bfseries\thepage}}
\cfoot{}
%\pagestyle{headings} % if fancy heading does not work

\maketitle

%\noindent COPYRIGHT (c) 1996 \\
THE REGENTS OF THE UNIVERSITY OF MICHIGAN \\
ALL RIGHTS RESERVED \\

PERMISSION IS GRANTED TO A SINGLE USER TO USE THIS SOFTWARE FOR
NONCOMMERCIAL EDUCATION AND RESEARCH PURPOSES.  THE USER IS GIVEN
PERMISSION TO INSTALL AND RUN THE SOFTWARE ON MULTIPLE PLATFORMS.  NO
PERMISSION IS GIVEN TO COPY OR REDISTRIBUTE THIS SOFTWARE IN ANY FORM
INCLUDING, BUT NOT LIMITED TO, COPYING SECTIONS OF THE SOURCE CODE OR
DOCUMENTATION AND DISTRIBUTING THE CODE TO OTHER USERS.  ALTHOUGH
SOURCE CODE IS DISTRIBUTED, USERS ARE RESTRICTED FROM MODIFYING THE
SOURCE CODE IN ANYWAY WITHOUT PRIOR CONSENT OF THE CENTER FOR SPACE
ENVIRONMENT MODELING AT THE UNIVERSITY OF MIGHIGAN.

THIS SOFTWARE IS PROVIDED AS IS, WITHOUT REPRESENTATION AS TO ITS
FITNESS FOR ANY PURPOSE, AND WITHOUT WARRANTY OF ANY KIND, EITHER
EXPRESS OR IMPLIED, INCLUDING WITHOUT LIMITATION THE IMPLIED WARRANTIES
OF MERCHANTABILITY AND FITNESS FOR A PARTICULAR PURPOSE. THE REGENTS OF
THE UNIVERSITY OF MICHIGAN SHALL NOT BE LIABLE FOR ANY DAMAGES,
INCLUDING SPECIAL, INDIRECT, INCIDENTAL, OR CONSEQUENTIAL DAMAGES, WITH
RESPECT TO ANY CLAIM ARISING OUT OF OR IN CONNECTION WITH THE USE OF
THE SOFTWARE, EVEN IF IT HAS BEEN OR IS HEREAFTER ADVISED OF THE
POSSIBILITY OF SUCH DAMAGES.




\tableofcontents

% Introduction
%\documentclass[a4paper,11pt]{article}
%\author{\bf Center for Space Environment Modeling, The University of Michigan}
%\title{\bf \Large Release Notes for the Milestone 7I and Reference Manual}
%\maketitle



\chapter{Introduction}

This document describes a working prototype of the NASA-funded Space
Weather Modeling Framework (SWMF) delivered to NASA to fulfill the
Milestone 11K requirements. The SWMF was developed to provide flexible
``plug and play" type simulation capabilities serving the Sun-Earth
modeling community.  In its current form the SWMF links together eight
models from the surface of the Sun to the upper atmosphere of the Earth: 
\begin{enumerate}
\item SC -- Solar Corona which includes the Eruptive Event Generator,
\item IH -- Inner Heliosphere
\item SP -- Solar Energetic Particles 
\item GM -- Global Magnetosphere 
\item IM -- Inner Magnetosphere
\item RB -- Radiation Belts
\item IE -- Ionosphere Electrodynamics
\item UA -- Upper Atmosphere
\end{enumerate}
In the future the SWMF may be extended to include even more 
physics domains: Cometary Environment, Interstellar
Neutrals, Outer Heliosphere, Plasmasphere, Planetary Satellites and
Polar Wind. 

The SWMF implementation is based on the component technology and
Object-Oriented Programming emulated in Fortran 90.  The SWMF parallel
communications are based on the MPI standard.  In its current
implementation the SWMF creates a single executable.

\section{Acknowledgments}

The SWMF was developed at the Center for Space Environment Modeling
(CSEM) of the University of Michigan under the NASA Earth Science
Technology Office (ESTO) Computational Technologies (CT) Project (NASA
CAN NCC5-614). The project is entitled as ``A High-Performance
Adaptive Simulation Framework for Space-Weather Modeling (SWMF)''.
The Project Director is Professor Tamas Gombosi, and the Co-Principal
Investigators are Professors Quentin Stout and Kenneth Powell.

The SWMF and many of the physics components were developed at CSEM
by the following individuals (in alphabetical order):
David Chesney,
Darren DeZeeuw, Tamas Gombosi, Kenneth Hansen, Kevin Kane, Ward (Chip)
Manchester, Robert Oehmke, Kenneth Powell, Aaron Ridley, Ilia Roussev,
Quentin Stout, Igor Sokolov, G\'abor T\'oth and Ovsei Volberg.

The core design and code development was done by G\'abor
T\'oth, Igor Sokolov and Ovsei Volberg:
\begin{itemize}
\item Component registration and layout was designed and implemented by 
      Ovsei Volberg and G\'abor T\'oth.
\item The session and time management support was designed and
      developed by G\'abor T\'oth.
\item The SWMF coupling toolkit was developed by Igor Sokolov.
\end{itemize}
The physics models were developed by the following research groups:
\begin{itemize}
\item
The Solar Corona (SC), Inner Heliosphere (IH) and the Global Magnetosphere 
(GM) components are based on \BATSRUS\ MHD code developed at CSEM. 
\BATSRUS\ is a 3-dimensional block-adaptive Cartesian code which is 
highly parallel.

\item
The Solar Energetic Particle (SP) component is the
K\'ota's SEP model which was developed at the University of Arizona.
It solves the equations for the advection and acceleration of
energetic particles along a magnetic field line in a 3D phase space.

\item
The Inner Magnetosphere (IM) component is the Rice Convection Model
(RCM) developed at Rice University.  This code is 2-dimensional and
serial.

\item
The Radiation Belt (RB) component is the Rice RBM
developed at Rice University.  This code is 2-dimensional and
serial.

\item
The Ionospheric Electrodynamics (IE) component is a 2-processor,
2-dimensional spherical electric potential solver developed at CSEM
(termed the ``Ridley Ionosphere'').  

\item
There are two versions of the Upper Atmosphere (UA) component:
the Global Ionosphere - Thermosphere Model (GITM) and its newer
version GITM2. Both versions are 3-dimensional spherical
models developed at CSEM.  They are fully parallel.

\end{itemize}
The transformation of physics models into physics components,
the coupling of components to the SWMF and each other and
all the testing were done at CSEM.

\section{What is New in Version 2.1}

The SWMF has been developed further since the second release of
version 2.0. Here is a partial list of improvements:
\begin{itemize}
\item The documentation has been improved and split into smaller parts.
\item The SWMF contains a new version for the UA component: GITM2.
\item Restarting SWMF is made easier with the Scripts/Restart.pl script.
\item The Solar Corona can be solved in a corotating frame.
\item The coupling between the GM and IE components has been rewritten
      with simpler and more accurate algorithms.
\item The coupling between the GM and IM components has been made more robust.
\item The geometry based grid refinement can be modified easily with
      the \#GRIDRESOLUTION command in the
      parameter file for the GM,IH,SC/BATSRUS components.

\end{itemize}

\section{What is New in Version 2.0}

The SWMF has been developed extensively since the first release of
version 1.0. Here are some of the highlights:
\begin{itemize}
\item The SWMF now contains 3 new components:
      the Solar Corona (SC), the Solar Energetic Particles (SP) 
      and the Radiation Belt (RB)
\item Some components now support dynamic memory allocation which 
      reduces the total memory required by a processing element. 
\item The directory structure of the SWMF has been greatly improved and 
      reorganized. 
\item This reorganization allows components to be used as stand alone 
      physics models without any modification in the source code.  The
      stand alone version only links to a small SWMF library.
\item The installation and configuration of SWMF has been greatly simplified
      with the aid of Perl scripts. 
\item Layout and input parameter can be checked with Scripts/TestParam.pl
\item User manual is produced from the XML description of the input parameters.
\item Unused components can be configured out completely.
\item The control module now fully supports steady state calculations 
      including component subcycling. 
\item The coupling toolkit provides means for extracting and following 
      the motion of field lines.
\end{itemize}

\section{The SWMF in a Few Paragraphs}

The SWMF is a structured collection of software building blocks that
can be used or customized to develop Sun-Earth system modeling
components, and to assemble them into applications. The SWMF consists
of utilities and data structures for coupling model components. The
SWMF contains a Control Module (CON), which is responsible for
component registration, processor layout for each component and
coupling schedules.  It controls initialization and execution of the
components. A component is adapted from user-supplied physics codes,
(for example \BATSRUS\ or RCM), by adding two relatively small units
of code:
\begin{itemize}
\item A wrapper, which provides the control functions, and
\item A coupling interface to perform the data exchange with other
components.
\end{itemize}
Both the wrapper and coupling interface are constructed from the
building blocks provided by the framework. From 
component software technology perspective both the wrapper and
coupling interface are component interfaces: the wrapper is an
interface with CON, and the coupling interface is an interface with
another component. A physics
model code and its wrapper, which comprise a component, share the
communication group.  The coupling interface uses the union
communicator of the two components that it links together.

An SWMF component is compiled into a separate library that resides in
the directory {\tt lib}, which is created as part of the installation
process described later in this document.  Currently the component
libraries are static libraries. The executable image is created in the
directory {\tt bin}, which is created during the compilation.  If a
user does not want to build some particular component, this component
should be substituted by an empty version of the component.

An important feature of the SWMF is the component registration.  A
component to be included in the run should be registered by the
framework.  Currently entering the line for the component in the input
file called {\tt LAYOUT.in} does the registration.  Thus the SWMF
performs the run-time registration of components.

The framework controls the initialization, execution, coupling and
finalization of components.  The execution is done in sessions. In
each session the parameters of the framework and the components can be
changed.  The parameters are read from the {\tt PARAM.in} file, which
may contain further included parameter files.  These parameters are
read and broadcast by CON and the component specific parameters are
sent to the components. The structure of the parameter file will be
described in detail.

If two components reside on different sets of processing elements
(PE-s) they can execute in an efficient concurrent manner.
This is possible, because the coupling times are
known in advance.  The components advance to the time of coupling and
only the processors involved in the coupling need to communicate with
each other. The components are also allowed to share some processing elements.
The execution is sequential for the components with overlapping layouts.
Of course this still allows the individual components to execute in parallel.
For steady state calculations the components are allowed to progress
at different rates towards steady state. Each component can be called
at different frequencies by the control module.

The coupling of the components is realized either with plain MPI
calls, or via the SWMF coupling toolkit, which can couple components
based on the following types of parallel distributed grids:
\begin{itemize}
\item 3-D Block adaptive (AMR) parallel grid
\item 2-D Spherical grid
\item Logically Cartesian uniform grid
\item Logically Cartesian non-uniform grid 
\end{itemize}
The SWMF coupling toolkit performs an efficient N to M parallel
coupling based on a router. The router is calculated in advance using
the domain decomposition and grid description obtained from the
components.  The router is updated only when the domain decompositions
or the grids of the components change, or when the mapping geometry
changes.  The coupling toolkit takes care of linear interpolation in
space based on the grid descriptor.  Temporal interpolation is not
supported by the current implementation.

The framework has been tested on the SGI Origin 3000, SGI Altix and 
Compaq ES45 machines, and on Linux Beowulf clusters with the NAG f95 
compiler. We have also run the framework with reasonable success under
Mac OS Darwin using the XLF and NAG f95 compilers, and under Linux with
the PGF90 compiler.

\section{System Requirements}

In order to install and run the SMWF the following minimum system
requirements apply.

\begin{itemize}
\item The SWMF runs only under the UNIX/Linux operating systems.  This now
  includes Macintosh system 10.x because it is based on BSD UNIX.  The
  SWMF does not run under any Microsoft Windows operating system.
\item A FORTRAN 77 and FORTRAN 90 compiler must be installed.
\item The Perl interpreter must be installed.
\item A version of the Message Passing Interface (MPI) library must be
  installed.
\item You may be able to compile the code and do very small test
runs on 1 or 2 processor machines.  However, to do most physically
meaningful runs the SWMF requires a
parallel processor machine with a minimum of 8 processors and a minimum of 8GB of
memory.
\item Very large runs require many more processors.
\item In order to generate the documentation you must have LaTex installed on
your system.  The PDF generation requires the {\tt dvips} and {\tt ps2pdf}
utilities.  To generate the HTML version you also must install the
{\tt latex2html} package. 

\end{itemize}


In addition to the above requirements, the SWMF output is designed to
be visualized using either IDL or Tecplot.  You may be able to
visualize the output with other packages, but formats and scripts have
been designed for only these two visualization softwares.




%-----------------------------------------------------------------------
% Chapter 2
%-----------------------------------------------------------------------

\chapter{Quick Start}

\section{A Brief Description of the SWMF Distribution}

The distribution in the form of the compressed tar image
includes the SWMF source code.
The top level directory contains the following subdirectories:
\begin{itemize}
\item {\tt CON}     - the directory of the framework's main building blocks
\item {\tt GM}      - Global Magnetosphere component       %^CMP IF GM
\item {\tt IE}      - Ionosphere Electrodynamics Component %^CMP IF IE
\item {\tt IH}      - Inner Heliosphere component          %^CMP IF IH
\item {\tt IM}      - Inner Magnetosphere component        %^CMP IF IM
\item {\tt RB}      - Radiation Belt component             %^CMP IF RB
\item {\tt SC}      - Solar Corona component               %^CMP IF SC
\item {\tt SP}      - Solar Energetic Particles component  %^CMP IF SP
\item {\tt UA}      - Upper Atmosphere component           %^CMP IF UA
\item {\tt Copyrights} - copyright files
\item {\tt Param}   - description of CON parameters, parameter and layout files
\item {\tt Scripts} - shell and Perl scripts
\item {\tt bin}     - scripts for installation, configuration and testing
\item {\tt doc}     - the documentation directory %^CMP IF DOC
\item {\tt share}   - shared scripts and source code
\item {\tt util}    - general utilities such as TIMING and NOMPI
\end{itemize}
and the following files
\begin{itemize}
\item {\tt README}           - a short instruction on installation and usage
\item {\tt Makefile}         - the main makefile
\item {\tt Configure.pl}     - Perl script for configuration %^CMP IF CONFIGURE
\item {\tt Configure.options} - default configuration options %^CMP IF CONFIGURE
\item {\tt SetSWMF.pl}     - Perl script for (un)installation and configuration
\end{itemize}

\section{General Hints}

\subsubsection{Getting help with scripts and the Makefile}
Most of the Perl and shell scripts that are distributed with the SWMF
provide help which can be accessed as follows using the {\tt -h} flag.
For example, 
\begin{verbatim}
  SetSWMF.pl -h
\end{verbatim}
will provide a detailed listing of the options and capabilities of the
{\tt SetSWMF.pl} script.  In addition, you can find all the possible
targets  that can be built by typing
\begin{verbatim}
make help
\end{verbatim}

\subsubsection{Input commands: PARAM.XML}
A very useful set of files to become familiar with are the {\tt PARAM.XML}
files.  Such a file exists for the SWMF itself and for each of the
physics components.  The file for the SWMF is found at
\begin{verbatim}
Param/PARAM.XML
\end{verbatim}
while the files for the physics components are found in the component's
subdirectory.  For example, the file for the GM/BATSRUS component can
be found at
\begin{verbatim}
GM/BATSRUS/PARAM.XML
\end{verbatim}
This file contains a complete list of all input commands for the
component as well as the allowed ranges for each of the input parameters.
Although the XML format makes the files a little hard to read, they are
extremely useful.  A typical usage is to cut and paste commands out of the
PARAM.XML file into the PARAM.in file for a run.

\subsubsection{Have the working directory in your path}
In order to run executable files in the UNIX environment you must have
the current working directory either your path or in the filename you
want to execute.  In UNIX the current working directory is represented
by the period (.).  For example
\begin{verbatim} 
./SWMF.exe
\end{verbatim}
will execute the SWMF.exe program if it is in your current directory.  If you
add the `.' to your path using
\begin{verbatim}
set path = (~/bin /usr/local/mpi/bin /usr/local/bin ${path} .)
\end{verbatim}
then you can simply type
\begin{verbatim} 
SWMF.exe
\end{verbatim}

\section{Installing the Code}

The first step in installing the SWMF is untarring the distribution.
If the tar program knows about the -z flag, you can open the gzipped
tar files with a single UNIX command:
\begin{verbatim}
  tar xzf SOMETARFILE.tgz
\end{verbatim}
If the tar program does not recognize the -z flag, two steps are needed:
\begin{verbatim}
  gunzip SOMETARFILE.tgz
  tar xf SOMETARFILE.tar
\end{verbatim}
In the following descriptions the shorter form is shown, but you may
need to use the two step procedure on certain platforms.

Untar the distribution using the command:
\begin{verbatim}
  tar xzf SWMF.tgz
\end{verbatim}

Change directories into the distribution:
\begin{verbatim}
  cd SWMF
\end{verbatim}

The SWMF needs to know what architecture you are running the code on
and what FORTRAN compiler will be used.  For most platforms and compilers,
tt can figure this out all by itself, but you have to run the command:
\begin{verbatim}
  SetSWMF.pl -i
\end{verbatim}
in the main directory. This creates {\tt Makefile.def} with
the correct absolute path to the base directory and {\tt Makefile.conf}
which contains the operating system and compiler specific part of
the Makefile. If the compiler is not the default one for a given
platform (e.g. not the NAG f95 compiler for a Linux platform) then
the compiler must be specified explicitly with the {\tt -c}
flag. If the MPI header file is not the default one, it can be
specified with the {\tt -m} flag. For example on the Altix machines
SWMF should be installed as
\begin{verbatim}
  SetSWMF.pl -i -c=ifort -m=Altix
\end{verbatim}
To uninstall SWMF type
\begin{verbatim}
  SetSWMF.pl -uninstall
\end{verbatim}
If the uninstallation fails (this can happen if some makefiles are missing)
force installation with
\begin{verbatim}
  SetSWMF.pl -install
\end{verbatim}
and then try uninstalling the code again.
When SWMF is installed, its configuration can be checked with
\begin{verbatim}
  SetSWMF.pl -s
\end{verbatim}
To get a list of the available component versions type
\begin{verbatim}
  SetSWMF.pl -l
\end{verbatim}
To get a complete description of the {\tt SetSWMF.pl}  script type
\begin{verbatim}
  SetSWMF.pl -h
\end{verbatim}

\section{Creating Documentation}

The documentation for SWMF can be generated from the distribution by
the command
\begin{verbatim}
  make PDF
\end{verbatim}
which creates the user manual
\begin{verbatim}
  doc/SWMF.pdf
\end{verbatim}
and several other documents in the Adobe PDF format.  
In order for this to work you must have
LaTex installed on your system (and dvips and ps2pdf).  
An on-line version can be created by
\begin{verbatim}
  make HTML
\end{verbatim}
The HTML version is generated from the LaTex using the command {\tt
  latex2html}.  You will have to install this if it does not already exist
  on your system.
The top level HTML file is in
\begin{verbatim}
  doc/HTML/index.html
\end{verbatim}
to point at with the browser.  This html file list the different
documentation files and what they contain.  To clean the intermediate files type
\begin{verbatim}
  cd doc/Tex
  make clean
\end{verbatim}
To remove all the created documentation type
\begin{verbatim}
  cd doc/Tex
  make cleanall
\end{verbatim}

\section{Building and Running an Executable}

At compile time, the user can select which physics components should be
compiled.  
Any component not compiled will not be available for
use at run time.  The physics components can be selected with the {\tt -v} flag
of the SetSWMF.pl script. For example typing
\begin{verbatim}
  SetSWMF.pl -v=SC/BATSRUS,IH/BATSRUS,SP/Kota
  SetSWMF.pl -v=GM/Empty,RB/Empty,IM/Empty,IE/Empty,UA/Empty
\end{verbatim}
will select BATSRUS for the SC and IH components and K\'ota's model for
the SP component.
The other components are set to Empty versions, which contain empty
subroutines for compilation, but cannot be used.
The default configuration includes a working version for all components, 
which takes up more memory, but is the most general.
The only exception is SC, which requires configuration, so the 
default version is Empty for the Solar Corona component.

The grid size of several components can also be set with the {\tt -g}
flag of the {\tt SetSWMF.pl} script. For example the 
\begin{verbatim}
  SetSWMF.pl -g=GM:8,8,8,400,100
\end{verbatim}
command sets the block size for the GM component to $8\times 8\times 8$ cells, 
the maximum number of blocks per processor to 400, 
and the maximum number of implicit blocks per processor to 100.
The SetSWMF.pl script actually runs the individual GridSize.pl
scripts in the component versions. These scripts can be run directly,
and they provide more options and more verbose information than SetSWMF.pl.
For example try
\begin{verbatim}
  cd GM/BATSRUS
  GridSize.pl -s
\end{verbatim}
Compilation flags, such as the precision and optimization 
level are stored in {\tt Makefile.conf}. This file is created on
installation of the SWMF and has defaults which are appropriate for
your system architecture.  The precision of reals
can be changed to single precision (for example) by typing
\begin{verbatim}
  SetSWMF.pl -p=single
\end{verbatim}
while the compiler flags can be edited in {\tt Makefile.conf} by hand.

Before compiling SWMF it is always a good idea to check its configuration
with
\begin{verbatim}
  SetSWMF.pl -s
\end{verbatim}

{\bf IMPORTANT NOTE:
On the Altix machine at NASA Ames (columbia)
you should load the 8.0.070 version 
of the Intel Fortran compiler 
with the command
\begin{verbatim}
  module load intel-comp.8.0.070
\end{verbatim}
You may wish to insert this line into the .cshrc file
so it executes at login time. 
Selecting the correct compiler version is 
necessary both to compile and to run the code.
Therefore the above line is needed in the job scripts
as well.}

To build the executable {\bf bin/SWMF.exe}, type:
\begin{verbatim}
  make
\end{verbatim} 
Depending on the configuration, the compiler settings and the machine 
that you are compiling on, this can take from 2 to up to 30 minutes.  
In addition, you may want to make the post processing
codes (for BATSRUS only) also:
\begin{verbatim}
  make PSPH
  make PIDL
\end{verbatim} 
These two commands will create the codes {\tt bin/PostSPH.exe}, for post
processing spherical Tecplot files, and {\tt bin/PostIDL.exe} 
for post processing IDL files.

The {\tt SWMF.exe} executable should be run in a sub-directory, since a large number
of files are created in each run.  To create this directory use the
command:
\begin{verbatim}
  make rundir
\end{verbatim} 
This command creates a directory called {\tt run}.  You can either
leave this directory as named, or {\tt mv} it to a different name.  It
is best to leave it in the same SWMF directory, since
keeping track of the code version associated with each run is quite
important.  The {\tt run} directory will contain links to the codes
which were created in the previous step as well as subdirectories
where input and output of the different components will reside.

Here we assume that the {\tt run} directory is still called {\tt
run}:
\begin{verbatim}
  cd run
\end{verbatim}
In order to run the SWMF you must have two input files:  LAYOUT.in and
PARAM.in.  The LAYOUT.in file defines the processor
layout for the components involved in the future run.  The PARAM.in
file contains the detailed commands for controlling what you want the
code to do during the run.  The default LAYOUT.in and PARAM.in
files in the run directory are suitable to perform the ``Start'' test
on 16 processors (PE-s). 

An example processor map file LAYOUT.in to run the executable with
five components on 16 processors is:
\begin{verbatim}
#COMPONENTMAP
GM    0    4    1
IE    5    6    1
IH    7   10    1
IM   11   11    1
UA   12   15    1
#END
\end{verbatim}
The file syntax is simple. It must start with the directive
\#COMPONENTMAP and end with another directive \#END. Each line between
these directives specifies the label for component, i.e. IE, GM and
etc., its first and last processor, all relatively to the world
communicator, and the stride. Thus GM will run on 5 processors from 0
to 4, and IM will run on only 1 processor, the processor 11.  If
stride is not equal to 1, the processors for the component will not be
neighboring processors.

It is strongly recommended to check the validity of the {\tt run/PARAM.in} and 
{\tt run/LAYOUT.in} files before running the code. If the
code will be run on 16 processors, type
\begin{verbatim}
Scripts/TestParam.pl -n=16
\end{verbatim}
in the main SWMF directory.
The Perl script reports inconsistencies and errors. 
If no errors are found, the script finishes silently.
Now you are ready to run the executable through submitting a batch job or, 
if it is possible on your computer, run the code interactively.  For
example, to run the SWMF interactively:
\begin{verbatim}
cd run
mpirun -np 16 SWMF.exe
\end{verbatim}
The SWMF provides example job scripts for several architectures and
machines used by the developers. These job scripts are found in 
\begin{verbatim}
CON/Scripts
\end{verbatim}
in the subdirectories named after the operating system. If the name
of the file in the appropriate subdirectory matches the 
name of the machine, the job script is copied into
the {\tt run} directory when it is created.
These job scripts serve as a starting point only, they must
be customized before they can be used for submitting a job.

To recompile the executable with different compiler settings you have
to use the command
\begin{verbatim}
make clean
\end{verbatim}
before recompiling the executables. It is possible to recompile
only a component or just one subdirectory if the {\tt make clean}
command is issued in the appropriate directory.

\section{Restarting a Run}

There are several reasons for restarting a run. A run may fail
due to a run time error, due to hardware failure, due to 
software failure (e.g. the machine crashes) or because the
queue limits are exceeded. In such a case the run can be continued from
the last saved state of SWMF. 

It is also possible that one builds up a complex simulation from multiple 
runs. For example the first run creates a steady state for the SC component.
The second run includes both the SC and IH components and it 
restarts from the results of the first run and creates a steady state
for both components. A third run may restart from this solution and include
the GM component, etc. 

The restart files are saved at the frequency determined in the PARAM.in file.
Normally the restart files are saved into the output restart directories
of the individual components and subsequent saves overwrite the previous ones
(to reduce the required disk space). A restart requires the modification
of the PARAM.in file: one needs to include the restart file for the
control module of SWMF as well as ask for restart by all the components.

The Scripts/Restart.pl script simplifies the work of the restart in
several ways:
\begin{enumerate}
\item The SWMF restart file and the individual output restart 
directories of the components are collected into a single directory tree, 
the {\bf restart tree}.
\item The default input restart file of SWMF and the default 
      input directories of the components can be linked to an existing
      restart tree.
\item The script can run continuously in the background and create
      multiple restart trees while SWMF is running. 
\item The script does extensive checking of the consistency 
      of the restart files.
\end{enumerate}
The Restart.pl script is copied into the run directory and it should
be executed in the run directory. Note that the PARAM.in file is not
modified by the script: it has to be modified with an editor as needed.

To demonstrate the use of the script, here are a few simple examples.
After a successful or failed run which should be continued, simply type
\begin{verbatim}
cd run
./Restart.pl
\end{verbatim}
to create a restart tree from the final output and to link to the tree for the
next run. The default name of the restart tree is based on the simulation time
for time accurate runs, or the time step for non-time accurate runs.
But you can also specify a name explicitly, for example
\begin{verbatim}
./Restart.pl RESTART_SC_steady_state
\end{verbatim}
If you wish to continue the run in another run directory, or on another
machine, transfer the restart tree as a whole into the new run
directory and type
\begin{verbatim}
./Restart.pl -i=RESTART_SC_steady_state
\end{verbatim}
where the {\tt -i} stands for ``input only'', i.e. the script links to
the tree, but it does not attempt to create the restart tree.

To save multiple restart trees repeatedly at an hourly frequency of 
wall clock time while the SWMF is running, type
\begin{verbatim}
./Restart.pl -r=3600 &
\end{verbatim}
To see all the options of the script type
\begin{verbatim}
./Restart.pl -h
\end{verbatim}

\section{What next?}

Hopefully this section has guided you through installing the SWMF and
given you a basic knowledge of how to run it.  However it has probably
also convinced you that the SWMF is quite a complex tool and that there
are many more things for you to learn.  So, what next?

We suggest that you read all of chapter \ref{chapter:basics}, which
outlines the basic features of the SWMF as well as some things you
really must know in order to use the SWMF.  Once you have done this you
are ready to experiment.  Chapter \ref{chapter:examples} gives several 
examples which are intended to make you familiar with the use of the
SWMF.  We suggest that you try them!

%\end{document}


% Basic User manual
\chapter{The Basics}

%  Copyright (C) 2002 Regents of the University of Michigan, portions used with permission 
%  For more information, see http://csem.engin.umich.edu/tools/swmf
\section{Configuration of SWMF}

Configuration refers to several different ways of controlling how the 
SWMF is compiled and run.  The most obvious is the setting of
compiler flags specific to the machine and version of FORTRAN
compiler.  The other methods refer to ways in which different physics
components are chosen to participate in or not participate in a run.
Inclusion of components can be controlled using one of several methods:

\begin{itemize}
\item The source code can modified so that all references %^CMP IF CONFIGURE
      to a subset of the components is removed. %^CMP IF CONFIGURE
      This method uses the Scripts/Configure.pl script. %^CMP IF CONFIGURE
      In a similar way, some physics components can be individually
      configured.
\item The user may select which version of a physics component,
      including the Empty version,
      should be compiled.  This is controlled using the Config.pl script.
\item When submitting a run, a subset of the non-empty (compiled) 
      components can be
      registered to participate in the run in the LAYOUT.in file.
\item Registered components can be turned off and on with the \#COMPONENT
      command in the PARAM.in file.
\end{itemize}
Each of these options have their useful application.

Finally, each physics component may have some settings which need to
(or can) be individually
configured, such as selecting user routines for the IH/BATSRUS or
GM/BATSRUS components.

%^CMP IF CONFIGURE BEGIN
\subsection{Scripts/Configure.pl}

The Scripts/Configure.pl script can build a new software package which
contains only a subset of the components. It is a simple interface
for the general share/Scripts/Configure.pl script. The configuration
can remove a whole component directory and all references to the component 
in the source code, in the scripts and the Makefiles.
This type of configuration results in a smaller software package.
The main use of this type of configuration is to distribute
a part of SWMF to users. For example one can create a 
software distribution which includes GM, IE and UA only by typing
\begin{verbatim}
  Scripts/Configure.pl -on=GM,IE,UA -off=SC,IH,SP,IM,PW,RB
\end{verbatim}
The configured package will be in the Build directory.  Type
\begin{verbatim}
  Scripts/Configure.pl -h
\end{verbatim}
to get complete usage information or read about this script 
in the reference manual.
%^CMP END CONFIGURE

\subsection{Selecting physics models with Config.pl}

The physics models (component versions) reside in the component 
directories CZ, EE, GM, IE, IH, IM, OH, RB, PS, PT, PW, SC, SP and UA.
Most components have only one working version and one empty version.
The empty version consists of a single wrapper file, which contains 
empty subroutines required by CON\_wrapper and the couplers.
These empty subroutines are needed for the compilation of the code,
and they also show the interface of the working versions.

The appropriate version can be selected with the {\tt -v} flag
of the {\tt Config.pl} script, which edits the Makefile.def file.
For example
\begin{verbatim}
  Config.pl -v=GM/BATSRUS,IM/RCM2,IE/Ridley_serial
\end{verbatim}
selects the BATSRUS, RCM2 and Ridley\_seriel models for
the GM, IM and IE components, respectively.
To see the current selectoin and the available models for all
the components type
\begin{verbatim}
  Config.pl -l
\end{verbatim}
The first column shows the currently selected models, the rest are the 
available alternatives.

If a physics component is not needed for a particular run, 
an Empty version of the component can be compiled.
Selecting the Empty version for unused components reduces
compilation time and memory usage during run time.
It may also improve performance slightly.
This is achieved with the {\tt -v} flag of the Config.pl script. 
For example the Empty UA component can be selected with
\begin{verbatim}
  Config.pl -v=UA/Empty
\end{verbatim}
It is also possible to select the Empty version for all components
with a few exceptions. For example
\begin{verbatim}
  Config.pl -v=Empty,GM/BATSRUS,IE/Ridley_serial
\end{verbatim}
will select the Empty version for all components except for GM and IE.
Note that the 'Empty' item has to be the first one.

\subsection{Clone Components}

The EE/BATSRUS, IH/BATSRUS, OH/BATSRUS and SC/BATSRUS models are special, 
since they use the same source code as GM/BATSRUS, which is stored 
in the CVS repository. We call the other BATSRUS models
{\bf clones} of the GM/BATSRUS code. The source code of the clone models
is copied over from the original files and then all modules,
external subroutines and functions are renamed. For example
ModMain.f90 is renamed to IH\_ModMain.f90 in IH/BATSRUS.
These steps are performed automatically when the clone model is selected
for the first time, for example by typing
\begin{verbatim}
Config.pl -v=IH/BATSRUS
\end{verbatim}
Once the source code is copied and renamed, the clone models work
just like any model. They can be configured, compiled, and used in runs.

It is important to realize that code development is always done
in the original source code, i.e. in GM/BATSRUS and in 
IH/BATSRUS/srcInterface/IH\_wrapper.f90.
If the source code of the clones should be refreshed, for example
after an update from the CVS respository, type
\begin{verbatim}
make cleanclones
Config.pl
\end{verbatim}
and the source code will be copied and renamed for the selected clones.
The source code of the clones is removed fully when the SWMF is
uninstalled with the
\begin{verbatim}
Config.pl -uninstall
\end{verbatim}
command. 

\subsection{Registering components with LAYOUT.in}

The components used in particular run has to be listed (registered)
in the LAYOUT.in file. 
Note that empty component versions cannot be registered at all.
Component registration allows to run the same executable with different 
subsets of the components. For example the GM and IE components 
can be selected with the following LAYOUT.in file
\begin{verbatim}
ID   first last  stride
#COMPONENTMAP
IE     0      1     1
GM     2   9999     1
#END
\end{verbatim}
The first column contains the component ID, the second is the index
of the first (root) processor for the component, the third column is the
last processor and the last column contains the stride 
that is typically set to 1.
In the example above IE will run on the first 2 PE-s,
while GM will run on the rest of the available PE-s.
Changing the LAYOUT.in file to
\begin{verbatim}
ID   first last  stride
#COMPONENTMAP
GM     0    999     1
#END
\end{verbatim}
will still use the same executable, but will not allow the IE 
physics component to participate in the run.

\subsection{Switching models on and off with PARAM.in}

Registered components can be switched on and off during a run
with the \#COMPONENT command in the PARAM.in file. 
This approach allows the component to be switched on in a later 
'session' of the run. For example, in the first session only GM 
is running, while in the second session it is coupled to IE. 
In this example the IE component can be switched off with the
\begin{verbatim}
#COMPONENT
IE              NameComp
F               UseComp
\end{verbatim}
in the first session and it can be switched on with the
\begin{verbatim}
#COMPONENT
IE              NameComp
T               UseComp
\end{verbatim}
command in the second session.

\subsection{Setting compiler flags}

The debugging flags can be switched on and off with
\begin{verbatim}
  Config.pl -debug
\end{verbatim}
and
\begin{verbatim}
  Config.pl -nodebug
\end{verbatim}
respectively. The maximum optimization level can be set to -O2 with
\begin{verbatim}
  Config.pl -O2
\end{verbatim}
The minimum level is 0, the maximum is 5. Note that not all compilers support
level 5 optimization. As already mentioned, the code needs to be cleaned 
and recompiled after changing the compiler flags:
\begin{verbatim}
  make clean
  make -j
\end{verbatim}
Note that not all the components take into account the selected
compiler flags. For example the IM/RCM2 component has to be compiled 
with the -save (or similar) flag, thus it uses the flags defined in the 
{\tt CFLAGS} variable. Also some of the compilers produce incorrect
code if they compile certain source files with high optimization level.
Such exceptions are described in the 
\begin{verbatim}
  Makefile.RULES.all
\end{verbatim}
files in the source code directories. The content of this file
is processed by {\tt Config.pl} into {\tt Makefile.RULES}
(according to the selected compiler and other parameters),  
which is then included into the main Makefile of the source
directory.

\subsection{Configuration of individual components}

Some of the components can be configured individually. 
The {\tt GM/BATSRUS} code, for example, can be configured to
use specific equation and user modules.
For example
\begin{verbatim}
cd GM/BATSRUS
Config.pl -e=MhdIonsPe
\end{verbatim}
will select the equation module for multiple ion fluids and separate
electron pressure. The same can be done with the {\tt Config.pl} script
in the main SWMF directory
\begin{verbatim}
Config.pl -o=GM:e=MhdIonsPe
\end{verbatim}
The grid sizes of the various components can be set with the 
{\tt -g} flag of the {\tt Config.pl} script.
For example the
\begin{verbatim}
  Config.pl -g=UA:36,36,50,16
\end{verbatim}
will set the blocks size to $36\times 36\times 50$ and the number of blocks to 
16 for the UA/GITM2 component. This command runs the {\tt Config.pl}
script of the selected UA component. 
On machines with limited memory it is especially important to
set the number of blocks correctly. 

Of course, the SWMF code has to be recompiled after any of these changes with
\begin{verbatim}
  make -j
\end{verbatim}
Note that in this case there is no need to type 'make clean', 
because the {\tt make} command knows which files need to be recompiled.

\subsection{Using stubs for all components}

It is possible to compile and run the SWMF without the physics components
but with place holders (stubs) for them that mimic their behavior.
This can be used as a test tool for the CON component, but it may
also serve as an inexpensive testbed for getting the optimal layout
and coupling schedule for a simulation. To configure SWMF with 
stub components, select the Empty version for all physics components
(with Config.pl -v=...) and edit the {\tt Makefile.def} file to
contain
\begin{verbatim}
#INT_VERSION = Interface
INT_VERSION = Stubs
\end{verbatim}
for the interface so that the real interface in {\tt CON/Interface}
is replaced with {\tt CON/Stubs}.
The resulting executable will run CON with 
the stubs for the physics components. For the stubs one can
specify the time step size in terms of simulation time and the
CPU time needed for the time step. The stub components communicate
at the coupling time, so the PE-s need to synchronize, but 
(at least in the current implementation) there is no net time taken
for the coupling itself. 

The stub components help development of the SWMF core, but it also
allows an efficient optimization of the LAYOUT and coupling
schedules for an actual run, where the physical time steps
and the CPU time needed by the components is approximately known.
In the test runs with the Stubs, one can reduce the CPU times by 
a fixed factor, so it takes less CPU time to see the efficiency of the 
SWMF for a given layout and coupling scheme.

An alternative way to test performance with different configurations is
to use the Scripts/Performance.pl script. See the help message of the
script for information on usage.

%  Copyright (C) 2002 Regents of the University of Michigan, portions used with permission 
%  For more information, see http://csem.engin.umich.edu/tools/swmf
%\documentclass{article}
%\begin{document}

\section{PARAM.in \label{section:param.in}}

The input parameters for the SWMF are read from the 
{\tt PARAM.in} file which must be located in the run directory.
This file, together with the LAYOUT.in file, controls the SWMF
and its components.
There are many include files in the {\tt Param} directory. These
can be included into the {\tt PARAM.in} files, or they can serve as
examples. 

In the PARAM.in file, 
the parameters specific to a component are given between
the \#BEGIN\_COMP ID and \#END\_COMP ID commands,
where the ID is the two character identifier of the component.
For example the GM parameters are enclosed between the 
\begin{verbatim}
#BEGIN_COMP GM
...
#END_COMP GM
\end{verbatim}
commands. We refer to the lines starting with a \# character as commands.
For example if the command string 
\begin{verbatim}
#END
\end{verbatim}
is present, it indicates the end of the run and lines following
this command are ignored. If the \#END command is not
present, the end of the PARAM.in file indicates the end of the run.

There are several features of the input parameter file syntax
that allow the user to easily run the code
in a variety of modes while at the same time being able to 
keep a library of useful parameter files that can be used
again and again.

The syntax and the content of the input parameter files
is defined in the PARAM.XML files. The commands controlling
the whole SWMF are described in the main directory in the
\begin{verbatim}
  PARAM.XML
\end{verbatim}
file. The component parameters are described by the PARAM.XML
file in the component version directory. For example the
input parameters for the GM/BATSRUS component are described in
\begin{verbatim}
  GM/BATSRUS/PARAM.XML
\end{verbatim}
These files can be read (and edited) in a normal editor.
The same files are used to produce much of this
manual with the aid of the {\tt share/Scripts/XmlToTex.pl} script. 
The {\tt Scripts/TestParam.pl} script also uses these files
to check the PARAM.in file.
Copying small segments of the {\tt PARAM.XML} files
into {\tt PARAM.in} can speed up the creation or modification of a 
parameter file. 

\subsection{Included Files, {\tt \#INCLUDE} \label{section:include}}

The {\tt PARAM.in} file can include other parameter files with the 
command
\begin{verbatim}
#INCLUDE
include_parameter_filename
\end{verbatim}
The include files serve two purposes: (i) they help
to group the parameters; (ii) the included files can be reused
for other parameter files. 
An include file can include another file itself.
Up to 10 include files can be nested.
The include files have exactly the same structure as {\tt PARAM.in}. 
The only difference is that the
\begin{verbatim}
#END
\end{verbatim}
command in an included file means only the end of the include file, 
and not the end of the run, as it does in {\tt PARAM.in}.

The user can place his/her
included parameter files into the main run directory or in any subdirectory
as long as the correct path to the file from the run directory is
included in the {\tt \#INCLUDE} command.

\subsection{Commands, Parameters, and Comments \label{section:commands}}

As can be seen from the above examples, the parameters are entered
with a combination of a {\bf command} followed by specific {\bf parameter(s)},
if any.
The {\bf command} must start with a hashmark (\#), which 
is followed by capital letters and underscores without space in between. 
Any characters behind the first space or TAB character are ignored
(the \#BEGIN\_COMP and \#END\_COMP commands are the only exception,
but these are markers rather than commands).
The parameters, which follow, must conform to 
requirements of the command. They can be of four types: logical, integer,
real, or character string. Logical parameters can be entered as 
{\tt .true.} or {\tt .false.} or simply {\tt T} or {\tt F}.
Integers and reals can be in any of the usual Fortran formats.  In
addition, real numbers can be entered as fractions (5/3 for example).
All these can be followed by arbitrary comments, typically separated
by space or TAB characters. In case of the character type input
parameters (which may contain spaces themselves), the comments must
be separated by a TAB or by at least 3 consecutive space characters.
Comments can be freely put anywhere between two commands as long
as they don't start with a hashmark.

Here are some examples of valid commands, parameters, and comments:
\begin{verbatim}

#TIMEACCURATE
F                       DoTimeAccurate

Here is a comment between two commands...

#DESCRIPTION
My first run            StringDescription (3 spaces or TAB before the comment)

#STOP
-1.                     tSimulationMax
100                     MaxIteration

#RUN ------------ last command of this session -----------------

#TIMEACCURATE
T                       DoTimeAccurate

#STOP
10.0                    tSimulationMax
-1                      MaxIteration

#BEGIN_COMP IH

#GAMMA
5/3                     Gamma

#END_COMP IH

\end{verbatim}

\subsection{Sessions \label{section:sessions}}

A single parameter file can control consecutive {\bf sessions}
of the run. Each session looks like
\begin{verbatim}
#SOME_COMMAND
param1
param2

...

#STOP
max_simulation_time_for_this_session
max_iter_for_this_session

#RUN
\end{verbatim}
while the final session ends like
\begin{verbatim}
#STOP
max_simulation_time_for_final_session
max_iter_for_final_session

#END
\end{verbatim}
The purpose of using multiple sessions is to be able to change parameters 
during the run. For example one can use only a subset of the
components in the first session, and add more components in the
later session. Or one can obtain a coarse steady state solution
on a coarse grid with a component in one session, and improve on the solution
with a finer grid in the next session. Or one can switch from 
steady state mode to time accurate mode. The SWMF remembers parameter
settings from all previous sessions, so in each session one should only
set those parameters which change relative to the previous session.
Note that the maximum number of iterations given in the {\tt \#STOP} command 
is meant for the entire run, and not for the individual sessions. 
On the other hand, when a restart file is read, the iterations prior to 
the current run do not count.

The {\tt PARAM.in} file and all included parameter files are read into 
a buffer at the beginning of the run, so even for multi-session runs, 
changes in the parameter files have no effect once {\tt PARAM.in} 
has been read. 

\subsection{The Order of Commands \label{section:order}}

In essence, the order of parameter commands within a
session is arbitrary, but there are some important restrictions.  
We should note that the order of the parameters following 
the command is not arbitrary and must exactly match what the code requires.  
Here we restrict ourselves to the restrictions on the commands read by
the control module of SWMF. There may be (and are) restrictions
for the commands read by the components, but those are described
in the documentation of the components.

The only strict restriction on the SWMF commands is related
to the 'planet' commands. The default values of the 
planet parameters are defined by the \#PLANET command.
For example the parameters of Earth can be selected with the
\begin{verbatim}
#PLANET
Earth            NamePlanet
\end{verbatim}
command. The true parameters of Earth can be modified or simplified
with a number of other commands which {\bf must occur after the
\#PLANET command}. These commands are (without showing their parameters)
\begin{verbatim}
#IDEALAXES
#ROTATIONAXIS
#MAGNETICAXIS
#MAGNETICCENTER
#ROTATION
#DIPOLE
\end{verbatim}
Other than this strict rule, it makes sense to follow a 'natural'
order of commands. This will help in interpreting, maintaining
and reusing parameter files.

If you want all the input parameters to be echoed back, the first
command in {\tt PARAM.in} should be
\begin{verbatim}
#ECHO
T                 DoEcho
\end{verbatim}
If the code starts from restart files, it usually reads in a
file which was saved by SWMF. The default name of the saved
file is RESTART.out and it is written into the run directory.
It should be renamed, for example to RESTART.in, so that it
does not get overwritten during the run. It can be included as
\begin{verbatim}
#INCLUDE
RESTART.in
\end{verbatim}
The SWMF will read the following commands (the parameter values are
examples only) from the included file:
\begin{verbatim}
#DESCRIPTION
Create startup for GM-IM-IE-UA from GM steady state.

#PLANET
EARTH                        NamePlanet

#STARTTIME
    1998                     iYear
       5                     iMonth
       1                     iDay
       0                     iHour
       0                     iMinute
       0                     iSecond
 0.000000000000              FracSecond
 
#NSTEP
    4000                     nStep
 
#TIMESIMULATION
 0.00000000E+00              tSimulation
 
#VERSION
 2.00                        VersionSwmf
 
#PRECISION
8                              nByteReal
\end{verbatim}
The \#PLANET command defines the selected planet.
The \#STARTTIME command defines the starting date and time of the whole
simulation. The current simulation time (which is relative to
the starting date and time) and the step number are
given by the \#TIMESIMULATION and \#NSTEP commands. Finally
the \#VERSION and \#PRECISION commands check the consistency
of the current version and real precision with the run which
is being continued. For sake of convenience, the \#IDEALAXES,
\#ROTATEHGR and \#ROTATEHGI commands are also saved 
into the restart file if they were set in the run.

As it was explained above, all modifications of the planet 
parameters should follow the \#PLANET command, i.e. they should be after 
the \#INCLUDE RESTART.in command. In case the description is
changed it should also follow, e.g.
\begin{verbatim}
#INCLUDE
RESTART.in

#DESCRIPTION
We continue the run for another 2 hours
\end{verbatim}
When the run starts from scratch, the PARAM.in file
should start similarly with the 
\begin{verbatim}
#DESCRIPTION
This is the start up run

#PLANET
SATURN

#STARTTIME
    2004                     iYear
       8                     iMonth
      15                     iDay
       1                     iHour
      25                     iMinute
       0                     iSecond
 0.000000000000              FracSecond
\end{verbatim}
commands (the parameters are examples only).
These commands are typically followed by the planet parameter
modifying commands, if any, and setting time accurate mode
(if changed from default true to false or relative to the previous session).
For example:
\begin{verbatim}
! Align the rotation and magnetic axes with Z_GSE
#IDEALAXES

#TIMEACCURATE
F                           DoTimeAccurate
\end{verbatim}
All the commands which are written into the RESTART.out file and all 
the planet modifying commands can only occur in the first session.
These commands contain parameters which should not change during a run.
In the PARAM.XML file these commands are marked with an 
{\tt if="\$\_IsFirstSession"} conditional.
If any of these parameters are attempted to be changed in later sessions, 
a warning is printed on the screen and the code stops running
(except when the code is in non-strict mode).

Most command parameters have sensible default values.
These are described in the PARAM.XML files,
and in chapter \ref{chapter:commands} (which was produced from them).
The {\tt PARAM.XML} file also defines which commands are required
with the {\tt required="T"} attribute of the {\tt <command...>} tag.
For the control module the only required command in every
session is the \#STOP command
(or this can be replaced with the \#ENDTIME command in the last session), 
which defines the final time step in steady state mode 
or the final time of the session in time accurate mode.

\subsection{Iterations, Time Steps and Time \label{section:frequency}}

In several commands the frequency or `time' of some action has
to be defined. This is usually done with a pair of parameters.
The first defines the frequency or time in terms of the number of time steps,
and the second in terms of the simulation time.
A negative value for the frequency means that it should not be taken 
into account. For example, in time accurate mode,
\begin{verbatim}
#SAVERESTART
T            DoSaveRestart
2000         DnSaveRestart
-1.          DtSaveRestart
\end{verbatim}
means that a restart file should be saved after every 2000th time step, while
\begin{verbatim}
#SAVERESTART
T            DoSaveRestart
-1           DnSaveRestart
100.0        DtSaveRestart
\end{verbatim}
means that it should be saved every 100 seconds in terms of physical time.
Defining positive values for both frequencies might be useful
when switching from steady state mode to time accurate mode.
In the steady state mode the DnSaveRestart parameter is used,
while in time accurate mode the DtSaveRestart if it is positive.
But it is more typical and more intuitive 
to explicitly repeat the command in the first 
time accurate session with the time frequency set.

The purpose of this subsection is to try to help the user understand 
the difference between the iteration number used for stopping the code
and the time step which is used to define the frequencies of various
actions. After using \BATSRUS\ over several years, it is clear to the
authors that this distinction is useful and the
most reasonable implementation. The SWMF has inherited these
features from the \BATSRUS\ code.

We begin by defining several different quantities and the variables that 
represent them in the code.  The variable {\tt nIteration}, 
represents the number of ``iterations'' 
that the simulation has taken since it began running.  
This number starts at zero every time the code is run, even if beginning 
from a restart file.
This is reasonable since most users know how many iterations the code can take
in a certain amount of CPU time and it is this number that is needed when 
running in a queue.
The quantity {\tt nStep} is a number of ``time steps'' that the code has 
taken in total.  This number starts at zero when the code is started from 
scratch, but when started from a restart file, this
number will start with the time step at which the restart file was written.
This implementation lets the user output data files at a regular interval, even
when a restart happens at an odd number of iterations.
The quantity {\tt tSimulation} is the amount of simulated, or physical, 
time that the code has run.  
This time starts when time accurate time stepping begins.
When restarting, it starts from the physical time for the restart.
Of course the time should be cumulative since it is the physically meaningful
quantity.  We will 
use these three phrases( ``iteration'', ``time step'', ``time'') 
with the meanings outlined above.

Now, what happens when the user has more than one session and he or she
changes the frequencies.  Let us examine what would happen in the following
sample of part of a {\tt PARAM.in} file.  For the following example we will
assume that when in time accurate mode, 1 iteration simulates 1 second of time.
Columns to the right indicate the values of {\tt nITER}, {\tt n\_step} and
{\tt time\_simulation} at which restart files will be written in each session.

\clearpage

\begin{verbatim}
                                             Restart Files Written at:
==SESSION 1                         Session   nITER   nStep    time_simulation
#TIMEACCURATE                       --------  ------  -------  --------------
F            DoTimeAccurate  

#SAVERESTART                             1     200      200             0.0  
T            DoSaveRestart               1     400      400             0.0
200          DnSaveRestart
-1.0         DtSaveRestart

#STOP
400          MaxIteration
-1.          tSimulationMax

#RUN ==END OF SESSION 1== 
                         
#SAVERESTART                             2     600      600             0.0
T            DoSaveRestart               2     900      900             0.0
300          DnSaveRestart
-1.0         DtSaveRestart
				
#STOP				
1000         MaxIteration				
-1.          tSimulationMax
				
#RUN ==END OF SESSION 2== 

#TIMEACCURATE
T            DoTimeAccurate  		
				
#SAVERESTART                             3    1100     1100           100.0
T            DoSaveRestart               3    1200     1200           200.0
-1           DnSaveRestart               3    1300     1300           300.0
100.0        DtSaveRestart
				
#STOP				
-1           MaxIteration				
300.0        tSimulationMax			
				
#RUN ==END OF SESSION 3== 
                          
#SAVERESTART                             4    1400     1400           400.0
T            DoSaveRestart               4    1800     1800           800.0
-1           DnSaveRestart               4    2000     2000          1000.0
400.0        DtSaveRestart
 				
#STOP				
-1           MaxIteration				
1000.0       tSimulationMax				
				
#END  ==END OF SESSION 4== 
\end{verbatim}
Now the question is how many iterations will be taken and when will restart
file be written out.  In session 1 the code will make 400 iterations and will
write a restart file at time steps 200 and 400.  Since the iterations 
in the {\tt \#STOP}
command are cumulative, the {\tt \#STOP} command in the second session will
have the code make 600 more iterations for a total of 1000.  Since the timing
of output is also cumulative, a restart file will be written at time step 600
and at 900.
After session 2, the code is switched to time accurate mode.  Since we
have not run in this mode yet the simulated (or physical) time is cumulatively
0.  The third session will run for 300.0 simulated seconds (which for the
sake of this example is 300 iterations).  The restart file will be written
after every 100.0 simulated seconds.
The {\tt \#STOP} command in Session 4 tells the code to simulate  700.0 more 
seconds for a total of 1000.0 seconds.  The code will make a restart file
when the time is a multiple of 400.0 seconds or at 400.0 and 800.0 seconds.
Note that a restart file will also be written at time 1000.0
seconds since this is the end of a run.

In the next example we want to restart from 1000.0 seconds 
and continue with a time accurate run.
\begin{verbatim}
                                             Restart Files Written at:
==SESSION 1                         Session   nITER   nStep    time_simulation
                                    --------  ------  -------  --------------
#INCLUDE                                 1       0     2000          1000.0
RESTART.in

#TIMEACCURATE
T            DoTimeAccurate  

#SAVERESTART                             1     200     2200          1200.0
T            DoSaveRestart
-1           DnSaveRestart
600.0        DtSaveRestart

#STOP
-1           MaxIteration
1400.0       tSimulationMax

#RUN ==END OF SESSION 1== 

#SAVERESTART                             2     700     2700          1500.0
T            DoSaveRestart               2    1000     3000          2000.0
-1           DnSaveRestart
750.0        DtSaveRestart

#STOP
-1           MaxIteration
2000.0       tSimulationMax

#END ==END OF SESSION 2 = 
                          
\end{verbatim}
In this example, we see that in time accurate mode the simulated, or
physical, time is always cumulative.  To make 400.0 seconds more simulation,
the original 1000.0 seconds must be taken into account.  The final output 
at 2000.0 seconds is written because the run ended.

Throughout this subsection, we have used the frequency of writing restart files
as an example.  The frequencies of coupling components and checking stop
files work similarly. In the SWMF, and potentially in any of the
components, the frequencies are handled by the general
\begin{verbatim}
  share/Library/src/ModFreq
\end{verbatim}
module which is described in the reference manual.

%\end{document}

%  Copyright (C) 2002 Regents of the University of Michigan, portions used with permission 
%  For more information, see http://csem.engin.umich.edu/tools/swmf
\section{Execution and Coupling Control}

The control module of SWMF controls the execution and coupling of
components. The control module is controlled by the user through the
input parameter file PARAM.in.
Defining the most efficient component layout, execution and coupling control
is not an obvious task. In the current version of SWMF the processor
layout of the components is static. This restriction is somewhat
mitigated by the possibility of restart, which allows to change
the processor layout from one run to another.

\subsection{Processor Layout}

Within one run the layout is determined by the \#COMPONENTMAP
command in the PARAM.in file. The command
is documented in the PARAM.XML file.
Here we provide several examples which will help to develop
a sense of using optimal layouts.  An optimal layout is one that 
maximizes the use of all processors and does not leave processors
with nothing do while waiting for other processors to finish their work.

First of all we have to define the processor rank:
it is a number ranging from 0 to $N-1$, where
$N$ is the total number of processors in the run. 
A component can run on a subset of the processors,
which is defined by the rank of the first (root) processor,
the rank of the last processor, and the stride. For
example if the root processor has rank 4, the last processor
has rank 8, and the stride is 2 than the component will
run on 3 processors with ranks 4, 6 and 8.

\subsubsection{One component}

In the simplest case a single component, say the Global
Magnetosphere (GM) is running. The layout should be
the following
\begin{verbatim}
ID   Proc0 ProcEnd Stride
#COMPONENTMAP
GM   0     -1    1
\end{verbatim}
Here the -1 is interpreted as the rank of the last processor,
which is $N-1$ if the SWMF is running on $N$ processors.

\subsubsection{One serial and one parallel component}

When two components are used, their layouts may or may not overlap.
An example for overlapping the layouts of the GM and the
Inner Magnetosphere (IM) components is
\begin{verbatim}
ID   proc0 last stride
#COMPONENTMAP
IM    0       0    1
GM    0      -1    1
\end{verbatim}
When the component layouts overlap, the two components can run
sequentially only. Since IM is using a single processor only
(because it is not a parallel code), all the other processors 
will be idling while IM is running. This can be rather inefficient,
especially if the CPU time required by IM is not negligible.
A more efficient execution can be achieved with a non-overlapping layout:
\begin{verbatim}
ID   proc0 last stride
#COMPONENTMAP
IM    0       0    1
GM    1      -1    1
\end{verbatim}
Note that this layout file will work for any number 
of processors from 2 and up.

\subsubsection{Two parallel components with different speeds}

It is not always possible, or even efficient to use non-overlapping
layouts. For example both the SC and IH components require a lot of memory,
but the IH component runs much faster (say 100 times faster) 
in terms of cpu time than the SC component (this is due to the 
larger cells and smaller wave speeds in IH).
If we tried to use concurrent execution on 101 processors,
SC should run on 100 and IH on 1 processors to get good load balancing.
However the IH component needs much more memory than available
on a single processor. It is therefore not possible to use a non-overlapping
layout for SC and IH on a reasonable number of processors.

Fortunately both the Solar Corona (SC) and Inner Heliosphere (IH)
components are modeled by \BATSRUS, which is a highly parallel code
with good scaling. The following layout can be optimal:
\begin{verbatim}
ID   proc0 last stride
#COMPONENTMAP
IH    0    -1    1
SC    0    -1    1
\end{verbatim}
Although IH and SC will execute sequentially, they both
use all the available CPU-s, so no CPU is left waiting for the others.

\subsubsection{Two parallel components with similar speeds}

If two parallel components need about the same CPU time/real time
on the same number of processors, the optimal layout can be
\begin{verbatim}
ID   proc0 last stride
#COMPONENTMAP
GM    0     -1    2
SC    1     -1    2
\end{verbatim}
Here GM is running on the processors with even rank,
while SC is running on the processors with odd ranks.
By using the processor stride, this layout works on
an arbitrary number of processors.

When more serial and parallel codes are executing together,
finding the optimal layout may not be trivial. 
It may take some experimentation to see which component
is running slower or faster, how much time is spent
on coupling two components, etc. It may be a good idea
to test the components separately or in smaller groups
to see how fast they can execute.

\subsubsection{A complex example with four components}

Here is an example with 4 components: the Ionospheric
Electrodynamics (IE) component can run on 2 processors and 
runs about 3 times faster than real time.
The serial Inner Magnetosphere (IM) component runs even faster,
on the other hand the coupling of GM and IM is rather
computationally expensive. The Upper Atmosphere (UA) component
can run on up to 32 processors, and it runs twice as fast
as real time. The Global Magnetosphere model (GM) needs
at least 32 processor to run faster than real time.
If we have a lot of CPU-s, we may simply create a non-overlapping
layout. Since GM has no restriction on the number of processors,
it can be the last component in the map
\begin{verbatim}
ID   proc0 last stride
#COMPONENTMAP
IM    0       0    1
IE    0       1    1
UA    2      33    1
GM   34     999    1
\end{verbatim}
This layout will be optimal in terms of speed for a large 
(more than 100) number of PE-s, and actually the maximum
speed is going to be limited by the components which do
not scale. On a more modest number of PE-s one can try
to overlap UA and GM:
\begin{verbatim}
ID   proc0 last stride
#COMPONENTMAP
IM    0       0    1
IE    1       2    1
UA    0      31    1
GM    3     999    1
\end{verbatim}

\subsubsection{Using OpenMP threads}

Some of the models, such as \BATSRUS, can use OpenMP threads
in addition to the MPI paralelization. Typically one should
run one OpenMP thread on each core, and the number of MPI
processes should be 1 or 2 (or possibly more) for each node.
The most efficient arrangement depends on the hardware architecture
and the model. The number of maximum threads MaxThread is set by the
environment variable OMP\_NUM\_THREADS. Typically one wants
to use nThread=MaxThread threads for the components that can use
OpenMP. This can be easily achieved by setting the stried and the
number of threads in the last (optional) column to -1:
\begin{verbatim}
ID   proc0 last stride nthread
#COMPONENTMAP
GM    0      -1    -1      -1
\end{verbatim}
For example, if the node has 56 cores split to two independent
slots, the optimal setting is likely to be OMP\_NUM\_THREADS=28.
In this case both stride and nthread will be 28.

If OMP\_NUM\_THREADS is not in advance, it is best to set
the root of the multithreaded component to proc0=0, so
that the stride is properly aligned with the cores of
the nodes. This means that other components that can
only use a fixed number of processors should be put
to the last processors, for example
\begin{verbatim}
ID   proc0 last stride nthread
#COMPONENTMAP
GM    0      -3    -1      -1
IE   -2      -1     1
\end{verbatim}
In this layout GM is running with multiple threads on
cores 0 to $N-3$, while IE is using cores $N-2$ and $N-1$.

\subsection{Steady State vs. Time Accurate Execution}

The SWMF can run both in time accurate (default) and
steady state mode. This sounds surprising first, 
since many of the components can run in time accurate 
mode only. Nevertheless, the SWMF can improve the convergence
towards a steady state by allowing the different components
to run at different speeds in terms of the physical time.
In \BATSRUS\ the same idea is used on a much smaller scale:
local time stepping allows each cell to converge towards
steady state as fast as possible, limited only by the local
numerical stability limit.

\subsubsection{Steady state session}

The steady state mode should be signaled with the 
\begin{verbatim}
#TIMEACCURATE
F                DoTimeAccurate
\end{verbatim}
command, usually placed somewhere at the beginning of the session.
Since the SWMF runs in time accurate mode by default,
this command is required in the first steady state session of the run.

When SWMF runs in steady state mode, the SWMF time is not
advanced and tSimulation usually keeps its default initial value,
which is zero. 
The components may or may not advance their own
internal times. The execution is controlled by the 
step number {\tt nStep}, which goes from its initial value 
to the final step allowed by the MaxIteration parameter
of the \#STOP command. The components are called at
the frequency defined by the \#CYCLE command. For example
\begin{verbatim}
#CYCLE
GM               NameComp
1                DnRun

#CYCLE
IM               NameComp
2                DnRun
\end{verbatim}
means that IM runs in every second time step of the SWMF.
By defining the DnRun parameter for all the components,
an arbitrary relative calling frequency can be obtained,
which can optimize the global convergence rate to steady state.
The default frequency is DnRun=1, i.e. the component is
run in every SWMF time step. 

The relative frequency can be important for numerical
stability too. When GM and IM are to be relaxed
to a steady state, the GM/BATSRUS code is running in 
local time stepping mode, while IM/RCM runs in time 
accurate mode internally. Since GM and IM are coupled
both ways, an instability can occur if both GM and IM
are run every time step, because the GM physical time
step is very small, and the MHD solution cannot relax
while being continuously pushed by the IM coupling.
This unphysical instability can be avoided by calling the
IM component less frequently.

The coupling frequencies should be set to be optimal
for reaching the steady state. If the components are
coupled too frequently, a lot of CPU time is spent
on the couplings. If they are coupled very infrequently,
the solution may become oscillatory instead of relaxing
into a (quasi-)steady state solution. For example
we used the
\begin{verbatim}
#COUPLE2
IM                      NameComp1
GM                      NameComp2
10                      DnCouple
-1.                     DtCouple
\end{verbatim}
command to couple the GM and IM components in both directions
in every 10-th SWMF iteration.
Note that according to the above \#CYCLE commands,
GM and IM do 10 and 5 steps between two couplings,
respectively. GM/BATSRUS uses 10 local time steps,
while IM advances by 5 five-second time steps.

Another example is the relaxation of SC and IH components.
Under usual conditions the solar wind is supersonic at the 
inner boundary of the IH component, thus the steady state SC
solution can be obtained first, and then IH can converge
to a steady state using the SC solution as the inner boundary 
condition. In this second stage SC does not need to run
(assuming that it has reached a good steady state solution),
it is only needed for providing the inner boundary condition for IH.
This can be achieved by
\begin{verbatim}
! No need to run SC too often, it is already in steady state
#CYCLE
SC                      NameComp
1000                    DnRun

! No need to couple SC to IH too often
#COUPLE1
SC                      NameSource
IH                      NameTarget
1000                    DnCouple
-1.0                    DtCouple
\end{verbatim}
Since SC and IH are always coupled at the beginning of the session,
further couplings are not necessary.

\subsubsection{Time accurate session}

The SWMF runs in time accurate mode by default. The
\begin{verbatim}
#TIMEACCURATE
T                       DoTimeAccurate
\end{verbatim}
command is only needed in a time accurate session following a 
steady state session.
In time accurate mode the components advance in time at
approximately the same rate. The component times are
only synchronized when necessary, i.e. when they
are coupled, when restart files are written, or 
at the end of session and execution. Since the time
steps (in terms of physical and/or CPU time) of the components can be 
vastly different, this minimal synchronization provides the 
most possibilities for efficient concurrent execution.

In time accurate mode the coupling times have to be defined
with the DtCouple arguments. For example
\begin{verbatim}
#COUPLE2
GM                      NameComp1
IM                      NameComp2
-1                      DnCouple
10.0                    DtCouple
\end{verbatim}
will couple the GM and IM components every 10 seconds. 

In some cases the models have to be coupled every time step.
An example is the coupling between the MHD model GM/BATSRUS and 
the Particle-in-Cell model PC/IPIC3D. This can be achieved with
\begin{verbatim}
#COUPLE2TIGHT
GM                      NameMaster
PC                      NameSlave
T                       DoCouple
\end{verbatim}
command. In this case the master component (GM) tells the slave
component (PC) the time step to be used. The tight 
coupling requires models and couplers that support this option.

By default the component time steps are limited by the
time of couplings. This means that if GM can take 4 second
times steps, and it is coupled with IE every 5 seconds,
then every second GM time step will be truncated to 1 second.
There are two ways to avoid this. One is to choose the
coupling frequencies to be round multiples of the time steps
of the two components involved. This works well if both components
have fixed time steps and/or much smaller time steps than the 
coupling frequency.

In certain cases the efficiency can be improved with the
\#COUPLETIME command, which can allow a component to 
step through the coupling time. For example
\begin{verbatim}
#COUPLETIME
GM                      NameComp
F                       DoCoupleOnTime
\end{verbatim}
will allow the GM component to use 4 second time steps even
if it is coupled at every 5 seconds. Of course this will
make the data transferred during the coupling be 
first order accurate in time.

\subsection{Coupling order}

The default coupling order is usually optimal for accuracy
and consistency, but it may not be optimal for speed.
In particular, the IE/Ridley\_serial component solves
a Poisson type equation for the data received from the 
other components (GM and UA). For sake of accuracy
IE always uses the latest data received from the other
components. If GM, UA and IE are coupled
in the default order
\begin{verbatim}
#COUPLEORDER
4             nCouple	  
GM IE         NameSourceTarget
UA IE         NameSourceTarget
IE UA         NameSourceTarget
IE GM         NameSourceTarget
\end{verbatim}
and the to-IE and from-IE coupling times coincide, e.g.
\begin{verbatim}
#COUPLE2
GM            NameComp1
IE            NameComp2
10.0          DtCouple
-1            DnCouple

#COUPLE2
UA            NameComp1
IE            NameComp2
10.0          DtCouple
-1            DnCouple
\end{verbatim}
then GM and UA will have to wait until IE solves
the Poisson equation, because IE receives new data
and it is required to produce results immediately.
With the reversed coupling order
\begin{verbatim}
#COUPLEORDER
4             nCouple	  
IE UA	      NameSourceTarget
IE GM	      NameSourceTarget
GM IE	      NameSourceTarget
UA IE	      NameSourceTarget
\end{verbatim}
IE will provide the solution from the previously received data,
and it will have time to work on the new data while GM and UA
are working on their time steps. The reversed coupling order
allows the concurrent execution of IE with other components.
The temporal accuracy, on the other hand, will be somewhat worse.

To demonstrate that the coupling order is important, here
is a very {\bf inefficient} coupling order
\begin{verbatim}
#COUPLEORDER
4             nCouple	  
GM IE         NameSourceTarget
IE GM         NameSourceTarget
UA IE         NameSourceTarget
IE UA         NameSourceTarget
\end{verbatim}
in case the coupling times with GM and UA coincide (always at the beginning
of a the sessions).
With this coupling order, IE first receives information from GM,
then solves the Poisson equation and returns the information based
on the solution to GM while GM is waiting. Then IE receives extra
information from UA, solves the Poisson equation again, and sends
back information to UA, while UA is waiting. 

An alternative way to achieve concurrent execution is to
stagger the coupling times. For example the
\begin{verbatim}
#COUPLE2SHIFT
GM                 NameComp1
IE                 NameComp2
-1                 DnCouple
10.0               DtCouple
-1                 nNext12
0.0                tNext12
-1                 nNext21
5.0                tNext21
\end{verbatim}
will schedule a GM to IE coupling at 0, 10, 20, 30, \ldots seconds,
and the IE to GM coupling at 5, 15, 25, \ldots seconds.
This provides IE half the GM time to solve the Poisson equations.
If IE runs at least twice as fast as GM, this solution will
produce concurrent execution. The temporal accuracy is
somewhat better than in the reversed coupling case.
Note that GM and IE will be synchronized at 0, 5, 10, \ldots seconds,
which works best if the GM time step is an integer fraction of 5 seconds.


% Example runs
\chapter{Example Runs}
The examples in this chapter are intended to make you familiar
with the use of the SWMF. By carefully following the steps you should
be able to do the tests as described. It is a good idea to read the
provided PARAM.in and LAYOUT.in files and try to understand how
the examples work. You may also experiment by changing these files
after copied from the originals. These examples should help you
in setting up your own runs.

For sake of simplicity we describe how to do all the example runs with the
same executable. In actual runs one would streamline the configuration
of the SWMF to reduce compilation time and memory use. If you work
on a machine with limited resources, you may wish to configure SWMF
differently. For example, you may set the Empty version for the unused
components. If the number of processors is limited, you will have to
change the LAYOUT.in file and overlap the components. You may also have to 
increase the allowed grid size per processor, or you can run the problem with 
a coarser grid resolution. To reduce the CPU time, you may shorten the 
run by changing the number of iterations or the final time in the 
PARAM.in file.

\section{Configuration and Compilation for the Examples}

Select the SC/BATSRUS component version. When this is done the first time
after installation, the SC/BATSRUS source code is created from the GM/BATSRUS
source code, which takes a few minutes:
\begin{verbatim}
Config.pl -v=SC/BATSRUS -g=SC:4,4,4,1000
\end{verbatim}
Set the grid size for the GM/BATSRUS and IH/BATSRUS\_share components:
\begin{verbatim}
Config.pl -g=GM:8,8,8,200,40
\end{verbatim}
You can select either the UA/GITM version with
\begin{verbatim}
Config.pl -v=UA/GITM -g=UA:9,9,25,2,2
\end{verbatim}
or the new (default) UA/GITM2 version with
\begin{verbatim}
Config.pl -v=UA/GITM2 -g=UA:9,9,25,4
\end{verbatim}
Note the difference in the grid size parameters.
You should also take care of using the parameter files appropriate for
the selected UA component version.  GITM2 is the default UA module
currently in the SWMF and is the most up to date.

Check the current settings with
\begin{verbatim}
Config.pl -show
\end{verbatim}
You should see the current directory, the operating system, the name
of the compiler and the following settings
\begin{verbatim}
The default precision for reals is double precision.
The selected component versions and grid sizes are:
GM/BATSRUS                   grid: 8,8,8,200,40
IE/Ridley_serial             
IH/BATSRUS                   grid: 8,8,8,200,40
IM/RCM                       
RB/RiceV5                    
SC/BATSRUS                   grid: 4,4,4,1000
SP/Kota                      grid: 1000,10,150
UA/GITM2                     grid: 9,9,25,4
\end{verbatim}
In case you selected the UA/GITM version, the last line will read
\begin{verbatim}
UA/GITM                      grid: 9,9,25,2,2
\end{verbatim}
If the settings differ you can change them with the Config.pl script.

Compile the main executable code bin/SWMF.exe. With the above
settings this may take an hour, but if you select Empty versions
for the unused components, it will take much less.
You should also compile the post processing code bin/PostIDL.exe,
and finally create a run directory and change to that directory
\begin{verbatim}
make
make PIDL
make rundir
cd run
\end{verbatim}
Note that the run directory contains subdirectories for all the
non-empty components. There is also a link to the Param directory
in the main SWMF directory. The Param directory contains all the
parameter and layout files used in the example runs.

\section{Example 1: Create Steady State for the Solar Corona}

This example involves the SC component only. It demonstrates
how a steady state solar corona can be obtained from
a magnetogram. The convergence to steady state is accelerated by 
a gradual grid refinement and a gradual application of the
more-and-more accurate numerical schemes. The final state
is a steady state to high accuracy. 

You can use the SWMF with the settings recommended above
but you only need the SC component and
all other component versions can be Empty.

Copy the PARAM.in and LAYOUT.in files
\begin{verbatim}
rm -f PARAM.in LAYOUT.in
cp Param/PARAM.in.test.start.SC.AMR8 PARAM.in
cp Param/LAYOUT.in.test.start.SC LAYOUT.in
\end{verbatim}
We recommend the AMR8 version of the parameter file because it creates
a smaller grid and reaches steady state in smaller number of iterations
than the higher resolution version (AMR9). For the milestone problem
we used the higher resolution.

Check the parameter and layout files with the TestParam.pl script. 
This script should be run from the main directory.
Define the number of CPU-s you plan to use with the -n=NUMBER flag.
For example
\begin{verbatim}
cd ..
Scripts/TestParam.pl -n=32
\end{verbatim}
The script should return without any warnings and error messages.
If there are error messages, fix them. 
For example if the grid is not large enough there will be an 
error message from the command \#CHECKGRIDSIZE with respect
to the parameter  MinBlockALL, which contains the minimum number
of blocks required. To fix the problem, you have to increase the number 
of CPU-s or increase the grid size for the SC component with the 
Config.pl script. For example set
\begin{verbatim}
Config.pl -g=SC:4,4,4,1500
\end{verbatim}
Keep fixing the problems until the Scripts/TestParam.pl runs silently.
Remember to recompile SWMF.exe if you change the grid size.

Run the code by submitting a job or do it interactively
\begin{verbatim}
cd run
mpirun -np 32 SWMF.exe | tee runlog.SC
\end{verbatim}
Here 'tee' is a Unix command which splits the output to the screen as
well as pipes it to the file 'runlog.SC'. 

Depending on the number of processors and the speed of the machine,
this run should take a few to several hours to complete.
You may check the progress on the screen, or look at the 
runlog.SC file, or look at the log file written by the code
\begin{verbatim}
tail SC/IO2/log*
\end{verbatim}
This file contains information determined by the \#SAVELOGFILE command
in the PARAM.in file. In this case the file contains the time step, the
time (which is zero, because this is a steady state run) and the
magnetic, kinetic and thermal energies integrated over the surface of 
two spheres of radii 4 and 10. As the code approaches steady state,
the integrated energies will change less and less. 

When the run finishes successfully (or even while the code is running), 
you can postprocess the plot files
\begin{verbatim}
cd SC
pIDL
pTEC p r
cd ..
\end{verbatim}
The 'p' and 'r' flags for the pTEC script mean that the raw ASCII data
files are preprocessed with preplot into .plt files and the ASCII data
files are removed. If the machine does not have the 'preplot' code
installed (preplot is a script that comes with the Tecplot software),
you can gzip the .dat files to save disk space and transfer time
\begin{verbatim}
pTEC g
\end{verbatim}
You can visualize the output files in run/SC/IO2 with your favored 
visualization software. For visualization with IDL you should read the
chapter on ``IDL visualization'' in the user manual in GM/BATSRUS/Doc.

The restart files were saved into RESTART.SC and the SC/restartIN directory.
These are not the default names, they were set in the PARAM.in file with
the \#RESTARTFILE command for the control module of the SWMF and 
with the \#RESTARTOUTDIR command for the SC/BATSRUS component.
This makes it easier to use the run in the following example run.

The alternative and more typical approach would be to write into the
default restart file RESTART.out and the default directory SC/restartOUT,
and use the Restart.pl script to create a restart tree, e.g.
\begin{verbatim}
Restart.pl RESTART_SC
\end{verbatim}

\section{Example 2: Create SC-IH Steady State}

This example run is built on the previous example. We restart SC from the
steady state created in the previous run and start the IH component from 
scratch and run the two components coupled until the IH component reaches
a steady state. 

First copy in the prepared parameter and layout files
\begin{verbatim}
rm -f PARAM.in LAYOUT.in
cp Param/PARAM.in.test.restart.SCIH  PARAM.in
cp Param/LAYOUT.in.test.restart.SCIH LAYOUT.in
\end{verbatim}
Look at the PARAM.in file to see how the convergence to 
steady state is accelerated.
First of all the SC component only provides the boundary conditions for IH,
so it only runs in every 100th iteration (see the \#CYCLE command in
the PARAM.in file). Second, the IH grid is built
up with a gradual grid refinement, and third the 
more-and-more accurate and expensive numerical schemes are 
applied in an optimal sequence. The final state
is a steady state for SC and IH to high accuracy. 

You can use the SWMF with the settings recommended for all the examples,
but you only need the SC and IH components so 
all other component versions can be Empty.

As usual, check the input parameters and the layout for the
number of CPU-s you plan to use, for example
\begin{verbatim}
cd ..
Scripts/TestParam.pl -n=32
\end{verbatim}
If there are error messages, fix them until the script runs silently.

Run the code by submitting a job, or interactively
\begin{verbatim}
cd run
mpirun -np 32 SWMF.exe | tee runlog.SCIH
\end{verbatim}
When the run finishes, postprocess the plot files for both components
\begin{verbatim}
cd SC
pIDL
pTEC
cd ../IH
pIDL
pTEC p r
cd ..
\end{verbatim}
Due to the commands \#RESTARTFILE and \#RESTARTOUTDIR (in the IH section)
the output restart file is saved into RESTART.SCIH and the IH component
saved the restart state into the IH/restartIN. The SC component saved
the restart state into the default SC/restartOUT. For a continuation run
one could move and link the directories in SC with the following Unix commands
\begin{verbatim}
cd SC
mv restartIN restart_SConly
mv restartOUT restart_SCIH
mkdir restartOUT
ln -s  restart_SCIH restartIN
cd ..
\end{verbatim}
As you can see this is a rather cumbersome and error prone procedure.
It is better to save the restart state into their default file and
directories and use the Restart.pl script to collect them into 
a restart tree and to link the input file and directories to them.

\section{Example 3: Create Initial Conditions for the Global Magnetosphere}

This example involves the GM component only. It demonstrates
how a reasonable global magnetosphere can be obtained.
The upwind boundary conditions are based on the solution obtained
for the IH component in the 2nd example run, but they
are intentionally modified to contain a discontinuity.
This is for demonstration purposes only to make a subsequent
GM-IH coupled run more dynamic. In a more typical run one would
use the upwind boundary condition based on satellite measurements 
or by coupling to the IH code.

The convergence to steady state is accelerated by 
a gradual grid refinement, and a gradual application of the
more-and-more accurate numerical schemes. The final state
is a reasonable global magnetosphere solution, but it is not 
a perfect steady state (that would require many more iterations).

First copy in the prepared parameter and layout files and check them
\begin{verbatim}
rm -f PARAM.in LAYOUT.in
cp Param/PARAM.in.test.start.GM  PARAM.in
cp Param/LAYOUT.in.test.start.GM LAYOUT.in
\end{verbatim}
The parameters of the \#SOLARWIND command were obtained
from the result of the SC-IH steady state. 
Otherwise this test can be run independently, 
no restart files are read.
You can use the SWMF with the settings recommended for all the tests, 
but you only need the GM component, so all other component versions 
can be set to Empty.

Check the parameters and layout before running the SWMF
\begin{verbatim}
cd ..
Scripts/TestParam.pl -n=16
\end{verbatim}
If there are error messages, fix them until the script reports no errors.
Run the code by submitting a job, or interactively
\begin{verbatim}
cd run
mpirun -np 16 SWMF.exe | tee runlog.GM
\end{verbatim}
Postprocess the plot files
\begin{verbatim}
cd GM
pIDL
pTEC p r
cd ..
\end{verbatim}
The output restart file is RESTART.GM and the GM component saved
its restart files into GM/restart.IN as determined by the PARAM.in file.

\section{Example 4: Create Initial Conditions for the GM, IM, IE 
         and UA Components}

This example involves the GM, IM, IE and UA components. It demonstrates
how to obtain a reasonable solution for the 
coupled global/inner magnetosphere, ionosphere and upper atmosphere.
The initial global magnetosphere solution is taken from Example 3.

Copy the layout file into the run directory
\begin{verbatim}
rm -f PARAM.in LAYOUT.in
cp Param/LAYOUT.in.test.restart.GMIMIEUA LAYOUT.in
\end{verbatim}
If you are using the UA/GITM2 component version, copy this parameter file:
\begin{verbatim}
cp Param/PARAM.in.test.restart.GMIMIEUA.GITM2 PARAM.in
\end{verbatim}
If you are using the UA/GITM component version, copy this parameter file:
\begin{verbatim}
cp Param/PARAM.in.test.restart.GMIMIEUA PARAM.in
\end{verbatim}
Have a look at the PARAM.in file.
The convergence to steady state is accelerated by component subcycling
(see the \#CYCLE commands in PARAM.in).
The components are called at different frequencies, which
allows the components to reach a reasonable solution
approximately at the same rate. Another optimization trick is the changing of 
the coupling order such that the IE component sends information before
it would receive it (see the \#COUPLEORDER command in PARAM.in),
so that IE component can solve for the electric potential concurrently with the
other components running. To reduce the time spent on field line
tracing, which is needed for the IM to GM coupling, the \#RAYTRACE command
in the GM section limits the frequency of traces to every 80th iteration.

Now have a look at the LAYOUT file:
\begin{verbatim}
#COMPONENTMAP
UA   0    31    1
IM   32   32    1
IE   33   34    1
GM   35  999    1
#END
\end{verbatim}
As you can see all the components are running concurrently.
Unless you have access to at least 64 processors, 
you probably want to modify the layout file.
For example for 32 processors you can overlap the UA component 
with the other components and run IE on a single processor:
\begin{verbatim}
#COMPONENTMAP
UA    0   31    1
IE    0    0    1
IM    1    1    1
GM    2   31    1
#END
\end{verbatim}
You can use the SWMF with the settings recommended for all the tests, 
but you only need the GM, IM, IE and UA components,
all other component versions can be Empty.

Check the parameters and layout for the number of processors you plan to use
\begin{verbatim}
cd ..
Scripts/TestParam.pl -n=32
\end{verbatim}
If there are error messages, fix them. 
Run the code by submitting a job, or interactively
\begin{verbatim}
cd run
mpirun -np 32 SWMF.exe | tee runlog.GMIMIEUA
\end{verbatim}
This run should take an hour or so on 32 processors.
After the run postprocess the plot files
\begin{verbatim}
cd GM
pIDL
pTEC p r
cd ../IE
pION
cd ..
\end{verbatim}
The pION script concatenates the northern and southern output files
produced by the component IE/Ridley\_serial.

In this run all output restart information is saved into the
default file (RESTART.out) and default directories.
You can use the RESTART.pl script to collect them into a restart tree.
For example
\begin{verbatim}
Restart.pl -o RESTART_GMIMIEUA
\end{verbatim}

\section{Example 5: Time Accurate Run with SC, IH and SP Components}

This example involves the Solar Corona, Inner Heliosphere
and Solar Energetic Particles components.
The run starts from a steady state of the SC and IH components
which was obtained in Example 2.

Copy in the prepared parameter and layout files
\begin{verbatim}
rm -f PARAM.in LAYOUT.in
cp Param/PARAM.in.test.restart.SCIHSP PARAM.in
cp Param/LAYOUT.in.test.restart.SCIHSP LAYOUT.in
\end{verbatim}
At the beginning of this time accurate run a CME is 
generated in the SC component. This demonstrates the
use of an Eruptive Event generator.
The SP component is coupled, and during the 4 hour evolution
of the CME, substantial particle acceleration is observed. 
By the end of the run the CME reaches the boundary 
between the SC and IH components, and partially enters the IH domain.
The test demonstrates the coupling between the SC-IH and SP
components in a challenging simulation.

In case you have limited computational resources, you can 
shorten the run by editing the PARAM.in file and changing
the \#STOP command at the end of the file. For example 
\begin{verbatim}
#STOP
-1                     MaxIteration
1800.0                 tSimulationMax
\end{verbatim}
will reduce the final simulation time to 30 minutes.

Also have a look at the layout file:
\begin{verbatim}
#COMPONENTMAP
SC    1 999    1
IH    1 999    1
SP    0   0    1 
#END
\end{verbatim}
The SC and IH components are overlapped while the SP component
runs concurrently. This is probably the optimal layout for any
number of CPU-s.

You can use the SWMF with the settings for all the tests, 
but you only need the SC, IH and SP components,
all other component versions can be Empty.

Check the parameters and layout before running the SWMF
\begin{verbatim}
cd ..
Scripts/TestParam.pl -n=32
\end{verbatim}
If there are error messages, fix them, then run the SWMF by submitting a job
or interactively
\begin{verbatim}
cd run
mpirun -np 32 SWMF.exe | tee runlog.SCIHSP
\end{verbatim}
This run may take a long time if run all the way to 4 hours.
After the run finishes, postprocess the plot files
\begin{verbatim}
cd SC
pIDL
pTEC p r
cd ../IH
pIDL
pTEC p r
cd ..
\end{verbatim}
The restart files can be collected into a restart tree 
with the Restart.pl script. For the full 4 hour run you could use
\begin{verbatim}
Restart.pl -o RESTART_SCIHSP_4hr
\end{verbatim}

\section{Example 6: Time Accurate Run with All Eight Components}

This run is described in the TESTING manual. 
The initial conditions were created with the runs described in
this chapter using the higher resolution SC grid (AMR9) and the full 4 hour
time accurate run with the SC-IH and SP components.

For sake of learning SWMF you can do this example with lower SC resolution
(AMR8) and do a shorter time accurate run with the SC, IH and SP components.

Copy the layout file into the run directory
\begin{verbatim}
rm -f PARAM.in LAYOUT.in
cp Param/LAYOUT.in.test.restart.8comp LAYOUT.in
\end{verbatim}
If you are using the UA/GITM2 component version, copy this parameter file:
\begin{verbatim}
cp Param/PARAM.in.test.restart.8comp.GITM2 PARAM.in
\end{verbatim}
If you are using the UA/GITM component version, copy this parameter file:
\begin{verbatim}
cp Param/PARAM.in.test.restart.8comp PARAM.in
\end{verbatim}
This example is desinged to run for 600 seconds only.
It is assumed that the input restart file RESTART.in contains
the simulation time corresponding to 4 hours (14400 seconds),
so the final time in the \#STOP command is 15000 seconds
in the PARAM.in file.
In case you are restarting from a different simulation time and/or
you wish to change the length of the run, edit the maximum
simulation time in the \#STOP command in the PARAM.in file.

To simplify this example run, you can remove some of the components.
To remove a component you need to edit the LAYOUT.in file and remove the
appropriate line, and edit the PARAM.in file, and remove the 
section belonging to the component and all the couplings with the 
component. In fact, in some cases it may not be necessary to remove 
anything from the PARAM.in file. You may simply add an \#END command
after the section belonging to the last included component,
since the lines following the \#END command are ignored.
For example to run with the SC and IH components only, you can put an
\#END command after the \#END\_COMP IH command but before the coupling
with the SP component:
\begin{verbatim}
#END_COMP IH ---------------------------------------

#END

#COUPLE1
SC                      NameSource
SP                      NameTarget
-1                      DnCouple
60.0                    DtCouple
...

\end{verbatim}
Test the parameters and the layout as usual
\begin{verbatim}
cd ..
Scripts/TestParam.pl -n=64
\end{verbatim}
and run the code
\begin{verbatim}
cd run
mpirun -np 32 SWMF.exe | tee runlog.8comp
\end{verbatim}
After the run finishes, postprocess the plot files
\begin{verbatim}
cd SC
pIDL
pTEC p r
cd ../IH
pIDL
pTEC p r
cd ..
cd ../GM
pIDL
pTEC p r
cd ../IE
pION
cd ..
\end{verbatim}
Since this example run is very short, no restart files are saved
(see the \#SAVERESTART command).


% Complete List of Input Commands
\chapter{Complete List of Input Commands}

\input{SWMFXML}

% Index for the commands
%^CFG COPYRIGHT UM
\documentclass[twoside,10pt]{book}

\input HEADER

\title{Space Weather Modeling Framework User Manual \\ 
       \hfill \\
       \large Code Version \SWMFVERSION}
\author{G\'abor T\'oth, Ovsei Volberg, Aaron Ridley\\
       \hfill \\
       {\it Center for Space Environment Modeling}\\
       {\it The University of Michigan}}

\makeindex

\begin{document}

\pagestyle{fancy}
\lhead[\fancyplain{}{\bfseries\thepage}]{\fancyplain{}{\bfseries\rightmark}}
\rhead[\fancyplain{}{\bfseries\leftmark}]{\fancyplain{}{\bfseries\thepage}}
\cfoot{}
%\pagestyle{headings} % if fancy heading does not work

\maketitle

%\input{SWMF_copyrights}

\tableofcontents

% Introduction
\input{SWMF_introduction}

% Basic User manual
\chapter{The Basics}

\input{SWMF_configuration}
\input{SWMF_param}
\input{SWMF_control}

% Example runs
\chapter{Example Runs}
\input{SWMF_example_runs}

% Complete List of Input Commands
\chapter{Complete List of Input Commands}

\input{SWMFXML}

% Index for the commands
\input{SWMF.ind}

\end{document}




\end{document}




\end{document}






\end{document}
