%^CFG COPYRIGHT UM
%\documentclass{article}

\chapter{Data Naming Standard}
%\author{D. Chesney, G. T\'oth, D. Zeeuw}

%\begin{document}
%\maketitle

\section{Introduction}

The Center for Space Environment Modeling (CSEM) is developing
a software framework which consists of science modules representing
various physical domains, and the framework itself, which connects
the science modules. It is important to develop some data naming
standards so that the independently developed science modules 
and the framework can be compiled together. Also it is useful
to have consistent naming when many developers work together
on the same part of the software.

The naming standard refers to all variable, procedure and file names
used in the framework and the science modules.
Here {\it variable name} means any constant, variable, type, structure, 
object, class or Fortran 90 module name, while {\it procedure name} 
means subroutine, function, or method names in general.

The purposes of the naming standard are the following:
\begin{itemize}
\item Avoid name conflicts between various modules of the framework.
\item Extensibility for case when intra-module dataflow name is modified
      to inter-module dataflow.
\item Characteristics of variables, procedures and files are
      explicitly understood from the name.
\item Allow consistent names within modules developed my more than
      one programmer.
\item Improved code readability allows outside programmers to be used.
\item Language (F90, C++, etc.) independence of variable and procedure names.
\item Representation in Backus-Naur Form (BNF), and thus, ability to apply
      syntax checking tools.
\end{itemize}

\section{Basic Considerations}

The science modules are independent in terms of variables, on the
other hand, in the current bottom-up design the procedures
of all modules will be compiled into a single executable. 
This means that modules must avoid procedure name conflicts. 

For example the B0 module will be present on all processors, and it
has to have procedure names that differ from the procedure names
of the magnetosphere, ionosphere, heliosphere etc. modules.
This restriction does not exclude the possibility of using
more than one instance of the same general module. 
For example, if both the magnetosphere and the heliosphere 
are modeled by the same general magnetohydrodynamic module, 
the procedure names will be the same, thus these module {\it must} 
run on separate sets of processors.

Since the science modules are mostly developed by scientists and not
by software engineers, it seems reasonable to minimize the constraints
on the data naming. The minimum requirements seem to be the following
\begin{itemize}
\item All procedure names of a module must contain a unique identifier
      for the module.
\item The control module should be able to define or modify all file names 
      (e.g. input parameter and output data file names) used by the other
      modules. 
\end{itemize}
The procedure names must start with the two-letter module name
acronyms, and the control module will use a directory name containing
the same acronym for the input and output files for each module.

An important simplification is that we allow for 
{\it intra- and inter-module} naming standard. 
When a single science module is being developed,
one may use the intra-module naming system. When the science modules
are coupled together, an automatic Perl script can add the unique
prefixes for each procedures in each modules.

\section{Physics and Software Modules}

Physics modules represent a certain physics domain. There are
only a well defined set of these modules, which all have two
character abbreviations:
\begin{itemize}
\item[CE]{Cometary Environment}
\item[EE]{Eruptive Events}
\item[GM]{Global Magnetosphere}
\item[IE]{Ionosphere Electrodynamics}
\item[IH]{Inner Heliosphere}
\item[IM]{Inner Magnetosphere}
\item[IN]{Interstellar Neutrals}
\item[IO]{Ionosphere}
\item[OH]{Outer Heliosphere}
\item[PL]{Plasmasphere}
\item[PS]{Planetary Satellites}
\item[PW]{Polar Wind}
\item[RB]{Radiation Belt}
\item[SC]{Solar Corona}
\item[TH]{Thermosphere}
\end{itemize}
We also define {\tt MH} for the generic magnetohydrodynamic
physics module, which can model GM, IH, SC, OH, and possibly EE.

In addition to physics modules there will be some general purpose
software modules. One example is the {\tt TIMING} module, which
can time nested procedure calls, and report the timing results
in various ways. Other examples can be general parallel linear
solvers, error handlers, etc. These modules also have identifiers
in all capital letters, but the identifier can have arbitrary
length. The control module of the framework is also a 
software module, which controls and provides communication
between the physics modules.

We note here that physics and software modules have nothing to do
with Fortran 90 modules. Fortran 90 modules are simply collections
of variables and procedures within a physics or software module.

\section{File Names}

File names of source files should reflect the content.
If a source file contains a single procedure, it
should be named the same as the intra-module name of
the procedure with an extension specific to the language.
For example a Fortran 90 source code file containing
{\tt subroutine calc\_flux} should be named
{\tt calc\_flux.f90}. If the source code is in Fortran 77 or C/C++,
the extension is {\tt .f} or {\tt .c} respectively.
If a source file contains a set of procedures,
it should be named by a group name that describes
the set of procedures, or by the main procedure in the group.
If a source file containes a Fortran 90 module, it should
be named accordingly, e.g. the file containing the ModMain module
should be named {\tt ModMain.f90}.

Source files will reside in separate directories for each
module, therefore it is not necessary to include the
module identifier into the file name.

\section{Inter-Module Procedure Names}

We distinguish intra-module procedure names and inter-module 
procedure names. Intra-module procedure names are used within a module,
while inter procedure names are used in the framework.
The conversion between these will be automatized, for example with 
the aid of a Perl script. This means that the modules can use 
arbitrary procedure names, still the name conflicts can be avoided. 

The inter-module procedure names start with the physics/software module 
abreviation (see the list in the previous section) 
in all capitals, followed by an undescore, then followed by the
intra-procedure name. 

Here are some possible examples for the full inter-procedure names:
\begin{verbatim}
   name           ! meaning
----------------------------------------------------------------
   GM_calc_flux   ! Global magenetosphere flux calculation
   IH_get_var     ! inner heliosphere variable extraction
   MH_get_var     ! generic MHD variable extraction
   MH_update_b0   ! generic MHD B0 update
   TIMING_start   ! TIMING software module starts timing
\end{verbatim}
Note that in the examples above the MHD module is not associated with 
a single physics module, thus it uses the module name {\tt MH}.
The control module will only call the subroutines for specific 
physics modules, e.g. for the inner heliosphere. These subroutines
have to be all defined as simple interfaces for the
generic {\tt MH\_*} subroutine.

\section{Inter-Module Variable Names}

Similarly to procedure names we distinguish intra-module and
inter-module variable names. Intra-module variable names 
are used within each science module, and also in most software
modules. The main exception is the control module, which must
communicate with many other modules, and thus it uses 
inter-module variable names.

Inter-module variable names start with a prefix, which
consists of the two character abbreviation of the physics module,
followed by a type indicator, an optional dimension indicator,
and an underscore. The prefix is then followed by the
intra-module variable name. The type indicator can be
the following:
\begin{itemize}
\item[b] boolean
\item[h] 2-byte-integer
\item[i] 4-byte-integer
\item[j] 8-byte-integer
\item[r] real
\item[d] double precision real
\item[c] character
\item[s] string
\item[e] enumerated value
\item[m] F90 module
\item[t] type/structure
\end{itemize}
The dimension indicator is a positive integer number, which
gives the number of indices for array variables. For scalar
variables it is omitted. Some examples for prefixes:
\begin{verbatim}
   GMr3_   - Global magnetosphere, real 3D array
   IHi_    - Inner heliosphere, integer scalar
\end{verbatim}

\section{Intra-Module Procedure Names for BATSRUS}

The intra-module procedure name consists of 
{\it procedure name parts} separated by undescores. 
A procedure name part starts with a lower case letter, followed
by an arbitrary number of lower case letters and numbers.
The use of lower case letters and undescores between the procedure 
name parts helps to distinguish procedure names from variable names, 
which use capitalization (see later).
The intra-procedure name should describe the action done
by the procedure, so it typically starts with a verb. Examples:
\begin{verbatim}
   name                ! meaning
----------------------------------------------------------------
   advance_impl        ! advance in time with implicit scheme
   calc_face_flux      ! calculate face fluxes
   set_b0              ! set the B0 magnetic field
   set_ics             ! set initial conditions
   read_restart_file   ! generic MHD B0 update
\end{verbatim}

\section{Intra-Module Variable Names for BATSRUS}

The following suggestions were developed by G. T\'oth and D. De Zeeuw
for the magnetosphere module (the main part of BATSRUS).
The original system was later improved by D. Chesney.
The suggested naming system tries to resemble the current somewhat chaotic 
naming system to minimize the difficulties of the transition. 
More importantly we tried to create logical, unique, distinct and easy 
to read and write variable names. This naming system has been used in 
some recently written subroutines, and our limited experience shows that
it works reasonably well, and definitely better than the current lack
of system. 

\subsection{Name parts}

Each intra-module variable name may consist of one or more {\it name parts}.
All the parts must start with a capital letter and continue
with lower case letters and numbers. There are two exceptions
to this rule: (i) if the first name part consists of a single 
character, it should be lower case; (ii) the last name
part may be one of the three scope descriptors {\tt BLK, PE}, and 
{\tt ALL}, which consists of all capital letters. 
In Fortran capitalization is ignored
by the compiler, so mistakes in the capitalization
have no effect on the correctness of the code. On the other hand
consistent capitalization is essential to improve readability of the code.
Examples of correct capitalizations:
\begin{verbatim}
  i
  iMax
  UseConstrainB
  DtALL
  nBlockExpl
  CflImpl
  R2Min
  InnerBcType
\end{verbatim}

\subsection{Indication of variable type}

There are strict rules in this data naming standard to indicate 
the type (module, real, integer, logical, character string) of 
a variable. These rules are made as easy to read and write
as possible:

Fortran 90 module names start with 
\begin{verbatim}
  Mod
\end{verbatim}
All integer variable names must start with one of the following name parts:
\begin{verbatim}
  i
  j
  k
  l
  m
  n
  Di
  Dj
  Dk
  Dl
  Dm
  Dn
  Min
  Max
  Int
\end{verbatim}
All character and character string type variable names must start with
any of 
\begin{verbatim}
  Name
  Type
  String
\end{verbatim}
All logical variable names must start with one of
\begin{verbatim}
  Do
  Done
  Is
  Use
  Used
  Unused
\end{verbatim}
Finally all real type variable names must start with a name part 
which was not listed in any of the above lists.

In addition to the consistent data names, we must make sure that the compiler 
checks the declaration and use of variables, thus
the {\bf implicit none} statement {\bf must be
used in every single procedure} in Fortran source code. 

These rules should also help to make the order of the name parts
less arbitrary. For example the minimum of the pressure should
be named {\tt pMin} and not {\tt MinP}, because it is a real
number which cannot start with the name part {\tt Min} which 
is reserved for integers.
Examples:
\begin{verbatim}
  UseConstrainB       - logical
  UnusedBlock         - logical
  nBlockUsed          - integer
  iProc               - integer
  RhoSwDim            - real
  xTest               - real
  TypeFlux            - character string
  NamePlotFile        - character string
\end{verbatim}

\subsection{Array variable names}

Array variable names are distinguished from scalar variable names
by the indication of indexes. The indexes are represented by 
an underscore followed by capital case letters at the end 
of the array variable name. Each capital letter represents
one or more well defined indexes, and their order must be the
same as the order of indexes in the declaration of the array variable.
The following index abbreviations are defined:
\begin{verbatim}
name    typical range                 meaning
-------------------------------------------------
A       (1:nBlockALL)                 global blocks
B       (1:nBlock)                    local blocks
C       (1:nI,1:nJ,1:nK)              physical cells
D       (1:nDim)                      dimensions
E       (1:2*nDim)                    edges
F       (1:nI+1,1:nJ+1,1:nK+1)        faces
G       (-1:nI+2,-1:nJ+2,-1:nK+2)     ghost cells
I                                     general index (none of the others)
K       (1:nI+1,1:nJ+1,1:nK+1)        corners
N       (-1:1,-1:1,-1:1)              neighbors
P       (1:nProc)                     processors
Q       (1:4)                         quadrants
S       (1:2)                         sides
V       (1:nVar)                      (flow) variables
X       (1:nI+1,1:nJ,1:nK)            X faces
Y       (1:nI,1:nJ+1,1:nK)            Y faces
Z       (1:nI,1:nJ,1:nK+1)            Z faces
\end{verbatim}
If there are many general indexes, one can use {\tt \_I3} instead of
{\tt \_III}. Examples:
\begin{verbatim}
  Dx_B(:)             - real array indexed by blocks
  GradRho_C(:,:,:)    - real array indexed by cells
  B0x_XB(:,:,:,:)     - real array indexed by the X faces and blocks
  nRoot_D(:)          - integer array indexed by the 3 directions (x,y,z)
\end{verbatim}

\subsection{Indication of scope of variables}

Some scalar variables or array elements refer to a single cell, 
of cell face, others refer to a whole block, processor, or even
all the processors used by the module. For example in the current
BATSRUS code {\tt dt} is the time step for the whole simulation
domain (i.e. all the processors), {\tt dt\_BLK(1)} is the smallest time step 
for the first block, {\tt time\_BLK(1,1,1,1)} is the time step for
a single cell in the first block, while {\tt TimeCell} is the time
step for the cell being updated. The current names are 
somewhat arbitrary. In the new naming system the different
scopes of a scalar variable or array element can be indicated by the 
following name parts:
\begin{verbatim}
  BLK     - one block
  PE      - one processing element
  ALL     - all the processing elements used by the module
\end{verbatim}
These name parts are all capital, and they must be the last name part
for scalars, and the last name part before the underscore for arrays.
In the new naming system the above mentioned variables will become
\begin{verbatim}
New name    Old name     Meaning
---------------------------------------------------
DtALL      dt           global time step
DtBLK_B    dt_BLK       minimum time step for blocks
Dt_CB      time_BLK     local time step for cells
Dt         TimeCell     local time step for one cell
\end{verbatim}
The last two names of the scalar {\tt dt} and the four dimensional 
array {\tt dt\_CB} only differ in the array index abbreviations.

When the scope is not explicitly indicated, we assume the 
smallest scope that makes sense. For example {\tt nBlock}
is the number of blocks for a processor, while {\tt nBlockALL}
is the total number of blocks used by the module.

\subsection{Named indexes}

In the planned rewrite of BATSRUS, many variables will be collected into
more general arrays. For example the conservative variables  
{\tt rho\_BLK, rhoUx\_BLK,... E\_BLK} will be put into a single array
{\tt Var\_GVB}, i.e. the variable array is indexed by (ghost) cells, 
variable, and block number. Similarly the coordinate arrays 
{\tt x\_BLK, y\_BLK, z\_BLK} will be merged into {\tt Xyz\_GDB},
i.e. the coordinate array is indexed by ghost) cells,
directions, and block number. 

To make the variable and dimension indexes easier to read, 
{\it named indexes} are introduced. A named index is
an integer constant (defined with the parameter statement in Fortran 90).
The named index consists of the usual name parts followed by
an underscore. The underscore is a reminder that named indexes
have to do with arrays, and it also makes the syntax of named
indexes different both from scalar and array variable names. 
Here is a list of named indexes (to be) introduced in BATSRUS:
\begin{verbatim}
name      value   meaning
-------------------------------------------------
x_        1       X index
y_        2       Y index
z_        3       Z index
Rho_      1       density index
RhoU_     1       momentum index
RhoUx_    2       X momentum index
RhoUy_    3       Y momentum index
RhoUz_    4       Z momentum index
B_        4       magnetic field index
Bx_       5       Bx index
By_       6       By index
Bz_       7       Bz index
e_        8       energy index
p_        9       pressure index
\end{verbatim}
Examples of use:
\begin{verbatim}
  ! F_x[rho] = rho*U_x
  Flux_FDV(i,j,k,x_,Rho_)=Var_FV(i,j,k,RhoUx_)

  ! F_i[rho] = rhoU_i
  Flux_FDV(i,j,k,iDim,Rho_)=Var_FV(i,j,k,RhoU_+iDim)
\end{verbatim}
Note the use of {\tt RhoU\_+iDim} which gives the index for the
{\tt iDim}-th component of the momentum. Both the flux and
the variable arrays are face centered.

When velocity is used instead of momentum and/or spherical components
are used instead of Cartesian, the following named indexes can be used:
\begin{verbatim}
name      value   meaning
-------------------------------------------------
U_        1       velocity index
Ux_       2       X velocity index
Uy_       3       Y velocity index
Uz_       4       Z velocity index
r_        1       Radial coordinate index
Phi_      2       Phi coordinate index
Theta_    3       Theta coordinate index
Ur_       2       Radial velocity index
Uphi_     3       Phi velocity index
Utheta_   4       Theta velocity index
Br_       2       Radial magnetic field index
Bphi_     3       Phi magnetic field index
Btheta_   4       Theta magnetic field index
\end{verbatim}

\subsection{Lists of name parts}

\subsubsection{General}

\begin{verbatim}
name      meaning
-------------------------------------------------
c         constant (real, parameter ::)
d         difference
Dim       dimensional/dimension
Do        do something or not (logical)
Dt        time step, time interval
Dn        difference in time step index
i         index or index in direction X
j         2nd index or index in direction Y
k         3rd index or index in direction Z
Min       minimum of
Max       maximum (number) of
n         number of
Par       parameter
Read      read input file
Write     write output file
Save      save output file
Coeff     coefficient
Crit      criteria
Test      test of
Use       use a module/scheme (logical)
\end{verbatim}

\subsubsection{Basic flow variables}

\begin{verbatim}
name      meaning
-------------------------------------------------
x         X direction/coordinate
y         Y direction/coordinate
z         Z direction/coordinate
xyz       coordinate in general
R         radial distance from origin
R2        radial distance from 2nd body
Time      time
Gamma     adiabatic index
Clight    speed of light
Rho       density
U         velocity
Ux        x component of velocity
Uy        y component of velocity
Uz        z component of velocity
P         pressure
T         temperature
E         energy
Ex        x component of electric field
Ey        y component of electric field
Ez        z component of electric field
B         magnetic field
Bx        x component of magnetic field
By        y component of magnetic field
Bz        z component of magnetic field
B0        split part of field
B0x       x component of split field
B0y       y component of split field
B0z       z component of split field
Jx        x component of current
Jy        y component of current
Jz        z component of current
Var       variable in general
Flux      flux
Source    source term
Sw        solar wind
Cme       coronal mass ejection
Arc       magnetic arcade in the solar corona 
\end{verbatim}

\subsubsection{Discretization}

\begin{verbatim}
name      meaning
--------------------------------------------------------
Inner     inner boundary at the surface of the body
Outer     outer boundary of the computational domain
Upstream  upstream boundary towards the sun
Iter      iterations for this run)
Step      time step for the whole (restarted) simulation
Stage     stage of the time discretization
Update    update of variables to the next time step
Expl      explicit time stepping
Impl      implicit time stepping
Bdf2      Backwad Difference Formula 2
Cfl       Courant-Friedrich-Lewy number (<1.0 for explicit)
Krylov    Krylov type iterative linear solver
Bicgstab  Bi-Conjugate Gradient STABilized scheme (Krylov type)
Gmres     General Minimum RESidual scheme (Krylov type)
Matvec    matrix vector multiplication in iterative solver
Newton    Newton iteration for non-linear problem
Schwarz   Schwarz type blockwise preconditioning
Precond   preconditioner to accelerate convergence
Mbilu     modified block incomplete lower-upper preconditioner
Rusanov   Rusanov's numerical flux function
Linde     Linde's numerical flux function
Roe       Roe's numerical flux function
Sokolov   Sokolov's numerical flux (artificial wind)
Grad      gradient
Limiter   function that limits slopes to avoid oscillations
Beta      Beta limiter
Minmod    Minmod slope limiter
Mc        Monotonized Central slope limiter
Eps       small number
Project   projection of B field to eliminate div B
Diff      diffusion of divergence of B
Powell    Powell's source term to advect div B
Constrain constrained transport to conserve div B=0
\end{verbatim}

\subsubsection{AMR grid}

\begin{verbatim}
name      meaning
-------------------------------------------------
Amr       Adaptive mesh refinement
Cell      cell
Coarsen   coarsening of grid
Face      face
Block     block
Nei       neighbour of
Lev       refinement level
Octree    octree of blocks
Prolong   prolongation of data for refinement
Refine    refinement of grid
Restrict  restriction of data for coarsening
Root      root of octree
Proc      processing element
Used      used block/node
Unused    unused block/node
\end{verbatim}


%\end{document}





