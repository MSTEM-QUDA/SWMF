%^CFG COPYRIGHT UM
\chapter{Introduction \label{chapter:introduction}}

\BATSRUS\ stands for Block Adaptive Tree Solar-wind Roe Upwind Scheme.
This name, while not complete in describing the code, especially in
its newer incarnations, points out some of \BATSRUS' main features.
Specifically, \BATSRUS\ originally solved the MHD equations using a
finite volume upwind Roe-type scheme.  Currently there are several
different solvers available.  The computational region in \BATSRUS\ is
made up of logically Cartesian blocks of cells that can be adaptively
refined to give higher resolution in a restricted part of the domain.
The division of blocks into smaller blocks creates a tree like
structure of blocks, where a divided block has eight children (in 3D), 
and the blocks are connected to other blocks much like the branches of a tree.
Finally, \BATSRUS\ is most commonly run to model the solar wind
interaction with solar system bodies.

More recently, \BATSRUS\ has been extended for high energy density plasma
(HEDP) applications. While the code is essentially the same, the applications
are drastically different, so the code used for HEDP is 
named CRASH (Code for Radiative Shock Hydrodynamics) 
and the corresponding development is supported and carried
out by the Center for Radiative Shock Hydrodynamics.

The \BATSRUS\ (CRASH) code is a first principles hydrodynamic (HD) and 
magnetohydrodynamic (MHD) model which has been used to simulate the Earth's 
magnetosphere, the solar convection zone, corona, inner and outer heliospheres, 
the magnetosphere of most of the planets, several moons and various comets.
The code can be extended for use to any problem for which the hydrodynamic
and MHD equations are a reasonable physical model. The CRASH code is used
to model radiative shocks generated by extremely strong laser pulses 
in laboratory plasmas.

The \BATSRUS\ code is the most important building block of the Space
Weather Modeling Framework (SWMF). The SWMF executes and couples a
number physics models, components as a single model.  \BATSRUS\ is
used in multiple roles in SWMF: it can model the Solar and Lower Corona 
(SC and LC components), the Inner and Outer Heliosphere (IH and OH components) 
and the the Global Magnetoshere (GM component).  
A lot of effort was spent on making sure
that the very same source code, scripts, test suites and makefiles are
used in the stand alone \BATSRUS\ and in the various components.  We
tried to change the behavior of the standalone version as little as
possible.

This document is aimed at providing the user detailed information about
installation, compilaton and execution of the code, and 
how one can change the physical and numerical parameters
to achieve the desired result. The tools provided for visualization
are also described.

The physics and the numerics contained in the code are described in
the DESIGN document. That document should help
the user understand the design philosophy behind the code, the 
available physics that the code contains and the numerical algorithms that
make the code work. This document may not be fully up-to-date regarding
software implementation. For more detailed description of the equations 
and algorithms read the published papers. A review paper describing
both the SWMF and \BATSRUS\ is Toth et al., 2011, Journal of Computational
Physics, doi:10.1016/j.jcp.2011.02.006. The algorithms of the CRASH
code are described by van der Holst et al., 2011, Astrophysical Journal
Supplements, doi:10.1088/0067-0049/194/2/23. Both papers contain numerous
references to earlier publications.

\section{Acknowledgments}

\BATSRUS\ was developed at the University of Michigan starting in 1996
with funding under the NASA High Performance Computing and Communications (HPCC)
Earth and Space Sciences (ESS) program (NASA ESS Cooperative Agreement 
Number: NCCS5-146).  Continued work is funded by various grants from
NFS, NASA, AFOSR, DoD, and for the CRASH code by DoE.
The Center for Space Environment Modeling is lead by Tamas Gombosi, while
the Center for Radiative Shock Hydrodynamics is lead by Paul Drake.

Contributions to the development of \BATSRUS\ fall roughly into three
categories: theoretical development, code development and scientific
investigations. The principle players and their involvement is as follows:

\begin{tabbing}
{\bf Principle Investigators and Theory} \\
Tamas Gombosi \hspace{.25in} \= 1994- \hspace{0.35in} \= 
                          Principle Investigator, MHD theory \\
Aaron Ridley   \> 2007-     \> Principal Investigator, magnetospheric physics \\
Paul Drake     \> 2008-     \> Principal Investigator, radiative shock theory \\
Ken Powell     \> 1994-     \> Co-Principle Investigator, numerics and algorithm development \\
Quentin Stout  \> 1994-     \> Co-Principle Investigator, 
                          parallel architecture and computer science \\
\> \> \\
{\bf \BATSRUS/CRASH Code Development} \\
Darren DeZeeuw \> 1996-     \>  \\
Hal Marshall   \> 1996-1998 \>  \\
Clinton Groth  \> 1997-1999 \>  \\ 
Gabor Toth     \> 1999-     \>  \\
Igor Sokolov   \> 2001-     \>  \\
Bart van der Holst \> 2008- \>  \\
Lars Daldorff  \> 2010-     \>  \\
\> \> \\
{\bf Code Development and Science} \\
Aaron Ridley   \> 1999-     \> Magnetosphere \\
Gabor Toth     \> 1999-     \> Magnetosphere, Extended MHD, CRASH \\
K.C. Hansen    \> 1999-     \> Planets, Moons, Comets \\
Chip Manchester\> 2000-     \> Eruptive Events, Corona, Helioshpere \\
Ilia Roussev   \> 2001-2004 \> Corona, Heliosphere \\
Igor Sokolov   \> 2001-     \> Corona, Heliosphere, CRASH \\
Bart van der Holst \> 2008- \> Eruptive Events, Corona, Heliosphere, CRASH \\
\end{tabbing}

\section{Web Pages for \BATSRUS\ and CRASH}
The web page of the Center for Space Environment Modeling can be found at
\begin{verbatim}
http://csem.engin.umich.edu/
\end{verbatim}
The web page of the Center for Radiative Shock Hydrodynamics is at
\begin{verbatim}
http://aoss-research.engin.umich.edu/crash/
\end{verbatim}
The SWMF, which includes \BATSRUS/CRASH, is publicly available at
\begin{verbatim}
http://csem.engin.umich.edu/swmf
\end{verbatim}
