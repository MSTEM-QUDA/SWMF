%^CFG COPYRIGHT UM
%\documentclass{article}
%\begin{document}

\section{PARAM.in \label{section:param.in}}

The input parameters for the \BATSRUS\ code are read from the 
{\tt PARAM.in} file which must be located in the run directory.
The file controls all of the \BATSRUS\ functionality and therefore
can become quite complicated. However,
there are several features of this file and of the \BATSRUS\
input scripting that allow the user to easily run the code
in a variety of modes while at the same time being able to 
keep a library of useful parameter files that can be used
again and again.

The user should be aware of and become intimately attached to the 
{\bf {\tt Param/FULL}} file.  This file contains the most detailed 
description and complete list of the all the input parameters
used by \BATSRUS. {\bf It is there as a reference only,
and it should not be included in its entirety into {\tt PARAM.in}!} 
Copying small segments of it
into {\tt PARAM.in} may be useful, however.  In fact, a typical use
of this file is to build a {\tt PARAM.in} file by cut and pasting
pieces of {\tt FULL}. 

The {\tt Param.in} file is ended with the command string
\begin{verbatim}
#END
\end{verbatim}
indicating that this is the end of the run.

\section{Included Files, {\tt \#INCLUDE} \label{section:include}}

The {\tt PARAM.in} file can include other parameter files with the 
command
\begin{verbatim}
#INCLUDE
include_parameter_filename
\end{verbatim}
The include files serve two purposes: (i) they help
to group the parameters; (ii) the included files can be reused
for other parameter files. 
And include file can include another file itself.
Up to 10 include files can be nested.
The include files have exactly the same structure as {\tt PARAM.in}. 
The only difference is that the
\begin{verbatim}
#END
\end{verbatim}
command in an included file means only the end of the include file, 
and not the end of the run, as it does in {\tt PARAM.in}.

The user can place his/her
included parameter files into the main run directory or in any subdirectory
as long as the correct path to the file from the run directory is
included in the {\tt \#INCLUDE} command.
There are many include files in the {\tt Param} directory. These
can be included into the {\tt PARAM.in} files, or they can serve as
examples. 

\section{Commands, Parameters, and Comments \label{section:commands}}

As can be seen from the above examples,  parameters are entered
with a combination of a {\bf command} followed by specific {\bf parameter(s)},
if any.
The {\bf command} must start with a hashmark (\#), which is followed by capital 
letters without space in between. Any characters behind the first 
space or TAB character are ignored. The parameters, which follow, must conform to 
requirements of the command. They can be of four types: logical, integer,
real, or character string. Logical parameters can be entered as 
{\tt .true.} or {\tt .false.} or simply {\tt T} or {\tt F}.
Integers and reals can be in any of the usual Fortran formats.
All these can be followed by arbitrary comments, typically separated
by space or TAB characters. In case of the character type input
parameters (which may contain spaces themselves), the comments must
be separated by a TAB or by at least 3 consecutive space characters.
Comments can be freely put anywhere between two commands as long
as they don't start with a hashmark.

Here are some examples of valid commands, parameters, and comments:
\begin{verbatim}
#RESTART
.false.                 restart (from a previous run)

Here is a comment between two commands...

#INNERBOUNDARY
ionosphereB0   innerBCtype (3 spaces or TAB before the comment)

#STOP
100		nITER

#RUN ------------ last command of this session -----------------

#BORIS
T                       boris_correction
0.10                    boris_cLIGHT_factor

\end{verbatim}

\section{Sessions \label{section:sessions}}

A single parameter file can control consecutive {\bf sessions}
of the run. Each session looks like
\begin{verbatim}
#SOME_COMMAND
param1
param2

...

#STOP
max_iter_for_this_session

#RUN
\end{verbatim}
while the final session ends like
\begin{verbatim}
#STOP
max_iter_for_final_session

#END
\end{verbatim}
The purpose of using multiple sessions is to be able to change parameters 
during the run. For example one can obtain a coarse steady state solution
with a low order scheme in the first session, improve on the solution
with a better scheme and finer grid in the second session, then switch
to time accurate mode in the third session. The code remembers parameter
settings from all previous sessions, so in each session one should only
set those parameters which change relative to the previous session.
Note that the maximum number of iterations given in the {\tt \#STOP} command 
is meant for the entire run, and not for the individual sessions. 
On the other hand, when a restart file is read, the iterations prior to 
the current run do not count.

The {\tt PARAM.in} file and all included parameter files are read into a buffer 
at the beginning of the run, so even for multi-session runs, changes in
the parameter files have no effect once {\tt PARAM.in} has been read.

\section{The Order of Commands \label{section:order}}

In essence, the order of parameter commands ({\tt \#COMMAND}) is arbitrary, 
but there are some important restrictions.  We should note that the 
order of the parameters following the command are not however arbitrary and must
exactly match what the code requires.  

If you want all the input parameters to be echoed back, the first
command in {\tt PARAM.in} should be
\begin{verbatim}
#TEST
read_inputs
\end{verbatim}
If other subroutines are to be tested or timed, the test string
may contain more words separated by spaces. In that case the
location and time of tests can be given with the {\tt \#TESTIJK},
{\tt \#TESTXYZ} and {\tt \#TESTTIME} commands.

After setting the test parameters, the next command should be
\begin{verbatim}
#PROBLEMTYPE
problem_type_number
\end{verbatim}
The problem type sets the default values for many parameters,
which may be overwritten by the following commands. If the 
problem type was set later it could overwrite the previous
settings with the default values!

There are several parameters which should not be changed once the code has begun
to execute.  In other words, these parameters can be defined only during the first
session.  These parameters typically either involve geometries which 
cannot be changed, or 
physical parameters for which there is no reasonable reason to change.
\begin{verbatim}
#PROBLEMTYPE 
#GRID 
#RESTART
#AMRINIT
#GAMMA
#SOLARWIND
#MAGNETOSPHERE
#SECONDBODY
#DIPOLE
#COROTATION
#HELEOSPHERE
#SHOCKTUBE
\end{verbatim}
If any of these parameters are attempted to be changed in later sessions, 
a warning is printed on the screen, and the command is ignored.

Another important restriction has to do with time accurate versus
steady state runs. The default is steady state mode. To switch
to time accurate mode, use for example
\begin{verbatim}
#TIMESTEPPING
T       time_accurate         (default is false)
2       nSTAGE                (default is 1 stage)
0.8     cfl                   (default is 0.8)
\end{verbatim}
For many commands, the parameters are slightly
different for the two modes. For example, the stop condition has
only one parameter in local time stepping (steady state) mode
\begin{verbatim}
#STOP
100      nITER Last/maximum iteration_number for this session.
\end{verbatim}
and in the time accurate mode
\begin{verbatim}
#STOP
1000     nITER Last/maximum iteration_number for this session.
100.     t_max (in seconds)
\end{verbatim}
In the latter case the code stops if either of the two limits 
(on the number of iterations and on the physical simulation time) is exceeded.
Similarly, the frequency of saving files 
({\tt \#SAVEPLOT}, {\tt \#SAVELOGFILE}, {\tt \#SAVERESTART})
 can be defined in terms of time
steps as well as physical time for time accurate mode. 
Therefore the time accurate mode must be set before any of these
commands are entered for a given session. 

A negative value for the frequency means that it should not be taken 
into account. For example, in time accurate mode,
\begin{verbatim}
#SAVERESTART
T            save_restartfile
2000         dn_output(restart_)
-1.          dt_output(restart_)
\end{verbatim}
means that a restart file should be saved after every 2000th time step, while
\begin{verbatim}
#SAVERESTART
T            save_restartfile
-1           dn_output(restart_)
100.0        dt_output(restart_)
\end{verbatim}
means that it should be saved every 100 second in terms of physical time.
Defining positive values for both frequencies is unlikely to be useful.
The default value is $-1$. for both dn\_output and dt\_output. Therefore,
if the code is switched from steady state to time accurate mode in a session, 
the frequency in terms of time steps remains valid. Of course, the frequencies
can be (and probably should be) changed explicitly for the time accurate
session after the {\tt \#TIMESTEPPING} and before the {\tt \#RUN}
 or {\tt \#END} commands.


\section{Command Defaults \label{section:defaults}}

A quick glance the the {\tt Param/FULL} file shows that there is an overwhelmingly
large number of input parameters to \BATSRUS.  Especially daunting is the long
list of parameters which controls details of the numerical methods that the
code uses.  Fortunately, many of these will never need to be set by the 
beginning user.  

\BATSRUS\ sets many of the parameters to reasonable values in the source file 
{\tt read\_inputs.f90}.  The two routines
{\tt set\_defaults} and {\tt set\_problem\_defaults} set the appropriate values.
The defaults have been chosen because they work for the vast
majority of problems for which the code has been run.  The {\tt FULL} file outlines
which commands are defaulted and the following section in the users manual
also indicates which commands are defaulted and which are required.  In general,
commands which deal with details of numerics are defaulted to reasonable values.
These may (and should) need to be changed eventually by most users to achieve the
desired code speed or solution accuracy, but will
work reasonably well with the defaults for most problems.
Physics parameters are also defaulted by the problem type and should work fine
when running the code on a previously run problem.  This will be commonly changed
by users as they begin to run their own simulations. Finally, geometry, problem type,
flow control (time stepping, start and stops, etc.) and plotting output 
are generally not defaulted.

\section{Iterations and Frequency of Output \label{section:frequency}}

The purpose of this section is to try to help the user understand what at
first may seem like an inconsistent use of stopping frequencies and output
frequencies.  After using \BATSRUS\ over several years, it is clear to the
authors that the code is used in specific ways and the frequencies are
very specifically designed to meet these needs and is the 
most reasonable implementation. The main ``inconsistencies''
come into play when the code is restarted, but let us begin by 
elaborating on time stepping and output frequencies in the code.

We begin by defining several different quantities and the variables that 
represent them in the code.  The variable {\tt nITER}, represents the number
of ``iterations'' that the simulation has taken since it began running.  This number
starts at zero every time the code is run, even if beginning from a restart file.
This is reasonable since most users know how many iterations the code can take
in a certain amount of CPU time and it is this number that is needed when running
in a queue.
The quantity {\tt n\_step} is a number of ``time steps'' that the code has 
taken in total.  This number starts at zero
when the code is started from scratch, but when started from a restart file, this
number will start with the time step at which the restart file was written.
This implementation lets the user output data files at a regular interval, even
when a restart happens at an odd number of iterations.
The quantity {\tt time\_simulation} is the amount of simulated, or physical, time
that the code has run.  This time starts when time accurate time stepping begins.
When restarting, it starts from the physical time for the restart.
Of course the time should be cumulative since it is the physically meaningful
quantity.  We will 
use these three phrases( ``iteration'', ``time step'', ``time'') with the meanings
outlined above.

As outlined in section~\ref{section:order}, commands that ask for a 
frequency of doing something (say 
writing the restart file) or for ending a session asks for either a number of
iterations, time steps or a physical (simulated) time, depending on whether the time
stepping mode is local or time accurate.
With the {\tt \#SAVERESTART} command, for example, in local time stepping mode
the command with its parameters would be
\begin{verbatim}
#SAVERESTART
T            save_restartfile
2000         dn_output(restart_)
\end{verbatim}
and a restart file would be written to disk every 2000 iterations. In
time accurate mode, the form must be 
\begin{verbatim}
#SAVERESTART
T            save_restartfile
2000         dn_output(restart_)
10.0         dt_output(restart_)
\end{verbatim}
Where the input for {\tt dn\_output} is an integer and for {\tt dt\_output}
is a real.
A negative value for the frequency means that it should not be taken 
into account. For example, in time accurate mode,
\begin{verbatim}
#SAVERESTART
T            save_restartfile
2000         dn_output(restart_)
-1.          dt_output(restart_)
\end{verbatim}
means that a restart file should be saved after every 2000th time step, while
\begin{verbatim}
#SAVERESTART
T            save_restartfile
-1           dn_output(restart_)
100.0        dt_output(restart_)
\end{verbatim}
means that it should be saved every 100 seconds in terms of physical time.
Defining positive values for both frequencies is unlikely to be useful but will
work.  The code would write a restart file at both frequencies.

Now, what happens when the user has more than one session and he or she
changes the frequencies.  Let us examine what would happen in the following
sample of part of a {\tt PARAM.in} file.  For the following example we will
assume that when in time accurate mode, 1 iteration simulates 1 second of time.
Columns to the right indicate the values of {\tt nITER}, {\tt n\_step} and
{\tt time\_simulation} at which restart files will be written in each session.
\begin{alltt}
                                             Restart Files Written at:
==SESSION 1       \hfill        Session   nITER   n_step   time_simulation
#TIMESTEPPING	  \hfill        --------  ------  -------  --------------
F       time_accurate  
1       nSTAGE         
0.8     cfl            

#SAVERESTART                      \hfill  1       200      200              0.0  
T            save_restartfile     \hfill  1       400      400              0.0
200          dn_output(restart_)

#STOP
400          nITER

#RUN ==END OF SESSION 1== 
                         
#SAVERESTART			  \hfill  2       600      600              0.0
T            save_restartfile	  \hfill  2       900      900              0.0
300          dn_output(restart_)
				
#STOP				
1000         nITER				
				
#RUN ==END OF SESSION 2== 
                          
#TIMESTEPPING 			
				
T       time_accurate  		
2       nSTAGE         		
0.8     cfl            		
				
#SAVERESTART			  \hfill  3      1100     1100            100.0
T            save_restartfile	  \hfill  3      1200     1200            200.0
-1           dn_output(restart_)  \hfill  3      1300     1300            300.0
100.0        dt_output(restart_)
				
#STOP				
-1           nITER				
300.0        t_max			
				
#RUN ==END OF SESSION 3== 
                          
#SAVERESTART                      \hfill   4      1400     1400            400.0
T            save_restartfile	  \hfill   4      1800     1800            800.0
-1           dn_output(restart_)  \hfill   4      2000     2000           1000.0
400.0        dt_output(restart_)
 				
#STOP				
-1           nITER				
1000.0       t_max				
				
#END  ==END OF SESSION 4== 
                           
                           
\end{alltt}
Now the question is how many iterations will be taken and when will restart
file be written out.  In session 1 the code will make 400 iterations and will
write a restart file at time steps 200 and 400.  Since the iterations 
in the {\tt \#STOP}
command are cumulative, the {\tt \#STOP} command in the second session will
have the code make 600 more iterations for a total of 1000.  Since the timing
of output is also cumulative, a restart file will be written at time step 600
and at 900.
After session 2, the code is switched to time accurate mode.  Since we
have not run in this mode yet the simulated (or physical) time is cumulatively
0.  The third session will run for 300.0 simulated seconds (which for the
sake of this example is 300 iterations).  The restart file will be written
after every 100.0 simulated seconds or in other words at time steps
1100, 1200 and 1300.
The {\tt \#STOP} command in Session 4  tells the code to simulate  700.0 more 
seconds for a total of 1000.0 seconds.  The code will make a restart file
when the time is a multiple of 400.0 seconds or at 400.0 and 800.0 seconds (1400 and
1800 time steps).  Note that a restart file will also be written at time 1000.0
seconds
(time step 2000) since this is the end of a run.

Hopefully it is clear how the simulation is stopped and when output is written.
Unfortunately, when restarting, things change just slightly. 
Let us say that we want to restart running from where our previous example left off.
We had written a final restart file at 1000.0 seconds of simulated time, which
was 2000 time steps.  We want to have the following {\tt PARAM.in} file
executed. 
\begin{alltt}
                                             Restart Files Written at:
==SESSION 1      \hfill        Session   nITER   n_step   time_simulation
                 \hfill        --------  ------  -------  --------------
#RESTART                         \hfill            0     2000           1000.0
T

#TIMESTEPPING
F       time_accurate  
1       nSTAGE         
0.8     cfl            

#SAVERESTART                      \hfill  1       200     2200           1000.0
T            save_restartfile
1100         dn_output(restart_)

#STOP
500          nITER

#RUN ==END OF SESSION 1== 

#SAVERESTART                      \hfill  2       700     2700           1000.0
T           save_restartfile	  \hfill  2      1000     3000           1000.0
300         dn_output(restart_)

#STOP
1000        nITER

#END ==END OF SESSION 2== 
                          
\end{alltt}
Since we switched to local time stepping we only have to worry about iteration
number and time steps.  Here we notice the difference in the {\tt \#STOP} command when
restarting and looking at the iteration number.  With a restart, 
the {\tt \#STOP} command does not consider the cumulative number of time steps but starts
again.  However, the output frequency is based on the cumulative time step.  
The simulation will make 500 iterations
in the first session.  This would cumulatively be 2500 time steps.  The restart
file will be written out at 2200 cumulative time steps or at 200 iterations into this
session.  The second session will make 500 more iterations for a total of 1000
in this run or 3000 time steps over all.  A restart file will be written out at multiples
of 300 time steps taken relative to the cumulative total number of time steps.  In other
words at 2700 and 3000 time steps over all, which is at 700 and 1000 iterations in
this run.

This example shows how iterations for stopping are cumulative within a run but
are not at restart, but that output is always based on the cumulative number of time steps.

Now let us take one last example.  We want to restart from 1000.0 seconds (2000 time steps)
just as in the previous example, but we want to continue with a time accurate run.
\begin{alltt}
                                             Restart Files Written at:
==SESSION 1      \hfill        Session   nITER   n_step   time_simulation
                 \hfill        --------  ------  -------  --------------
#RESTART                         \hfill            0     2000           1000.0
T

#TIMESTEPPING
T       time_accurate  
2       nSTAGE         
0.8     cfl            

#SAVERESTART                      \hfill  1       200     2200           1200.0
T            save_restartfile
-1           dn_output(restart_)
600.0        dt_output(restart_)

#STOP
-1           nITER
1500.0       t_max

#RUN ==END OF SESSION 1== 

#SAVERESTART                      \hfill  2       700     2700           1500.0
T            save_restartfile	  \hfill  2      1000     3000           2000.0
-1           dn_output(restart_)
750.0        dt_output(restart_)

#STOP
-1           nITER
2000.0       t_max

#END ==END OF SESSION 2 = 
                          
\end{alltt}
In this example, we see that in time accurate mode the simulated, or
physical, time is always cumulative.  To make 500.0 seconds more simulation,
the original 1000.0 seconds must be taken into account.  In this example,
since each second is 1 iteration, the restart file would be written at the
same time steps as in the previous example.  The final output (at 2000.0 seconds)
in this case
is not because a frequency was hit but because
the run ended.

Throughout this section, we have used the frequency of writing restart files
as an example.  The frequencies of writing plot files, writing logfiles and
doing AMR work similarly.
When some of these files are written, they have in the file name a time step
number.  This number is always the cumulative number of time steps.

%\end{document}








