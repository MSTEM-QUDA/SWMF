
Once you have the output from GITM in a bunch of output files, there
are a few IDL programs that you can use to visualize them:
\begin{enumerate}
\item {\bf thermo\_plotsat:} This is plotting 1D results.  It is the
  most commonly used program for plotting 1D.  It can be used for
  plotting satelite files and for plotting 1D simulations.  It is
  relatively straight forward to use, but experimentation can be help.
  This is an actual program, so you have to {\tt .run} it.
\item {\bf thermo\_gui:} This is a graphical user interface code for
  plotting 3D results that is somewhat simplistic.  The filename has
  to be entered manually in the upper left.  You then have to press
  the button for loading the file.  Variables appear on the left side,
  and you can select which one you want to plot.  You then select
  which plane you would like to look at (lat/lon, lat/alt, lon/alt).
  You can scroll through which altitude/latitude/longitude you want to
  look at.  You can also add wind vectors on, plot in polar
  coordinats, and plot the log of the variable.
\item {\bf thermo\_batch\_new:} This code will let you look at at 3D
  files exactly the same way as thermo\_gui, but is all scripted.
  There are a few features that this has that thermo\_batch doesn't
  have: (1) you can use wildcards for the file name, so it can read in
  a list of files.  The postscript file name will be appended with
  numbers. (2) When plotting a lat/alt plane, you can do a zonal
  average. (3) You can do a global average.
\item {\bf thermo\_plotter:} All of the above plotting codes will only
  plot one plot per page.  This code will plot many more than one plot
  per page.  You can plot multiple variables on the same page, or multiple
  files with the same variable, or both.
\end{enumerate}


