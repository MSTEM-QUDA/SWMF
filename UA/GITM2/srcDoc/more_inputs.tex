
\section{Auxiliary Input Files}


\subsection{IMF and Solar Wind}

This file controls the high-latitude electric field and aurora when
using models that depend on the solar wind and interplanetary magnetic
field (IMF).  It allows for dynamically controlling these quantities.
You can create realistical IMF files or hypothetical ones.  For
realistic IMF files, we typically use CDF files downloaded from the
CDAWEB ftp site, and IDL code that merges the solar wind and IMF files
to create one single file.  The IDL code also propagates the solar
wind and IMF from L1 to 32 Re upstream of the Earth.  You can use the
{\tt DELAY} statement to shift the time more (e.g.,in the example
below, it shifts by an additional 15 minutes).  The IDL code to
process the CDF files is called {\tt cdf\_to\_mhd.pro}.  It requires
both a solar wind file and an IMF file. For example, the IMF file
would be {\tt ac\_h0\_mfi\_20011231\_v04.cdf} and the solar wind file
would be {\tt ac\_h0\_swe\_20011231\_v06.cdf}.  The code assumes that
the data starts at {\tt \#START}, and ends when it encounters an
error.  This can mean that if there is an error in the data somewhere,
the code will only read up to that point.  To validate that the solar
wind and IMF is what you think it is, it is recommended that you use
the IDL code {\tt imf\_plot.pro}.

Here is an example file:
\begin{verbatim}

This file was created by Aaron Ridley to do some
wicked cool science thing.

The format is:
 Year MM DD HH Mi SS mS   Bx  By   Bz     Vx   Vy   Vz    N        T

Year=year
MM = Month
DD = Day
HH = Hour
Mi = Minute
SS = Second
mS = Millisecond
Bx = IMF Bx GSM Component (nT)
By = IMF By GSM Component (nT)
Bz = IMF Bz GSM Component (nT)
Vx = Solar Wind Vx (km/s)
Vy = Solar Wind Vy (km/s)
Vz = Solar Wind Vz (km/s)
N  = Solar Wind Density (/cm3)
T  = Solar Wind Temperature (K)

#DELAY
900.0

#START
 2000  3 20  2 53  0  0  0.0 0.0  2.0 -400.0  0.0  0.0  5.0  50000.0
 2000  3 20  2 54  0  0  0.0 0.0  2.0 -400.0  0.0  0.0  5.0  50000.0
 2000  3 20  2 55  0  0  0.0 0.0  2.0 -400.0  0.0  0.0  5.0  50000.0
 2000  3 20  2 56  0  0  0.0 0.0  2.0 -400.0  0.0  0.0  5.0  50000.0
 2000  3 20  2 57  0  0  0.0 0.0  2.0 -400.0  0.0  0.0  5.0  50000.0
 2000  3 20  2 58  0  0  0.0 0.0  2.0 -400.0  0.0  0.0  5.0  50000.0
 2000  3 20  2 59  0  0  0.0 0.0  2.0 -400.0  0.0  0.0  5.0  50000.0
 2000  3 20  3  0  0  0  0.0 0.0 -2.0 -400.0  0.0  0.0  5.0  50000.0
 2000  3 20  3  1  0  0  0.0 0.0 -2.0 -400.0  0.0  0.0  5.0  50000.0
 2000  3 20  3  2  0  0  0.0 0.0 -2.0 -400.0  0.0  0.0  5.0  50000.0
 2000  3 20  3  3  0  0  0.0 0.0 -2.0 -400.0  0.0  0.0  5.0  50000.0
 2000  3 20  3  4  0  0  0.0 0.0 -2.0 -400.0  0.0  0.0  5.0  50000.0
\end{verbatim}


To actually read in this file, in {\tt UAM.in}, use the command:

\begin{verbatim}
#MHD_INDICES
filename
\end{verbatim}

\subsection{Hemispheric Power}

These files describe the dynamic variation of the auroral power going
into each hemisphere.  Models such as \cite{fuller87} use the
Hemispheric Power to determine which level of the model it should use.
The Hemispheric Power is converted to a Hemispheric Power Index using the 
formula (check):
\begin{equation}
HPI = 2.09log(HP)^{1.0475}
\end{equation}

Example File 1:

\begin{verbatim}
# Prepared by the U.S. Dept. of Commerce, NOAA, Space Environment Center.
# Please send comments and suggestions to sec@sec.noaa.gov 
# 
# Source: NOAA POES (Whatever is aloft)
# Units: gigawatts

# Format:

# The first line of data contains the four-digit year of the data.
# Each following line is formatted as in this example:

# NOAA-12(S)  10031     9.0  4    .914

# Please note that if the first line of data in the file has a
# day-of-year of 365 (or 366) and a HHMM of greater than 2300, 
# that polar pass started at the end of the previous year and
# ended on day-of-year 001 of the current year.

# A7    NOAA POES Satellite number
# A3    (S) or (N) - hemisphere
# I3    Day of year
# I4    UT hour and minute
# F8.1  Estimated Hemispheric Power in gigawatts
# I3    Hemispheric Power Index (activity level)
# F8.3  Normalizing factor

2000
NOAA-15(N)  10023    35.5  7    1.085
NOAA-14(S)  10044    25.3  7     .843
NOAA-15(S)  10114    29.0  7     .676
NOAA-14(N)  10135   108.7 10    1.682
NOAA-15(N)  10204    36.4  7    1.311
.
.
.
\end{verbatim}

The second example shows a file format that is much better.  This
format started in 2007, while all of the files before this time are of
the first example type.

\begin{verbatim}
:Data_list: power_2010.txt
:Created: Sun Jan  2 10:12:58 UTC 2011


# Prepared by the U.S. Dept. of Commerce, NOAA, Space Environment Center.
# Please send comments and suggestions to sec@sec.noaa.gov 
# 
# Source: NOAA POES (Whatever is aloft)
# Units: gigawatts

# Format:

# Each line is formatted as in this example:

# 2006-09-05 00:54:25 NOAA-16 (S)  7  29.67   0.82

# A19   Date and UT at the center of the polar pass as YYYY-MM-DD hh:mm:ss
# 1X    (Space)
# A7    NOAA POES Satellite number
# 1X    (Space)
# A3    (S) or (N) - hemisphere
# I3    Hemispheric Power Index (activity level)
# F7.2  Estimated Hemispheric Power in gigawatts
# F7.2  Normalizing factor

2010-01-01 00:14:37 NOAA-17 (N)  1   1.45   1.16
2010-01-01 00:44:33 NOAA-19 (N)  1   1.45   1.17
.
.
.
\end{verbatim}


\subsection{Solar Irradiance}


More to come here.

\subsection{Satellites}


\begin{verbatim}
#SATELLITES 
2                    nSats 
guvi.2002041623.in 
15.0                 SatDtPlot 
stfd.fpi.in 
60.0                 SatDtPlot

\end{verbatim}

Here is a sample satellite input file:

\begin{verbatim}
year mm dd hh mm ss msec long lat alt
#START
2002 4 16 23 34 25 0 299.16 -2.21 0.00 
2002 4 16 23 34 25 0 293.63 -1.21 0.00 
2002 4 16 23 34 25 0 291.28 -0.75 0.00 
2002 4 16 23 34 25 0 289.83 -0.45 0.00 
2002 4 16 23 34 25 0 288.79 -0.21 0.00 
2002 4 16 23 34 25 0 287.98 -0.01 0.00 
2002 4 16 23 34 25 0 287.32  0.16 0.00 
2002 4 16 23 34 25 0 286.76  0.31 0.00 
2002 4 16 23 34 25 0 286.26  0.46 0.00 
2002 4 16 23 34 25 0 285.81  0.60 0.00 
2002 4 16 23 34 25 0 285.39  0.74 0.00
\end{verbatim}

At this time, GITM ignores the altitude, and just outputs the entire column.

