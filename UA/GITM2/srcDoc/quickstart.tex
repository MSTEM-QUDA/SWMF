\section{Extracting the code from a tar file}

Create a new and empty directory, and open the tar file you received,
e.g.:
\begin{verbatim}
mkdir Gitm 
cd Gitm 
mv ../gitm.tgz . 
tar -xvzf gitm.tgz 
\end{verbatim}

\section{Checking out the code with CVS}

If CVS (Concurrent Versions System) is available on your computer and
you have an account on the CVS server machine herot.engin.umich.edu,
you can use CVS to install the current or a particular version of the
code. First of all have the following environment variables:
\begin{verbatim}
setenv CVSROOT UserName@herot.engin.umich.edu:/CVS/FRAMEWORK
setenv CVS_RSH ssh
\end{verbatim}
where UserName is your user name on herot. Here it is assumed that you
use csh or tcsh. Also put these settings into your .cshrc file so it
is automatically executed at login. Alternatively, use
\begin{verbatim}
CVSROOT=UserName@herot.engin.umich.edu:/CVS/FRAMEWORK 
export CVSROOT 
CVS_RSH=ssh 
export CVS_RSH
\end{verbatim}
under sh, ksh and bash shells, and also put these commands into your
.bashrc or .profile file so it is automatically executed at login.

Once the CVS environment variables are set, you can download the
current (HEAD) version of the GITM distribution with
\begin{verbatim}
cvs checkout GITM2
\end{verbatim}

If you want a particular version, use
\begin{verbatim}
cvs checkout -r v2_0 GITM2
\end{verbatim}
where v2\_0 is the it tag associated with the version. To download bug
fixes or new features, the
\begin{verbatim}
cvs update 
\end{verbatim}
command can be used. See {\tt man cvs} for more information.

A lot of times, you don't really want the {\tt GITM2} directory to
stay that name, since you might download a couple different version
(maybe one for development and one for runs).  Therefore, typically
you will:
\begin{verbatim}
mv GITM2 GITM2.Development
\end{verbatim}

\section{Configuring and Making GITM}

In order to compile GITM, you have to configure it first.  The
configure script is inherited from the Space Weather Modeling
Framework.  There are two primary reasons you need to do the
configure: (1) put the right {\tt Makefile} in the right place,
specifying the compiler and the version of MPI that you will use to
link the code; (2) put the right MPI header in the right place.  It
also does some things like hard-codes the path of the source code into
the {\tt Makefile}. Currently the configure script is not capable of detecting the system and compilers available.  Some examples for commonly used set-ups are shown below.

Installing on Nyx:
\begin{verbatim}
./Config.pl -install -compiler=ifortmpif90 -earth
\end{verbatim}

Installing on a Mac with gfortran and openmpi:
\begin{verbatim}
./Config.pl -install -compiler=gfortran -earth
\end{verbatim}

Installing on a computer with gfortran and not using MPI:
\begin{verbatim}
./Config.pl -install -compiler=gfortran -earth -nompi
\end{verbatim}

Sometimes people have a hard time with the ModUtilities.F90 file.  If
you have errors with this file, try (for example):
\begin{verbatim}
./Config.pl -uninstall
./Config.pl -install -compiler=gfortran -earth -noflush
\end{verbatim}

Don't forget, after configuring the Makefiles, you must still compile the code!
\begin{verbatim}
make
make test_earth
make install
\end{verbatim}

\section{Running the Code}

GITM requires a bunch of files to be in the right place in order to
run.  Therefore, it is best to use the makefile to create a run
directory:
\begin{verbatim}
make rundir
mv run myrun
\end{verbatim}
where {\tt myrun} can be whatever you want.  I will use {\tt
myrun} as an example.  You can actually put this directory where ever
you want.  On many systems (such as nyx), there is a {\tt nobackup} scratch disk that
you are supposed to use for runs, instead of your home directory.  If you need to ensure that your home directory doesn't use too much space, moving the run directory onto a disk with more free space can solve the problem:

\begin{verbatim}
make rundir
mv run /nobackup/myaccount/gitm/myrun
ln -s /nobackup/myaccount/gitm/myrun .
\end{verbatim}

This creates a shortcut to the {\tt myrun} directory location on {\tt nobackup} in your GITM working directory.  It allows you to treat the run directory as if it were a local directory, but it isn't!  It also means that you don't have to compile and install GITM on the scratch disk, where program storage may not be allowed.

Once you have created the run directory, you can run the default simulation, by:

\begin{verbatim}
cd myrun
mpirun -np 4 GITM.exe
\end{verbatim}
This, hopefully should run GITM for Earth for 5 minutes.  If it
doesn't work, then you might have mpi set up incorrectly.  The default
is to allow you to run 4 blocks per processor, and the default {\tt
UAM.in} file is set up for 4 blocks, so you could try just running
GITM without mpi, just to see if it works at all:
\begin{verbatim}
./GITM.exe
\end{verbatim}
If that doesn't work, then it probably didn't compile correctly.
Hopefully, it just worked!

\section{Post Processing}
\label{post_process.sec}

GITM, by default, produces one file per block per output.  If you are outputting often and you are running with many blocks, you can produce a huge number of files.  To post process all of these files, simply:

\begin{verbatim}
cd UA
./pGITM
\end{verbatim}

This merges all of the files for one time period, for one file type into the same file.  You can actually running this while the code is running, since GITM doesn't use old files, unless you are using the APPENDFILE option.  As implied by the option's name, APPENDFILE opens an existing file and appends the most recent data to it.  This feature is typically used only when running a satellite track though GITM.  More information on the APPENDFILE option is located in Chapter~\ref{input.ch} Section~\ref{def_out.sec}.

If you are NOT using satellites and NOT using APPENDFILE, then you are free and clear to use pGITM as often as you want during a run.  To avoid deleting a file that GITM is currently writing to, it is recommended that pGITM be run as part of a script in which there is a five minute pause between executions.

Another useful script, when running GITM on another system, is given in the block below.  It occasionally executes an rsync between the computer that you run GITM on and your home computer.  This allows you to bring over and evaluate the output files as they become available.  To do this, execute pGITM at a set cadence (60 seconds in the example) while GITM is running and then rsync the remote and home directories (excluding all unprocessed files).  Finally, remove the processed and rsynced files to prevent the remote directory from filling up.  Be sure to replace yourname@home.computer with your user and computer names!  This is a very simple, but very useful script.

\begin{verbatim}
#!/bin/csh
rm -f stop
set LOC=$1

if (-f remoteloc) set LOC=`cat remoteloc`

while (!(-f stop))
  rsync -vrae ssh log.* UAM.* imf* yourname@home.computer:$LOC
  cd UA ; ./pGITM ; rsync --exclude '*.[bsh][0123ae][0123456789at]*' -vrae ssh d
ata yourname@home.computer:$LOC ; cd ..
  sleep 60
end
\end{verbatim}

\section{The Code Won't Compile!!}
I'm sorry. I tried to make this work on many different platforms, but
sometimes machines are very specific, and it just doesn't work out of
the box. Here are some ideas on how to quickly get this thing
compiling.

\subsubsection{Can't find the right Makefile.whatever}

If make does not work, then there is probably a problem with not
finding the FORTRAN 90 compiler. The platform and machine specific
Makefiles are in srcMake. If you type:
\begin{verbatim}
uname 
ls srcMake
\end{verbatim}

If you don't see a file named something like Makefile.uname (where
uname is the output of the uname command), then you will have to build
a proper general Makefile.

You will need a little a little information about your computer, like
what the mpif90 compiler is called and where it is located. Take a
look at srcMake/Makefile.Linux, and try to figure out what all of the
flags are for your system. Then create a srcMake/Makefile.uname with
the correct information in it.

\subsubsection{The compiler doesn't recognize flag -x}

You have an operating system that is recognized, but probably a
different compiler. In the srcMake/Makefile.uname file (where uname is
the output of the uname command), there is a line:

OSFLAGS	= -w -dusty

You need to change this line to something more appropriate for your
compiler. Try deleting the flags and compile. If that doesn't work,
you will have to check the man pages of your compiler.

\subsubsection{src/ModHwm.90 doesn't compile}

Certain versions of gfortran (4.6 and later) may give the following error:

\begin{verbatim}
src/ModHwm.f90:168.22:

        call HWMupdate(input,last,gfs,gfl,gfm,gvbar,gwbar,gbz,gbm,gzwght,glev,u
                      1
Error: Dummy argument 'ebz' of procedure 'hwmupdate' at (1) has an attribute
 that requires an explicit interface for this procedure

src/ModHwm.f90:168.22:

        call HWMupdate(input,last,gfs,gfl,gfm,gvbar,gwbar,gbz,gbm,gzwght,glev,u
                      1
Error: Dummy argument 'ebz' of procedure 'hwmupdate' at (1) has an attribute
 that requires an explicit interface for this procedure
\end{verbatim}

This is caused by the inputs in HWM.  The latest incarnations of gfortran don't allow optional inputs that are not declared.  More information about this can be found at:

\begin{verbatim}
http://cosmocoffee.info/viewtopic.php?p=5136
\end{verbatim}

A solution to this problem is currently being sought.
