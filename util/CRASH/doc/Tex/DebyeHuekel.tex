\section{Debye-Huekel approximation of electrostatic energy}
{\bf Debye-Huekel approximation} may also be used to account for Coulomb interactions.
The extra term in the free energy which accounts for the
potential of the Coulomb field created in the point of the given particle location
by all the particles except for the given one is as follows (see Eq.(78.11) in \cite{ll}):
\begin{equation}\label{DH}
F_{DH}=-\frac{VT}{12\pi R_D^3},\,\,\,\frac1{R_D^2} = \frac{4\pi q_e^2(N_e+\sum_i{i^2N_i})}{VT}.
\end{equation}
Accordingly, in Eq.(\ref{equili}) the term $\partial F_{DH}/\partial N_i -\partial F_{DH}/\partial N_{i+1}-\partial F_{DH}/\partial N_e$ will 
give the contribution of $(i+1)q^2_e/(TR_D)$ into the left hand side. The partition function with the Coulomb interaction, thus becomes:
\begin{equation}\label{pfDH}
\frac{N_{i}}{g_{i}}=\frac{N_0}{g_0}(g_e)^ie^{\frac{(i^2+i)}{2R_DT}-\frac{\sum_{j=0}^{i-1}I_j}T}.
\end{equation}
This may be interpreted as the ionization potential lowering which results from the Coulomb interaction: each of the potentials, $I_i$, is reduced
by $(i+1)q_e^2/R_D$. The total reduction in the energy of the ion of sort $i$ is, hence, $(i^2+i)q_e^2/2R_D$, where the terms proprotional to $i^2$ and $i$ 
stand for ion and electrons electrostatic energy correspondingly. This effect shifts the ionization equilibrium towards higher ionization decrees, for a given 
temperature and atomic density. 

With these ion partition functions and with added correction as in Eq.(\ref{DH}) added
one can rewrite Eq.(\ref{equile}) in the following form:
\begin{equation}\label{FDH}
F = -TN_a\log\left[\frac{eV}{N_a}\left(\frac{MT}{2\pi \hbar^2}\right)^{3/2}\right]-TN_a\log S +\Omega_e +\frac{VT}{24\pi R_D^3}, 
\end{equation}
where
\begin{equation}\label{RDdef}
\frac1{R_D^2} = \frac{4\pi q_e^2N_a\langle i+i^2\rangle}{VT},
\end{equation}
and the equation for the electron statistical weigh now reads:
\begin{equation}
Z=\langle i\rangle=g_{e1}{\rm Fe}_{1/2}(g_e).
\end{equation}
Here the subscript $DH$ is to emphasize that both averages and the statistical sum
should be calculated with the partition functions as in
Eq.(\ref{pfDH}), which depend not only on $T,g_e$, but also on $R_D$.
Again, it is easy to check that the partial derivatives of
Eq.(\ref{FDH}) over $g_e$ and over $1/R_D$ both vanish.

To simplify further formulas derivation and to make them independent of system of units,
we introduce a dimensionless variable $L$:
\begin{equation}
L_{CGS} = \frac{q_e^2}{6 R_D T},\qquad
L = \frac{Ry[eV]}{3 T[eV] R_D[a]},
\end{equation}
where $R_D[a]$ stands for the Debye radius expressed in the units of Bohr radius,
so that we obtain a simpler formula for $R_D$:
\begin{equation}
\frac{1}{R_D^3} = 24 \pi n_a \langle i^2+i \rangle L.
\end{equation}

By differentiating the free energy over temperature we find the internal energy and by differentiating over the volume we find the pressure:
\begin{equation}\label{EDH}
{\cal E}={\cal E}_i + {\cal E}_e, \qquad
{\cal E}_i = \frac32 Tn_a, \qquad
{\cal E}_e = n_a \left[ \frac32 TZR^+ + \langle E^*_i\rangle \right],
\end{equation}
\begin{equation}\label{PDH}
P = P_i+P_e, \qquad
P_i = n_a T, \qquad
P_e = n_a T (1 + ZR^+ - L \langle i^2+i \rangle),
\end{equation}

{\bf Taking exact partition functions} leads us to the following equations on differentials of $g_e$, $T$, $V$ and $1/R_D$:
\begin{eqnarray}
\frac{dg_e}{g_e} \left[ \langle \delta^2 i \rangle + ZR^- \right] +
dT \left[ \frac{\langle \delta E^*_i \delta i \rangle}{T^2} - \frac32 \frac{Z}T \right] + \\
\nonumber + d\left( \frac{1}{R_D} \right) 3L R_D \langle \delta(i^2+i) \delta i \rangle = \frac{Z}V dV.
\end{eqnarray}
Here we should substitute $d(1/R_D)$ obtained by differentiation of Eq.(\ref{RDdef}):
\begin{equation}
d \left( \frac1{R_D} \right) = \frac{1}{2 R_D} \frac{-\frac{dV}{V} +
\frac{dg_e}{g_e} \frac{\langle \delta(i^2+i) \delta i \rangle}{\langle i^2+i \rangle} +
\frac{dT}{T} \left( \frac{\langle \delta(i^2+i) \delta E^*_i \rangle}{T \langle i^2+i \rangle} - 1 \right) }
{1 - \frac32L \frac{\langle \delta^2 (i^2+i) \rangle}{\langle i^2+i \rangle}},
\end{equation}
so that the coefficients in Eq.(\ref{diffstruct}) become those as follows:
\begin{equation}
A_{g_e} = \langle \delta^2 i \rangle + ZR^- + \frac{\langle \delta(i^2+i) \delta i \rangle^2}
{\frac{2}{3L} \langle i^2+i \rangle - \langle \delta^2 (i^2+i) \rangle}, \qquad
A_V = Z + \frac{\langle \delta(i^2+i) \delta i \rangle}
{\frac{2}{3L} - \frac{\langle \delta^2(i^2+i) \rangle}{\langle i^2+i \rangle}},
\end{equation}
\begin{equation}\nonumber
A_T = \frac32 Z - \frac{\langle \delta E^*_i \delta i \rangle}{T} +
\frac
{\langle \delta(i^2+i) \delta i \rangle \left( \langle i^2+i \rangle - \frac{\langle \delta(i^2+i) \delta E^*_i \rangle}{T} \right) }
{\frac{2}{3L} \langle i^2+i \rangle - \langle \delta^2(i^2+i) \rangle}.
\end{equation}


%{\bf In the limit of strong plasma non-ideality,} specifically, if the 'ionosphere radius', $a=(3/4\pi n_a)^{1/3}$, is not small as compared with the Debye radius: 
%$a\ge R_D$, the  Debye-Huekel theory is not applicable. While treating this limiting case, however, it should be mentioned, that if the condition $a\gg R_D$, or, the same,
%$a\ll Z^2q_e^2/T$ is achieved at the cost of high atomic concentration, then the pressure ionization becomes dominant. The latter effect is significant, if the ionosphere 
%radius is less than the radius of the dominant ion as in the Thomas-Fermi model: $a\le R_{TF}$. Under this condition the approach of the present would be not directly 
%applicable, because the ion energy levels, $E_i$ in this case are strongly affected by the compression, resulting ultimately in the total disappearance of the bound 
%states for electron in the limit of highest pressures.  To neglect this effect we have to assume that the density is limited from above:
%\begin{equation}
%a\gg R_{TF}(Z,i_{\max}).
%\end{equation}  
%Under this condition, which is essentially the condition for the thermal ionization dominating the pressure ionization, the plasma strong non-ideality may occur only at low 
%temperature. In this case we can use the Madelung estimate for the electrostatic energy, $-9i^2 q_e^2/10a$ - see \cite{PD} for more detail. This estimate seems to be the
%asymptotic exppression for high $i\gg1$, particularly, for $i=1$ the positive contribution fron the electron-electron interaction estimated as $\sim +i^2$ 
%should in fact turn to zero ($=i(i-1)$?). Therefore, while considering the transition from the Debye-Huekel electrostatic interaction energy per atom,  
%$-(i^2+i)q_e^2/2R_D$, to the Madelung one we evaluate 
%the latter for simlicity as as $-9(i^2+i)q_e^2/10a$. To achieve this, the only thing we pratically need is to change $R_D\rightarrow \max(R_D,5a/9)$ in 
%Eqs.(\ref{FDHApprox},\ref{ZDHApprox},\ref{EDHApprox},\ref{PDHApprox}).

