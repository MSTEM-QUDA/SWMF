%  Copyright (C) 2002 Regents of the University of Michigan, portions used with permission 
%  For more information, see http://csem.engin.umich.edu/tools/swmf
\section{Excited states of atoms and ions}
{\bf To account for excitation of atoms and ions} we need not only to involve the
distribution over the ionization states, $i$, but also to quantify the
ion distribution over the ground and 
excited levels. 

For simplicity,  multiple excitation and autoionization are neglected. We
count as the separate excited levels only those for which the principal 
quantum number, 
$n$, of the outermost electron, exceeds that for the atom or ion in its ground state, 
$n_{gr}$, by at least one. The excited states, therefore, can be enumerated using two indexes,
namely, the ion charge, $i=0,1,2...$, and the principal quantum number 
$n(i)=n_{gr}(i),n_{gr}(i)+1,...$. 

The partition function, $p_i$, describing the ion distribution over charge state $i$ 
(recall that $\sum_i{p_i}=1$), 
is now split into smaller populations, $p_{i,n}$, with each relating to a particular 
excitation level, $n$. Note that:
\begin{equation}
\sum_n{p_{i,n}}=p_i,\qquad \sum{\frac{p_{i,n}}{p_i}}=1.	
\end{equation}
Which is why the statistical weights, $w_{i,n}$, and, hence, the statistical sum and the partition functions
become more complex:
\begin{equation}
w_{i,n} = g_{i,n} \cdot g_e^i \exp \left( -\frac{E^*_{i,n}}{T} \right),\qquad
S = \sum_{i=0}^{i_{max}} \sum_{n=n_{gr}}^{\infty} w_{i,n},\qquad
p_{i,n} = \frac{w_{i,n}}{S},
\end{equation}
where $g_{i,n}$ stands for the excited level degeneracy, and the energy of the excited state,
\begin{equation}
E^*_{i,n}=E^*_i + E^{exc}_{i,n},
\end{equation} 
now includes the contribution from the ionization energy, the excitation energy and, generally 
speaking, the contribution from the electrostatic energy.

We take the values of the excitation energy, $E^{exc}_{i,n}$, from several sources:
\cite{spectrw3}, \cite{nistdb}, \cite{RefAlMartin}, \cite{RefXeSaloman}.
When some values are missing in those sources, we calculate them using the following formula:
\begin{equation}\label{estarin}
E^{exc}_{i,n} = I_i - Ry \frac{(i+1)^2}{n^2}.
\end{equation}

For the degeneracies of excited levels we assume $g_{i,n} = 2 n^2$. To calculate the
ground level degeneracy we use the electron configurations from \cite{Carlson}.
Some of the values of degeneracies have been modified to match the data from \cite{allenAQ}.

Up to this point mean values and covariances were being calculated over $i_{max}+1$ possible
values of $i$, but now $i$ is not the only variable that the energy levels depend on.
Accounting for excitation levels in the way described above leads to energy levels
being a function of $i$ and $n$.
This only increases the number of terms in the statistical sum, but does not change
anything in principle. To obtain the thermodynamic variables and the thermodynamic functions
we can just replace the mean values over $i$ with mean values over $i$ and $n$ in the
Eqs.(\ref{generalE}), (\ref{generalP}), (\ref{generalCv}),
(\ref{generalPT}), (\ref{generalCompr}) and substitute the new expression, as in Eq.(\ref{estarin}), 
for energy levels.

To calculate the covariances over $i$ and $n$ we use Eq.(\ref{sepcov}).

The following table is intended to compare the reference data with
the values of excitation energies obtained using Eq.(\ref{estarin}). Here for each
charge state of nitrogen ion, $i$, and principal quantum number of the excited
electron, $n$, we have the value given by the formula, which does not depend on the
orbital momentum of the excited electron, $l$, or the reference data (Table is removed to
because its style is outdated).

%\newcolumntype{x}[1]{>{\centering\hspace{0pt}}p{#1}}
\begin{tabular}{|x{1cm}|x{1cm}||x{2cm}||x{2cm}|x{2cm}|x{2cm}|x{2cm}|x{2cm}|}
\hline
\multirow{2}{*}{i} & \multirow{2}{*}{n} & \multirow{2}{*}{formula} & \multicolumn{5}{c|}{database, for different values of $l$} \tabularnewline
\cline{4-8}
 & & & s & p & d & f & g \tabularnewline
\hline
\hline
\multirow{4}{*}{  0} 
 &   2 &   11.1 &    0.0 &    2.4 &    0.0 &    0.0 &    0.0 \tabularnewline
 &   3 &   13.0 &   10.3 &   11.8 &   13.0 &    0.0 &    0.0 \tabularnewline
 &   4 &   13.7 &   12.8 &    0.0 &   13.7 &    0.0 &    0.0 \tabularnewline
 &   5 &   14.0 &   13.6 &    0.0 &   14.0 &    0.0 &    0.0 \tabularnewline
\hline
\multirow{4}{*}{  1} 
 &   2 &   16.0 &    0.0 &    0.0 &    0.0 &    0.0 &    0.0 \tabularnewline
 &   3 &   23.6 &   18.5 &   20.4 &   23.2 &    0.0 &    0.0 \tabularnewline
 &   4 &   26.2 &   24.4 &   25.1 &   26.0 &    0.0 &    0.0 \tabularnewline
 &   5 &   27.4 &   26.6 &    0.0 &   27.3 &    0.0 &    0.0 \tabularnewline
\hline
\multirow{4}{*}{  2} 
 &   2 &   16.8 &    0.0 &    0.0 &    0.0 &    0.0 &    0.0 \tabularnewline
 &   3 &   33.8 &   27.4 &   30.5 &   33.1 &    0.0 &    0.0 \tabularnewline
 &   4 &   39.8 &   37.3 &   38.6 &   39.4 &   39.7 &    0.0 \tabularnewline
 &   5 &   42.6 &   41.4 &   41.9 &   42.4 &   42.5 &    0.0 \tabularnewline
\hline
\multirow{4}{*}{  3} 
 &   2 &   23.0 &    0.0 &    8.3 &    0.0 &    0.0 &    0.0 \tabularnewline
 &   3 &   53.3 &   46.8 &   50.1 &   52.1 &    0.0 &    0.0 \tabularnewline
 &   4 &   63.9 &   60.5 &   62.4 &   63.3 &   64.0 &    0.0 \tabularnewline
 &   5 &   68.8 &   67.4 &   68.1 &   68.4 &   68.8 &    0.0 \tabularnewline
\hline
\multirow{4}{*}{  4} 
 &   2 &   12.9 &    0.0 &   10.0 &    0.0 &    0.0 &    0.0 \tabularnewline
 &   3 &   60.1 &   56.6 &   59.2 &   60.1 &    0.0 &    0.0 \tabularnewline
 &   4 &   76.6 &   75.1 &   76.3 &   76.6 &   76.6 &    0.0 \tabularnewline
 &   5 &   84.3 &   83.5 &   84.1 &   84.3 &   84.3 &   84.3 \tabularnewline
\hline
\multirow{4}{*}{  5} 
 &   2 &  429.6 &  419.8 &  426.3 &    0.0 &    0.0 &    0.0 \tabularnewline
 &   3 &  497.6 &  494.9 &  496.7 &  497.6 &    0.0 &    0.0 \tabularnewline
 &   4 &  521.4 &  520.3 &  521.1 &  521.4 &  521.5 &    0.0 \tabularnewline
 &   5 &  532.5 &  531.9 &  532.3 &  532.5 &  532.5 &    0.0 \tabularnewline
\hline
\multirow{4}{*}{  6} 
 &   2 &  500.4 &  500.3 &  500.2 &    0.0 &    0.0 &    0.0 \tabularnewline
 &   3 &  593.0 &  592.9 &  592.9 &  593.0 &    0.0 &    0.0 \tabularnewline
 &   4 &  625.4 &  625.4 &  625.4 &  625.4 &  625.4 &    0.0 \tabularnewline
 &   5 &  640.4 &  640.4 &  640.4 &  640.4 &  640.4 &  640.4 \tabularnewline
\hline
\end{tabular}

