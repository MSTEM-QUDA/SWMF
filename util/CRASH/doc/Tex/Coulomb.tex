%  Copyright (C) 2002 Regents of the University of Michigan, portions used with permission 
%  For more information, see http://csem.engin.umich.edu/tools/swmf
\section{Madelung approximation of electrostatic energy}
{\bf Coulomb interactions} may be accounted for within the Madelung approximation.
According to this model, we assume that the net charge of an ion and all of the
electrons bound to it is concentrated in one point. For an ion in the charge state $i$,
$i$ free electrons are considered to be coupled with it, being uniformly distributed over its {\it ion sphere,} which is a sphere of volume $1/n_a$, with the ion at its center.
While calculating the electrostatic energy we neglect the Coulomb interactions between
charges related to different ion spheres. As a result, we only need to calculate the electrostatic energy
of an ion with free electrons coupled to it, which is equal to the energy of
electron-electron interactions and the energy required to put the ion
at the center of the ion sphere:
\begin{equation}
E_{E} = \frac12 \int_{r=0}^{r_{iono}}{q_e n_e \varphi_e(r) dV} - q_e i \varphi_e(0) =
\frac35 \frac{q_e^2 i^2}{r_{iono}} - \frac32 \frac{q_e^2 i^2}{r_{iono}} =
-\frac{9}{10} \frac{q_e^2 i^2}{r_{iono}},
\end{equation}
where $r_{iono}$ stands for the radius of the ion sphere, and the potential of
the electrostatic field of electrons is as follows:
\begin{equation}
\varphi_e(r) = ei \left( -\frac32 \frac{1}{r_{iono}} + \frac12 \frac{r^2}{r_{iono}^3} \right).
\end{equation}

{\bf Helmholtz free energy.}
The extra term in the free energy, which accounts for the
electrostatic energy, is as follows (see also Eq.(3.50) in \cite{drake}):
\begin{equation}\label{fterm}
F_M=-E_M \sum_{i=0}^{i_{max}} i^2 N_i,\qquad E_M=\frac{9}{10} \frac{q_e^2}{r_{iono}},\qquad
r_{iono} = \left( \frac{4 \pi}{3} n_a \right)^{-\frac13}.
\end{equation}
Here the Madelung energy,
\begin{equation}
E_M=\frac9{10}\frac{q_e^2}{r_{iono}}=\frac{1.8Ry}{(r_{iono}/a)},
\end{equation}
characterizes the electrostatic energy related per atomic cell. It is conveniently expressed in terms of the Rydberg constant,  $Ry=\frac{q_e^2}{2a}\approx 13.60$ eV, as long as
the ion sphere radius, $r_{iono}$, is related to the Bohr radius, $a=\hbar^2/m_e q_e^2\approx0.5\cdot10^{-10}$ m.

{\bf Ionization equilibrium.} Accordingly, in the requirement for ionization equilibrium, $\partial F/\partial N_i - \partial F/\partial N_{i+1} - \partial F/\partial N_e = 0$ (with respect to the reaction $(i)\leftrightarrow(i+1)+e$),
the term $\partial F_{M}/\partial N_i -\partial F_{M}/\partial N_{i+1}$ will 
give the contribution of $(2i+1) E_M$ to the left side:
\begin{equation}
-T \log \left[ \frac{g_i}    {N_i}     e^{\sum_{j=0}^{i-1} I_j/T} \right] - i^2 E_M
+T \log \left[ \frac{g_{i+1}}{N_{i+1}} e^{\sum_{j=0}^i     I_j/T} \right] + (i+1)^2 E_M
-\mu_e = 0.
\end{equation}
The solution of the ionization equilibrium, hence, reads:
\begin{equation}
\frac{N_{i+1}}{g_{i+1}} = \frac{N_i}{g_i} \exp\left(-\frac1T \left(I_i - (2i+1) E_M + \mu_e \right)\right),
\end{equation}
or, applying recursively and reducing the sum $\sum_{j=0}^{i-1} (2j+1) = i^2$,
\begin{equation}\label{pfM}
\frac{N_i}{g_i}=\frac{N_0}{g_0}(g_e)^i \exp \left( i^2 \frac{E_M}{T} -\frac{\sum_{j=0}^{i-1}I_j}T \right) .
\end{equation}
This may be interpreted as the ionization potential lowering caused by the Coulomb interaction.
Each of the potentials, $I_i$, is reduced by $(2i+1)E_M$.
Energy of the ion of the charge state $i$ is:
\begin{equation}\label{estar}
E_i^* = \sum_{j=0}^{i-1}I_j - i^2 E_M.
\end{equation}
This effect shifts the ionization equilibrium towards higher ionization degrees for a given temperature and atomic density.

{\bf Partition function.} The common multiplier, $\frac{N_0}{g_0}$, in each of Eqs.(\ref{pfM}) may be also represented as $\frac{N_a}{S}$.
From the normalization condition, $\sum N_i = N_a$, we find that $S$ is a statistical sum:
\begin{equation}\label{MadS}
S=\sum_{i=0}^{i_{max}} g_i (g_e)^i \exp\left(-\frac{E_i^*}T\right),
\end{equation}
so that:
\begin{equation}\label{ni}
p_i = \frac{N_i}{N_a} = \frac{1}S g_i (g_e)^i \exp \left( -\frac{E_i^*}T \right).
\end{equation}

{\bf Differentials along the curve of ionization equilibrium} obey the following equations:
\begin{equation}\label{diffstruct}
A_{g_e} \frac{dg_e}{g_e} = A_V \frac{dV}{V} + A_T \frac{dT}{T},
\end{equation}
\begin{equation}
A_{g_e} = \langle \delta^2 i \rangle + ZR^-(g_e), \qquad
A_T     = \frac32 Z - \frac{\langle \delta i \delta E^*_i \rangle}{T}, \qquad
A_V     = Z + L \langle \delta(i^2) \delta i \rangle,
\end{equation}
where 
\begin{equation}
L   = \frac{E_M}{3T} = \frac35 \frac{Ry[eV]}{T[eV] r_{iono}[a]}.
\end{equation}

Again, we express the result in terms of covariances,
$\langle \delta a \delta b \rangle = \langle (a - \langle a \rangle) (b - \langle b \rangle) \rangle$,
and mean values, which are now being calculated
using the modified partition functions.
Differentiation of mean values, which is necessary for derivation of the above equation on
differentials, is not a complicated problem with the following formula:
\begin{equation}
d \langle f_i \rangle = \left\langle \delta f_i \delta\left( \frac{dp_i}{p_i} \right) \right\rangle,
\end{equation}
where $f_i$ is a function of the only argument $i$, for example, $iE_i$ or $i^2+i$.

{\bf The ionization equilibrium} can be solved using the old technique, i.e. the Newton-Rapson
iterations, defined in Eq.(\ref{iter}).

{\bf The full Helmholtz free energy} now includes the contribution of the electrostatic field energy as in Eq.(\ref{fterm}):
\begin{equation}\label{fe1}
F=-T
\sum_{i=0}^{i_{max}}{
N_i\log\left[g_i
  \frac{eV}{N_i}\left(\frac{MT}{2\pi \hbar^2}\right)^{3/2}\exp \left(-\sum_{j=0}^{i-1}\frac{I_j}T \right)\right]}+F_e
  - \sum_{i=0}^{i_{max}} {N_ii^2 E_M}.
\end{equation}  

With the ion partition functions in Eq.(\ref{ni}) one can rewrite Eq.(\ref{fe1}) in the following form:
\begin{equation}\label{ffullm}
F = -TN_a\log\left[\frac{eV}{N_a}\left(\frac{MT}{2\pi \hbar^2}\right)^{3/2}\right]-TN_a\log S + \Omega_e, 
\end{equation}
where, again, $\Omega_e = F_e - \mu_e N_a \sum i p_i = F_e - \mu_e N_a \langle i \rangle $.

{\bf Plasma thermodynamics and Equation-Of-State.} 
While differentiating Eq.(\ref{ffullm}) with respect to $T$ and $V$, again we see that the derivatives
by $g_e$ from the second and third terms cancel 
each other: $g_e(\partial \log S/\partial g_e)=\langle i\rangle=Z$,
which is evident from $d \log S = \langle d (\log p_i) \rangle$,
and $-g_eg_{e1}{\rm Fe}^\prime_{3/2}(g_e)=g_{e1}{\rm Fe}_{1/2}=Z$.
Accordingly, for the internal energy density,
${\cal E}$, and for the pressure, $P$, we find the following general expressions:
\begin{equation}\label{generalE}
{\cal E} = -\frac{T^2}V\left(\frac{\partial}{\partial T}\left(\frac F T\right)\right)=
{\cal E}_i+{\cal E}_e,\qquad{\cal E}_i=\frac32Tn_a,\qquad
{\cal E}_e = n_a\left[ \frac32 T Z R^+ + \langle E^*_i \rangle \right],
\end{equation}
\begin{equation}\label{generalP}
P = -\frac{\partial F}{\partial V}=P_i+P_e,\quad
P_i = n_aT,\quad
P_e = n_a \left[ T ZR^+ - V \langle \frac{\partial E^*_i}{\partial V} \rangle \right],
\end{equation}
where we assume a general dependence $E^*_i=E^*_i(V)$, such as that in Eq.(\ref{estar}).

In the above equations we add the Madelung corrections to the energy of the electron gas, ${\cal E}_e$, and
to the electron pressure, $P_e$, because those corrections are controlled by the electron temperature.

The thermodynamic derivatives can also be expressed in a general form for $E^*_i=E^*_i(V)$
in such a way that one can find the specific heat in an isochoric process, per the unit volume:
\begin{equation}\label{generalCv}
C_{Ve}=\frac{\partial {\cal E}_e}{\partial T}=n_a\left[\frac{\langle\delta^2 E^*_i \rangle}{T^2}+\frac{15}4ZR^+
-\frac{\left(\frac32Z-\frac{\langle\delta E^*_i \delta i\rangle}T\right)^2}{\langle\delta^2i\rangle+ZR^-}\right],
\end{equation}
the temperature derivative of pressure:
\begin{equation}\label{generalPT}
\frac {\partial P_e}{\partial T}=
n_a\left[
	\frac52 Z R^+ -
	\left( Z+\frac{V}{T} \langle \delta \frac{\partial E^*_i}{\partial V} \delta i \rangle \right)
		\frac{\frac32Z-\frac{\langle\delta E^*_i \delta i\rangle}T}{\langle\delta^2i\rangle+ZR^-} -
	\frac{V}{T^2} \langle \delta \frac{\partial E^*_i}{\partial V} \delta E^*_i \rangle
\right],
\end{equation}
as well as the isothermal compressibility:
\begin{equation}\label{generalCompr}
V\frac{\partial P_e}{\partial V}=
n_a T \left[ -\frac{\left(Z + \frac{V}{T} \langle \delta i \delta \frac{\partial E^*_i}{\partial V} \rangle \right)^2}
{\langle \delta^2 i \rangle + ZR^-} +
\frac{V^2}{T^2} \langle \delta^2 \frac{\partial E^*_i}{\partial V} \rangle -
\frac{V^2}{T} \langle \frac{\partial^2 E^*_i}{\partial V^2} \rangle
\right].
\end{equation}
Again, for simplicity in the above equations the contributions due to ion translational motions,
\begin{equation}
C_{Vi}=\frac32n_a, \qquad
\frac{\partial P_i}{\partial T}=n_a, \qquad
V\frac{\partial P_i}{\partial V}=-n_aT,
\end{equation}
are omitted.

To apply the Madelung theory we calculate the first and the second partial derivatives
of the energy levels over volume using Eq.(\ref{estar}):
\begin{equation}\label{estarprime}
\frac{V}{T} \frac{\partial E^*_i}{\partial V} = L i^2, \qquad
\frac{V^2}{T} \frac{\partial^2 E^*_i}{\partial V^2} = -\frac43 L i^2.
\end{equation}

Eqs.(\ref{estar},\ref{estarprime}) allow for specification of all averages and covariances in the 
expressions for thermodynamic variables and derivatives.

{\bf Simulation results.}
The following table shows the values of $Z$ calculated for Xenon
at various electron temperatures
(given in electron-volts -- the value of $k_{B}T_{e}$, where $k_{B}$
is in eV/K) and heavy particle concentrations, given in number of
particles per $cm^{3}$. The "no" columns contain the same values
calculated without Coulomb interation taken into account.

\begin{center}
\begin{tabular}{|c||c|c|c|c|c|c|}
\hline
Te[eV]\textbackslash \textbackslash Na[$1/cm^3$] & $10^{18}$ & $10^{19}$ & $10^{20}$ & $10^{21}$ & $10^{22}$ & $10^{23}$\tabularnewline
\hline
\hline
   5. &     3.4 &     3.0 &     2.5 &     1.9 &     1.4 &     2.9\tabularnewline
\hline
  10. &     6.4 &     5.4 &     4.6 &     3.7 &     3.2 &     3.9\tabularnewline
\hline
  15. &     7.3 &     6.9 &     6.3 &     5.4 &     4.6 &     4.8\tabularnewline
\hline
  20. &     9.3 &     7.9 &     7.1 &     6.5 &     5.7 &     5.6\tabularnewline
\hline
  25. &    12.0 &     9.7 &     8.0 &     7.1 &     6.5 &     6.2\tabularnewline
\hline
  30. &    14.0 &    11.9 &     9.5 &     7.8 &     7.0 &     6.7\tabularnewline
\hline
  35. &    15.9 &    13.7 &    11.3 &     8.9 &     7.6 &     7.2\tabularnewline
\hline
  40. &    17.4 &    15.3 &    12.9 &    10.2 &     8.3 &     7.8\tabularnewline
\hline
  45. &    18.2 &    16.7 &    14.2 &    11.6 &     9.2 &     8.5\tabularnewline
\hline
  50. &    19.2 &    17.7 &    15.5 &    12.9 &    10.3 &     9.3\tabularnewline
\hline
  55. &    20.7 &    18.4 &    16.7 &    14.0 &    11.4 &    10.1\tabularnewline
\hline
  60. &    22.2 &    19.3 &    17.5 &    15.1 &    12.4 &    11.0\tabularnewline
\hline
  65. &    23.5 &    20.6 &    18.1 &    16.0 &    13.4 &    11.9\tabularnewline
\hline
  70. &    24.6 &    21.9 &    18.8 &    16.8 &    14.3 &    12.7\tabularnewline
\hline
  75. &    25.4 &    23.0 &    19.7 &    17.5 &    15.1 &    13.5\tabularnewline
\hline
  80. &    25.7 &    24.0 &    20.7 &    18.0 &    15.8 &    14.2\tabularnewline
\hline
  85. &    25.9 &    24.8 &    21.8 &    18.6 &    16.5 &    14.9\tabularnewline
\hline
  90. &    25.9 &    25.3 &    22.7 &    19.2 &    17.0 &    15.5\tabularnewline
\hline
  95. &    26.0 &    25.6 &    23.6 &    20.0 &    17.5 &    16.0\tabularnewline
\hline
 100. &    26.1 &    25.8 &    24.3 &    20.8 &    18.0 &    16.5\tabularnewline
\hline
 105. &    26.2 &    25.9 &    24.9 &    21.6 &    18.4 &    16.9\tabularnewline
\hline
 110. &    26.5 &    26.0 &    25.3 &    22.4 &    18.9 &    17.3\tabularnewline
\hline
 115. &    27.0 &    26.0 &    25.5 &    23.1 &    19.5 &    17.7\tabularnewline
\hline
 120. &    27.6 &    26.1 &    25.7 &    23.7 &    20.1 &    18.1\tabularnewline
\hline
 125. &    28.4 &    26.3 &    25.8 &    24.3 &    20.7 &    18.4\tabularnewline
\hline
\end{tabular}

\par\end{center}

\clearpage
