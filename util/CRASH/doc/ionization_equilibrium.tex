%% LyX 1.5.5 created this file.  For more info, see http://www.lyx.org/.
%% Do not edit unless you really know what you are doing.
\documentclass[english]{revtex4}
\usepackage[T1]{fontenc}
\usepackage[latin9]{inputenc}
\usepackage{nicefrac}

\makeatletter

%%%%%%%%%%%%%%%%%%%%%%%%%%%%%% LyX specific LaTeX commands.
%% Because html converters don't know tabularnewline
\providecommand{\tabularnewline}{\\}

%%%%%%%%%%%%%%%%%%%%%%%%%%%%%% User specified LaTeX commands.
%% LyX 1.5.5 created this file.  For more info, see http://www.lyx.org/.
%% Do not edit unless you really know what you are doing.



\usepackage{nicefrac}


\makeatletter

%%%%%%%%%%%%%%%%%%%%%%%%%%%%%% LyX specific LaTeX commands.
%% Because html converters don't know tabularnewline


%%%%%%%%%%%%%%%%%%%%%%%%%%%%%% User specified LaTeX commands.
%% LyX 1.5.5 created this file.  For more info, see http://www.lyx.org/.
%% Do not edit unless you really know what you are doing.



\usepackage{nicefrac}



\makeatletter

%%%%%%%%%%%%%%%%%%%%%%%%%%%%%% LyX specific LaTeX commands.
%% Because html converters don't know tabularnewline


%%%%%%%%%%%%%%%%%%%%%%%%%%%%%% User specified LaTeX commands.
%% LyX 1.5.5 created this file.  For more info, see http://www.lyx.org/.
%% Do not edit unless you really know what you are doing.



\usepackage{nicefrac}




\makeatletter

%%%%%%%%%%%%%%%%%%%%%%%%%%%%%% LyX specific LaTeX commands.
%% Because html converters don't know tabularnewline


%%%%%%%%%%%%%%%%%%%%%%%%%%%%%% User specified LaTeX commands.
%% LyX 1.5.5 created this file.  For more info, see http://www.lyx.org/.
%% Do not edit unless you really know what you are doing.



\usepackage{nicefrac}





\makeatletter

%%%%%%%%%%%%%%%%%%%%%%%%%%%%%% LyX specific LaTeX commands.
%% Because html converters don't know tabularnewline


%%%%%%%%%%%%%%%%%%%%%%%%%%%%%% User specified LaTeX commands.
%% LyX 1.5.5 created this file.  For more info, see http://www.lyx.org/.
%% Do not edit unless you really know what you are doing.




\makeatletter

%%%%%%%%%%%%%%%%%%%%%%%%%%%%%% LyX specific LaTeX commands.
%% Because html converters don't know tabularnewline


%%%%%%%%%%%%%%%%%%%%%%%%%%%%%% User specified LaTeX commands.
%% LyX 1.5.5 created this file.  For more info, see http://www.lyx.org/.
%% Do not edit unless you really know what you are doing.




\makeatletter
%%%%%%%%%%%%%%%%%%%%%%%%%%%%%% User specified LaTeX commands.
 


\makeatother

\makeatother





\makeatother





\makeatother





\makeatother





\makeatother





\makeatother

\usepackage{babel}

\begin{document}

\title{{\Large Calculating the Ionization Equilibrium in an Ideal Plasma}}

\maketitle

\section{Ionization Equilibrium at a Given Temperature\label{sec:Ionization-Equilibrium}}


\subsection{The Basics }

Population of an ion ($P_{i}$) in a plasma is the probability of
finding that particle in the state possessing a certain charge, i.
Similarly, if we let $N_{a}$ be the heavy particle density -- the
total number of atoms and ions per unit of volume in a plasma -- then,
$N_{i}=P_{i}*N_{a}$, where $N_{i}$ is the concentration of i-level
ions (ions that lost i electrons), and $P_{i}$, their populations.
Furthermore, since the total of probabilities of all the alternative
events is always equal to one, we get 
\begin{equation}
\sum_{i=0}^{Z}P_{i}=1,
\end{equation}
where X is the atomic number of the element -- hence the maximum
charge it can attain. Or, similarly: 
\begin{equation}
\sum_{i=0}^{Z}N_{i}=N_{a},
\end{equation}
which is the same equation multiplied by $N_{a}$. Furthermore, since
in a single-component plasma, each of the $N_{i}$ ions looses i electrons,
which is further true for each of the X types of ions, we obtain the
formula for the total number of free electrons per unit volume, $N_{e}$:
\begin{equation}
N_{e}=\sum_{i=0}^{X}iN_{i},
\end{equation}
which is the condition of plasma neutrality -- the total positive
charge on the ions per unit of volume in a plasma must be equal to
the total number of free electrons in that volume. If we divide this
equation by $N_{a}$, we get: 
\begin{equation}
<Z>=\sum_{i=0}^{X}iP_{i},
\end{equation}
where <Z> = $N_{e}/N_{a}$ -- the average positive charge per ion
in the plasma (measured in elementary charge units). From all these
formulas we can see that to calculate practically any property of
a plasma, we need to know the populations of all the levels of ions.


\subsection{Boltzmann Distribution}

To calculate the population of an i-level ion at a certain temperature,
we need to know for how many atoms/ions, the energy per atomic cell
is greater than or equal to the energy level of the $i^{th}$ ion.
Thus, we need to know the energy distribution between the atoms and
ions for a given temperature T. Here, recall the Maxwell-Boltzmann
Distribution, which formulates that the probability of finding a particle
with certain energy in some volume of gas (population of that energy
state) is proportional to the Boltzmann factor: 
\begin{equation}
g_{i}e^{-\frac{E_{i}}{k_{B}T}},
\end{equation}
where the exponent gives the probability of finding a particle in
a specific state with energy $E_{i}$, while $g_{i}$ -- degeneracy
-- is the number of such distinct states that have the same desired
energy. In the case of a plasma, however, the degeneracy value includes
not only the number of states that the i-charged ion might have, but
also the number of distinct states possible for all of the i electrons
that have separated from the original atom. Since each of these electrons
has the same degeneracy value, $g_{e}$, the total number of possible
states for all of these electrons will be $(g_{e})^{i}$. Therefore,
in Eq.(5), $g_{i}$ will be replaced by $g_{i}*(g_{e})^{i}$. Furthermore,
recalling that the sum of the populations of all alternative states
must equal one, we can find the proportionality constant for the Boltzmann
factor. Dividing each of these factors by the value of their sum,
we will normalize them so that their sum after the transformation
will be equal to unity (Eq. 1). Hence, if we also use the ion energy
levels for $E_{i}$ (energy needed to ionize a neutral atom to an
i-charged ion), then the ion populations become: 
\begin{equation}
P_{i}=\frac{g_{i}(g_{e})^{i}e^{-\frac{E_{i}}{k_{B}T}}}{{\displaystyle \sum_{i=0}^{X}g_{i}(g_{e})^{i}e^{-\frac{E_{i}}{k_{B}T}}}},\label{eq:6}
\end{equation}
 Here, the sum in the denominator is called the partition function.
The main difficulty with finding the actual populations using this
equation comes from the $g_{e}$ and $g_{i}$ values. Through statistical
and quantum mechanics we can find that the value of $g_{e}$ is: 
\begin{equation}
g_{e}=\frac{2}{N_{e}\Lambda^{3}}\ \ \ \ \ \ \ \ \ \ \ \ \ \Lambda=\sqrt{\frac{h^{2}}{2\pi m_{e}k_{B}T}},
\end{equation}
where $\Lambda$ is the thermal De Broglie wavelength of an electron.
Note that to calculate the value for $g_{e}$, we need to know $N_{e}$,
which is found from Eq.(3) (or 4), which in turn takes the populations
as an input parameter. Hence, we will have to solve a system of equations,
instead of a single equation, to find the correct population values.
This will be addressed later on in the paper. In regards to the $g_{i}$
values, on the other hand, note that in most cases in a realistic
plasma, while the value of $g_{e}$ might be of the order of $10^{15}$,
the value of $g_{i}$ is usually not grater than 3. Hence, after normalization,
the effect of $g_{i}$ on the final population values will be negligibly
small in comparison. Furthermore, since in most cases we are not interested
in the actual population values, but rather in properties of plasma,
such as <Z> (Eq.4), the slight inaccuracies in the populations will
be evened out by the summation often used to calculate such properties,
making the approximation even less evident. Hence, from now on, we
will ignore the $g_{i}$ values in our equations, approximating them
by $g_{i}\approx1$.
\subsection{Saha Ionization Equation and Calculation Speed Improvement}
Another similar approach to finding the population values (if the
value of $g_{e}$ is known) is using the Saha Ionization Equation.
This equation is obtained by taking the ratio of the populations of
the two consecutive level ions as given by Eq.(6): 
\begin{equation}
\frac{P_{i+1}}{P_{i}}=g_{e}\frac{g_{i+1}}{g_{i}}e^{-\frac{E_{i+1}-E_{i}}{k_{B}T_{e}}}=\frac{2}{N_{e}\Lambda^{3}}\frac{g_{i+1}}{g_{i}}e^{-\frac{I_{i+1}}{k_{B}T_{e}}},
\end{equation}
or, similarly: 
\begin{equation}
\frac{N_{i+1}N_{e}}{N_{i}}=\frac{2}{\Lambda^{3}}
\frac{g_{i+1}}{g_{i}}e^{-\frac{I_{i+1}}{k_{B}T_{e}}},
\end{equation}
which is a better known form of the Saha Equation. Note that here
$I_{i+1}=E_{i+1}-E_{i}$ denotes the ionization energy -- energy required
to remove a single electron from an i-charged ion, forming (i+1)-level
ion. These equations can provide an advantage in finding the populations
if the speed of calculation is important. First, realize that if such
is the case, than it is better to avoid the complex exponents of Eq.(6),
which would take a long time to calculate. To do this, we could instead
start with the natural logarithm of Eq.(8) (now ignoring the ratio
$g_{i+1}/g_{i}\approx1$): 
\begin{equation}
\ln P_{i+1}-\ln P_{i}=\ln g_{e}-\frac{I_{i+1}}{k_{B}T_{e}},
\end{equation}
Then, we will let $P_{0}=1$ (ln $P_{0}$ = 0), and using this relationship,
build up and array of non-normalized values of $P_{1}$, $P_{2}$,
$P_{3}$, ..., $P_{X}$. The values of this array will vary greatly,
and some will be very large. To avoid having to find exponents of
huge numbers, we can normalize this array so that the population of
the most probable ion $P_{m}ax$ = 1 (ln $P_{m}ax$ = 0). Note that
normalizing an array of logarithms is done by subtraction, rather
than division, since log ($P_{i}$/C) = log $P_{i}$ - log C, and
hence we simply need to subtract ln $P_{m}ax$ from every term. After
this is done, to simplify the calculations even further, we can get
rid of all the values of the array <-5, for example, because the actual
populations corresponding to these values will be negligibly small
-- $P_{i}<e^{-5}=0.0067$, and can be approximated as zeros. Finally,
at this point, it should not take long to find the exponents of the
remaining part of the array, and normalize them as before so that
their sum equals one. This method can then be used instead of Eq.(6)
any time the population values are needed.
\subsection{The System of Equations}
Both of the methods for finding the populations presented above can
only be used if the value of $g_{e}$ is known. However, $g_{e}$
depends on $N_{e}$, which in turn depends on the populations themselves,
and hence we find ourselves faced with solving a system of two equations
-- Eq.(3) (or 4) and Eq.(6) must both be satisfied in order to obtain
the correct population values. To solve this system, we first need
to rewrite the term $g_{e}$ as a function of Z: 
\begin{equation}
g_{e}=\frac{2}{N_{e}\Lambda^{3}}=\frac{2}{ZN_{a}\Lambda^{3}}=\frac{C_{1}}{Z},\end{equation}
 $C_{1}$, which is independent of Z, is introduced here only to simplify
this ratio. Hence, the Boltmznn factor from Eq.(5) becomes: 
\begin{equation}
P_{i}\propto\left(\frac{C_{1}}{Z}\right)^{i}e^{-\frac{E_{i}}{k_{B}T_{e}}}=p_{i},\end{equation}
 Since this term appears frequently in the succeeding discussion,
we will call it $p_{i}$ to simplify the notation. Now, we can combine
equations (4) and (6) into a single equation in Z: 
\begin{equation}
Z=\sum_{i=0}^{X}iP_{i}=\sum_{i=0}^{X}\left(i\frac{p_{i}}{\sum_{j=0}^{X}p_{j}}\right)=\frac{\sum_{i=0}^{X}ip_{i}}{\sum_{i=0}^{X}p_{i}},
\end{equation}
\subsection{Iterative Approach}
Finally, all the terms in the above equation are known. However, this
expression is too complex to solve by simple algebraic methods and
requires an iterative approach. Direct iterations will converge to
the answer very slowly, if at all. Hence, we will use the Newton-Rapson
iteration method: 
\begin{equation}
x_{n+1}=x_{n}-\frac{f(x_{n})}{f^\prime(x_{n})},
\end{equation}
 This method is used to approximate zeros of the function f. If $x_{n}$
is an approximation of the solution (or a trial solution), than $x_{n+1}$
will be a much better approximation. This method does have limitation
as to whether it will converge to the answer, but if $f'(x_{n})\geq1$,
then convergence is guaranteed. Further, since this method works only
for finding zeros of the function, considering equation (13), let
\begin{equation}
f(Z)=Z-\frac{\sum i\left(\frac{C_{1}}{Z}\right)^{i}e^{-\frac{E_{i}}{k_{B}T_{e}}}}{\sum\left(\frac{C_{1}}{Z}\right)^{i}e^{-\frac{E_{i}}{k_{B}T_{e}}}},
\end{equation}
then when f(Z)=0, Z will have the desired value. Now, to use the
method as in Eq.(14), we need to know f '(Z). Let's first find the
derivative \[
\frac{\partial}{\partial Z}\left[\sum\left(\frac{C_{1}}{Z}\right)^{i}e^{-\frac{E_{i}}{k_{B}T_{e}}}\right]=\sum i\left(\frac{C_{1}}{Z}\right)^{i-1}\ \frac{-C_{1}}{Z^{2}}e^{-\frac{E_{i}}{k_{B}T_{e}}}=\]
 \begin{equation}
=-\frac{C_{1}}{Z^{2}}\frac{Z}{C_{1}}\sum i\left(\frac{C_{1}}{Z}\right)^{i}e^{-\frac{E_{i}}{k_{B}T_{e}}}=-\frac{1}{Z}\sum ip_{i},\end{equation}
 and similarly: \begin{equation}
\frac{\partial}{\partial Z}\left[\sum ip_{i}\right]=-\frac{1}{Z}\sum i^{2}p_{i},\end{equation}
 Now, combining these answers to find f '(Z), we get: \[
\frac{\partial f}{\partial Z}=1-\frac{-\frac{1}{Z}\sum i^{2}p_{i}}{\sum p_{i}}-\frac{-\sum ip_{i}}{\left(\sum p_{i}\right)^{2}}\left(-\frac{1}{Z}\sum ip_{i}\right)=\]
 \begin{equation}
=1+\frac{1}{Z}\left(\frac{\sum i^{2}p_{i}}{\sum p_{i}}-\left(\frac{\sum ip_{i}}{\sum p_{i}}\right)^{2}\right),\end{equation}
 Therefore, looking back on Eq.(14) and combining Eq.(15) and (18),
we get the overall resulting equation that can be used to find Z:
\begin{equation}
Z_{n+1}=Z_{n}-\frac{Z_{n}-\frac{\sum ip_{i}}{\sum p_{i}}}{1+\frac{1}{Z}_{n}\left(\frac{\sum i^{2}p_{i}}{\sum p_{i}}-\left(\frac{\sum ip_{i}}{\sum p_{i}}\right)^{2}\right)},\end{equation}
 where the values for $p_{i}$ can be found from equation (12), using
$Z_{n}$ for Z and all sums are across the entire span of possible
ion charges. Further, notice that this equation is equivalent to:
\begin{equation}
Z_{n+1}=Z_{n}-\frac{Z_{n}-<Z>_{n}}{1+\frac{1}{Z}_{n}(<Z^{2}>_{n}-<Z>_{n}\ ^{2})},\end{equation}
 where <Z> is the approximate average charge per ion, and <$Z_{2}$>
is the approximate average of the squares of charges. This equation
can also be used to find Z by first, finding the population values
through either of the presented methods, using $Z_{n}$ for Z in the
calculation of $g_{e}$, and then find the values of $<Z>_{n}$ and
$<Z^{2}>_{n}$ using the formulas: \begin{equation}
<Z>=\sum_{i=1}^{X}iP_{i}\ \ \ \ \ \ \ \ <Z^{2}>=\sum_{i=1}^{X}i^{2}P_{i},\end{equation}
 Also, from Eq.(20), it can be seen that since the average of the
squares is always greater than the square of the average, $<Z^{2}>-<Z>^{2}\geq0$,
and hence $f'(Z)\geq1$, which then necessarily means that the iterations
will converge. Hence, with these equations, we can fairly quickly
find Z, if our initial guess is reasonably close.


\subsection{Initial Approximation}

In order to make the above iterative approach quick and efficient,
it is important to find a decent first approximation for what the
value of <Z> is. Since in most cases in plasma, there is a single
ion whose population is much higher than that of all the others, a
good first guess would be that <Z> is approximately the same as the
charge of that ion. Hence, we need to find the value of i that maximizes
$P_{i}$, or ln $P_{i}$ (which will occur at the same i, but is easier
to find). Recall that at a local extrema of any function, its first
derivative is equal to 0. Since the populations are not defined by
a continuous function, we can similarly say that $\Delta(\ln P_{i})/\Delta i$
is closest to 0 at the maximum value of ln $P_{i}$. Eq.(10) essentially
gives an expression for $\Delta(\ln P_{i})/\Delta i$. Hence, since
we decided to initially approximate $Z\approx i_{P}{}_{max}$, we
can build a sequence of values of what $\Delta(\ln P_{i})/\Delta i$
would be if the current i was equal to $i_{P_{max}}$, and hence to
Z. Hence, we will approximate $g_{e}=C_{1}/Z\approx C_{1}/i$. Further,
when this derivative of the ln $P_{i}$ sequence is $\approx0$, $P_{i}$
will be $\approx P_{i+1}$, and hence i+1 can be used for Z here just
as well. Hence we get: \begin{equation}
\frac{\Delta(\ln P_{i})}{\Delta i}=\frac{\ln P_{i}-\ln P_{i-1}}{1}=\ln(\frac{C_{1}}{i})-\frac{I_{i}}{k_{B}T_{e}},\end{equation}
 Then, the value of i for which the right side of this equation is
closest to 0, will be a reasonable initial approximation for Z. Further,
since, as mentioned, i as well as i-1 could be used for Z here, a
better initial approximation would actually be between the two: $Z\approx i-0.5$.
A separate case arises when the plasma is weak and the vast majority
of the atoms are not ionized ($Z\approx0$). This implies that the
populations of higher level ions will be negligible compared to those
of $P_{0}$, or even $P_{1}$, and hence, Eq.(13) becomes: \begin{equation}
Z=\frac{{\displaystyle \sum_{i=0}^{X}ip_{i}}}{{\displaystyle \sum_{i=0}^{X}p_{i}}}\approx\frac{p_{1}}{p_{0}}=\left(\frac{C_{1}}{Z}\right)e^{-\frac{I_{1}}{k_{B}T_{e}}},\end{equation}
 This can then be easily solved for Z: \begin{equation}
Z\approx\sqrt{C_{1}e^{-\frac{I_{1}}{k_{B}T_{e}}}},\end{equation}
 which is a very good and simple approximation. Hence, when the method
in Eq.(22) yields $i_{P}{}_{max}=1$, the method in Eq.(24) can be
used instead, but only as long as it yields a value for Z less than
1 (if it does not, than $Z\approx1$ and is too large for this method).


\subsection{Section Summary}

Finally, we have concluded that in order to find the ion populations
in a plasma, we can use the following procedure:

\begin{enumerate}
\item Approximate the value of <Z> (average charge per ion) by first, finding
the value of i for which \[
\ln(\frac{C_{1}}{i})-\frac{I_{i}}{k_{B}T_{e}}\]
 is closest to 0, and then approximating $<Z>\approx i-0.5$. Note
that if the i value found is =1, than <Z> might be very close to zero,
in which case use: \[
Z\approx\sqrt{C_{1}e^{-\frac{I_{1}}{k_{B}T_{e}}}}\]


\begin{itemize}
\item $I_{i}$ is the ionization energy (energy required to remove one electron
from (i-1)-level ion, creating i charged ion) 
\item $k_{B}$ is Boltzmann constant; 
\item $T_{e}$ is the electron temperature; 
\item $C_{1}=2/(N_{a}\Lambda^{3})$

\begin{itemize}
\item $N_{a}$ is the total concentration of atoms and ions in the plasma 
\item $\Lambda=\sqrt{\frac{h^{2}}{2\pi m_{e}k_{B}T}}$, which is the thermal
De Broglie wavelength of an electron.

\begin{itemize}
\item h is Plank's constant 
\item $m_{e}$ is the electron mass 2. 
\end{itemize}
\end{itemize}
\end{itemize}
\item Then, use \[
Z_{n+1}=Z_{n}-\frac{Z_{n}-<Z>_{n}}{1+\frac{1}{Z}_{n}(<Z^{2}>_{n}-<Z>_{n}\ ^{2})}\]
 to iteratively get a better approximation of Z.

\begin{itemize}
\item To find $<Z>_{n}$ and $<Z^{2}>_{n}$, first find the population values
using the equation in the next step, and using $Z_{n}$ for Z. Then,
use the formulas:\[
<Z>=\sum_{i=1}^{X}iP_{i}\ \ \ \ \ \ \ \ <Z^{2}>=\sum_{i=1}^{X}i^{2}P_{i}\]

\end{itemize}
\item Finally, calculate the populations using: \[
P_{i}=\frac{\left(\frac{C_{1}}{Z}\right)^{i}e^{-\frac{E_{i}}{k_{B}T_{e}}}}{{\displaystyle \sum_{i=0}^{X}\left(\frac{C_{1}}{Z}\right)^{i}e^{-\frac{E_{i}}{k_{B}T_{e}}}}}\]

\item If speed of calculation is important, use the method in section 3
(Eq. 10) instead on the previous step. 
\end{enumerate}
Following is the table that was constructed using the above algorithm.
It shows the Z values calculated for Xenon at various electron temperatures
(given in electron-volts -- the value of $k_{B}T_{e}$, where $k_{B}$
is in eV/K) and heavy particle concentrations (given in number of
particles per $cm^{3}$). Note that the unphysical negative values
of Z are used here to mark the domain, in which the electron degeneracy
$g_{e}$ is not >\,>1, as assumed in section II, and hence the ion
degeneracies $g_{i}$ are not negligible in comparison. Hence, finding
Z at those points would then require more complex computations.

\begin{center}
\input{Table1.tex} 
\par\end{center}


\section{Plasma Thermodynamic Parameters at a Given Temperature}


\subsection{Internal Energy}

After having set the populations of the ions as above, we can now
calculate some important thermodynamic characteristics of the plasma.
One such characteristic is the average internal energy per atomic
cell ($U$) -- the energy of the total combination of particles that
originally comprised a whole neutral atom. In fact, experimentally,
this energy, which fundamentally depends on T, is usually much easier
to find than the temperature itself, and hence the value of T actually
has to be calcualted from this energy. This calcualation will be further
discussed in section \ref{sec:Temperature-Calculation}. First, our
goal is to find U as a function of T for plasma. Recall that for an
ideal gas, the average internal kinetic energy per particle is given
by:

\begin{equation}
U=\frac{3}{2}k_{B}T\label{eq:25}\end{equation}


Hence, to find the internal energy of a whole atomic cell in a plasma,
we have to find the sum of the kinetic energies of the ion and all
the electrons that have separated from it, as well as the energy that
was used to separate these electrons ($E_{i}$). Then, averaging this
out across the various charge ions that are present in the plasma,
we get:

\begin{equation}
U=\frac{3}{2}k_{B}T+\sum_{i=1}^{X}P_{i}(\frac{3}{2}ik_{B}T+E_{i})=\frac{3}{2}k_{B}T(1+Z)+<E_{i}>\label{eq:26}\end{equation}


Note that this also implies that in plasma, ions have the same average
kinetic energy as electrons do, despite the immense difference in
mass. Furthermore, notice that because of the statistical averages
used in this equation ($Z$ and $<E_{i}>$), which depend on the population
values that in turn depend on the temperature, this equation cannot
be solved for T by simple algebraic methods, and hence finding the
temperature will once again require an iterative approach.


\subsection{Heat Capacity}

Another important thermodynamic parameter of a plasma is its heat
capacity at a constant volume. This parameter is needed to study heat
transmission through the plasma and the variation of temperature and
internal energy of various parts of the plasma with time. This value
will also be needed for the temperature calculation in the next section.
First of all, realize that if volume is constant and $dV=0$, then
the change in internal energy of a system is:

\[
dU=dQ-dW=dQ-PdV=dQ,\ \text{and}\]


\begin{equation}
\frac{dU}{dT}=\frac{dQ}{dT}=C_{V},\label{eq:28}\end{equation}


where dQ is the heat added to the system, while dW is the work done
by that system. 

Hence, our goal now is to find the partial derivative of U, as given
by Eq.\ref{eq:26}, with respect to T. For simplicity, from now on
we will use the value of the temperature measured in electron-volts
-- $T_{eV}=k_{B}T$, where $k_{B}$ is in eV/K. First, begin by taking
the derivative of Eq.\ref{eq:26} only:\[
C_{V}=\frac{\partial U}{\partial T_{eV}}=\frac{3}{2}+\sum\left[P_{i}\left(\frac{3}{2}i\right)+\frac{\partial P_{i}}{\partial T_{eV}}\left(\frac{3}{2}T_{eV}i+E_{i}\right)\right]=\]
 \begin{equation}
=\frac{3}{2}(1+Z)+\sum\left[\frac{\partial P_{i}}{\partial T_{eV}}\left(\frac{3}{2}T_{eV}i+E_{i}\right)\right]\label{eq:29}\end{equation}


Now, we need to find the partial derivative of the populations w.r.t.
temperature. For this we can use the expression from Eq. \ref{eq:6},
written as:

\begin{equation}
P_{i}=\frac{p_{i}}{\sum p_{i}},\ \text{where}\ \ \ \ \ \ \ \ \ \ p_{i}=\left(C_{0}\frac{(T_{eV})^{\nicefrac{3}{2}}}{Z}\right)^{i}e^{-\frac{E_{i}}{T_{eV}}},\label{eq:30}\end{equation}


where $C_{0}$ is now independent of $T_{eV}$. Thus:

\begin{equation}
\frac{\partial P_{i}}{\partial T_{eV}}=\frac{p_{i}'}{\sum p_{i}}-\frac{p_{i}\sum p_{i}'}{\left(\sum p_{i}\right)^{2}},\label{eq:31}\end{equation}


where primes denote the derivatives with respect to $T_{eV}$. Also,
recall that Z is also a function of T, hence for $p_{i}'$ we have:

\[
p_{i}'=\left(C_{0}\frac{(T_{eV})^{\nicefrac{3}{2}}}{Z}\right)^{i}e^{-\frac{E_{i}}{T_{eV}}}\left(\frac{E_{i}}{(T_{eV})^{2}}\right)+\left(C_{0}^{\ \ i}\right)i\left(\frac{(T_{eV})^{\nicefrac{3}{2}}}{Z}\right)^{i-1}\left(\frac{\frac{3}{2}(T_{eV})^{\nicefrac{1}{2}}}{Z}-\frac{(T_{eV})^{\nicefrac{3}{2}}}{Z^{2}}Z'\right)e^{-\frac{E_{i}}{T_{eV}}}=\]


\begin{equation}
=p_{i}\left(\frac{E_{i}}{(T_{eV})^{2}}+i\frac{Z}{(T_{eV})^{\nicefrac{3}{2}}}\left(\frac{\frac{3}{2}(T_{eV})^{\nicefrac{1}{2}}}{Z}-\frac{(T_{eV})^{\nicefrac{3}{2}}}{Z^{2}}Z'\right)\right)=p_{i}\left(\frac{E_{i}}{(T_{eV})^{2}}+i\left(\frac{3}{2T_{eV}}-\frac{Z'}{Z}\right)\right)\label{eq:32}\end{equation}


Also, since\begin{equation}
Z=\frac{\sum ip_{i}}{\sum p_{i}},\ \text{it follows that}\ \ \ \ \ \ \ Z'=\frac{\sum ip_{i}'}{\sum p_{i}}-\frac{\sum ip_{i}\sum p_{i}'}{\left(\sum p_{i}\right)^{2}},\label{eq:33}\end{equation}


where we can then plug in the expression found for $p_{i}'$ in Eq.\ref{eq:32}.
Let's first simplify $\sum p_{i}'$:

\begin{equation}
\sum p_{i}'=\sum\left[\frac{E_{i}p_{i}}{(T_{eV})^{2}}+ip_{i}\left(\frac{3}{2T_{eV}}-\frac{Z'}{Z}\right)\right]=\frac{\sum E_{i}p_{i}}{(T_{eV})^{2}}+\left(\frac{3}{2T_{eV}}-\frac{Z'}{Z}\right)\sum ip_{i}\label{eq:34}\end{equation}
 Further note that for any expression $a$ it is true that its statistical
average is $<a>={\displaystyle \sum_{i=0}^{X}aP_{i}=\frac{\sum ap_{i}}{\sum p_{i}}}$.
Hence, using these formulas in Eq.\ref{eq:33} we get:

\[
Z'=\frac{1}{(T_{eV})^{2}}\frac{\sum iE_{i}p_{i}}{\sum p_{i}}+\left(\frac{3}{2T_{eV}}-\frac{Z'}{Z}\right)\frac{\sum i^{2}p_{i}}{\sum p_{i}}-\frac{1}{(T_{eV})^{2}}\frac{\sum ip_{i}\sum E_{i}p_{i}}{\left(\sum p_{i}\right)^{2}}-\left(\frac{3}{2T_{eV}}-\frac{Z'}{Z}\right)\frac{\left(\sum ip_{i}\right)^{2}}{\left(\sum p_{i}\right)^{2}}=\]
 \[
=\frac{<iE_{i}>-<i><E_{i}>}{(T_{eV})^{2}}+\left(\frac{3}{2T_{eV}}-\frac{Z'}{Z}\right)\left(<i^{2}>-<i>^{2}\right)\]
 \[
Z'\left(\frac{Z+<i^{2}>-<i>^{2}}{Z}\right)=\frac{<iE_{i}>-<i><E_{i}>}{(T_{eV})^{2}}+\frac{3}{2T_{eV}}\left(<i^{2}>-<i>^{2}\right)\]


\begin{equation}
\frac{Z'}{Z}=\frac{\frac{3}{2}T_{eV}\left(<i^{2}>-<i>^{2}\right)+<iE_{i}>-<i><E_{i}>}{(T_{eV})^{2}\left(Z+<i^{2}>-<i>^{2}\right)}\label{eq:35}\end{equation}


Finally, we have reached the bottom of this derivative -- everything
in the above equation can be found using the methods presented in
Part I. Now, we can significantly simplify this whole derivative by
working our way up to the beginning. However first, we need to introduce
the following transformation, which is true for any two expressions
$a$ and $b$:

\begin{equation}
<ab>-<a><b>=<\delta a\ \delta b>,\ \text{where}\ \ \ \ \delta a=a-<a>\label{eq:36}\end{equation}
 \[
\text{proof:}\ \ \ <(a-<a>)\ (b-<b>)>=<ab-a<b>-b<a>+<a><b>>=\]
 \[
=<ab>-2<a><b>+<a><b>=<ab>-<a><b>\]


This transformation makes the computation simpler, as well as, in
some cases, gives better accuracy for the value of $<\delta a\ \delta b>$,
which can determine the difference between a zero and a non-zero value.
Hence, using this transformation and combining Eq.\ref{eq:35} and
Eq.\ref{eq:32} we get:

\[
p_{i}'=p_{i}\left(\frac{E_{i}}{(T_{eV})^{2}}+i\ \frac{\frac{3}{2}T_{eV}\left(Z+<(\delta i)^{2}>\right)-\frac{3}{2}T_{eV}<(\delta i)^{2}>-<\delta i\ \delta E_{i}>}{(T_{eV})^{2}\left(Z+<(\delta i)^{2}>\right)}\right)=\]
 \begin{equation}
=\frac{p_{i}}{(T_{eV})^{2}}\left(E_{i}+i\ \frac{\frac{3}{2}T_{eV}Z-<\delta i\ \delta E_{i}>}{Z+<(\delta i)^{2}>}\right),\ \text{and let this}=\alpha p_{i}\label{eq:37}\end{equation}


Here $\alpha$ is introduced only to simplify the notation for the
next step, which then transforms Eq.\ref{eq:31} as:

\[
\frac{\partial P_{i}}{\partial T_{eV}}=\frac{\alpha p_{i}}{\sum p_{i}}-\frac{p_{i}\sum\alpha p_{i}}{\left(\sum p_{i}\right)^{2}}=P_{i}(\alpha-<\alpha>)=P_{i}(\delta\alpha)=\]
 \begin{equation}
=\frac{P_{i}}{(T_{eV})^{2}}\left(\delta E_{i}+\delta i\left(\frac{\frac{3}{2}T_{eV}Z-<\delta i\ \delta E_{i}>}{Z+<(\delta i)^{2}>}\right)\right),\ \text{and let this}=\frac{P_{i}}{(T_{eV})^{2}}\left(\delta E_{i}+\beta\delta i\right)\label{eq:38}\end{equation}


Here, again, we introduce $\beta$ only for simplicity of notation.
At this point, we can finally come back to our original Eq.\ref{eq:29}.
First, however, note that the transformation given in Eq.\ref{eq:36}
can also be written as:\begin{equation}
<\delta a\ b>=<ab-<a>b>=<ab>-<a><b>=<\delta a\ \delta b>\label{eq:39}\end{equation}


Using this and Eq.\ref{eq:38} we can now transform Eq.\ref{eq:29}
as follows:

\[
C_{V}=\frac{\partial U}{\partial T_{eV}}=\frac{3}{2}(1+Z)+\sum\left[\frac{P_{i}}{(T_{eV})^{2}}\left(\delta E_{i}+\beta\delta i\right)\left(\frac{3}{2}T_{eV}i+E_{i}\right)\right]=\]


\[
=\frac{3}{2}(1+Z)+\frac{1}{(T_{eV})^{2}}<\left(\delta E_{i}+\beta\delta i\right)\left(\frac{3}{2}T_{eV}\delta i+\delta E_{i}\right)>=\]


\[
=\frac{3}{2}(1+Z)+\frac{1}{(T_{eV})^{2}}<(\delta E_{i})^{2}+\left(\frac{3}{2}T_{eV}+\beta\right)\delta i\ \delta E_{i}+\frac{3}{2}T_{eV}\beta(\delta i)^{2}>=\]


\[
=\frac{3}{2}(1+Z)+\frac{1}{(T_{eV})^{2}}\Biggl(<(\delta E_{i})^{2}>+\left(\frac{\frac{3}{2}T_{eV}\left(Z+<(\delta i)^{2}>\right)+\frac{3}{2}T_{eV}Z-<\delta i\ \delta E_{i}>}{Z+<(\delta i)^{2}>}\right)<\delta i\ \delta E_{i}>+\]


\[
+\frac{3}{2}T_{eV}\frac{\frac{3}{2}T_{eV}Z-<\delta i\ \delta E_{i}>}{Z+<(\delta i)^{2}>}<(\delta i)^{2}>\Biggr)=\]


\[
=\frac{3}{2}(1+Z)+\frac{1}{(T_{eV})^{2}}\left(<(\delta E_{i})^{2}>+\frac{3T_{eV}Z<\delta i\ \delta E_{i}>-<\delta i\ \delta E_{i}>^{2}+\frac{9}{4}(T_{eV})^{2}Z<(\delta i)^{2}>}{Z+<(\delta i)^{2}>}\right)=\]


\begin{equation}
=\frac{3}{2}(1+Z)+<\left(\frac{\delta E_{i}}{T_{eV}}\right)^{2}>+\frac{3Z\left(\frac{3}{4}<(\delta i)^{2}>+<\delta i\ \frac{\delta E_{i}}{T_{eV}}>\right)-<\delta i\ \frac{\delta E_{i}}{T_{eV}}>^{2}}{Z+<(\delta i)^{2}>}=C_{V}\label{eq:Cv}\end{equation}


This equation can then be used to find the heat capacity at constant
volume when the temperature is known. Following is a table built using
this equation, as well as Eq. \ref{eq:26}, that shows the values
of average internal energy {[}eV] and heat capacity per atomic cell,
calculated at various values of $T_{eV}$ and $N_{a}$. Once again,
the unrealistic negative values mark the domain where our assumption
that $g_{e}\gg1$ fails, and hence a different method is required
for finding the correct plasma parameters.

\begin{center}
\input{Table2.tex} 
\par\end{center}


\section{Temperature Calculation from a Given Internal Energy\label{sec:Temperature-Calculation}}

In realistic experiments, it is usually not possible, or not easy,
to measure the temperature directly. It is, however, possible to measure
the total internal energy of a volume of plasma, which can then, using
a known heavy particle density value, be converted to the average
internal energy per atomic cell ($U$). This energy can then be used
to find the value of the temperature, however, as can be seen from
Eq. \ref{eq:26}, this cannot be done algebraically and hence, we
will once again use the Newton-Rapson iterations method. As before,
since this approach can only be used to iteratively find the zeros
of a function, this function needs to be \begin{equation}
f(T)=\frac{3}{2}k_{B}T(1+Z)+<E_{i}>-U,\label{eq:27}\end{equation}
 where U is a given constant. Then when $f(T)=0$, we will have the
desired value for T. Furhtermore, to use this mehtod, we also need
to have an expression for $f'(T)$, which now becomes simmply $\partial U/\partial T=C_{V}$.
Hence, we have everything needed to prroceed with the Newton's method:

\[
T_{n+1}=T_{n}-\frac{U(T_{n})-U_{given}}{C_{V}(T_{n})},\ \text{where}\ \ U=\frac{3}{2}k_{B}T(1+Z)+<E_{i}>\text{and}\]


\[
C_{V}=\frac{3}{2}(1+Z)+<\left(\frac{\delta E_{i}}{T_{eV}}\right)^{2}>+\frac{3Z\left(\frac{3}{4}<(\delta i)^{2}>+<\delta i\ \frac{\delta E_{i}}{T_{eV}}>\right)-<\delta i\ \frac{\delta E_{i}}{T_{eV}}>^{2}}{Z+<(\delta i)^{2}>},\]


while for any expression $a$, $<a>={\displaystyle \sum_{i=0}^{X}aP_{i}}$
and $\delta a=a-<a>$, where the populations are found using the trial
value of T and the method presented in section \ref{sec:Ionization-Equilibrium}. 

However, unlike with iterations on Z, here convergence is not guaranteed
when starting at any value of T. Although in most cases the iterations
will eventually converge, they will do so slowly and before approaching
the correct value, they might make huge jumps from one side of the
correct T to another, which can lead to T becoming negative at one
of the iterations. Since to calculate the populations and Z, which
is necessary to do for every iteration, we need to at some point find
the value of $\sqrt{T}^{3}$, when T becomes negative, this has no
real answer and will make further iterations impossible. To make matters
worse, because the relationship between U and T is so complex, there
is no nice way to initially approximate the value of T so as to speed
up the iterations and avoid huge jumps.

Therefore, the only way left to avoid this problem is to hard-code
some limits on the iteration jumps, such as not letting the iterations
jump lower than half or higher than twice the trial value of T. This
way, the Newton-Rapson equation will initially only point out the
general direction of the correct value of T relative to the trial
value, while the iterations will only half or double the trial value
until it gets close enough to the actual T. Following is a table of
the temperature values {[}eV] calculated at the given internal energies
{[}eV] and heavy particle densities {[}$1/cm^{3}$]. The number of
iterations it took to find each temperature value is also given and
the negative values are used as before.

\begin{center}
\input{Table3.tex} 
\par\end{center}
\end{document}
