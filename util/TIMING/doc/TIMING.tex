%^CFG COPYRIGHT UM
\section{util/TIMING module}

\subsection{Introduction}

This module was developed by G. T\'oth (2001-).
It can be used for timing and profiling Fortran 90 codes.

It is platform and compiler independent and very easy to use.
It can provide profiling information while the code is running.
The amount and type of information can be easily manipulated.

Profiling with a 'real' profiler is compiler and platform dependent, 
it can only be done after the run is finished, and the amount and
type of information is not always easy to manipulate. On the other hand
a profiler may provide more accurate and detailed information than 
this timing module, and it does not require changes of the code.

\subsection{Usage}

The TIMING module can time anything identified by a name string.
Here is a short example of usage:
{\footnotesize
\begin{verbatim}
  call timing_version(on,vname,vnum)! check the version
  ....
  call timing_active(.true.)        ! Activate timing
  call timing_step(0)               ! Initialize step value
  call timing_start('main')         ! Start timing the main code
  ...
  do nstep=1,100
     call timing_step(nstep)        ! Put step into timing module
     call timing_start('whatever')  ! Start timing
     ...                            ! Do whatever
     call timing_stop('whatever')   ! Stop timing
     call timing_show('whatever',1) ! Show last timing for whatever 
     if (mod(nstep,10)==0) then
        write(*,*) &                ! Obtain and write speed
            'speed of whatever is',
            1./timing_func_d('sum/iter',1,'whatever','main'),&
	    ' iterations/sec'
        call timing_report          ! Show tree of timings for last 10 steps
        call timing_reset_all       ! Reset timing
     end if
  end do
  call timing_stop('main')          ! Stop timing the main code
  call timing_report_total          ! Show all timings as a sorted list
  call timing_report_style('tree')  ! Change report style
  call timing_report_total          ! Show all timings in calling tree
\end{verbatim}
}
The timing\_version returns three values: the first logical variable
'on' is true for a functional timing module, and false for
the empty timing module. The second string variable 'name' (of length 40)
returns the name and the author of the module, and the last real variable
'number' returns the version number. 

The timing is activated by timing\_active. For parallel runs
one should usually activate the timing module for one processor only.
When the timing module is inactive, the timing commands are executed 
but do not time and do not provide output.

The timing is done by a pair of timing\_start and timing\_stop calls.
The string name arguments of the two calls must match. 
Timings can be inside loops, and nested arbitrarily. 
Note, however, that timing inside a recursive procedure does not work.
The timings are distinguished by the name as well as by the nesting level.

The timing clocks can be reset, and the results of timings can be printed 
to the screen or returned into variables in various formats as discussed 
in the following sections. 

\subsubsection{Clocks and resets}

There are 3 clocks started and stopped by timing\_start
and timing\_stop. Clock 1 always measures the latest timings,
clock 2 measures cummulative timings since the last reset,
while clock 3 typically measures cummulative timings for the whole run.

The clocks can be reset by
\begin{verbatim}
  call timing_reset('whatever',2)
\end{verbatim}
which resets clocks 1 and 2 for the timing of 'whatever'. 
The first string argument is 'name', and the second integer
argument is 'nclock', i.e. the number of clocks to be reset
starting with clock 1.

If the 'name' argument is set to '\#all', 
then the clocks 1 to nclock are reset for all names.
The particularly useful and typical call 
\begin{verbatim}
  call timing_reset('#all',2)
\end{verbatim}
is identical with the shortcut version
\begin{verbatim}
  call timing_reset_all
\end{verbatim}
Note that active timings (started but not yet stopped) are not stopped 
by the reset, but the start time is overwritten
so that only the time after the reset is measured.

Beside measuring cummulative timings, clocks 2 and 3 also count the number 
of 'calls', and the number of 'iterations' for each timing entry. 
The iterations are distinguished by the current step number 
(a monotonically increasing positive integer) which can be set by calling
timing\_param\_put\_i or timing\_step, which are
described in the next subsection.

\subsubsection{Putting parameters: \\
{\tt timing\_param\_put\_i, timing\_step, timing\_depth, 
timing\_report\_style}}

The integer parameters for the timing module can be set with the generic
subroutine call
\begin{verbatim}
  call timing_param_put_i('depth',2,error)
\end{verbatim}
where the first string argument is the name of the function,
the second integer is the value, and the third integer argument
returns 0 if the parameter was set successfully, or -1 if it failed.

There are only two integer parameters for the timing module:
'step' gives the current step, while 'depth' is the maximum
depth of nested timings. For these two parameter settings the 
following short cuts are provided:
{\small
\begin{verbatim}
  timing_step(value)  ! same as timing_param_put_i('step',value,error)
  timing_depth(value) ! same as timing_param_put_i('depth',value,error)
\end{verbatim}
}
The default value for 'step' is 0. It is expected to be set to
a positive integer value which is monotonously increasing in
successive calls.

The default value of 'depth' is -1, which 
means that the timings can be nested arbitrarily deep. 
If depth is set to 0, then no timing is done at all, while
if depth is set to 1, only the main code is timed.

The style of the report shown by the generic 
{\tt timing\_report} and  {\tt timing\_report\_total} subroutines
is determined by the {\tt report\_style}. The default style is
'cumm', which produces cummulative timings sorted by the timing values.
The 'list' style also gives sorted timings, but timings with different calling
parents are distinguished. Finally the 'tree' style gives the timings
in the format of a nested calling tree.

\subsubsection{Reading the timings: {\tt timing\_func\_d}}

The current timing value of clock 2 for 'whatever' called from 'main'
can be obtined by the function call
\begin{verbatim}
  timing_func_d('sum',2,'whatever','main')
\end{verbatim}
The first string argument 'func\_name' determines the function
to be returned. The available values are 'sum', 'sum/iter' and 'sum/call'.
The latter two functions only make sense for clocks 2 and 3.
The second integer argument 'iclock' selects the clock.
The third string argument 'name' selects the timing,
which is further specified by the last string argument 'parent\_name'.
The parent is the timing that was started last but not stopped when
the timing for 'whatever' is started. The parent of the first timing
is itself, so
\begin{verbatim}
  write(*,*)'Elapsed time=',timing_func_d('sum',1,'main','main')
\end{verbatim}
prints out the total time spent by 'main' since the last reset. 
The parent is needed to distinquish between timings called from 
different places. If the names do not match, no output is produced.

\subsubsection{Show individual timings: {\tt timing\_show}}

Results for a certain timing can be printed with the timing\_show command.
The first string argument 'name' is the name of the timing to be shown, 
the second integer argument 'iclock' is the selected clock number.

For clock 1 the name, the calling parent, and the very last timing are shown:
\begin{verbatim}
  call timing_show('calc_gradients',1)
  Last timing for calc_gradients (advance_expl):    0.01 sec
\end{verbatim}
For clock 2 the cummulative timing since the last reset is given.
All timings matching the name (but called from different parents) are shown.
The timing per iteration and per call and the percentage
with respect to the parent are also shown:
{\small
\begin{verbatim}
  call timing_show('calc_gradients',2)
  Timing for calc_gradients from step      15 to      20 :
      0.55 sec,    0.111 s/iter   0.011 s/call   26.66 % of advance_expl
  Timing for calc_gradients from step      15 to      20 :
      0.01 sec,    0.008 s/iter   0.008 s/call    0.32 % of timing_test
\end{verbatim}
}
For clock 3 the total timing is reported:
{\small
\begin{verbatim}
call timing_show('calc_gradients',3)
Timing for calc_gradients at step      20 :
    1.11 sec,    0.111 s/iter   0.011 s/call   26.69 % of advance_expl
Timing for calc_gradients at step      20 :
    0.01 sec,    0.008 s/iter   0.008 s/call    0.16 % of timing_test
\end{verbatim}
}

\subsubsection{Timing reports and profiling: \\
      {\tt timing\_sort, timing\_tree, timing\_report}}

For most purposes one can use the following two generic subroutines
\begin{verbatim}
  timing_report  
  ! same as timing_sort(2,-1,.true.)  if style is 'cumm'
  ! same as timing_sort(2,-1,.false.) if style is 'list'
  ! same as timing_tree(2,-1)         if style is 'tree'

  timing_report_total
  ! same as timing_sort(3,-1,.true.)  if style is 'cumm'
  ! same as timing_sort(3,-1,.false.) if style is 'list'
  ! same as timing_tree(3,-1)         if style is 'tree'
\end{verbatim}
In the following the general timing\_tree and timing\_sort
subroutines are described in detail. 

The timings of all or some of the subroutines can be reported
in various ways. The most complete information is obtained by
\begin{verbatim}
  call timing_tree(2,-1)
\end{verbatim}
where 2 is the clock number, and the second argument is the maximum
depth of the tree to be shown (-1 means to show the whole tree).
The output is shown in Table~\ref{t-timingtree1}.
\begin{table}
\caption{Output of {\tt timing\_tree(2,-1)}}
{\footnotesize
\begin{verbatim}

-------------------------------------------------------------------------------
TIMING TREE from step      15 to step      20
name                  #iter  #calls      sec   s/iter   s/call  percent
-------------------------------------------------------------------------------
timing_test               1       1     2.54    2.536    2.536   100.00
-------------------------------------------------------------------------------
advance_expl              5       5     2.02    0.404    0.404    79.65
  calc_gradients          5      50       0.50    0.100    0.010    24.79
  calc_facevalues         5      50       1.02    0.203    0.020    50.32
  #others                                 0.50    0.101             24.88
calc_gradients            1       1     0.01    0.008    0.008     0.31
save_output               1       1     0.20    0.201    0.201     7.92
#others                                 0.32    0.315             12.42
-------------------------------------------------------------------------------

\end{verbatim}
}
\label{t-timingtree1}
\end{table}
The header indicates that the timing tree was generated at time step 20
by clock 2 which was restarted at step 15. So the timings refer to
5 time steps. 

The table consists of seven columns and several rows:
\begin{enumerate}
\item 
The 1st column gives the name of the timing. 
    The very first row is the top of the tree, usually refers the main program.
    The names below the first row are indented according to the calling depth:
    timings called directly from the top timing are not indented, 
    timings called from these are indented by 2 spaces, 
    timings called from these are indented by 4 spaces, etc.
\item
The 2nd column gives the number of iterations when a timing call was maede.
\item
The 3rd column gives the number of timing calls for an item.
\item
The 4th through 6th columns give the actual timings in seconds: 
    total time, time/iteration and time/call.
\item
The 7th column gives the percentage with respect to the 
    calling 'parent'. The consecutive lines at the
    same indentation level should always add up to 100\%, because
    the last row with name '\#other' contains the untimed part of any 
    given level.
\end{enumerate}

In the example presented in Table~\ref{t-timingtree1} 
timing\_test took 2.54 seconds to run from step 15 to 20. 
Roughly 80\% of the time was spent in advance\_expl, and 8\% on save\_output.
Advance\_expl itself took 2.01 seconds or 0.4 sec/step. 50\% of this time
was spent on calc\_facevalues, 25\% on calc\_gradients, and 25\% on 
other things.
Note that calc\_gradients occurs twice, because it is called from
the main program and advance\_expl as well. 

The amount of detail can be decreased by giving a maximum depth. For example
\begin{verbatim}
  call timing_tree(2,2)
\end{verbatim}
will produce a table without the indented 3rd to 5th rows. 
The timing\_tree cannot be used with clock 1, because clock 1 does
not accumulate timings, which makes the information of the table
rather difficult to interpret. Clock 1 timings are better presented
by the 'timing\_sort' subroutine, which is discussed next.

Another way of representing the timing results is
\begin{verbatim}
  call timing_sort(1,-1,.true.)
\end{verbatim}
which shows the full uniquely sorted timings for clock 1. 
The first argument 'iclock' selects the clock, the second argument 
'show\_length' defines the maximum number of timings shown 
(-1 means show all), 
and the third argument 'unique' determines whether
the timings for identical names but different calling parents
should be added up or not. 
When clock 1 is used the timings are given for the very last call.
A sample output is shown in Table~\ref{t-timinglist1}.
\begin{table}
\caption{Output of {\tt timing\_sort(1,-1,.true.)}}
{\footnotesize
\begin{verbatim}

-------------------------------------------------------------------------------
SORTED TIMING at step=      20
name                     sec percent
-------------------------------------------------------------------------------
timing_test             2.54  100.00
-------------------------------------------------------------------------------
advance_expl            0.40   15.77
save_output             0.01    0.31
calc_facevalues         0.02    0.87
calc_gradients          0.01    0.34
initialize              0.00    0.00
-------------------------------------------------------------------------------

\end{verbatim}
}
\label{t-timinglist1}
\end{table}
The percentages are with respect to the longest timing in the first row.

If clock 2 or 3 is used, the table contains the cummulative timings 
and the number of steps and calls are also indicated. 
For example the first four of the uniquely sorted timings of clock 2 
can be obtained with
\begin{verbatim}
  call timing_sort(2,4,.true.)
\end{verbatim}
which gives an output as shown in Table~\ref{t-timinglist2}.
\begin{table}
\caption{Output of {\tt timing\_sort(2,4,.true.)}}

{\footnotesize
\begin{verbatim}

-------------------------------------------------------------------------------
SORTED TIMING from step=      15 to step=      20
name                     sec percent   #iter  #calls
-------------------------------------------------------------------------------
timing_test             2.71  100.00       1       1
-------------------------------------------------------------------------------
advance_expl            2.14   78.88       5       5
calc_facevalues         1.04   38.17       5      50
calc_gradients          0.60   21.96       6      51
#others                 0.20    7.40
-------------------------------------------------------------------------------

\end{verbatim}
}
\label{t-timinglist2}
\end{table}
Note that calc\_gradients was called 51 times altogether. The last row
with name '\#others' contains the sum of timings that were not included
into the first 4 rows. Also note that the total percentage exceeds 
100\% since the timings at different depths overlap.

Finally the original timings can be sorted without adding up values
for the same subroutine. In this case the parents are also indicated,
so that timings with identical names can be distinguished:
\begin{verbatim}
  call timing_sort(3,-1,.false.)
\end{verbatim}
results in Table~\ref{t-timinglist3}.

\begin{table}
\caption{Output of {\tt timing\_sort(3,-1,.false.)}}
{\footnotesize
\begin{verbatim}

-------------------------------------------------------------------------------
SORTED TIMING at step=      20
name                (parent)                 sec percent   #iter  #calls
-------------------------------------------------------------------------------
timing_test         (timing_test)           5.21  100.00       1       1
-------------------------------------------------------------------------------
advance_expl        (timing_test)           4.23   81.23      10      10
calc_facevalues     (advance_expl)          2.05   39.38      10     100
calc_gradients      (advance_expl)          1.16   22.26      10     100
initialize          (timing_test)           0.30    5.76       1       1
save_output         (timing_test)           0.20    3.85       1       1
calc_gradients      (timing_test)           0.01    0.22       1       1
-------------------------------------------------------------------------------

\end{verbatim}
}
\label{t-timinglist3}
\end{table}

\newpage
\subsection{List of subroutines and functions}

\begin{verbatim}
option_timing(on,name)

timing_active(value)

timing_param_put_i(name,value,error)
timing_step(value)  ! == timing_param_put_i('step',value,error)
timing_depth(value) ! == timing_param_put_i('depth',value,error)

timing_report_style(value)

timing_start(name)

timing_stop(name)

timing_reset(name,nclock)
timing_reset_all    ! == timing_reset('#all',2)

timing_show(name,iclock)

timing_sort(iclock,show_length,unique)

timing_tree(iclock,show_depth)

timing_report       ! == timing_sort(2,-1,.true.)  for style 'cumm'
                    ! == timing_sort(2,-1,.false.) for style 'list'
                    ! == timing_tree(2,-1)         for style 'tree'
timing_report_total ! == timing_sort(3,-1,.true.)  for style 'cumm'
                    ! == timing_sort(3,-1,.false.) for style 'list'
                    ! == timing_tree(3,-1)         for style 'tree'

real*8 function timing_func_d(func_name,iclock,name,parent_name)
\end{verbatim}

\newpage
\subsection{Files and make targets}

A complete list of make targets can be listed with
\begin{verbatim}
  make help
\end{verbatim}
The actual TIMING module consists of
\begin{verbatim}
  ModTiming.f90
  timing.f90
  timing_cpu.f90
\end{verbatim}
The last file contains the call to the actual timing function,
e.g. MPI\_WTIME, or SYSTEM\_CLOCK.
The three source files can be compiled into one library module:
\begin{verbatim}
  libTIMING.a
\end{verbatim}
with the command
\begin{verbatim}
  make LIB
\end{verbatim}
Compiler options can be edited in the 
\begin{verbatim}
  Makefile
\end{verbatim}
or given as command line options e.g. as
\begin{verbatim}
  make LIB FTN=f95 CFLAG='-c -O4' LFLAG='-O4'
\end{verbatim}
where FTN defines the compiler, CFLAG the compile flags, and LFLAG the
link flags. This is usefule when the library is compiled from another
Makefile. The interface and an empty version of the TIMING module
is defined by
\begin{verbatim}
  timing_empty.f90
\end{verbatim}
When the TIMING module is not needed for the main code, 
libTIMING.a should be replaced by
\begin{verbatim}
  timing_empty.o
\end{verbatim}
which can be compiled with
\begin{verbatim}
  make EMPTY
\end{verbatim}
The empty module does not use any memory, most subroutine calls 
return directly to the caller without any output. The exceptions
are timing\_version, which tells the calling program that the
empty timing routine is not functional, and timing\_active, which
writes a warning message if an attempt is made to activate 
the empty timing module.

The use of the TIMING module is fully demonstrated in 
\begin{verbatim}
  timing_test.f90
\end{verbatim}
which can be compiled both to a serial and a parallel code, both with
the real and the empty timing module. These four combinations
provide 4 tests, which can be all executed with
\begin{verbatim}
  make TEST
\end{verbatim}
A sample output can be found in
\begin{verbatim}
  EXAMPLE_OUTPUT
\end{verbatim}
which was obtained by
\begin{verbatim}
  make TEST > EXAMPLE_OUTPUT
\end{verbatim}

This manual was produced from
\begin{verbatim}
  MAN_TIMING.tex
\end{verbatim}
with
\begin{verbatim}
  make MAN
\end{verbatim}
The distribution can be cleaned with
\begin{verbatim}
  make clean
  make distclean
\end{verbatim}
and a distribution produced with
\begin{verbatim}
  make dist
\end{verbatim}

