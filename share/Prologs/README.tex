\documentclass[10pt]{article}

\title{Protex Templates for Developing SWMF and Physics Components}

\author{G\'abor T\'oth}

\begin{document}

\maketitle

\tableofcontents

\section{Introduction to the Usage of Protex}

This directory contains files which should be used to create self documented
modules, subroutines and functions using the ProTex documentation system.
In the following text {\it procedure} means a Fortran subroutine or function.

Please use the templates provided in the following files:
\begin{verbatim}
   program.f90              - template for an F90 program
   module.f90               - template for an F90 module
   subroutine.f90           - template for an F90 procedure
   internal_subroutine.f90  - template for an F90 internal procedure
   Makefile                 - an example for scripts and Makefile-s
\end{verbatim}
The templates contain compilable source code (except for the use statements), 
and they can be converted to Protex. 

When writing actual source code, the templates should be followed as closely 
as possible, but only the actually relevant tags should be used.
Here is a list of the tags that are used in SWMF:

\newpage

\begin{verbatim}
Tags                       Where to use it
----------------------------------------------------------------------------
!INTERFACE:              - in all documented modules/procedures

!MODULE:                 - in all documented modules

!ROUTINE:                - in all documented procedures

!IROUTINE:               - in all documented internal procedures

!DESCRIPTION:            - in all documented module/procedure 
                           whose purpose is not trivial

!USES:                   - in a module/procedure which uses other modules

!PUBLIC DATA MEMBERS:    - in a module which defines public constants
                           (Fortran parameters with public attribute)

!PUBLIC TYPES:           - in a module which defines public types

!PUBLIC MEMBER FUNCTIONS: - in a module which contains public procedures

!INPUT ARGUMENTS:         - in a procedure with intent(in) argument

!OUTPUT ARGUMENTS:        - in a procedure with intent(out) argument

!INPUT/OUTPUT ARGUMENTS:  - in a procedure with intent(inout) argument

!RETURN VALUE:            - in a function
 
!LOCAL VARIABLES:         - in a procedure/module with local variables 
                            which are worth documenting

!BOC ... !EOC             - in a procedure/module with source code
                            which is worth including into the manual

!REVISION HISTORY:        - in a procedure/module which has 
                            its own revision history

!QUOTE:                   - in a file which needs extra LaTex commands
\end{verbatim}

\newpage

\section{Documentation Produced from the Templates}

The rest of this documents shows the Protex documentation
produced from the templates. To learn how Protex should be
used, study the templates themselves. This documentation
shows the result.

\input{PROTEX}

\end{document}

===============================================================================
The files below contain information that was provided with the original 
protex code. It is somewhat obsolete and not to be used in the development 
of SWMF and its components.
===============================================================================

PrologIntro.txt    is to be used for the  occasional introduction 
			to a long file

PrologModule.txt   is to be used for your f90 module definition

PrologIRoutine.txt is to be used for subroutines/functions that come 
			after a CONTAINS statement, i.e., routines 
			internal to a Module or subroutine.

PrologRoutine.txt  is to be used for your usual subroutine/functions (read: 
			not internal)
			Routines can either be Fortran90 subroutines, 
			functions or main programs, which are not part 
			of a module (for routines inside a module use 
			!IROUTINE:).

