%$Id$
%

\chapter{List of input commands}

Input commands are set in logical blocks, which are
marked by lines \texttt{\#NAME} - beginning of 
a block {\bf NAME} - and \texttt{\#endNAME} - end
of that block. Input commands are listed below 
sorted by blocks.



%%%%%%%%%%%%%%%%%%%%%%%%%%%%%%%%%%%%%%%%%%%%%%%%%%%%%%%%%%%%%%%%
\section{Main}

The section should start with {\tt \#main} and finish with {\tt \#endmain}. 

\begin{itemize}

\item {\bf SpeciesList}={\it species-list}

Definition of the species used in the simulation, names of the species should
be the same as they are used in the source code. \\
{\tt SpeciesList=Na,H,O2PLUS,O\_THERMAL}

\item {\bf makefile}

Obsolete but still can be used. Introduces changes in the Makefile.\\
{\tt makefile MPIRUN=mpirun -np 16}\\
{\tt makefile SPICE=/Users/MyUser/MySpiceFolder}

\item {\bf ForceRepeatableSimulationPath}=[off,on] \\ Force AMPS to run a repeatable calculations by disabling all code optimizations/procedures that are executed based on the actual execution time \\ {\tt ForceRepeatableSimulationPath=on} \\ {\tt  ForceRepeatableSimulationPath=off}  

\item {\bf DebuggerMode}=[off,on]

Switches ON/OFF debugger mode of execution

\item {\bf CouplerMode}=[off,swmf,ices]

Defines the mode of coupler execution, three are available:
\\{\tt CouplerMode=off} - no coupling, 
\\{\tt CouplerMode=swmf} - coupling within SWMF, 
\\{\tt CouplerMode=ices} - coupling using ICES tool.

\item {\bf SourceDirectory}={\it source-directory}

The directory where the main source code is located, usually it is
{\tt src} folder.\\
{\tt SourceDirectory=src}

\item {\bf ProjectSourceDirectory}={\it project-source-directory}

The directory where the source code for particular project is located.\\
{\tt ProjectSourceDirectory=srcEuropa}

\item {\bf WorkingSourceDirectory}={\it working-source-directory}

The directory where the code will be assembled and compiled
(choose the one that doesn't exist in the repository in order
to avoid deletion of source code files!)\\
{\tt WorkingSourceDirectory=srcTemp}

\item {\bf TrajectoryIntersectionWithBlockFaces}

\item {\bf StdOutErrorLog}=[off,on] \\ Switch (on/off) that controls output of the error messages on screen

\item {\bf TimeStepMode}=[SingleGlobalTimeStep,SpeciesGlobalTimeStep, \\ SingleLocalTimeStep,SpeciesLocalTimeStep]

Defines the way the time step is performed, four are available:
\\{\tt TimeStepMode=SingleGlobalTimeStep} - the same time step
for the entire domain and all species,
\\{\tt TimeStepMode=SpeciesGlobalTimeStep} - time step is different for
different species, but the same for the entire domain,
\\{\tt TimeStepMode=SingleLocalTimeStep} - time step is the same for all species, 
but differs within the domain,
\\{\tt TimeStepMode=SpeciesLocalTimeStep} - time step is different for different
species and differs within the domain.

\item {\bf ParticleWeightMode}=[SingleGlobalParticleWeight,SpeciesGlobalParticleWeight, \\ SingleLocalParticleWeight,SpeciesLocalParticleWeight]

Defines the way the statistical weights of particles are defined, for are available:
\\{\tt  ParticleWeightMode=SingleGlobalParticleWeight},
\\{\tt  ParticleWeightMode=SpeciesGlobalParticleWeight},
\\{\tt  ParticleWeightMode=SingleLocalParticleWeight},
\\{\tt  ParticleWeightMode=SpeciesLocalParticleWeight}.

\item {\bf ParticleWeightCorrectionMode}=[off,on] \\
Switches ON/OFF.

\item {\bf ErrorLog}={\it file-name}

Name of the file to write error information to.
\\{\tt ErrorLog=error.log}

\item {\bf Prefix}={\it prefix} \\ The prefix used for on screen output of AMPS  \\ {\tt Prefix=AMPS}

\item {\bf DiagnosticStream}=[screen,{\it file-name}] \\ The location of the diagnostic information output. The run-time diagnostic can be printed either on screen or in a file: \\  {\tt DiagnosticStream=amps.log} \\ {\tt DiagnosticStream=screen }  

\item {\bf OutputDirectory}={\it output-directory} \\ Name of the folder to write results to. \\ {\tt OutputDirectory=out}


\end{itemize}

%%%%%%%%%%%%%%%%%%%%%%%%%%%%%%%%%%%%%%%%%%%%%%%%%%%%%%%%%%%%%%%%
\section {Species}

The file contains database of the physical parameters of the model species. During compilation the configuration script extract the parameters only of the species used in the current model runs.

The file must begin with {\tt \#species} and ends with {\tt \#endspecies}. The following are other commands 

\begin{itemize}

\item {\bf \#component}={\it specie-symbol} \\ Defines the symbolic name for the specie \\{\tt \#component=O}
\item {\bf mass}={\it molecule/atom mass-mass} \\ The molecule/atom mass [kg] \\ {\tt mass=16*\_AMU\_} 
\item {\bf charge}={\it electric-charge} \\ Electric charge [e] \\ {\tt charge=0}
\end{itemize}



%%%%%%%%%%%%%%%%%%%%%%%%%%%%%%%%%%%%%%%%%%%%%%%%%%%%%%%%%%%%%%%%
\section{Include}

Parts of the input parameters can be stored in separate files and included when the code is compiling using command {\tt \#include}. 

Example {\tt \#include species.input}.


%%%%%%%%%%%%%%%%%%%%%%%%%%%%%%%%%%%%%%%%%%%%%%%%%%%%%%%%%%%%%%%%
\section{UserDefinitions}
Obsolete but still can be used. Execution of AMPS can be controlled through files "UserDefinition.Exosphere.h", "UserDefinition.meshAMR.h", "UserDefinition.PIC.h", and "UserDefinition.PIC.PhysicalModelHeaderList.h" that contains some settings of the model.

The switch determines which of the files will be included into the sources of AMPS during compiling.

\begin{itemize}
\item {\bf Mesh} = [On/Off]
\item {\bf PIC} = [On/Off]
\end{itemize}




%%%%%%%%%%%%%%%%%%%%%%%%%%%%%%%%%%%%%%%%%%%%%%%%%%%%%%%%%%%%%%%%
\section{General}


%%%%%%%%%%%%%%%%%%%%%%%%%%%%%%%%%%%%%%%%%%%%%%%%%%%%%%%%%%%%%%%%
\section{ParticleCollisions}
The section described the particle collision model used in the simulation. The block has to begin with {\tt \#ParticleCollisions} and end with {\tt endParticleCollisions}. Additional commands:
\begin{itemize}
\item {\bf model} = [off,HS] \\ the name of the model that will be used
\item {\bf SampleCollisionFrequency} = [on,off] \\ the switch turns sampling of the collision frequentcy
\item{\bf CollisionCrossSection}=[const,function] \\ defines a collision cross section \\ {\tt CollisionCrossSection=function {\it function-name}} 

{\tt CollisionCrossSection=const} \\ when constant collision cross section is used its value has to be defined for each pair of species with a constant or a function \\ {\tt const (H2O,H2O)= 2.0E-19} \\ if a collision cross section is not defined for a pair that its value is assumed being zero (no collisions between the species)
\end{itemize}


%%%%%%%%%%%%%%%%%%%%%%%%%%%%%%%%%%%%%%%%%%%%%%%%%%%%%%%%%%%%%%%%
\section{Sampling}
The section controls the non-standard AMPS sampling routines. The section begins with {\tt \#Sampling} and ends with {\tt \#endSampling}.

\begin{itemize}
\item {\bf SampleParallelTangentialKineticTemperature}=[on,off], direction=[] \\
Turn on sampling of the parallel and tangential temperatures. The direction can be defined by a constant vector or a function \\ {\tt direction=const(lx,ly,lx)} \\ {\tt direction=function({\it function-name})}

\end{itemize}



%%%%%%%%%%%%%%%%%%%%%%%%%%%%%%%%%%%%%%%%%%%%%%%%%%%%%%%%%%%%%%%%
\section{IDF}
The section defines the model of the internal degrees of freedom. The section has to begin with {\tt \#IDF} and ends with {\tt \#endIDF}. 

\begin{itemize}
\item {\bf Model}=[off,LB,qLB]
\item {\bf vtRelaxation}=[on,off]
\item {\bf rtRelaxation}=[on,off]
\item {\bf vvRelaxation}=[on,off]
\item {\bf nVibModes} \\ the number of the vibrational modes \\ Example: {\tt nVibModes(CO2=1,H2O=2)} 
\item {\bf nRotModes} \\ the number of the rotational modes \\ Example: {\tt nRotModes(CO2=2,H2O=3)}
\item {\bf RotZnum} \\ the rotational $Z_\nu$ number. $1/Z_\nu$ is the probability of the rotationa-translational energy exchange during a collision event. \\ {\tt RotZnum(H2O=3,CO2=2)}
\item {\bf VibTemp} \\ the characteristic vibrational temperature \\ {\tt  VibTemp(H2O=2000,CO2=1000)}

\end{itemize}




%%%%%%%%%%%%%%%%%%%%%%%%%%%%%%%%%%%%%%%%%%%%%%%%%%%%%%%%%%%%%%%%
\section{UnimolecularReactions}


%%%%%%%%%%%%%%%%%%%%%%%%%%%%%%%%%%%%%%%%%%%%%%%%%%%%%%%%%%%%%%%%
\section{Additional Blocks}

\subsection{Exosphere}








